\section{Proofs of Propositions in Section~\ref{sec:semantics}}
\label{app:semantics_proofs}

% \subsection{Proof of Proposition~\ref{prop:pv}}
% \label{app:proof:pv}

% \proppv*

% \begin{proof}
%     The proof is analogous to that for the semantics of well-defined Bayesian Logic Programs (BLPs)~\cite[Theorem 4.9]{kersting2000bayesian}. They show that such a probability measure exists over a non-empty set of random variables if the ancestor structure of the random variables is acyclic and every random variable has a finite set of ancestors, which are exactly  conditions W2 and W1 in Definition~\ref{def:well-defd-facts}. The key idea is that under these conditions, 
%     for each finite subset of random variables closed under the ancestor relation, the joint distribution on that set has the form of a Bayesian network, and factorizes into the product of the individual variables' distributions. 
%     This family of distributions forms the basis of the unique measure over the potentially infinite set $\randomvariableset$. We refer to \cite[Theorem 4.9]{kersting2000bayesian} for  technical details.  
% \end{proof}

% Note that while BLPs also use LP syntax to define the random variables and structure of a Bayesian network, the way they use that syntax is fundamentally different from ours.



\subsection{Proof of Proposition~\ref{prop:omegaf}}
\label{app:proof:omegaf}

\new{

\propomegaf*


\begin{proof}
    Consider the set of comparison atoms $\comparisonfacts$=   $\{ \kappa_1,  \kappa_2, \dots\}$. 
    Each $\kappa_i$ depends on a finite subset $\randomvariableset_i$ of random variables, namely  those mentioned in $\kappa_i$.
    We write $\randomvariableset_{\leq n}=\bigcup_{1\leq j\leq n} \randomvariableset_j $ for the union of random variables that the first $n$ atoms in the enumeration depend on.
    We obtain the set of all random variables from the following limit:
    \begin{align}
        \mathcal{V}_\comparisonfacts = \lim_{n\rightarrow \infty} \mathcal{V}_{\leq n}.
    \end{align}
    We construct the sample space of $\randomvariableset_{\comparisonfacts}$  with a (countable) Cartesian product
    \begin{align}
        \samplespace_{\comparisonfacts}=\prod_{\nu \in \mathcal{V}_{\comparisonfacts}} \samplespace_\nu.
    \end{align}
\end{proof}
}




\subsection{Proof of Proposition~\ref{prop:pfsigma}}
\label{app:proof:pfsigma}



\new{



\proppfsigma*





\begin{proof}
    We construct the following cylinder set for each comparison atom in the set $\comparisonfacts=\{ \kappa_1,  \kappa_2, \dots \}$:
    \begin{align}
        K_j &= \{  \omega \in \samplespace_{\comparisonfacts} \mid \kappa_j(\omega)=\top \}.
        % \\
        % K_k^\bot &= \{  \omega \in \samplespace_{\comparisonfacts} \mid \mu_k(\omega)=\bot \}.
    \end{align}
    Here we use $\kappa_j(\omega)$ to explicitly denote the evaluation of the comparison atom at $\omega$. We denote the set of all such cylinder sets by $\mathcal{K}_{\comparisonfacts}= \lim_{n\rightarrow \infty} \bigcup_{j=1}^n K_j$.
    
    Finally, we form the sigma-algebra $\Sigma_{\comparisonfacts}$ as the sigma-algebra generated by the collection of cylinder sets $\mathcal{K}_{\comparisonfacts}$:
    \begin{align}
        \Sigma_{\comparisonfacts} = \sigma(\mathcal{K}_{\comparisonfacts}).
    \end{align}
    where $\sigma(\mathcal{K}_{\comparisonfacts})$ denotes the intersection of all sigma-algebras containing $\mathcal{K}_{\comparisonfacts}$.
    % Given that $\mathcal{U}_{\leq n}$ is closed under countable unions, intersection, and complements, $\Sigma_{\comparisonfacts}$ is also the smallest sigma-algebra.
    Given that $\mathcal{K}_{\comparisonfacts} \subseteq \sigmaalgebra_\distdb$ we also have that $\sigmaalgebra_\comparisonfacts \subseteq \sigmaalgebra_\distdb$.
    % This concludes the proof as we provide an explicit construction of $ \Sigma_{\comparisonfacts}$.
\end{proof}







    % If we now associate to each element in $\mathcal{K}_{\leq n}$ an element from the index set $I_{\leq n}= \{1, \dots, 2n  \}$, we can construct a larger collection of cylinder sets using the powerset of $I_{\leq n}$ (denoted by $\mathcal{I}_{\leq n}$):
    % \begin{align}
    %     \mathcal{U}_{\leq n }
    %     &=
    %     \bigcup_{J \in \mathcal{I}_{\leq n}  }
    %     \bigcap_{j \in J} 
    %     \bigg\{ \omega \in \samplespace_{\leq n} \bigg| \omega \in \mathcal{K}_{\leq n,j}  \bigg\}
    %     \\
    %     &=
    %     \bigcup_{J \in \mathcal{I}_{\leq n}  }
    %     \bigcap_{j \in J} 
    %     \bigg\{  \omega \in \mathcal{K}_{\leq n,j}  \bigg\},
    % \end{align}
    % where $\mathcal{K}_{\leq n,j}$ denotes an element from $\mathcal{K}_{\leq n}$.
    % Note that the intersection in the equation above denotes a set intersection and the union denotes a collection union. Moreover, as a finite intersection of cylinder sets is again a cylinder set we can rewrite the equation above as:



    % \begin{align}
    %     \mathcal{U}_{\leq n }
    %     =
    %     \bigcup_{J \in \mathcal{I}_{\leq n}  }
    %     \bigg\{ \omega \in  \bigwedge_{j \in J} \mathcal{K}_{\leq n, j}  \bigg\}, 
    % \end{align}
    % where $\bigwedge_{j \in J} \mathcal{K}_{\leq n, j}$ is a cylinder set for every $J \in \mathcal{I}_{\leq n}$.

    % It is now straightforward to show that the $\mathcal{U}_{\leq n }$ is closed under countable union, intersection and complements. This can easily be seen by realizing that our construction of $\mathcal{U}_{\leq n}$ makes sure that all the $2^n$ cylinder sets  $\{\omega \in \samplespace \mid \mu_1(\omega)=b_1, \dots , \mu_n(\omega)=b_n \}$ with $b_i\in \{\bot, \top\}$ are included in the collection, as well as arbitrary subsets thereof, \ie cylinder sets with arbitrarily ordered comparison atoms $\mu_i$ with a number of comparison atoms less or equal to $n$.




}


\subsection{Proof of Proposition~\ref{prop:pf}}
\label{app:proof:pf}


\new{

\proppf*

\begin{proof}
    To show existence of the  measure $\measurecomparisonfacts$, we need to show that 
    \begin{enumerate}
        \item non-negativity: $\probabilitymeasure_\comparisonfacts(A)\geq0 , \quad \forall A \in \sigmaalgebra_\comparisonfacts$
        \item normality: $\probabilitymeasure_\comparisonfacts(\samplespace_\comparisonfacts)=1$
        \item countably additivity: for any collection $\{A_i\}_{i=1}^\infty$ of disjoint sets in $\sigmaalgebra_\comparisonfacts$ we have
        \begin{align}
            \probabilitymeasure_\comparisonfacts
            \left(
                \bigcup_{i=1}^\infty A_i
            \right)
            =
            \sum_{i=1}^\infty \probabilitymeasure_\comparisonfacts(A_i)
        \end{align} 
    \end{enumerate}
    Using the fact that $\sigmaalgebra_\comparisonfacts\subseteq \sigmaalgebra_\distdb$ it is straightforward to show these three properties hold. Uniqueness of $\probabilitymeasure_\comparisonfacts$ is also inherited from the uniqueness of $\probabilitymeasure_\distdb$.
\end{proof}


}


% we fix an arbitrary enumeration  $\mu_1$, $\mu_2$, \dots, of the atoms in $\comparisonfacts$. 
% Each $\mu_i$ \emph{depending} on a subset $\randomvariableset_i\subseteq \randomvariableset$ of random variables, namely  those mentioned in $\mu_i$, as well as their ancestor sets. We write $\randomvariableset_{\leq n}=\bigcup_{1\leq j\leq n} \randomvariableset_j $ for the union of random variables that the first $n$ atoms in the enumeration depend on. By $\probabilitymeasure_{\randomvariableset_{\leq n}}$ we denote the measure restricted to this set.  

% By definition, all queries $\mu_i\in \comparisonfacts$ are  Lebesgue-measurable, and we thus get a family of distributions
% \begin{align}
% \measurecomparisonfacts^{(n)}(\mu_1=\boolval_1,\ldots,\mu_n=\boolval_n) = \int_{\samplespace(\randomvariableset_{\leq n})} \mathbf{1}_{[\mu_1=\boolval_1\wedge\ldots\wedge \mu_n=\boolval_n]}(\samplefunction(\randomvariableset_{\leq n})) \differential{\probabilitymeasure_{\randomvariableset_{\leq n}}}
% \end{align}
% where the $\boolval_i$ belong to the set  $\{ \bot, \top\}$, $\probabilitymeasure_{\randomvariableset_{\leq n}}$ factorizes over the random variables in $\randomvariableset_{\leq n}$, $\samplespace(\randomvariableset_{\leq n})$ denotes the space of possible assignments for variables in $\randomvariableset_{\leq n}$, and $\mathbf{1}_{[\varphi]}$ is the indicator function, \ie, equals $1$ if $\varphi$ is true and $0$ otherwise. The definition in terms of an indicator function and the measurability of the underlying Boolean queries ensures that this family of distributions is of the form required for the distribution semantics, \ie they are well-defined probability distributions satisfying the compatibility condition: $\measurecomparisonfacts^{(n)}$ can be obtained from $\measurecomparisonfacts^{(n+1)}$ by summing out $\mu_{n+1}$. There thus exists a completely additive probability measure $\measurecomparisonfacts$ over the space of truth value assignments to $\comparisonfacts$ such that for any $n$, we have
% \begin{align}
% \measurecomparisonfacts(\mu_1=\boolval_1,\ldots,\mu_n=\boolval_n) =\measurecomparisonfacts^{(n)}(\mu_1=\boolval_1,\ldots,\mu_n=\boolval_n)
% \end{align}





% \begin{proof}
%     We fix an arbitrary enumeration  $\langle \mu_1, \mu_2, \dots \rangle$, of the atoms in $\comparisonfacts$.
%     Each $\mu_i$ \emph{depending} on a subset $\randomvariableset_i\subseteq \randomvariableset$ of random variables, namely  those mentioned in $\mu_i$, as well as their ancestor sets. We write $\randomvariableset_{\leq n}=\bigcup_{1\leq j\leq n} \randomvariableset_j $ for the union of random variables that the first $n$ atoms in the enumeration depend on.
%     Furthermore, let $I_{\leq n}$ be the finite index set for the $n$ first atoms,  and $\mathcal{I}_{\leq n}$ its powerset.  


%     For an element $K\in \mathcal{I}_{\leq n}$ we define the following cylinder set:
%     \begin{align}
%         Cyl(K) = \{ \omega \in \samplespace_\randomvariableset \mid \forall k \in K: \mu_k(\omega)=\top \},  \quad K\in \mathcal{I}_{\leq n}  
%     \end{align}
%     where $\mu_k(\omega)$ evaluates the $k$-th comparison atom for the variable assignment $\omega$. We denote the collection of all such cylinder sets as follows:
%     \begin{align}
%         \mathcal{U}_{\leq n} = \bigcup_{K\in \mathcal{I}_{\leq n}} Cyl(K)
%     \end{align}
    

    
    

    

        
% \end{proof}






% \begin{proof}
%     We fix an arbitrary enumeration  $\langle \mu_1, \mu_2, \dots \rangle$, of the atoms in $\comparisonfacts$.
%     Each $\mu_i$ \emph{depending} on a subset $\randomvariableset_i\subseteq \randomvariableset$ of random variables, namely  those mentioned in $\mu_i$, as well as their ancestor sets. We write $\randomvariableset_{\leq n}=\bigcup_{1\leq j\leq n} \randomvariableset_j $ for the union of random variables that the first $n$ atoms in the enumeration depend on.
%     Furthermore, let $I_n$ be the finite index set for the $n$ first atoms,  and $\mathcal{I}_n$ its powerset.  

%     Let $I$ be an index set and $J\subseteq I$ be a finite subset. For each $j\in J$, let $A_j \subseteq \samplespace_j$. A cylinder set in $\samplespace$ is a set of the form:

%     For an element $i\in \mathcal{I}_n$ we define the following cylinder set:
%     \begin{align}
%         Cyl(i) = \{ \omega \in \samplespace_\randomvariableset \mid \in J: \mu_j(\omega)=\top \}    
%     \end{align}

    

%     we fix an arbitrary enumeration  $\langle \mu_1, \mu_2, \dots \rangle$, of the atoms in $\comparisonfacts$. 
%     Each $\mu_i$ \emph{depending} on a subset $\randomvariableset_i\subseteq \randomvariableset$ of random variables, namely  those mentioned in $\mu_i$, as well as their ancestor sets. We write $\randomvariableset_{\leq n}=\bigcup_{1\leq j\leq n} \randomvariableset_j $ for the union of random variables that the first $n$ atoms in the enumeration depend on.
    
%     We can now write the set of assignments to the random variables in $\randomvariableset_{\leq n}$ that respect the first $n$ comparison facts as:
%     \begin{align}
%         Cyl(\randomvariableset_{\leq n})
%         =
%         \{\omega \in \samplespace(\randomvariableset_{\leq n}) \mid \mu_1(\omega)=\top  \land \dots \land \mu_n(\omega) = \top  \}
%     \end{align}
    
    
%     We then denote by
%     \begin{align}
%         \boolval^{w_n} = \langle \boolval^{w_n}_1, \dots, \boolval^{w_n}_n \rangle \in \{\bot,\top\}^n, \quad 1\leq {w_n} \leq 2^n
%     \end{align}
%     the enumeration of Boolean values the $\mu_i$ take. We can now write the set of assignments to the random variables in $\randomvariableset_{\leq n}$ that respect the 

%     \begin{align}
%         Cyl(\randomvariableset_{\leq n}, \boolval^{w_n})
%         =
%         \{\omega \in \samplespace(\randomvariableset_{\leq n}) \mid \mu_1(\omega)=\boolval^{w_n}_1  \land \dots \land \mu_n(\omega) = \boolval^{w_n}_n  \}
%     \end{align}
    
%     The collection of all cylinder sets is:
%     \begin{align}
%         \mathcal{U}_\mathcal{V} =  \bigcup_{j=0}^n \bigcup_{w_j=1}^{2^j} Cyl(\randomvariableset_{\leq j}, \boolval^{w_j})
%     \end{align}
%     Here we also include the case $j=0$ in order to include the empty set. 
    
%     \begin{align}
%         \sigmaalgebracomparisonfact
%         =
%         \sigma (\mathcal{U}_\mathcal{V})
%     \end{align}
    
    

        
% \end{proof}







\subsection{Proof of Proposition~\ref{prop:pp}}
\label{app:proof:pp}

\proppp*


\new{
\begin{proof}
% To show this, we consider two cases. 
% If $\distdb$ is empty, \ie \dfprogram does not define any random variables,  the semantics of \dfprogram is the well-founded model of $\logicprogram$. Thus, normal logic programs (with total well-founded models) are a special case of \dfplpsty.

To show this, we follow Sato's construction to obtain the probability measure $\probabilitymeasure_\dfprogram$ over Herbrand interpretations from $\measurecomparisonfacts$. 
To this end we denote the set of atoms in the Herbrand base by $\mu_1, \mu_2, \dots$, which also includes those in $\comparisonfacts$.
As \dfprogram is valid, for every consistent comparison database $\comparisonfacts_{\samplefunction(\randomvariableset)}$ (\cf Definition~\ref{def:consistent-fact-db}), the logic program
$\comparisonfacts_{\samplefunction(\randomvariableset)}\cup \logicprogram$
has a total well-founded model $M_{\samplefunction(\randomvariableset)}$, and we can define   
\begin{align}
\probabilitymeasure_\dfprogram(\mu_1=\boolval_1,\mu_2=\boolval_2, \ldots)
:=
\measurecomparisonfacts
    \left(
    \left\{
    \samplefunction(\randomvariableset)
        ~|~
        M_{\samplefunction(\randomvariableset)}
    \right\}
    \right)
\end{align}
% It follows again that there is a completely additive probability measure $\probabilitymeasure_\dfprogram$ over Herbrand interpretations.
What remains, is to show that the set
$
    \left\{
    \samplefunction(\randomvariableset)
        ~|~
        M_{\samplefunction(\randomvariableset)}
    \right\}
$
is an element of the sigma-algebra $\sigmaalgebra_\comparisonfacts$. To this end, we rewrite the set as:

\begin{align}
    \Bigl\{
    \samplefunction(\randomvariableset)
        ~|~
        \mu_1(\samplefunction(\randomvariableset))=\boolval_1\land \mu_2(\samplefunction(\randomvariableset))=\boolval_2\land \ldots
    \Bigr\}
    \label{eq:proof:prop:semantics}
\end{align}
where we have that, 
\begin{align}
    M_{\samplefunction(\randomvariableset)} \models
    \mu_1(\samplefunction(\randomvariableset))=\boolval_1\land \mu_2(\samplefunction(\randomvariableset))=\boolval_2\land \ldots
\end{align}
We rewrite the set in Equation~\ref{eq:proof:prop:semantics} as


\begin{align}
    \bigcap_{j=1}^n
    \left\{
    \samplefunction(\randomvariableset)
        ~|~
        \mu_j(\samplefunction(\randomvariableset))={\boolval_j}       
    \right\}.
\end{align}
We now retain only those $ \mu_j$'s that depend on (a subset of) $\randomvariableset$, which we denote by $ \kappa_j$:
\begin{align}
    \bigcap_{j=1}
    \left\{
        \samplefunction(\randomvariableset)
            ~|~
            \kappa_j(\samplefunction(\randomvariableset))={\boolval_j}      
        \right\}.
\end{align}
The last line is an intersection of elements from $\comparisonfacts$ (or their complements) and thereby trivially part of $\sigmaalgebra_\comparisonfacts$, which concludes the proof.
\end{proof}

}
