%\documentclass[preprint]{elsarticle}
\documentclass[final]{elsarticle}

\makeatletter
\def\ps@pprintTitle{%
 \let\@oddhead\@empty
 \let\@evenhead\@empty
 \def\@oddfoot{}%
 \let\@evenfoot\@oddfoot}
\makeatother

\usepackage{lineno}
\modulolinenumbers[5]

\usepackage{hyperref}






%% `Elsevier LaTeX' style
% \bibliographystyle{elsarticle-num}
\bibliographystyle{plainnat}
% \biboptions{authoryear}



\newcommand\hmmax{0}
\newcommand\bmmax{0}





%%%%%%%%%%%%%%%%%%%%%%
\usepackage{amsmath}
\usepackage{amssymb,amsthm}

\usepackage{stmaryrd}
\usepackage[noend]{algpseudocode}
\usepackage[dvipsnames,table,xcdraw]{xcolor}
\usepackage{xspace}
% \usepackage[cachedir=minted-cache,newfloat]{minted}
% \usepackage[finalizecache=true,cachedir=minted-cache,newfloat]{minted}
\usepackage[frozencache=true,cachedir=minted-cache,newfloat]{minted}
% \usepackage[utf8]{inputenc}
\usepackage{caption}
\usepackage{listings}
\usepackage{nicefrac}
\usepackage{multirow}
\usepackage{textcomp} % nice greek alphabet
\usepackage{textgreek}
\usepackage{changepage}
\usepackage{cancel}
\usepackage{subcaption}
\usepackage{tikz}
\usetikzlibrary{arrows, calc,shapes.geometric,shapes.multipart,positioning,tikzmark}
\usetikzlibrary{arrows,shapes,trees}
\usepackage{enumitem}
\usepackage{multicol}
\usepackage{bm}

\usepackage[T1]{fontenc}
\usepackage{txfonts}
\usepackage{mdframed}
\usepackage{scrextend}
\usepackage{blindtext}
\usepackage[section]{placeins}

\usepackage{xassoccnt}
\usepackage{chngcntr}
\usepackage{thm-restate}
\usepackage{xpatch}
\usepackage{fixmath}
\usepackage{soul}

\usepackage[htt]{hyphenat}

\hyphenation{data-base ground-ers}

\usepackage[linesnumbered,ruled,vlined,algosection]{algorithm2e}
\DontPrintSemicolon
\SetKwInput{KwInput}{Input}                % Set the Input
\SetKwInput{KwOutput}{Output}              % set the Output
\newcommand\mycommfont[1]{\footnotesize\ttfamily\textcolor{gray}{#1}}
\SetCommentSty{mycommfont}

% \usepackage[bbgreekl]{mathbbol}



\graphicspath{{figures/}}

%%%%%%%%%%%%%%%%%%%%%%
\usepackage[utf8]{inputenc} % allow utf-8 input
\usepackage[T1]{fontenc}    % use 8-bit T1 fonts
\usepackage{hyperref}       % hyperlinks
\usepackage{url}            % simple URL typesetting
\usepackage{booktabs}       % professional-quality tables
\usepackage{amsfonts}       % blackboard math symbols
\usepackage{nicefrac}       % compact symbols for 1/2, etc.
\usepackage{microtype}      % microtypography
\usepackage{xcolor}         % colors

\usepackage{multirow}
\usepackage{courier}
\usepackage{listings, lstautogobble,amsfonts}
\usepackage{amsmath,amssymb,amsthm}
\usepackage{mdframed}
\usepackage{mathtools}
\usepackage{xspace}
\usepackage{xcolor}
\usepackage{caption}
\usepackage{multicol}
\usepackage{thmtools}
\usepackage{bm}
\usepackage{thm-restate}
\usepackage[inline]{enumitem}
\usepackage{soul}
\usepackage{physics}
\usepackage{caption}
\usepackage{dirtytalk}
\usepackage{enumitem}
\usepackage{mathrsfs}
\usepackage{stmaryrd}
% \usepackage{tikz}
\usetikzlibrary{decorations.pathreplacing,calligraphy,calc,hobby,intersections,through}
\usetikzlibrary{spn}
\pgfplotsset{compat=1.18}

\usepackage{nicefrac}
\usepackage{bbold}
\usepackage{wrapfig}
\usepackage{multicol}

%%%%%%%%%%%%%%%%%%%%%%%%%%%%%%%%%%%%%%%%%%%%%%%%%%%%%%%%%%%%%%%%%%%%%%%%%%%%%%%%%
%%%%%%%%%%%%%%%%%%%%%%%%%%%%%%%%%%%%%%%%%%%%%%%%%%%%%%%%%%%%%%%%%%%%%%%%%%%%%%%%%
%%%%%%%%%%%%%%%%%%%%%%%%%%%%%%%%%%%%%%%%%%%%%%%%%%%%%%%%%%%%%%%%%%%%%%%%%%%%%%%%%
%%%%%%%%%%%%%%%%%%%%%%%%%%%%%%%%%%%%%%%%%%%%%%%%%%%%%%%%%%%%%%%%%%%%%%%%%%%%%%%%%

\newcommand{\cf}{cf.\xspace}
\newcommand{\eg}{e.g.\xspace}
\newcommand{\ie}{i.e.\xspace}

\newcommand{\todo}[1]{\textcolor{red}{\textbf{TODO: #1}}}


%%%%%%%%%%%%%%%%%%%%%%%%%%%%%%%%%%%%%%%%%%%%%%%%%%%%%%%%%%%%%%%%%%%%%%%%%%%%%%%%%
%%%%%%%%%%%%%%%%%%%%%%%%%%%%%%%%%%%%%%%%%%%%%%%%%%%%%%%%%%%%%%%%%%%%%%%%%%%%%%%%%
%%%%%%%%%%%%%%%%%%%%%%%%%%%%%%%%%%%%%%%%%%%%%%%%%%%%%%%%%%%%%%%%%%%%%%%%%%%%%%%%%
%%%%%%%%%%%%%%%%%%%%%%%%%%%%%%%%%%%%%%%%%%%%%%%%%%%%%%%%%%%%%%%%%%%%%%%%%%%%%%%%%

% \newtheorem{theorem}{Theorem}[section]
% \theoremstyle{theorem}
% \newtheorem{definition}[theorem]{Definition}
% \theoremstyle{theorem}
% \newtheorem{lemma}[theorem]{Lemma}
% \theoremstyle{theorem}
% \newtheorem{corollary}[theorem]{Corollary}
% \theoremstyle{theorem}
% \newtheorem{proposition}[theorem]{Proposition}
\theoremstyle{theorem}
\newtheorem{example}[theorem]{Example}
% \theoremstyle{theorem}
\newtheorem{conjecture}[theorem]{Conjecture}




%%%%%%%%%%%%%%%%%%%%%%%%%%%%%%%%%%%%%%%%%%%%%%%%%%%%%%%%%%%%%%%%%%%%%%%%%%%%%%%%%
%%%%%%%%%%%%%%%%%%%%%%%%%%%%%%%%%%%%%%%%%%%%%%%%%%%%%%%%%%%%%%%%%%%%%%%%%%%%%%%%%
%%%%%%%%%%%%%%%%%%%%%%%%%%%%%%%%%%%%%%%%%%%%%%%%%%%%%%%%%%%%%%%%%%%%%%%%%%%%%%%%%
%%%%%%%%%%%%%%%%%%%%%%%%%%%%%%%%%%%%%%%%%%%%%%%%%%%%%%%%%%%%%%%%%%%%%%%%%%%%%%%%%

\newcommand{\expect}{\ensuremath{\mathbb{E}}}

\newenvironment{talign}
{\let\displaystyle\textstyle\align}
{\endalign}


%%%%%%%%%%%%%%%%%%%%%%%%%%%%%%%%%%%%%%%%%%%%%%%%%%%%%%%%%%%%%%%%%%%%%%%%%%%%%%%%%
%%%%%%%%%%%%%%%%%%%%%%%%%%%%%%%%%%%%%%%%%%%%%%%%%%%%%%%%%%%%%%%%%%%%%%%%%%%%%%%%%
%%%%%%%%%%%%%%%%%%%%%%%%%%%%%%%%%%%%%%%%%%%%%%%%%%%%%%%%%%%%%%%%%%%%%%%%%%%%%%%%%
%%%%%%%%%%%%%%%%%%%%%%%%%%%%%%%%%%%%%%%%%%%%%%%%%%%%%%%%%%%%%%%%%%%%%%%%%%%%%%%%%

\newcommand{\circuit}{\ensuremath{\xi}}


\newcommand{\pcircuit}{\ensuremath{p}}
\newcommand{\Pcircuit}{\ensuremath{P}}


\newcommand{\ocircuit}{\ensuremath{o}}
\newcommand{\Ocircuit}{\ensuremath{O}}

% \newcommand{\vcircuit}{\ensuremath{v}}
\newcommand{\Vcircuit}{\ensuremath{V}}


\newcommand{\Qcircuit}{\ensuremath{Q}}
\newcommand{\tildeQcircuit}{\ensuremath{\widetilde{Q}}}

\newcommand{\qcircuit}{\ensuremath{q}}



\newcommand{\qop}{\ensuremath{\Phi}}
\newcommand{\kraus}{\ensuremath{K}}


\newcommand{\diagmat}{\ensuremath{\text{diagmat}}}
\newcommand{\diagvec}{\ensuremath{\text{diagvec}}}






\newcommand{\inputs}{\ensuremath{\text{in}}}
\newcommand{\scope}{\ensuremath{\phi}}
\newcommand{\component}{\ensuremath{{\kappa}}}



\newcommand{\Xvars}{\ensuremath{\mathbf{X}}}
\newcommand{\xvars}{\ensuremath{\mathbf{x}}}
\newcommand{\Xvar}{\ensuremath{X}}
\newcommand{\xvar}{\ensuremath{x}}

\newcommand{\Yvars}{\ensuremath{\mathbf{Y}}}
\newcommand{\yvars}{\ensuremath{\mathbf{y}}}
\newcommand{\Yvar}{\ensuremath{Y}}
\newcommand{\yvar}{\ensuremath{y}}


\newcommand{\Zvars}{\ensuremath{\mathbf{Z}}}
\newcommand{\zvars}{\ensuremath{\mathbf{z}}}
\newcommand{\Zvar}{\ensuremath{Z}}
\newcommand{\zvar}{\ensuremath{z}}


\newcommand{\bigO}{\ensuremath{\mathcal{O}}}


\newcommand{\weight}{\ensuremath{w}}
\newcommand{\nweight}{\ensuremath{\omega}}
\newcommand{\nparams}{\ensuremath{\mathbf{\theta}}}


\newcommand{\iverson}[1]{\llbracket#1\rrbracket}


\newcommand{\numevents}{I}
\newcommand{\numvar}{N}
\newcommand{\numbond}{B}
\newcommand{\numcomponents}{C}
\newcommand{\samplespacesize}{S}
\newcommand{\subcompletemeasrure}{M}


\newcommand{\msocs}{$\mu$SOCS\xspace}
\newcommand{\msocss}{$\mu$SOCSs\xspace}



%%%%%%%%%%%%%%%%%%%%%%%%%%%%%%%%%%%%%%%%%%%%%%%%%%%%%%%%%%%%%%%%%%%%%%%%%%%%%%%%%
%%%%%%%%%%%%%%%%%%%%%%%%%%%%%%%%%%%%%%%%%%%%%%%%%%%%%%%%%%%%%%%%%%%%%%%%%%%%%%%%%
%%%%%%%%%%%%%%%%%%%%%%%%%%%%%%%%%%%%%%%%%%%%%%%%%%%%%%%%%%%%%%%%%%%%%%%%%%%%%%%%%
%%%%%%%%%%%%%%%%%%%%%%%%%%%%%%%%%%%%%%%%%%%%%%%%%%%%%%%%%%%%%%%%%%%%%%%%%%%%%%%%%

\newcommand{\smalllinewidth}{0.6pt}
\newcommand{\midlinewidth}{1.0pt}
\newcommand{\midlinewidthx}{2.0pt}
\newcommand{\largelinewidth}{1.7pt}
\newcommand{\middist}{24pt}
\newcommand{\middistt}{20pt}
\newcommand{\middisttt}{28pt}
\newcommand{\largedist}{30pt}
\newcommand{\hugedist}{50pt}
\newcommand{\smalldist}{20pt}
\newcommand{\smalldistt}{4pt}
\newcommand{\tinydist}{5pt}
\newcommand{\intermiddist}{30pt}
\newcommand{\sqintermiddist}{15.5pt}
\newcommand{\halfdist}{4pt}
%%%%%%%%%%%%%%%%%%%%%%%%%%%%%%%%%%%%%%%%%%%%%%%%%%%%%%%%%%%%%%%%%%%%%%%%%%%%%%%%%
%%%%%%%%%%%%%%%%%%%%%%%%%%%%%%%%%%%%%%%%%%%%%%%%%%%%%%%%%%%%%%%%%%%%%%%%%%%%%%%%%
%%%%%%%%%%%%%%%%%%%%%%%%%%%%%%%%%%%%%%%%%%%%%%%%%%%%%%%%%%%%%%%%%%%%%%%%%%%%%%%%%
%%%%%%%%%%%%%%%%%%%%%%%%%%%%%%%%%%%%%%%%%%%%%%%%%%%%%%%%%%%%%%%%%%%%%%%%%%%%%%%%%

% Redefine the proof environment to modify margins
\makeatletter
\renewenvironment{proof}[1][\proofname]{\par
	\pushQED{\qed}%
	\normalfont\topsep0pt \partopsep0pt % Adjust the vertical spacing above
	\trivlist
	\item[\hskip\labelsep
	            \itshape
	            #1\@addpunct{.}]\ignorespaces
}{%
	\popQED\endtrivlist\@endpefalse
	\vskip 1ex  % Add some flexible glue for the bottom margin
}
\makeatother
%%%%%%%%%%%%%%%%%%%%%%%%%%%%%%%%%%%%%%%%%%%%%%%%%%%%%%%%%%%%%%%%%%%%%%%%%%%%%%%%%
%%%%%%%%%%%%%%%%%%%%%%%%%%%%%%%%%%%%%%%%%%%%%%%%%%%%%%%%%%%%%%%%%%%%%%%%%%%%%%%%%
%%%%%%%%%%%%%%%%%%%%%%%%%%%%%%%%%%%%%%%%%%%%%%%%%%%%%%%%%%%%%%%%%%%%%%%%%%%%%%%%%
%%%%%%%%%%%%%%%%%%%%%%%%%%%%%%%%%%%%%%%%%%%%%%%%%%%%%%%%%%%%%%%%%%%%%%%%%%%%%%%%%

\newcommand{\poc}{{\color{red}POX\xspace}}
\newcommand{\pocs}{{\color{red}POXs\xspace}}
\newcommand{\pvc}{{\color{red}PVX\xspace}}
\newcommand{\pvcs}{{\color{red}PVXs\xspace}}
\newcommand{\snpc}{NPC\textsuperscript{2}\xspace}
\newcommand{\snpcs}{NPC\textsuperscript{2}s\xspace}
\newcommand{\smpc}{MPC\textsuperscript{2}\xspace}
\newcommand{\smpcs}{MPC\textsuperscript{2}s\xspace}



\newcommand{\punc}{PUnC\xspace}
\newcommand{\dpunc}{D-PUnC\xspace}
\newcommand{\sdpunc}{SD-PUnC\xspace}
\newcommand{\puncs}{PUnCs\xspace}
\newcommand{\dpuncs}{D-PUnCs\xspace}
\newcommand{\sdpuncs}{SD-PUnCs\xspace}

\newcommand{\noisepunc}{NoisePUnC\xspace}
\newcommand{\noisepuncs}{NoisePUnCs\xspace}




\newcommand{\icol}[1]{% inline column vector
	\left(\begin{smallmatrix}#1\end{smallmatrix}\right)%
}

\newcommand{\irow}[1]{% inline row vector
	\begin{smallmatrix}(#1)\end{smallmatrix}%
}
%%%%%%%%%%%%%%%%%%%%%%%







\numberwithin{equation}{section}


%Commands definitions
\newcommand{\setbackgroundcolour}{\pagecolor[rgb]{0.19,0.19,0.19}}
\newcommand{\settextcolour}{\color[rgb]{0.77,0.77,0.77}}
\newcommand{\invertbackgroundtext}{\setbackgroundcolour\settextcolour}

%Command execution. 
%If this line is commented, then the appearance remains as usual.
% \invertbackgroundtext

\setlist[itemize]{itemsep=0pt}
\setlist[enumerate]{itemsep=0pt}


\begin{document}


\setcounter{page}{1}
\renewcommand{\thepage}{\roman{page}}

\setcounter{section}{0}
\renewcommand{\thesection}{R\arabic{section}}

\renewcommand{\theequation}{R\arabic{section}.\arabic{equation}}

% 



\section*{ \Huge Rebuttal}
Firstly, we would like to thank all reviewers for taking their time and giving such detailed comments. We understand that our manuscript is rather long and that reviewing it takes considerable effort. We would like to stress that we found the reviews detailed and very helpful in improving the manuscript. Please find below our detailed rebuttal.



\section*{General Comments}

In the following sections we address the points raised by the reviewers one by one. We highlight in \fixedrebuttal{green} where we agree with the reviewers' opinion and where we made adequate changes in the manuscript. We highlight in \commentrebuttal{blue} parts where we do not necessarily agree with a certain opinion and give our justification. Within the paper there are passages that are highlighted in \new{purple}, which indicates new material not present in the previous version of the paper.

We would first like to highlight the three major changes made to the manuscript compared to the initial submission:
\begin{enumerate}
    \item As asked by Reviewer 1 we have added a proof that the $sigma$-algbera of the probability measure $P_F$ is indeed (as suggested by the reviewer) a cylinder measure. This let (in our opinion) to a more streamlined exposition of the semantics. In the draft of the paper this manifests itself by the three new Propositiona \ref{prop:omegaf} \ref{prop:pfsigma} and \ref{prop:pf}.
    \item We have relegated large parts of the syntactic sugar discussion to the appendix (as suggested by Reviewer 3). We hope this increases readability and accessibility of the manuscript. Specifically, we only introduce the syntactic sugar by example and discuss the technical details in the appendix.
    \item We have added experimental evidence (\cf Section~\ref{sec:experimental}) that our proposed algorithm for the discrete-continuous domain ( (Symbolic IALW)) results in improvements over state-of-the-art approaches.
\end{enumerate}


\section{Reviewer 1}


\subsection{Sigma Algebra}
Overall I believe the paper deserves publication but there is one point that should be clarified: in the proof of
proposition 3.20 that a valid DF-PLP program induces a unique probability distribution over Herbrand interpretations,
eq (C.3), it is not obvious that the set $\{F_\omega|M_\omega(V)\models \mu_1^{b_1} \wedge ... \wedge \mu_n^{b_n}\}$ is
measurable according to $P_F$. In proposition 3.15 the authors proved that $P_F$ exists and is unique but provide no
information on the $\sigma$-algebra of the probability measure. I think the algebra of $P_F$ is the sigma algebra of
cylinder sets of value assignments to $F$ but in this case this is not the powerset so it is not evident that
the set above belongs to the sigma algebra of $P_F$. This should be proved.

\fixedrebuttal{We agree with the reviewer that this was missing. We took the opportunity and restructure the construction of the probability space (using the new Definition~\ref{def:distDB}), which then led to a more explicit construction of the meaure $\probabilitymeasure_\comparisonfacts$. See also our first comment in the general comments above. }

\subsection{Minor comments}

\begin{enumerate}
    \item Example 5.3 and following: why not using a program clause instead of the mathematical definitions of $\nu_1$ and$ \nu_2$?
    \\
    \commentrebuttal{We strictly restricted the clause notation to writing explcit programs. We used math notation everywhere else. This is due to the fact that using clause notation makes it occasionally hard to read formal expression, especially when we make use of indexing.}

    \item Example 5.4: $\eta=1-> \eta=\top$
    \\
    \fixedrebuttal{Fixed.}
    \item Note 3: for a program to be range restricted, all the variables in the head should occur in POSITIVE literals in the body.
    \\
    \fixedrebuttal{Fixed.}
    \item Line 802: Definition 4.14 repeated
    \\
    \fixedrebuttal{Fixed.}
    \item Eq 7.3: I believe $N$ should be $|S|$
    \\
    \fixedrebuttal{Fixed.}
    \item Eq 7.10 $(s,m)->(t,m)$
    \\
    \fixedrebuttal{Fixed.}
    \item line 920 $n>m$ -> n
    \\
    \fixedrebuttal{Fixed.}
    \item line 966 $\phi$->$\varphi$
    \\
    \fixedrebuttal{Fixed.}
    \item line 999: this seems to be removed
    \\
    \fixedrebuttal{Fixed.}
    \item note 5: "(otherwise case in Definition 7.12)"; Def 7.12 does not speak of circuits, why is it mentioned here?
    \\
    \fixedrebuttal{Fixed, Indeed we referenced the wrong definition. We changed the footnote to referencing Definition~\ref{def:sample_labeling_function}.}
    \item Algorithm 7.19: it should return a probability but it actually returns an infinitesimal number
    \\
    \fixedrebuttal{Fixed. We also slightly adapted the text describing the final step of the algorithm (just above Algorithm 7.19.)}
    \item line 1000: computeS
    \\
    \fixedrebuttal{Fixed.}
    \item Figure 7.4: what is the meaning of 1 in \texttt{size\_1(1)} and \texttt{size\_0(1)}? the part in parenthesis does not seem necessary
    \\
    \fixedrebuttal{Fixed. Indeed the part in the parenthesis is not necessary. This was a remnant from a previous version. We removed the parenthesis and its argument.}
    \item Page 44:
    $Eval(5)=\alpha_IALW(1=1)$->$\alpha_{IALW}(size_0=0.4)$
    \\
    \fixedrebuttal{Fixed.}
    \item line 1029 DC-ProbLog->ProbLog
    \\
    \fixedrebuttal{Fixed.}
    \item page 46 Eval formulas: some formulas have equalities with random variables, other equalities with constants
     
    $Eval(4)=..\alpha_{SIALW}(1=1)$->$Eval(4)=..\alpha_{SIALW}(m=1)$
     

     $Eval(2)=..\alpha_{SIALW}(0.3)$->$Eval(2)=..\alpha_{SIALW}(M=1)$
     
     $Eval(5)=\alpha_{IALW}(1=1)$->$\alpha_{IALW}(size_0=0.4)$
     \\
     \fixedrebuttal{Fixed.}
     \item line 1069 inference task repeated
     \\
     \fixedrebuttal{Fixed.}
    \item line 1075 remove "views"
    \\
    \fixedrebuttal{Fixed.}
    \item eq C.3 $F_\omega->F_\omega(V)$
    \\
    \fixedrebuttal{Fixed.}
    \item eq F.11 in the numberator $|S|$ is missing from the limit, in the denominator $N$ should be replaced by $|S|$
    \\
    \fixedrebuttal{Fixed.}
\end{enumerate}

\section{Reviewer 2}





\subsection{Weaknesses}

\begin{enumerate}
    \item The contribution of this work is more conceptual with a focus on extending Problog. It would be much better and more complete if the paper provides concrete evidence of how the extension could benefit real-life applications, and, especially how DC-ProbLog with IALW performs in comparison with related approaches. For example, whether the proposed language can be applied to perform reasoning of mixed variables at a large scale such as images and their labels. There are several attempts with neurosymbolic approach which deserve to be discussed in this paper.

    \fixedrebuttal{We have added an experimental section demonstrating the benefits of the proposed infernce algorithm (\cf Section~\ref{sec:experimental})}

    \fixedrebuttal{In the context of neurosymbolic AI we have already deployed our semantics by developing the neurosymbolic programming language DeepSeaProblog \citep{desmet2023neural}. We discuss this now in the related work section. Note that without the work performed in this paper, developing DeepSeaProblog would have constituted a rather intricate affair.}


    The paper should clarify the complexity of the approach, e.g. whether it is scalable in terms of presentation (space complexity) and inference (time complexity). For example, in example 2.2, if the depth of the relation children-parents-ancestors increases the number of distributional facts would grow exponentially.

    \commentrebuttal{Indeed in Example 2.2, the number of distributional facts would grow exponentially. Hinting at the computational hardness of the problem. Note, however, that here we are intersted in defining the semantics of a Turing complete probabilistic programming language. Meaning that performing inference in \dcproblogsty is actually an undecidable problem. While we do not state this explicitly this follows immediately from \dcproblogsty being Turing-complete. 
        
    In practice, however, we are concerned with programs that will become ground eventually. In this respect we already state explicitly that computing algebraic model counts is \#P-hard (\cf Section~\ref{sec:ALWviaKC}).

    With these two points we believe we have sufficiently delineated the complexity of inference in \dcproblogsty.
    }



    \item The paper's presentation needs to be significantly improved in order to make it more easily comprehensible for a wider range of readers.

    \commentrebuttal{Following the suggestion from Reviewer 3 concerning the syntactic sugar we moved a large part from the main body of the manuscript to the appendix. We hope this improves readability.}

    The numbering is not well-formatted, it is mixed between examples, definitions, propositions, theorems, etc.

    \commentrebuttal{The AIJ author guide does not provide any details on how to number environments (e.g, propositions and theorems)\footnote{\url{https://www.sciencedirect.com/journal/artificial-intelligence/publish/guide-for-authors}}. In our opinion having different environments sharing the same numbering helps to navigate the paper as everything is in order. We would like to note that this is not uncommon. For instance, the template for the "International Conference on Machine Learning" follows a similar convention. }

    Some definitions are not well-written. For example Definition 4.14 is more like a construction/procedure. There are two definitions (3.7 and 4.17) for Parent and Ancestor. Please check other definitions as well.

    \fixedrebuttal{We renamed Definition~\ref{def:df_ancestor} from (Parent, Ancestors) to (Parents and Ancestors of Random Variables). Furthermore, we renamed Definition 4.17 (now Definition~\ref{def:parentancestor2}) from (Parent, Ancestors) to (Parents and Ancestors of Random Terms)}

    \commentrebuttal{While we agree that Definition 4.14 (now Definitino~\ref{def:elim-ad}) and Definition 4.25 (now Definition~\ref{def:adfree-to-core}) have a procedural character we think this is acceptable. Especially, as they define rewrite rules to be applied to a program.}


    

    There are too many definitions.

    \commentrebuttal{We hope that moving parts of the syntactic sugar discussion to the appendix sufficiently reduced the number of definitions.}


    The paper may be difficult for those who do not have a deep background in probabilistic logic programming. It is a good idea to have the "A Panoramic Overview" section early. However, it goes straight to the CD-Problog concept without laying the ground for readers to understand the basis of probabilistic logic programming.

    \fixedrebuttal{Fixed: we now give an introductory example in Section~\ref{sec:introduction} (Example~\ref{example:intro}).}

\end{enumerate}


\subsection{Other comments}

\begin{enumerate}
    \item Abstract: DC-ProbLog --> DC has not been defined yet.
    \\
    % \commentrebuttal{\dcproblogsty is the name of the language and therefore a proper name. We do not believe that it necessistates stating in the abstract that this stands for  "distributional clauses probabilistic Prolog", where "Prolog" stands for the French expression "programmation en logique".}
    \fixedrebuttal{Fixed.}
    \item 5-10: can be used represent --> to represent
    \\
    \fixedrebuttal{Fixed.}
    \item 55-60: such e.g --> such as
    \\
    \fixedrebuttal{Fixed.}
    \item 120-125: distributional facts in disguise --> please make the statement more formal
    \\
    \fixedrebuttal{Fixed. We now say: Note how probabilistic facts are actually syntactic sugar for distributional facts.}
    \item 135-14: DF-PLP: first time introduced, please provide the detailed name.
    \\
    \fixedrebuttal{Fixed.}
    \item Example 4.26: T2a and T2b are only mentioned once, the purpose of these notations should be well explained.
    \\
    \fixedrebuttal{Fixed. This was indeed a remnant from an earlier draft and the T2a and T2b are referring to the rules in Definition~\ref{def:adfree-to-core}. We now refer to these rules as CR1 and CR2. Note this part of the manuscript moved from the main body to the appendix (Line ~\ref{line:comment_r2_t2a_rule}).}
    \item 555-560: comparison literals --> check the grammar
    \\
    \commentrebuttal{We believe the grammar is correct. Note this part has moved to the Appendix in Line \lineref{line:comment_grammar_r2}}
    \item 555-560: maintains measurability of the latter --> the measurability
    \\
    \fixedrebuttal{Fixed.}
    \item before 1060: lsited bellow --> listed
    \\
    \fixedrebuttal{Fixed.}
\end{enumerate}







\section{Reviewer 3}










\subsection{Weaknesses}

\begin{enumerate}
    \setcounter{enumi}{0}

\item The proposal seems very similar to an approach called Bayesian Logic
Program, but with a different syntax. The semantics look very similar, and
the basic results are inherited from that work. The authors should clarify
the similarities and differences between their approach and Bayesian Logic
Programs, and provide sufficient reasons for introducing a new language.
This comparison, as far as I can see is not included in the related work
section at the end of the paper.

\fixedrebuttal{We have added a discussion on the relationship to Bayesian logic programs to the related work section (Section~\ref{sec:blp}). Following the suggestion from Reviewer 1 we reworked the theory in Section~\ref{sec:semantics}. We now substantially generaliize BLPs and do not inherit any of the results from \Citet{kersting2000bayesian} anymore. We discuss the differences in Section~\ref{sec:blp}.}

\item The most novel (and interesting) part of the paper concerns the infinites-
imal semantics, which comes very late in the paper. I have some doubts
about the formalisation. If I’m correct this could be a problem. (see
comments below)

\commentrebuttal{we discuss the reviewer's doubt in more detail in the technical comments section below. While pointing out some minor issues, we were able to readily resolve them.}

\item I have found the part on “syntactic sugar” not very relevant. It’s ok to
put it in a manual. It does not introduce new important concepts, and it
distracts the reader from the main message of the paper.

\fixedrebuttal{We relegated a large portion from the "syntactic sugar" discussion to the Appendix. In the main body of the paper we only discuss the syntax, while studying the semantics and specific details in the appendix.}

\item No experimental evaluation is provided.

\fixedrebuttal{We have now included an experimental comparison (Section~\ref{sec:experimental}). Furthermore, we would like to point out the experiments  we have performed in a(n) (already published) follow-up paper in the field of neurosymbolic AI \citep{desmet2023neural}.}



\end{enumerate}

\subsection{Readability}
\begin{enumerate}
    \setcounter{enumi}{7}

\item The paper is not easy to read. I believe it is not well organised. The most
important part is at the end of the paper. I believe the paper should concentrate mostly on basic primitives and leave the rewriting of ”syntactic
sugar primitives” later.

\fixedrebuttal{as mentioned above we have relegated large parts of the syntactic sugar discussion to the appendix.}

\item the paper is too long. There are many examples, even for elementary
concepts. The reader gets lost. I think this style is good for a manual,
not a scientific paper.

\commentrebuttal{removing the syntactic sugar discussion from the main body has significantly reduced the paper's length. Concerning the examples, we refer the reader to our comment on the next point (\cf AIJ author guidelines)}

\item I have found the panoramic overview not very interesting. It’s good to have
it in a manual but not in a scientific article. A more compact example
would do the job.

\commentrebuttal{It seems that this is a question of taste. Reviewer 1 and 2 seem to have a different opinion. Furtheermore, AIJ demands from papers to be accessible to a wider audience. Quoting from the AIJ author guidelines\footnote{https://www.sciencedirect.com/journal/artificial-intelligence/publish/guide-for-authors}: "Papers that are heavily mathematical in content are welcome but should include a less technical high-level motivation and introduction that is accessible to a wide audience and explanatory commentary throughout the paper." We believe the overview section and the examples throughout the paper have this function.}

\item The notation sometimes is heavy, I think that some simplifications are
possible.

\commentrebuttal{
We put considerable work into keeping the notation as light as possible yet consistent throughout the paper and also precise. The latter two affect of course, occasionally, readability. However, if there are concrete suggestions on how to improve the notation we would eagerly incorporate them into the manuscript.
} 
\end{enumerate}

\subsection{Technical Issues}


\begin{enumerate}
    \setcounter{enumi}{11}

\item I don’t fully understand why you consider “countable” many random variables instead of just a finite set of random variables. It looks to me that
all the results. I’m not an expert in probability theory but in this way,
you can represent random processes as for instance $X_i \sim N (\mu = X_i , \sigma)$,
and this might introduce rather complex aspects. Perhaps you should
comment on the case of infinite random variables.

\commentrebuttal{We are not entirely sure what the reviewer is referring to: first the reviewer talks about "countably many" (which can be an infinite number but still countable) then in the last sentence the reviewer mentioned infinite random variables. We are not sure whether the reviewer means  countable or uncountable infinity here.

Concerning the distribution above ($X_i \sim N (\mu = X_i , \sigma)$). Here we have a cycle: $X_i$ is used as its own mean and we do not have a valid random variable to begin with. 
}

\item Definition 3.10 Well-Defined Distributional Database. This coincides with
the definition of Bayesian network. I might have missed some details but
I don’t see the difference when the set of random variables is finite. When
they are infinite, the restriction that every formula has a finite number of
ancestors


\commentrebuttal{We substantially altered the definition of a distritbutional database (Definition~\ref{def:distDB}) and the reviewer's comment does directly apply anymore. We point, however, to the new Section~\ref{sec:fintiedistdb}, which we hope clarifies possible remaining doubts.
}




\item Condition DC1 This condition is semantically defined. How is it possible
to check it at the syntactic level without running the program itself? The
conditions might depend on the entire part of the program e.g., Suppose
that you have two statements
\begin{align*}
    x \sim \delta_1 \lpif A.
    \\
    x \sim \delta_2 \lpif B.
\end{align*}
2and there are two subpart of the program that imposes conditions on A
and B may depend on other random variables. As a matter of fact, you
have to ensure that $P(A \land B)$ is equal to $0$. I believe you have to provide
some sufficient “syntactic” condition that guarantees such that $A \land B$ is
not satisfiable.

\commentrebuttal{Indeed, when implementing the language one could think about introducing syntactic sugar that would guarantee validity.
The problem with such syntactic constructs is that they might be sufficient but not necessary to guarantee validity and thereby they would limit the user.

In our opinion ensuring that a set of distributional clauses is well-defined lies therefore with the user of the language itself or with the programmer who implements the language. In this paper we are, however, concerned with the semantics and therefore refrain from making such restrictions. 
}

\item Definition 7.4 states that $e^\oplus = (0, 0)$ is the neutral element for $\oplus$, So I
expect that $(r, n) \oplus  e^\oplus  = e^\oplus  \oplus  (r, n) = (r, n)$, But according to Definition
7.2 when $n > 0$, $(r, n) \oplus  e^\oplus  = e^\oplus  \oplus  (r, n) = e^\oplus $ . From this definition it
looks like a neutral element for $\oplus $ is $(0, \infty)$ but you have to include $\infty$ to
the set $\mathbb{Z}$. However, this implies that the inverse w.r.t. the $\otimes$ does not
exist. I have probably missed something here, it looks like all the following
relies on the fact that this structure is a commutative semiring. \citet{jacobs2021paradoxes} introduces explicitly the operation of difference and inverse, without
considering a neutral elements. Perhaps, the fact that is a semiring is not
so important for the paper.


\fixedrebuttal{indeed this was a mistake and the neutral element should have been defined as $e^\oplus =(0, \infty)$. We have fixed this in the manuscript. Note, however, that this has no serious consequences on any other part of the paper, except minor for the inverse elements. Specifically, we strike (as the reviewer suggested the definition of the inverse elements and define immediately subtraction and division \cf Definition~\ref{def:subdiv}.)
}

\commentrebuttal{Having no explicit inverse for $\otimes$ and $\oplus$ is not a problem: semirings do not necessitate inverse elements (\cf Definition~\ref{def:comm_semiring}).}

\item The solution to the problem of conditioning with $0$-probability events are
solved only partially. I.e., only in the case in which the $0$ probable events
are explicitly stated in an atom by using the primitive delta interval.
However, there are many other situations of events that are still possible
but with $0$ probability. As a simple example, the conjunction of the two
constraints $v \leq 1$ and $v \geq 1$ is equivalent to the constraint $v = 1$, and
I think that in a pure declarative approach, like the one that is pursued
by ProbLog they should have the same semantics. While the latter is
interpreted in terms of infinitesimal number the latter is not. How would
the infinitesimal approach generalize to the other comparison atoms? A
possible solution to this problem can be obtained by rewriting $x \leq y$ into
$(x < y ) \overline{\vee} (x=y)$
˙ (where $ \overline{\vee}$ denotes the xor operator).

\commentrebuttal{Here we have to disagree with the following statement: "While the latter is
interpreted in terms of infinitesimal number the [former] is not". This is not correct. Neither of the two are interpreted as infinitesimal numbers!
We would like to stress here that $x=y$ is not interpreted as an infinitesimal interval but simply as an indicator function where the equality has to hold strictly. This means that the semantics align between $v \leq 1 \land v \geq 1$ and  $v = 1$ as they are equivalent. Note that both of these expressions are, however, not equivalent to $v \doteq 1$, which is the actual infinitesimal interval. We already discuss this issue in Example~\ref{ex:conditional_prob}.
}

\item I also think that the semantics based on infinitesimals introduces other
paradoxes, For instance, if $x \sim \mathcal{N} (0, 1)$ $P (x < 0 | x = 1)$ is equal to $(0, 0)$.
However, by infinitesimal semantics this is equal to $( \frac{2}{y}
, -1)$ where $y$ is
the value of $(0, 1)$ in 1 (which is different from $(0, 0)$).

\commentrebuttal{Indeed, following the definition of a conditional probability with a zero probability conditioning event (using a limit), gives us
$P (x < 0 | x \doteq 1)=0 = (0\epsilon^0) = (0,0)$. Note that here we use the "$\doteq$" notation to indicate that we condition on an infinitesimal interval.

However, using infinitesimal numbers does not yield $(0,-1)$. If we write the conditional probability in question using infinitesimal numbers we get the following result:
\begin{align}
    \frac{
        \sum_{i}^{|S|} \alpha_{SIALW}(\ive{x^{(i)}<0}) \otimes \alpha_{SIALW}(\ive{x^{(i)} \doteq 1})
    }
    {
        \sum_{i}^{|S|} \alpha_{SIALW}(\ive{x^{(i)} \doteq 1})
    }
\end{align}
Because $x^{(i)} \doteq 1$ denotes an observation, all the samples $x^{(i)}$ are forced to take the value $1$. Plugging this into above expression gives:
\begin{align}
    \frac{
        \sum_{i}^{|S|}  (0,0) \otimes (p(1),1)
    }
    {
        \sum_{i}^{|S|}   (p(1),1)
    }
    \label{eq:rev3idiot}
    &=
    \frac{
        \sum_{i}^{|S|} (0,1)    
    }{
        \sum_{i}^{|S|}  (p(1),1)
    }
    \\
    &=
    \frac{
        (0,1)
    }
    {
        (|S|  p(1),1)
    }
    \\
    &=
    (0,0)
    \label{eq:r3:misconception}
\end{align}
That means if we straightforwardly apply the correct expression (the one that we derived in the manuscript) to compute the conditional probability we do obtain the result we expect.
}

\item Theorem 7.11 and Proposition 7.14 are not well formulated because the
meaning of $(x, 0) \approx (y, n)$ when $n \neq 0$ is not defined. Consider the previous
example $P (x < 0 | x  1) = (0, 0)$, however it is possible that S contains
at least one sample $S^{(i)}$ such that $S^{(i)}_x = 1$. In this case, the summation
at the denominator will be some value $(y, 1)$ while at the numerator you
will have an empty summation, which I suppose it is equal to $(0, 0)$. The
$\frac{(0,0)}{(x,1)}$
is equal to $(0, -1)$.

\commentrebuttal{Let us consider the left-hand side of Equation~\ref{eq:rev3idiot} above:
\begin{align}
    \frac{
        \sum_{i}^{|S|}  (0,0) \otimes (p(1),1)
    }
    {
        \sum_{i}^{|S|}   (p(1),1)
    }
\end{align}
Instead of multiplying out the numerator we can also cancel out the $(p(1),1)$ expression to obtain:
\begin{align}
    \frac{
        \sum_{i}^{|S|}  (0,0) \otimes (p(1),1)
    }
    {
        \sum_{i}^{|S|}   (p(1),1)
    }
    &
    =
    \frac{
        (0,0) \otimes (p(1),1) \otimes \cancel{\sum_{i}^{|S|} (1,0)}  
    }
    {
        (p(1),1) \otimes  \cancel{\sum_{i}^{|S|} (1,0)}
    }
    \\
    &=
    \frac{
        (0,0) \otimes \cancel{(p(1),1) } 
    }
    {
        \cancel{(p(1),1) }
    }
    \\
    &=
    (0,0)
\end{align}


While here we show that the particular case (pointed out by the reviewer) does not cause any problems (if one correctly applies the defined algebra), we also prove this for the general case in Subsection~\ref{app:proof:alwapproximation}.
We conclude that the apparent misbehavior never arises when the definitions are applied correctly.

Also note that for the case that we do not condition on $x\doteq 1$ but on $x=1$, we almost surely compute the correct result. We state this explicitly in Proposition~\ref{prop:alw_consistency} already.
}


\end{enumerate}


\subsection{Minor Points}
\begin{enumerate}
    \setcounter{enumi}{18}
\item line 150 “on” → “one”
\\
\fixedrebuttal{Fixed.}
\item line 175 “mutually marginally independent” → “mutually independent”
why you use marginally.
\\
\fixedrebuttal{We removed "marginally."}


\item Definition 3.3: “regular ground term” → “regular ground atom”
\\
\commentrebuttal{No, these are indeed ground terms and not ground atoms.}




\item Paragraph 901 explains the order of sampling, and introduces the term
“ancestral sample”. This is confusing because every sample must be an
ancestral sample since you need the value of the parent variables in order
to sample a variable. I suggest not to use this term and to explain before
Property 7.1, that to obtain a sample for all the variables we have to
proceed according to the partial order on the variables.

\commentrebuttal{With using the term "ancestral sampling" we are stressing the fact we perform sampling using ancestral sampling. This is in order to distinguish the samples from other sampling strategies, such as Gibbs sampling or any other MCMC method. As this MCMC methods are ubiquitous in probabilistic programming we us "ancestral sample" to make the sample strategy explicit.}


\item line 510: $L > 5$ → $\Lambda \geq 5$
\\
\fixedrebuttal{Fixed.}
\item Definition 4.21. in the first bullet, you assume that only one random
variable occurs in a random term, however in general this is not the case.
\\
\commentrebuttal{We do not believe that we make this assumption. We actually explicitly state "a distribution term that involves exavtly $k$ different random terms". Note this definition has moved and is now called Definition~\ref{def:elim-dc}.}
\item Definition 4.28: Clarify what is $P^{DF,*}$.
\\
\fixedrebuttal{We now clearly state this at the end of Definitino~\ref{def:adfree-to-core}.}
\item Definition 7.9: revise English and “evaluate” → “denotes”. The defini-
tion is hard to read. The reference to literals and negated literals is a
bit confusing. For instance, what happens to the negation of the comparison atom $v =/= 3$? Using the notation $[[s_k = x]]$ could also improve
readability.
\\
\fixedrebuttal{We rewrote the Definiton~\ref{def:sample_labeling_function}, hopefully improving the readability. We also use the Iverson bracket $\ive{\cdot}$ as suggested.}
\item Lines 1029-1031 it seems that you compare DC problog with itself. Is the
system introduced in [Fierence et al. 2015] also called DC-ProbLog?
\\
\fixedrebuttal{The language introduced by \citep{fierens2015inference} is called \problogsty. This was a typo on our side, and we change "\dcproblogsty" to "\problogsty",}
\item Definition 7.24: “given is” → “given”
\\
\fixedrebuttal{Fixed.}
\item It’s not completely clear if $\neg(x =/= 2)$ should be considered a negative
literal, or $x =/= 2$ is a negative literal of the literal $x=2$.
\\
\commentrebuttal{We discuss the issue of how to interpret negation in detail in Appendix~\ref{sec:non-mixture-dc} and Appendix~\ref{sec:dcproblog-dc}. Especially as \dcproblogsty and \dcsty have different semantics for negation. In short: when distributional clauses are present $x =/= 2$ is not necessriöy equivalent to $\neg(x=:=2)$ otherwise they are equivalent.}
\item Paragraph 185: is very obscure at this point of the paper. I believe it
could be put in a proper related work section.
\\
\fixedrebuttal{We removed the discussion in this paragraph and give now forward pointers to the related work Section~\ref{sec:related} and to Appendix~\ref{sec:dcproblog-dc}.}
\item The definition of infinitesimal interval is not very intuitive. In order to
grasp it I had to read the paper \citep{jacobs2021paradoxes}. I suggest reporting his explanation provided below the definition 5.2 of that paper.
\\
\fixedrebuttal{We added a reference to \Citet[Section 5.2]{jacobs2021paradoxes} just before Definition~\ref{def:inf_number}.}
\item Bibliography format is not correct.

\fixedrebuttal{Fixed.}


\end{enumerate}



\subsection{Reccomendations}

\begin{enumerate}
    \setcounter{enumi}{32}


\item The paper is potentially interesting but I think it needs to be improved in
the readability, i.e., make it shorter by concentrating on the major points,
and optimizing the usage of examples. The technical part should also
be double-checked. I might be wrong with my comments, in such a case
the authors should better explain the technical part, or if I’m right the
authors have to fix and revise (if necessary) the proofs. Since the paper
does not contain major technical theoretical results, most of the results
are applications/simple generalization of existing results, the paper should
provide some experiments that show the potential of the approach. I think
that to address all the above observations the paper needs a major revision,
which is my suggestion.

\comment{We hope that the changes we made to the paper addresses the concerns raised by the reviewer. Specifically,
\begin{enumerate}
    \item we substantially reduced the section on syntactic sugar
    \item we corrected our mistake regarding the neutral elements for the addition in our semiring and adapted the manuscript accordingly,
    \item we added experimental evidence that the developed SIALW algorithm exhibits advantageous behavior when compares to similar algorithms in the presence of low probability events. 
\end{enumerate}

We would also argue that our technical contribution does not constitute a simple extension of existing ideas. This is exemplified by the fact that a series of papers \citep{kersting2000bayesian,gutmann2010extending,gutmann2011magic,azzolini2021semantics} have indeed tackled the problem of unifying discrete and continuous random variables for probabilistic logic programming but none have succeeded in linking these discrete-continuous languages to the inference method of knowledge compilation -- the de facto standard for exact inference in the discrete domain. We achieve this using our IALW semiring. While we do deploy proof techniques developed in other papers, considerable effort has gone into correctly formulating them for the discrete-continuous domain. We are of the strong opinion that our theoretical contributions are not applications but strong generalizations of existing work.
}
\end{enumerate}









% \clearpage

\setcounter{page}{1}
\renewcommand{\thepage}{\arabic{page}}

\setcounter{section}{0}
\renewcommand{\thesection}{\arabic{section}}
\renewcommand{\theequation}{\arabic{section}.\arabic{equation}}




\begin{frontmatter}

    \title{Declarative Probabilistic Logic Programming\\in Discrete-Continuous Domains}
    % \tnotetext[mytitlenote]{}

    \author[orebro]{Pedro Zuidberg Dos Martires}
    \author[orebro,kuleuven,leuvenai]{Luc De Raedt}
    \author[kuleuven,leuvenai]{Angelika Kimmig}


    \address[orebro]{Centre for Applied Autonomous Sensor Systems, Örebro University, Sweden}
    \address[kuleuven]{Department of Computer Science, KU Leuven, Belgium}
    \address[leuvenai]{Leuven.AI, Belgium}



    % \begin{abstract}
    % A core strength of probabilistic logic programming languages, which belong to the best understood probabilistic programming languages in the scientific literature, are their declarative semantics. However, this is only true for programs limited to random variables with finite sample spaces. When extending PLP languages to the discrete-continuous domain, procedural semantics have usually been preferred over declarative semantics.
    % We introduce \dcproblogsty, a PLP language capable of representing random variables in the discrete-continuous domain, which we equip with purely declarative semantics.
    % Additionally, we develop the first inference algorithm for a probabilistic programming language in the discrete-continuous domain based on knowledge compilation. As such, we generalize conventional PLP in a twofold fashion. On the hand, \dcproblogsty extends the declarative semantics to random variables with infinite (and uncountable) sample spaces, instead of finite ones only. On the other hand, we generalize the state-of-the-art knowledge compilation approach towards inference in conventional PLP languages to the discrete-continuous domain.
    % \end{abstract}


    \begin{abstract}
        Over the past three decades, the logic programming paradigm has been successfully expanded
        to support probabilistic modeling, inference and learning. The resulting paradigm
        of probabilistic logic programming (PLP) and its programming languages owes much of its success to a declarative semantics,
        the so-called distribution semantics. However, the distribution semantics is limited to discrete random variables only.
        While PLP has been extended in various ways for supporting hybrid, that is, mixed discrete and continuous
        random variables, we are still lacking a declarative semantics for hybrid PLP that not only generalizes
        the distribution semantics and the modeling language but also the standard inference algorithm
        that is based on knowledge compilation.
        We contribute the {\em measure semantics} together with the hybrid PLP language
        \dcproblogsty (where DC stands for distributional clauses) and its inference engine {\em infinitesimal algebraic likelihood weighting} (IALW).
        These have the original distribution semantics, standard PLP languages
        such as \problogsty, and standard inference engines for PLP based on knowledge compilation as special cases.
        Thus, we generalize the state of the art of PLP towards hybrid PLP in three different aspects: semantics, language and inference.
        Furthermore, IALW is the first inference algorithm for hybrid probabilistic programming based on knowledge compilation.
    \end{abstract}




    





    \begin{keyword}
        Probabilistic Programming \sep Declarative Semantics \sep Discrete-Continuous Distributions \sep Likelihood Weighting \sep Logic Programming \sep Knowledge Compilation \sep Algebraic Model Counting
    \end{keyword}

\end{frontmatter}

% \linenumbers




\section{Introduction}
\label{sec:introduction}

Probabilistic logic programming (PLP) is at the crossroads of two parallel developments in artificial intelligence and machine learning.
On the one hand, there are the probabilistic programming languages with built-in support for machine learning. These languages can be used to represent very expressive -- Turing equivalent
-- probabilistic models, and they provide primitives for inference and learning.
On the other hand, there is the longstanding open question for integrating the two main frameworks
for reasoning, that is logic and probability, within a common framework \citep{russell2015unifying,deraedt:mc16}.
Probabilistic logic programming~\citep{de2015probabilistic,riguzzi2018foundations} fits both paradigms and goes back to at least the early 90s
with seminal works by \citet{sato1995statistical} and \citet{poole1993probabilistic}. 
\citeauthor{poole1993probabilistic} introduced ICL, the Independent Choice Logic, an elegant
extension of the Prolog programming language, and \citeauthor{sato1995statistical} introduced
the {\em distribution semantics} for probabilistic logic programs in conjunction with a learning algorithm based on expectation maximization (EM).
The PRISM language \citep{sato1995statistical}, which utilizes the distribution semantics and the EM learning algorithm constitutes, to the best of the authors' knowledge, the very first probabilistic programming language with support for machine learning.


Today, there is a plethora of probabilistic logic programming languages, most of which are based
on extensions of the ideas by \citeauthor{sato1995statistical} and~\citeauthor{poole1993probabilistic} ~\citep{sato1997prism,kersting2000bayesian,vennekens2004logic,de2007problog}. However, the vast majority of them is restricted to discrete, and more precisely finite categorical, random variables. 
When merging logic with probability, the restriction to discrete random variables
is natural and allowed Sato to elegantly extend the logic program semantics
into the celebrated distribution semantics.   




\begin{example}[Probabilistic Logic Program]
  \label{example:intro}

  \new{
  Consider the probabilistic logic program below (written in \problogsty syntax~\citep{fierens2015inference}), where we model the behavior of two machines. We first state that there are two machines (Line~\ref{line:exintro:1}). Subsequently, we say that the temperature has a probability of $0.8$ to be low (Line \ref{line:exintro:temp}) and that the cooling of the machines works with probability $0.99$ and $0.95$ respectively (Lines \ref{line:exintro:cool1} and \ref{line:exintro:cool2}) These labeled facts are called \textit{probabilistic facts}. We also model that the machines themselves work: either if the cooling is working (Line~\ref{line:exintro:work1}) or if the temperature is low (Line~\ref{line:exintro:work2}).
  }
  \begin{problog*}{linenos}
machine(1). machine(2). @\label{line:exintro:1}@
0.8::temperature(low). @\label{line:exintro:temp}@
0.99::cooling(1).@\label{line:exintro:cool1}@
0.95::cooling(2).@\label{line:exintro:cool2}@

works(N):- machine(N), cooling(N). @\label{line:exintro:work1}@
works(N):- machine(N), temperature(low). @\label{line:exintro:work2}@

    \end{problog*}
    \new{
We can now, for instance, ask for the conditional probability of the first machine working given that the second one works:
    $$
    P(\mathprobloginline{works(1)}=\top \mid \mathprobloginline{works(2)}=\bot).
    $$
The (exact) inference algorithm currently implemented in \problogsys~\citep{fierens2015inference,dries2015problog2} then returns as answer the probability $\approx 0.998$.
}
\end{example}

While \citeauthor{sato1995statistical}'s extension of logic programming to the probabilistic domain is elegant, it also imposes an important restriction to random variables with countable sample spaces. This raises the question of how to extend the distribution semantics towards hybrid, \ie discrete-continuous, random variables.

Defining the semantics of probabilistic programming language with support for random variables with infinite and possibly uncountable sample spaces is a much harder task. This can be observed when looking at the development of important imperative and functional probabilistic programming languages~\citep{goodman2008church,mansinghka2014venture} that support  continuous random variables. 
These works initially focused on inference, typically using  a particular Monte Carlo approach, yielding  an operational or procedural semantics. It is only follow-up work that started to address a declarative semantics for such hybrid probabilistic programming languages.
~\citep{staton2016semantics,wu2018discrete}.
%\footnote{Aside from some subtle differences (\cf~\citep{kimmig2017probabilistic}), in non-logical probabilistic programming languages declarative semantics are usually called denotational and procedural semantics are referred to as operational.}

The PLP landscape has experienced similar struggles. First approaches for  hybrid PLP languages were achieved by restricting the language~\citep{gutmann2010extending,gutmann2011magic,islam2012inference} or via recourse to procedural semantics~\citep{nitti2016probabilistic}.  The key contributions of this paper are:

\begin{enumerate}[label={\bf C\arabic*}]
  \setcounter{enumi}{0}
\item We introduce the {\em measure semantics} for mixed discrete-continuous probabilistic logic programming.
Our {\em measure semantics} (based on measure theory) extends \citeauthor{sato1995statistical}'s distribution semantics and supports:
\begin{itemize}
    \item  \label{item:k1} a countably infinite number of random variables,
     \item a uniform treatment of discrete and continuous random variables,
      \item a clear separation between probabilistic dependencies and logical dependencies by extending the ideas of \citet{poole2010probabilistic} to the hybrid domain.
    \end{itemize}
\item \label{item:k3} We introduce \dcproblogsty, an expressive PLP language in the discrete-continuous domain,
which incorporates the{\em measure semantics}. 
 \dcproblogsty has standard discrete PLP, \eg ProbLog~\citep{fierens2015inference}, as a special case (unlike other hybrid PLP languages~\citep{gutmann2011magic,nitti2016probabilistic}).
    \item \label{item:k2}  We introduce a novel inference algorithm, {\em infinitesimal algebraic likelihood weighting} (IALW), for hybrid PLPs,
which extends the standard knowledge compilation approach used in PLP towards mixed discrete continuous distributions, and  which 
provides an operational semantics for hybrid PLP.
\end{enumerate}


In essence, our contributions ~\ref{item:k1} and ~\ref{item:k3} generalize both  Sato's distribution semantics and discrete PLP such that in the absence of random variables with infinite sample spaces we recover the \problogsty language and declarative semantics. It is noteworthy that our approach of disentangling probabilistic dependencies and logical ones, allows us to express more general distributions than state-of-the-art approaches such as~\citep{gutmann2011magic,nitti2016probabilistic,azzolini2021semantics}. 
Contribution \ref{item:k2} takes this generalization to the inference level: in the exclusive presence of finite random variables our IALW algorithm reduces to \problogsty's current inference algorithm~\citep{fierens2015inference}.












% Probabilistic logic programming (PLP) constitutes a well-studied field within computer science, with a rich tradition dating back to the early 1990s. In particular, \citet{dantsin1990probabilistic}, \citet{ng1992probabilistic}, and \citet{poole1993probabilistic} generalized earlier ideas of \citet{nilsson1986probabilistic} and \citet{pearl1988probabilistic} on probabilistic logic towards probabilistic logic programming. In \citeyear{sato1995statistical}, \citeauthor{sato1995statistical} then presented his seminal work on {\em distribution semantics} for probabilistic logic programs. All these early works on PLP restrict random variables to finite categorical (usually binary) random variables.

% An advantage of restricting a probabilistic programming language to finite random variables is the relative straightforwardness of defining a declarative semantics for programs, i.e giving meaning to programs regardless of the underlying inference algorithm used in the implementation of the language.
% In PLP, this is often done by falling back on \citeauthor{sato1995statistical}'s distribution semantics~\citep{sato1997prism,kersting2000bayesian,vennekens2004logic,de2007problog}.

% Defining declarative semantics for a language including random variables with infinite and possibly uncountable sample spaces is a much harder task. This can be observed when looking at the development of non-logic probabilistic programming languages~\citep{goodman2008church,mansinghka2014venture}, whose focus is on continuous random variables and which were only defined with regard to a specific Monte Carlo inference algorithm. Only follow-up works started tackling the issue of fully declarative semantics for non-logic probabilistic programming languages in discrete-continuous domains~\citep{staton2016semantics,wu2018discrete}.\footnote{Aside from some subtle differences (\cf~\citep{kimmig2017probabilistic}), in non-logical probabilistic programming languages declarative semantics are usually called denotational and procedural semantics are referred to as operational.}

% The PLP landscape has experienced similar struggles. First attempts of extending the traditionally finite sample space to discrete-continuous domains (also called hybrid) were achieved by restricting the language~\citep{gutmann2010extending,gutmann2011magic,islam2012inference} or via recourse to procedural semantics~\citep{nitti2016probabilistic}. This brings us to our first key contribution:


% \begin{enumerate}[label={\bf C\arabic*}]
%   \setcounter{enumi}{0}
% \item
% We introduce \dcproblogsty, a fully expressive PLP language in the discrete-continuous domain, which we equip with fully declarative semantics based on \citeauthor{sato1995statistical}'s distribution semantics. \dcproblogsty is a PLP language with:
%     \begin{itemize}
%         \item \label{item:k1} a countably infinite number of random variables
%         \item a uniform treatment of discrete and continuous random variables
%         \item a clear separation between probabilistic dependencies and logical dependencies by extending the ideas of \citet{poole2010probabilistic} to the hybrid domain
%         \item standard discrete PLP, \eg ProbLog~\citep{fierens2015inference}, as a special case (unlike other hybrid PLP languages~\citep{gutmann2011magic,nitti2016probabilistic}).
%     \end{itemize}
% \end{enumerate}
% Our second key contribution concerns extending probabilistic inference in discrete PLP, \eg \problogsty~\citep{fierens2015inference}, to discrete-continuous domains. 
% \begin{enumerate}[label={\bf C\arabic*}]
%   \setcounter{enumi}{1}
%     \item \label{item:k2} We present {\em infinitesimal algebraic likelihood weighting} (IALW), a novel inference algorithm that allows for the combination of knowledge compilation~\citep{darwiche2002knowledge} and sampling based approaches in hybrid domains. 
% \end{enumerate}

% In essence, our contribution ~\ref{item:k1} generalizes Sato's distribution semantics such that in the absence of random variables with infinite sample spaces we recover the \problogsty language.
% It is noteworthy that our approach of disentangling probabilistic dependencies and logical ones, allows us to express more general distributions than state-of-the-art approaches, \eg~\citep{gutmann2011magic,nitti2016probabilistic,azzolini2021semantics}. 
% Contribution \ref{item:k2} takes this generalization to the inference level: in the exclusive presence of finite random variables our ALW algorithm reduces to \problogsty's current inference algorithm~\citep{fierens2015inference}.




\section{A Panoramic Overview}
\label{sec:panorama}

Before diving into the technical details of the paper we first give a high-level overview of the \dcproblogsty language. This will also serve us as roadmap to the remainder of the paper.  
We will first introduce, by example, the \dcproblogsty language (Section~\ref{sec:panorama_semantics}). The formal syntax and semantics of which are discussed in Section~\ref{sec:semantics} and Section~\ref{sec:dcproblog}.
In Section~\ref{sec:panorama_inference} we demonstrate how to perform probabilistic inference in \dcproblogsty by translating a queried \dcproblogsty program to an algebraic circuit~\citep{zuidbergdosmartires2019transforming}. 
Before giving the details of this transformation in Section~\ref{sec:dc2smt} and Section~\ref{sec:alw}, we define conditional probability queries on \dcproblogsty programs (Section~\ref{sec:inference-tasks}).
The paper ends with a discussion on related work (Section~\ref{sec:related}) and concluding remarks in Section~\ref{sec:conclusions}.

Throughout the paper, we assume that the reader is familiar with basic concepts from logic programming and probability theory. We provide, however, a brief refresher of basic logic programming concepts in Appendix~\ref{app:lp_new}. In Appendix~\ref{app:table} we give a tabular overview of notations used, and in the remaining sections of the appendix we give proofs to propositions and theorems or discuss in more detail some of the more subtle technical issues.

\subsection{Panorama of the Syntax and Semantics}
\label{sec:panorama_semantics}


\begin{example}
    \label{example:sweets_dc}
    A shop owner creates random bags of  sweets with two independent random binary properties (\emph{large} and \emph{balanced}).  He first picks the number of red sweets from a Poisson distribution whose parameter is 20 if the bag is large and 10 otherwise, and then the number of yellow sweets from a Poisson whose parameter is the number of red sweets if the bag is balanced and twice that number otherwise. His favorite type of bag contains more than 15 red sweets and no less than 5 yellow ones. We model this in \dcproblogsty as follows:
\begin{problog*}{linenos}
0.5::large.
0.5::balanced.

red ~ poisson(20) :- large.
red ~ poisson(10) :- not large.

yellow ~ poisson(red) :- balanced.
yellow ~ poisson(2*red) :- not balanced.

favorite :- red > 15, not yellow < 5.
\end{problog*}

In the first two lines we encounter {\em probabilistic facts}, a well-known modelling construct in discrete PLP languages (\eg~\citep{de2007problog}). Probabilistic facts, written as logical facts labeled with a probability, express Boolean random variables that are true with the probability specified by the label. For instance, \probloginline{0.5::large} expresses that \probloginline{large} is true with probability \probloginline{0.5} and false with probability \probloginline{1-0.5}.

In Lines 4 to 8, we use {\em distributional clauses} (DCs); introduced by~\citet{gutmann2011magic} into the PLP literature. DCs are of the syntactical form \probloginline{v~d:-b} and define random variables \probloginline{v} that are distributed according to the distribution \probloginline{d}, given that \probloginline{b} is true.
For example, Line 4 specifies that when \probloginline{large} is true, \probloginline{red} is distributed according to a Poisson distribution. 
We call the left-hand argument of a \predicate{~}{2} predicate in infix notation a {\em random term}. The random terms in the program above are \probloginline{red} and \probloginline{yellow}.

Note how random terms reappear in three distinct places in the \dcproblogsty program. First, we can use them as parameters to other distributions, \eg \probloginline{yellow ~ poisson(red)}. 
Second, we can use them within arithmetic expression, such as \probloginline{2*red} in Line 8.
Third, we can use them in comparison atoms (\probloginline{red>15}) in Line 10. The comparison atoms appear in the bodies of logical rules that express logical consequences of probabilistic event, for example having more than 15 red sweets and less than 5 yellow ones.
\end{example}




Probabilistic facts and distributional clauses are the main modelling constructs to define random variables in probabilistic logic programs.
As they are considered to be fundamental building blocks of a PLP language, the semantics of a language are defined in function of these syntactical constructs (\cf.~\citep{fierens2015inference,gutmann2011magic}). 
We now make an important observation: probabilistic facts and distributional clauses can be deconstructed into a much more fundamental concept, which we call the {\em distributional fact}.  
Syntactically, a distributional fact is of the form \probloginline{v~d}. That is, a distributional clause with an empty body.
As a consequence, probabilistic facts and distributional clauses do not constitute fundamental concepts in PLP but are merely special cases, \ie while helpful for writing concise programs, they are only of secondary importance when it comes to semantics.


\begin{example}\label{ex:sweets_df}
We now rewrite the program in Example~\ref{example:sweets_dc} using distributional facts only. Note how probabilistic facts are actually syntactic \fixed{sugar for distributional facts}. The random variable is now distributed according to a Bernoulli distribution (\probloginline{flip}) and the atom of the probabilistic fact is the head of a rule with a probabilistic comparison in its body (\eg Lines 1 and 2 in the program below). Rewriting distributional facts is more involved. The main idea is to introduce a distinct random term for each distributional clause. Take for example, the random term \probloginline{red} in Example~\ref{example:sweets_dc}. This random term encodes, in fact, two distinct random variables, which we denote in the program below \probloginline{red_large} and \probloginline{red_small}. We now have to propagate this rewrite through the program and replace every occurrence of \probloginline{red} with \probloginline{red_large} and \probloginline{red_small}. This is why we get instead of two distributional clauses for \probloginline{yellow}, four distributional facts. It explains also why we get instead of one rule for \probloginline{favorite} in Example~\ref{example:sweets_dc} four rules now.

    \begin{problog*}{linenos}
rv_large ~ flip(0.5).
large :- rv_large=:=1.
rv_balanced ~ flip(0.5).
balanced :- rv_balanced=:=1.

red_large ~ poisson(20).
red_small ~ poisson(10).

yellow_large_balanced ~ poisson(red_large).
yellow_large_unbalanced ~ poisson(2*red_large).
yellow_small_balanced ~ poisson(red_small).
yellow_small_unbalanced ~ poisson(2*red_small).

favorite :- large, red_large > 15, 
              balanced, not yellow_large_balanced < 5.
favorite :- large, red_large > 15, 
              not balanced, not yellow_large_unbalanced < 5.
favorite :- not large, red_small > 15, 
              balanced, not yellow_small_balanced < 5.
favorite :- not large, red_small > 15, 
              not balanced, not yellow_small_unbalanced < 5.
\end{problog*}
\end{example}

The advantage of using probabilistic facts and distributional clauses is clear. They allow us to write much more compact and readable programs. However, as they do not really constitute fundamental building blocks of PLP, defining the semantics of a PLP language is much more intricate. For this reason we adapt a two-stage approach to define the semantics of \dcproblogsty. We first define the semantics of \dfplpsty (distributional fact PLP), a bare-bones language with no syntactic sugar only relying on distributional facts to define random variables. This happens in Section~\ref{sec:semantics}. During the second stage we define the program transformations to rewrite syntactic sugar (\eg distributional clauses) as distributional facts. The semantics of \dfplpsty and the program transformations then give us the \dcproblogsty language (\cf Section~\ref{sec:dcproblog}).








\subsection{Panorama of the Inference}
\label{sec:panorama_inference}

The part of the paper concerning inference consists of three sections. First, we start in Section~\ref{sec:inference-tasks} to define a query to a \dcproblogsty program.
For instance, we might be interested in the probability
$$
P(\mathprobloginline{favorite}=\top, \mathprobloginline{large}=\bot)
$$
In other words, the joint probability of \probloginline{favorite} being true and \probloginline{large} being false. While the example query above is simply a joint probability, we generalize this in Section~\ref{sec:inference-tasks} to conditional probabilities (possible with zero-probability events in the conditioning set).

Second, we map the queried ground program to a labeled Boolean formula. Contrary to the approach taken by~\citet{fierens2015inference} the labels are not probabilities (as usual in PLP) but indicator functions. This mapping to a labeled Boolean formula happens again in a series of program transformations, which we describe in Section~\ref{sec:dc2smt}. One of these steps is obtaining the relevant ground program to a query. For example, for the query above 
only the last two rules for  \probloginline{favorite} matter.
\begin{problog*}{}
favorite :- not large, rs > 15, balanced, not ysb < 5.
favorite :- not large, rs > 15, not balanced, not ysu < 5.
\end{problog*}
Here, we abbreviated \probloginline{red_small} as \probloginline{rs} and \probloginline{yellow_small_balanced} and \probloginline{yellow_small_unbalanced} as \probloginline{ysb} and \probloginline{ysu}, respectively. We can further rewrite these rules by replacing \probloginline{large} and \probloginline{balanced} with equivalent comparison atoms and pushing the negation into the comparisons:
\begin{problog*}{}
favorite :- rvl=:=0, rs > 15, rvb=:=1, ysb >= 5.
favorite :- rvl=:=0, rs > 15, rvb=:=0, ysu >= 5.
\end{problog*}
Again using abbreviations: \probloginline{rvl} for \probloginline{rv_large} and \probloginline{rvb} for \probloginline{rv_balanced}.

In Section~\ref{sec:alw} we then show how to compute the expected value of the labeled propositional Boolean formula corresponding to these rules by compiling it into an algebraic circuit, which is graphically depicted in Figure~\ref{fig:circuit:panorama}. In order to evaluate this circuit and obtain the queried probability (expected value), we introduce the IALW algorithm.

The idea of IALW is the following: sample the random variables dangling at the bottom of the circuit by sampling parents before children. For instance, we first sample from $\mathprobloginline{poisson}(10)$ (at the very bottom) before sampling from $\mathprobloginline{poisson}(rs)$ using the sampled value of the parent as the parameter of the child. Once we reach the comparison atoms (\eg $ysb\geq5$) we stick in the sampled values for the mentioned random variables. This evaluates the comparisons to either $1$ or $0$, for which we then perform the sums and products as prescribed by the circuit. We get a Monte Carlo estimate of the queried probability by averaging over multiple such evaluations of the circuit.

The method, as sketched here, is in essence the probabilistic inference algorithm Sampo presented in~\citep{zuidbergdosmartires2019exact}. The key contribution of IALW, which we discuss in Section~\ref{sec:alw}, is to extend Sampo such that conditional inference with zero probability events is performed correctly.  

\begin{figure}
	\resizebox{\linewidth}{!}{%

		\tikzstyle{distribution}=[rectangle, text centered, fill=white, draw, dashed,thick]
		\tikzstyle{leaf}=[rectangle, text centered, fill=gray!10, draw,thick]



		\tikzstyle{negate}=[
			rectangle split,
			rectangle split parts=3, 
			rectangle split horizontal,
			text centered,
			rectangle split part fill={gray!10,white,gray!10},
			draw,
			rectangle split draw splits=false,
			anchor=center,
			align=center,
			thick
		]
		\newcommand{\minus}{  ${\bm e^\otimes}$ \nodepart{second} ${{\bm \ominus}}$ \nodepart{third}  \phantom{${\bm e^\otimes}$}}

		\tikzstyle{sumproduct}=[
			rectangle split,
			rectangle split parts=3, 
			rectangle split horizontal,
			text centered,
			rectangle split part fill={gray!10,white,gray!10},
			draw,
			rectangle split draw splits=false,
			anchor=center,
			align=center,
			thick
		]
		\newcommand{\supr}[1]{  \phantom{${\bm e^\otimes}$} \nodepart{second} ${{\bm #1}}$ \nodepart{third} \phantom{${\bm e^\otimes}$}}
		
		\tikzstyle{circuitedge}=[ultra thick, thick,->]
		\tikzstyle{distributionedge}=[thick,->, dashed, in=-90]
		
		\tikzstyle{indexnode}=[draw,circle, inner sep=1pt]				
		
		\begin{tikzpicture}[remember picture]
			
			\node[sumproduct] (r) at (200.54bp,378.0bp) {\supr{\otimes}};
			\draw[ultra thick, thick,->] (r.90) to  ([shift={(0,1)}]r.90);
			
			\node[sumproduct] (l1) [below left = of r] {\supr{\oplus}};
			\node[sumproduct] (r1) [below right = of r,  yshift=-5.5cm]  {\supr{\otimes}};

			\node[leaf] (r21) [below right=of r1,  xshift=-1.3cm] {$\subnode{var_rvl}{rvl}=0$};
			\node[leaf] (r22) [below left=of r1, xshift=1.3cm] {$\subnode{var_rs}{rs}>15$};




			\node[sumproduct] (l21) [below left=of l1, xshift=0.5cm] {\supr{\otimes}};
			\node[sumproduct] (l22) [below right=of l1, xshift=-0.5cm] {\supr{\otimes}};


			\node[leaf] (l31) [below right=of l21,  xshift=-1.5cm] {$\subnode{var_rvb}{rvb}=1$};
			\node[leaf] (l32) [below left=of l21,  xshift=1.5cm] {$\subnode{var_ysb}{ysb}\geq5$};

			\node[distribution] (rvb)  [below=of l1,  yshift=-3.5cm,xshift=-0.5cm] {$\mathprobloginline{flip}(0.5)$};
			\node[distribution] (ysb)  [left=of rvb, xshift=0.4cm ]{$\mathprobloginline{poisson}(\subnode{poisson_rs_ysb}{rs})$};
			\node[distribution] (ysu)  [right=of rvb, xshift=-0.4cm] {$\mathprobloginline{poisson}(2\times \subnode{poisson_rs_ysu}{rs})$};

			\node[leaf] (l33) [below left=of l22,  xshift=1.5cm] {$\subnode{var_rvbbis}{rvb}=0$};
			\node[leaf] (l34) [below right=of l22,  xshift=-1.5cm] {$\subnode{var_ysu}{ysu}\geq5$};

			\node[distribution] (rvl)  [below=of r21] {$\mathprobloginline{flip}(0.5)$};
			\node[distribution] (rs)  [below=of rvb, yshift=-2cm] {$\mathprobloginline{poisson}(10)$};

			\draw[circuitedge] (l1) to  (r.mid);
			\draw[circuitedge] (r1) to  (r.three south |- r.mid);

			\draw[circuitedge] (l21) to  (l1.mid);
			\draw[circuitedge] (l22) to  (l1.three south |- l1.mid);

			\draw[circuitedge] (l21) to  (l1.mid);
			\draw[circuitedge] (l22) to  (l1.three south |- l1.mid);

			\draw[circuitedge] (l32) to  (l21.mid);
			\draw[circuitedge] (l31) to  (l21.three south |- l21.mid);

			\draw[circuitedge] (l33) to  (l22.mid);
			\draw[circuitedge] (l34) to  (l22.three south |- l22.mid);

			\draw[circuitedge] (r22) to  (r1.mid);
			\draw[circuitedge] (r21) to  (r1.three south |- r1.mid);

			\draw[distributionedge,out=90]  (ysb) to (var_ysb);
			\draw[distributionedge,out=90]  (ysu) to (var_ysu);
			\draw[distributionedge,out=90]  (rvb) to (var_rvb);
			\draw[distributionedge,out=90]  (rvb) to (var_rvbbis);

			\draw[distributionedge,out=90]  (rs) to (poisson_rs_ysb);
			\draw[distributionedge,out=90]  (rs) to (poisson_rs_ysu);
			\draw[distributionedge,out=90]  (rs) to (var_rs);

			\draw[distributionedge,out=90]  (rvl) to (var_rvl);


			

			% \draw[circuitedge] (9) to  (14.mid);
			% \draw[circuitedge] (13) to  (14.three south |- 14.mid);



			% \node[leaf] (l41) [below left=of l31,  xshift=1.3cm] {$\subnode{var_l_rv}{rv\_b}=0$};
			% \node[leaf] (l42) [below right=of l32, xshift=-1.3cm] {$\subnode{var_rs}{rs}>15$};

			% \node[sumproduct] (12)  [below  = of 13] {\supr{\oplus}};
			
			% \node[sumproduct] (8)  [below=of 12]  {\supr{\otimes}};
			
			% \node[indexnode, left=of 14, xshift=0.9cm, yshift=0.4cm] {\footnotesize {$6$}};			
			% \node[indexnode, left=of 13, xshift=0.9cm, yshift=0.4cm] {\footnotesize {$5$}};			
			% \node[indexnode, left=of 8, xshift=0.9cm, yshift=0.4cm] {\footnotesize {$2$}};			
			% \node[indexnode, left=of 12, xshift=0.9cm, yshift=0.4cm] {\footnotesize {$4$}};			
			% \node[indexnode, left=of 9, xshift=0.9cm, yshift=0.4cm] {\footnotesize {$3$}};			
			% \node[indexnode, left=of m1, xshift=0.9cm, yshift=0.4cm] {\footnotesize {$1$}};	
			
			
			% \node[leaf, below= of 8, xshift=-0.2cm] (4)   {$\subnode{var_m0}{$m$}=1$};
			% \node[leaf, below= of 8, xshift=1.8cm] (2) {$\subnode{var_m1}{$m$}=0$};				
			% \node[leaf] (size1obs) [left=of 4, xshift=0cm]  {$\subnode{var_s_11_un}{size_{1}(1)} \doteq 0.4$};
			% \node[leaf] (size0obs) [left=of 4, xshift=-4cm] {$\subnode{var_s_01_un}{size_{0}(1)} \doteq 0.4$};				

			% \node[distribution] (size0)  [below=of size0obs] {\probloginline{beta(2,3)}};
			% \node[distribution] (size1)  [below=of size1obs] {\probloginline{beta(4,2)}};
			% \node[distribution] (flip)  [below= of 4, xshift=1cm] {\probloginline{flip(0.3)}};				


			% \draw[distributionedge,out=90]  (size1) to (var_s_11_un);
			% \draw[distributionedge,out=90] (size0) to  (var_s_01_un);
			% \draw[distributionedge,out=160] (flip) to  (var_m0);
			% \draw[distributionedge] (flip) to  (var_m1);


			% %https://tex.stackexchange.com/questions/447989/anchor-node-names-for-tikz-rectangle-split-horizontal			
			% \draw[circuitedge] (9) to  (14.mid);
			% \draw[circuitedge] (13) to  (14.three south |- 14.mid);
			% \draw[circuitedge] (m1) to  (9.mid);
			% \draw[circuitedge] (12) to  (13.three south |- 13.mid);
			% \draw[circuitedge] (size0obs) to   (m1.three south |- m1.mid);	
			% \draw[circuitedge] (2) to  (12.three south |- 12.mid);				
			% \draw[circuitedge] (size1obs) to   (8.mid);
			% \draw[circuitedge] (4) to  (8.three south |- 8.mid);	
			% \draw[circuitedge] (8) to  (9.three south |- 9.mid);	
			% \draw[circuitedge] (8) to   (12.mid);
			% \draw[circuitedge] (size0obs) to  (13.mid);

		\end{tikzpicture}
	}
    \captionof{figure}{Graphical representation of the computation graph (\ie algebraic circuit) used to compute the  probability $(\mathprobloginline{favorite}=\top, \mathprobloginline{large}=\bot)$ using the IALW algorithm introduced in Section~\ref{sec:alw}.}
    \label{fig:circuit:panorama}
\end{figure}



\section{\dfplpsty}\label{sec:semantics}

Sato's distribution semantics~\citep{sato1995statistical} start from a probability measure over a countable set of facts $\comparisonfacts$, the so-called \emph{basic distribution}, which he then extends this to a probability measure over the Herbrand interpretations of the full program.
It is worth noting that the basic distribution is defined independently of the logical rules and that the random variables are mutually independent.

In our case, the set $\comparisonfacts$ consists of ground Boolean comparison atoms over the random variables, for which we drop the mutual independence assumption. These comparison atoms form an interface between the random variables (represented as terms) and the logical layer (clauses) that reasons about truth values of atoms.
\fixed{This is inspired by the work of \citet{gutmann2011magic} and \citet{nitti2016probabilistic} on the \dcsty language. We discuss this relationship in more detail in the related work Section~\ref{sec:related} and in Appendix~\ref{sec:dcproblog-dc}.
}
% While~\citet{gutmann2011magic} used the same principle to define the distribution semantics for \dcsty, they did not support negation. \citep{nitti2016probabilistic} extended the fixed point semantics for hybrid probabilistic logic programs (also introduced by \citet{gutmann2011magic}) to stratified programs with negation . However, by doing so \citet{nitti2016probabilistic} introduced a procedural element to the semantics.


In this section we introduce the syntax and declarative semantics of \dfplpsty -- a probabilistic programming language with a minimal set of built-in predicates and functors. 
We do this in three steps.
Firstly, we discuss distributional facts and the probability measure over random variables they define (Section~\ref{sec:dist_facts_rand_var}).
Secondly, we introduce the Boolean comparison atoms that form the interface layer between random variables and a logic program (Section~\ref{sec:boolean_comparison_atoms}).
Thirdly, we add the logic program itself (Section~\ref{sec:log_cons_bool_comp}).
Fourht, we discuss practical considerations for constructing sets of distributional facts (Section~\ref{sec:fintiedistdb}).
An overview table of the notation related to semantics is provided in Appendix~\ref{app:table}.

\begin{definition}[Reserved Vocabulary]
    \label{def:reserved_vocabulary}
We use the following {\em reserved} vocabulary (built-ins), whose intended meaning is fixed across programs:
\begin{itemize}
    \item a finite set $\distributionfunctors$ of \emph{distribution functors}.
    \item a finite set $\arithmeticfunctors$ of \emph{arithmetic functors}.
    % \footnote{
    %     As logic programming generally operates on a countable universe, we restrict the arithmetic functors to the rational numbers (constructed from the natural numbers and appropriate functors). Similarly to implementations of \prologsty, an implementation of \dcproblogsty can support integers and floating point numbers using external arithmetic engines~\cite[Chapter 9]{sterling1994art}. We will use those in our examples for ease of readability.
    % }
    \item A finite set $\comparisonpredicates$ of binary \emph{comparison predicates}, 
    \item the binary predicate \predicate{~}{2} (in infix notation).
\end{itemize}
\end{definition}

Examples of distribution functors that we have already seen in Section~\ref{sec:panorama} are \functor{poisson}{1} and \functor{flip}{1} but may also include further functors such as \functor{normal}{2} to denote normal distributions. Possible arithmetic functors are \functor{*}{2} (\cf Example~\ref{example:sweets_dc}) but also \functor{max}{2}, \functor{+}{2}, \functor{abs}{1}, \etc.
Binary comparison predicates (in Prolog syntax and infix notation) are \predicate{<}{2}, \predicate{>}{2}, \predicate{=<}{2}, \predicate{>=}{2}, \predicate{=:=}{2}, \predicate{=\=}{2}.
The precise definitions of $\distributionfunctors$, $\arithmeticfunctors$ and $\comparisonpredicates$ are left to system designers implementing the language.

\begin{definition}[Regular Vocabulary] \label{def:regular_vocabulary}
    We call an atom $\mu(\rho_1, \dots, \rho_k)$ whose predicate \mathfunctor{\mu}{k} is not part of the reserved vocabulary a \emph{regular} atom. The set of all regular atoms constitutes the regular vocabulary. 
\end{definition}


As a  brief comment on notation: in the remainder of the paper we will usually denote logic program expressions in teletype font (\eg \probloginline{4>x}) when giving examples. When defining new concepts or stating theorems and propositions, we will use the Greek alphabet.


\subsection{Distributional Facts and Random Variables}\label{sec:dist_facts_rand_var}

\begin{definition}[Distributional Fact]\label{def:distfact}
A distributional fact is of the form $\nu \sim \delta$, with $\nu$ a regular ground term, and $\delta$ a ground term whose functor is in $\distributionfunctors$. 
The distributional fact states that the ground term $\nu$ is interpreted as a random variable distributed according to $\delta$.
\end{definition}

\begin{definition}[Sample Space]\label{def:samplespace}
    Let $\nu$ be be a random variable distributed according to $\delta$. The set of possible samples (or values) for $\nu$ is the sample space denoted by $\samplespace_\nu$ and which is determined by $\delta$. We denote a sample from $\samplespace_\nu$ by $\samplefunction(\nu)$, where $\samplefunction$ is the {\em sampling} or {\em value} function. 
\end{definition}


\begin{definition}[Distributional Database]\label{def:distDB}
A \emph{distributional database} is a \new{(not necessarily countable)} set $\distdb=\{\nu_1 \sim \delta_1,\nu_2 \sim \delta_2, \ldots \}$ of distributional facts, with distinct $\nu_i$. We let $\randomvariableset= \{\nu_1, \nu_2, \dots \}$ denote the set of random variables.
\end{definition}


\begin{example}\label{ex:dist_db} 
The following distributional database encodes a Bayesian network with  normally distributed random variables, two of which serve as parameters in the distribution of another one. We thus have $\randomvariableset= \{ \text{\probloginline{x}}, \text{\probloginline{y}},\text{\probloginline{z}} \}$.
\begin{problog*}{linenos}
% distributional facts $\distdb$
x ~ normal(5,2).
y ~ normal(x,7).
z ~ normal(y,1).
\end{problog*}
\end{example}



\new{
\begin{definition}[Well-Defined Distributional Database]
    \label{def:well-distdb}
    We call a distributional database $\distdb$ well-defined if and only if the product of probability spaces induced by the random variables in $\distdb$ is unique. We denote this product probability space by
    $\probabilityspace_\distdb=(\samplespace_\distdb, \sigmaalgebra_\distdb, \probabilitymeasure_\distdb)$, where $\samplespace_\distdb$ is the sample space, $\sigmaalgebra_\distdb$ the sigma-algebra, and  $\probabilitymeasure_\distdb$ the probability measure. 
\end{definition}

\new{
Note that in Definition~\ref{def:distDB} we do not impose the restriction of having a countable set of random variables. This allows for modelling a rich class of probabilistic models, including random processes that are described via uncountable sets of random variables. We discuss the construction of well-defined databases in Section~\ref{sec:fintiedistdb}.
}


}



% \begin{restatable}{proposition}{proppv}
% 	\label{prop:pv}
% A well-defined distributional database~$\distdb$  defines a unique probability measure $\measurerandomvariableset$ on value assignments $\samplefunction(\randomvariableset)$.
% \end{restatable}

% \begin{proof}
%     See Appendix~\ref{app:proof:pv}.
% \end{proof}

%%%%%% comparison atom layer %%%%%%%%%%

\subsection{Boolean Comparison Atoms over Random Variables} \label{sec:boolean_comparison_atoms}

















Starting from a well-defined distributional database, we now introduce the \new{probability space $\probabilityspace_\comparisonfacts$ induced by Boolean comparison atoms}, which corresponds to the basic (discrete) distribution in Sato's distribution semantics.

\begin{definition}[Boolean Comparison Atoms]\label{def:comparison-atoms-set}
Let $\distdb$ be a well-defined distributional database. 
A binary \emph{comparison atom} $\gamma_1 {\bowtie} \gamma_2$ over $\distdb$ is a ground atom with predicate $\bowtie \in \comparisonpredicates$. The ground terms $\gamma_1$  and $\gamma_2$ are either random variables in $\randomvariableset$ or terms whose functor is in $\arithmeticfunctors$.
We denote by $\comparisonfacts$ a countable set of \new{$\probabilitymeasure_\distdb$-measurable} 
Boolean comparison atoms over $\distdb$.
\end{definition}

\begin{example}
    Examples of Boolean comparison atoms over the distributional database of Example~\ref{ex:dist_db} include   \probloginline{z>10},  \probloginline{x<y}, \probloginline{abs(x-y)=<1},  and \probloginline{7*x=:=y+5}. 
\end{example}

\new{
\begin{restatable}{proposition}{propomegaf}
    \label{prop:omegaf}
The Boolean comparison atoms $\comparisonfacts$ induce a product sample space $\samplespace_\comparisonfacts$.
\end{restatable}

\begin{proof}
    See Appendix~\ref{app:proof:omegaf}.
\end{proof}


}

\new{

\begin{restatable}{proposition}{proppfsigma}
    \label{prop:pfsigma}
    The Boolean comparison atoms $\comparisonfacts$ induce a sigma-algebra $\sigmaalgebra_\comparisonfacts{\subseteq}\sigmaalgebra_\distdb$.
\end{restatable}


\begin{proof}
    See Appendix~\ref{app:proof:pfsigma}.
\end{proof}

}

\new{

\begin{restatable}{proposition}{proppf}
    \label{prop:pf}
Let $\distdb$ be a well-defined distributional database, the function $\probabilitymeasure_\comparisonfacts$ defined via
$
\probabilitymeasure_\comparisonfacts(A)
=
\frac
{
    \probabilitymeasure_\distdb(A)
}
{
    \probabilitymeasure_\distdb(\samplespace_\comparisonfacts)
}
$ defines a unique probability measure over the sample space $\samplespace_\comparisonfacts$ and the sigma algebra $\sigmaalgebra_\comparisonfacts$.
\end{restatable}

\begin{proof}
    See Appendix~\ref{app:proof:pf}.
\end{proof}





}












\new{


\begin{proposition}
    The triplet $\probabilityspace_\comparisonfacts= (\samplespace_\comparisonfacts, \sigmaalgebra_\comparisonfacts, \probabilitymeasure_\comparisonfacts)$ forms a probability space.
\end{proposition}

\begin{proof}
    This follows immediately from Propositions \ref{prop:omegaf}, \ref{prop:pfsigma}, and \ref{prop:pf}.
\end{proof}
Note that, while \citeauthor{sato1995statistical} refers to $\probabilitymeasure_{\comparisonfacts}$ as the \textit{basic distribution}, we use the term  \textit{basic measure}. This is due to the fact that we do not necessarily have a distribution.
}


%%%%%% logic program layer %%%%%%%%%%

\subsection{Logical Consequences of Boolean Comparisons}\label{sec:log_cons_bool_comp}
We now define the semantics of a \dfplpsty program, \ie, extend the basic measure $\measurecomparisonfacts$ over the comparison atoms to a measure over the Herbrand interpretations of a logic program.

\begin{definition}[\dfplpsty Program]\label{def:core-prog}
A \dfplpsty program $\dfprogram= \distdb \cup \logicprogram $ consists of a well-defined distributional database $\distdb$ (Definition~\ref{def:well-distdb}), comparison atoms $\comparisonfacts$ (Definition~\ref{def:comparison-atoms-set}), and a normal logic program $\logicprogram$ where clause heads belong to the regular vocabulary (cf. Definition~\ref{def:regular_vocabulary}), and which can use comparison atoms from $\comparisonfacts$ in their bodies.
\end{definition}

\begin{example}\label{ex:random_choice_dependency}
We further extend the running example.
\begin{problog*}{linenos}
% distributional facts $\distdb$
x ~ normal(5,2).
y ~ normal(x,7).
z ~ normal(y,1).
% logic program $\logicprogram$
a :- abs(x-y) =< 1. 
b :- not a, z>10.
\end{problog*}
The logic program defines two logical consequences of Boolean comparisons, where \probloginline{a} is true if the absolute difference between random variables \probloginline{x} and \probloginline{y} is at most one, and \probloginline{b} is true if \probloginline{a} is false, and the random variable \probloginline{z} is greater than $10$. 
\end{example}

In order to extend the basic measure to logical consequences, \ie logical rules, we require the notion of a {\em consistent comparisons database} (CCD).
The key idea is that samples of the random variables in $\distdb$ jointly determine a truth value assignment to the comparison atoms in $\comparisonfacts$.


\begin{definition}
    A {\em value assignment}
    $\samplefunction(\randomvariableset)$ is a combined value assignment to all random variables $\randomvariableset=\{\nu_1,\nu_2,... \}$, \ie,  $\samplefunction(\randomvariableset)=(\samplefunction(\nu_1),\samplefunction(\nu_2),\ldots)$.
\end{definition}


\begin{definition}[Consistent Comparisons Database]\label{def:consistent-fact-db}
    Let $\distdb$ be a well-defined distributional database,  $\comparisonfacts=\{\kappa_1, \kappa_2,\ldots\}$ a set of measurable Boolean comparison atoms, and  $\samplefunction(\randomvariableset)$ a  value assignment to all random variables $\randomvariableset=\{ \nu_1,\nu_2,... \}$. 
    We define $I_{\samplefunction(\randomvariableset)}(\kappa_i)=\top$ if $\kappa_i$ is true after setting all random variables to their values under $\samplefunction(\randomvariableset)$, and $I_{\samplefunction(\randomvariableset)}(\kappa_i) = \bot$ otherwise. $I_{\samplefunction(\randomvariableset)}$ induces the consistent comparisons database $\comparisonfacts_{\samplefunction(\randomvariableset)}=\{\kappa_i \mid I_{\samplefunction(\randomvariableset)}(\kappa_i) = \top\}$.
\end{definition}

To define the semantics of a \dfplpsty program $\dfprogram$, we now require that, given a CCD $\comparisonfacts_{\samplefunction(\randomvariableset)}$, the logical consequences in $\dfprogram$ are uniquely defined.
As common in the PLP literature, we achieve this by requiring the program to have a two-valued well-founded model~\citep{van1991well} for each possible value assignment $\samplefunction(\randomvariableset)$.


\begin{definition}[Valid \dfplpsty Program]\label{def:core-valid}
A \dfplpsty program $\dfprogram=\distdb \cup\logicprogram $ is called \emph{valid} if and only if for each CCD $\comparisonfacts_{\samplefunction(\randomvariableset)}$, the logic program $\comparisonfacts_{\samplefunction(\randomvariableset)} \cup  \logicprogram$ has a two-valued well-founded model. 
\end{definition}



We follow the common practice of defining the semantics with respect to ground programs; the semantics of a program with non-ground $\logicprogram$ is defined as the semantics of its grounding with respect to the  Herbrand universe.

\begin{restatable}{proposition}{proppp}
\label{prop:pp}
A valid \dfplpsty program \dfprogram  induces a unique probability measure $\probabilitymeasure_\dfprogram$ over Herbrand interpretations.
\end{restatable}

\begin{proof}
    See Appendix~\ref{app:proof:pp}.
\end{proof}


\begin{definition}
    We define the declarative semantics of a \dfplpsty program \dfprogram to be the probability measure $\probabilitymeasure_\dfprogram$, and we call this the {\em measure semantics}.  
\end{definition}

In contrast to the imperative semantics of~\citet{nitti2016probabilistic}, in \dfplpsty the connection between comparison atoms and the logic program is purely declarative. That is, logic program negation on comparison atoms negates the (interpreted) comparison. For example, if we have a random variable \probloginline{n}, then \probloginline{n>=2} is equivalent to \probloginline{not n<2}. Such equivalences do not hold in the stratified programs introduced by~\citet{nitti2016probabilistic}.
This then allows the programmer to refactor the logic part as one would expect. 



\subsection{Constructing Distributional Databases}
\label{sec:fintiedistdb}

\new{
In the definition of the measure semantics, we simply assumed that the distributional database $\distdb$ was valid, and we forewent an explicit construction of such databases.
When implementing \dcplpsty we would, however, ideally have a prescription for constructing $\distdb$.
A practical choice that is often made in probabilistic logic programming~\citep{kersting2000bayesian,fierens2015inference} restricts the distributional database to be a finite (and consequently countable) set.
For such distributional databases with a countable number of random variables to be meaningful, they have to encode a unique joint distribution over the variables~$\randomvariableset$.
}

We will now provide the conditions under which a finite distributional database is valid, but first we define the concepts of parents and ancestors that, which also hold for non-finite distributional databases.
\begin{definition}[Parents and Ancestors of Random Variables]
    \label{def:df_ancestor}
Let $\distdb$ be a distributional database. For facts $\nu_p\sim\delta_p$ and $\nu_c\sim\delta_c$  in $\distdb$.  The random variable $\nu_p$  is a \emph{parent} of the child random variable $\nu_c$ if and only if  $\nu_p$ appears in $\delta_c$. We define \emph{ancestor} to be the transitive closure of \emph{parent}. A node's ancestor set is the set of all its ancestors.
\end{definition}

\begin{example}[Ancestor Set]
    We graphically depict the ancestor set of the distributional database in Example~\ref{ex:dist_db} in Figure~\ref{fig:ex:dist_db_ancestor}.

  
    \begin{figure}[h]
        \centering
    \tikzstyle{node}=[circle, text centered, draw,thick]
        \begin{tikzpicture}[remember picture]
			

            \node[node] (x) {\probloginline{x}};
            \node[node] (y) [right=of x] {\probloginline{y}};
            \node[node] (z) [right=of y] {\probloginline{z}};

			\draw[->,thick] (x) to  (y);
			\draw[->,thick] (y) to  (z);


        \end{tikzpicture}
        \caption{Directed acyclic graph representing the ancestor relationship between the random variables in Example~\ref{ex:dist_db}. The ancestor set of \probloginline{x} is the empty set, the one of \probloginline{y} is $\{\mathprobloginline{x} \}$ and the one of \probloginline{z} is $\{\mathprobloginline{x}, \mathprobloginline{y} \}$.}
        \label{fig:ex:dist_db_ancestor}
    \end{figure}

\end{example}





% \new{
% \begin{definition}[Ancestor Chain]
%     \label{def:df_ancestor_chain}
% An ancestor chain is an ordered set of random variables $\nu_1, \nu_2, \dots$ for which we have that $\nu_{i+1}$ is a parent of $\nu_{i}$. 
% \end{definition}
% }







\begin{definition}[Well-Defined Finite Distributional Database]\label{def:well-defd-facts}
A finite distributional database~$\distdb$  is called \emph{well-defined} if and only if it satisfies the following criteria:
\begin{description}
    \item[W!] Each $\nu \in \randomvariableset$ has a finite set of ancestors.
    \item[W1] The ancestor relation on the variables~$\randomvariableset$ is acyclic.
\item[W3] If $\nu \sim \delta \in \distdb$ and the parents of $\nu$ are $\{ \nu_1, \dots, \nu_m \}$, then replacing each occurrence of $\nu_i$ in $\delta$ by a sample $\samplefunction(\nu_i)$ always results in a well-defined distribution for~$\nu$. 
\end{description}  
\end{definition}

The distributional database in Example~\ref{ex:dist_db} is well-defined: the ancestor relation is acyclic and finite, and as normally distributed random variables are real-valued, using such a variable as the mean of another normal distribution is always well-defined. The database would no longer be well-defined after adding \probloginline{w ~ poisson(x)}, as not all real numbers can be used as a parameter of a Poisson distribution. 

\new{
Constructing valid finite distributional databases can be viewed as building Bayesian networks over a finite set of variables, where nodes represent (conditional) random variables and where each node's distribution is parameterized by the node's parents.
This approach was, for instance, taken by~\citep{kersting2000bayesian}.
More recently, \citet{wu2018discrete} extended these finite Bayesian network to allow also (under certain conditions) for infinite and uncountable numbers of nodes in these Bayesian networks. They dub this extension {\em measure theoretic Bayesian network} (MTBNs). 

While our measure semantics allow for MTBNs as the distributional database, we will restrict ourselves in the remainder of the paper to distributional databases that are finite. We leave it as future work on how to effectively use MTBNs in probabilistic logic programming. This would necessitate developing novel syntax and inference algorithms able to handle, for example, stochastic processes.
}
%
\section{\dcproblogsty Semantics}\label{sec:dcproblog}
While DC-PLP is an expressive probabilistic programming language for which we have defined fully declarative semantics, using DC-PLP in practice might turn out to be challenging. The syntax is simply too cumbersome. Therefore, we will now introduce semantics for the higher-level \dcproblogsty language whose syntax we already introduced in Section~\ref{sec:syntax}.




A \dcproblogsty program consists of a countable set of ground distributional facts, a countable set of ground probabilistic facts, and a logic program (i.e. a finite set of facts and normal clauses), where bodies of normal clauses can use comparison predicates.   


\begin{definition} A \dcproblogsty program \dcpprogram is {\bf valid} if and only if it encodes a unique probability distribution.
\end{definition} 

This raises the question: what are the conditions we have to impose on \dcproblogsty programs such that they are valid, or in other words such that they encode unique probability distributions? We will answer this question in Theorem~\ref{theo:valid_dcproblog}. But first we are going to define a program transformation that maps a \dcproblogsty program to a DC-PLP program.  





\begin{definition} \label{def:dcproblog2dcplp}


\todo{ make transformations formal with substitutions}

\pedro{@Angelika, I believe you mentioned you had this transformation ready already?}
\ak{see copy from old below for probabilistic facts; LP stays the same (modulo dropping the test-wrapping?); distributional facts should be just rv(v,d) to v\probloginline{~}d (the complicated part would be directly mapping full DC)}

\end{definition}

\ak{begin copy}
\subsubsection{ProbLog with probabilistic facts only}
ProbLog with probabilistic facts (no ADs) restricts $P$ to discrete distributions over Boolean truth values, and $Q$ to equality comparisons between a random variable and the constant \verb|true|. The translation is modular, mapping each normal logic clause to itself, and each ground probabilistic fact \verb|p::f| to a program fragment
\begin{verbatim}
rv(f,flip(p)).
f :- holds(f=true).
\end{verbatim}
where \verb|flip(p)| denotes the distribution that returns \verb|true| with probability \verb|p| and \verb|false| otherwise.\footnote{This assumes that all probabilistic facts are syntactically different and associated to a single probability; for the multi-set view of the current system, the name of the random variable has to be made unique through different means, which is straightforward.} The correspondence in semantics is straightforward. 

\subsubsection{ProbLog with ADs}
Adding ADs to ProbLog adds discrete distributions over finite sets $\{1,...,n\}$ to $P$, and equality comparisons between random variables and the elements of these sets to $Q$.
A ground AD \verb|p1::h1;...;pn::hn :- body| maps to a program fragment
\begin{verbatim}
rv(id,[p1:1,...pn:n]).
h1 :- body, holds(id=1).
...
hn :- body, holds(id=n).
\end{verbatim}
where \verb|id| is a fresh term not used anywhere else. The correspondence in semantics is again straightforward. 
\ak{end copy}




\begin{theorem} \label{theo:valid_dcproblog}
A \dcproblogsty program is valid if the following two validity conditions hold.
	\begin{itemize}
	    \item[V1] No two distributional facts can introduce the same random variable, \ie use the same term on the left hand side.
		\item[V2] The program \dcpprogram is distribution stratified, that is, there exists a function $rank(\cdot)$ that maps ground terms to $\mathbb{N}$ and that satisfies the following properties:
		\begin{enumerate}			
			\item for each ground instance of a distributional fact of the form \linebreak \probloginline{rand~dist(d@$_1$@,...,d@$_n$@)} it holds that   $rank(\text{\probloginline{rand}})>rank(\text{\probloginline{d@$_i$@}})$ for every $i$ ($1\leq i \leq n$).
			\item for each ground instance of another program clause \probloginline{h:- b@$_1$@,...,b@$_n$@} it holds that $rank(\text{\probloginline{h}})\geq rank(\text{\probloginline{b@$_i$@}})$, for all $i$.
			\item for each ground term \probloginline{b} that contains a random variable \probloginline{rand}, $rank(\text{\probloginline{b}})\geq rank(\text{\probloginline{rand~dist}})$ (with \probloginline{rand~dist} the head of the distributional clause defining \probloginline{rand}).
		\end{enumerate}
	    \item[V3] \todo{Lebesgue measurabiltiy of comparisons}
	\end{itemize}
\end{theorem}


\proof{\todo{}

\begin{itemize}
    \item use transformation from \dcproblogsty to DC-PLP and show uniqueness under conditions
    \item we need to show that the transformation is unique (order does not matter) and results in a valid DC-PLP program
\end{itemize}
}


\todo{Discuss complexity/decidability of determining whether program is valid, cf. \citep[Chapter 4.5]{milch2006probabilistic}. discuss sufficiency not necessity of validity conditions. maybe disucss in dc-plp section }








\section{\dcproblogsty with Distributional Clauses}



\begin{example}[\dcproblogsty Program] \label{ex:dcproblog_distributional_clause}
	This example program models the correct working of a machine. The probability distribution of the \probloginline{temperature} of the machine depends on whether it is a \probloginline{hot} day or not.
	\begin{problog*}{linenos}
machine(1).

0.2::hot.
0.99::cooling(1).

temperature ~ normal(27,5):- hot.
temperature ~ normal(20,5):- \+hot. 

works(N):- machine(N), cooling(N).
works(N):- machine(N), temperature<25.0. 

query(works(1)).
	\end{problog*}
	Note that we introduced {\em distributional clauses} in this example, which let us write down conditional probability distributions. 
\end{example}






\begin{definition}
	Distributional clauses are syntactic sugar to concisely write down conditional probability distributions of random variables. \todo{more elaborate}

	
\end{definition}




\begin{definition}




\todo{formal transformation: \dcproblogsty with distributional clauses to \dcproblogsty without}
	A \dcproblogsty program containing a random variable \probloginline{rand} involved in $N$ distributional clauses
    \begin{problog}
rand ~ dist@$_i$@ :- b@$_i$@
    \end{problog}
	with mutually logically inconsistent bodies $\text{\probloginline{b@$_i$@}}$ ($1{\leq} i {\leq} N$)
	is equivalent to a \dcproblogsty program with $N$ fresh random variables \probloginline{rand@$_i$@} ($1{\leq} i {\leq} N$) involved in $N$ distinct distributional facts
    \begin{problog}
rand@$_i$@ ~ dist@$_i$@.
    \end{problog}
    Each clause initially containing \probloginline{rand} is replaced in the equivalent program that does not contain distributional clauses for \probloginline{rand} by $N$ clauses, where \probloginline{rand} is substituted by $\text{\probloginline{rand@$_i$@}}$ and its body is conjoined with \probloginline{b@$_i$@} for every i with $1{\leq} i {\leq} N$.

\end{definition}



Mapping distributional clauses to distributional facts is akin to mapping intentional probabilistic facts (e.g \probloginline{0.2::a:- b.}) to probabilistic facts (and deterministic rules). Note that \dcproblogsty allows, just as \problogsty, the usage of intentional probabilistic facts, with the same semantics as in \problogsty. In absence of any distributional facts and clauses, a \dcproblogsty program reduces to a \problogsty program.


\begin{theorem}\label{theo:valid_dcproblog_w_clauses} A {\bf \dcproblogsty program} \dcpprogram{} {\bf with distributional clauses} is {\bf valid} if it can be mapped to an equivalent valid \dcproblogsty program. Such a mapping exists if the following three validity conditions hold.
	\begin{itemize}
		\item[V1] The bodies of ground distributional clauses with the same random variable in their heads are pairwise logically inconsistent.
		\item[V2] The program \dcpprogram is distribution stratified, that is, there exists a function $rank(\cdot)$ that maps ground terms to $\mathbb{N}$ and that satisfies the following properties:
		\begin{enumerate}			
			\item for each ground instance of a distributional clause of the form \linebreak \probloginline{rand~dist(d@$_1$@,...,d@$_n$@):-b} it holds that   $rank(\text{\probloginline{rand}})>rank(\text{\probloginline{d@$_i$@}})$ for every $i$ ($1\leq i \leq n$).
			\item for each ground instance of a distributional clause \probloginline{rand~dist:-b@$_1$@,...,b@$_n$@} it holds that $rank(\text{\probloginline{rand ~ dist}})>rank(\text{\probloginline{b@$_i$@}})$, for all $i$.
			\item for each ground instance of another program clause \probloginline{h:- b@$_1$@,...,b@$_n$@} it holds that $rank(\text{\probloginline{h}})\geq rank(\text{\probloginline{b@$_i$@}})$, for all $i$.
			\item for each ground term \probloginline{b} that contains a random variable \probloginline{rand}, $rank(\text{\probloginline{b}})\geq rank(\text{\probloginline{rand~dist}})$ (with \probloginline{rand~dist} the head of the distributional clause defining \probloginline{rand}).
		\end{enumerate}
	    \item[V3] \todo{Lebesgue measurabiltiy of comparisons}
	\end{itemize}

\proof{transform distributional clauses to distributional facts.

}
\end{theorem}

The criterion V1 in Theorem~\ref{theo:valid_dcproblog_w_clauses} is equivalent to the validity criterion V1 given in~\citep[Definition 3]{gutmann2011magic}, which requires that there is a unique ground distribution for each ground random variable \probloginline{rand}. V2 is a carbon copy of the second validity criterion of~\citep[Definition 3]{gutmann2011magic}, with the exception of the first point of V2. We added it, as distributional clauses with cyclic functional dependencies were not {\em explicitly} disallowed in ~\citep{gutmann2011magic}.

\subsection*{Difference to Distributional Clauses}

\ak{discussion on differences to DC: in how far is this semantics vs implementation? I believe we silently assumed (but never stated) that only defined RVs would be used, but I don't remember how Davide handled this when introducing negation to the semantics...}

This is also a good point to illustrate a major difference between the semantics of random variables in Distributional Clauses and \dcproblogsty. Take the code snippet below:
\begin{problog*}{linenos}
x ~ normal(20,2).

q(1):- x>20.
q(2):- y>20.

query(q(1)).
query(q(2)).
\end{problog*}
The first query succeeds and returns as an answer $p(\text{\probloginline{query(q(1)))}}) = 0.5$. The second query, however, throws an error as there is no term of type \type{RandomVariable} declared that is named \probloginline{y}. The second query would also fail if there were an \type{Atom} term named \probloginline{y}. In this case the lefthand side of \probloginline{y>20} would be ill-typed.

Let us now write the same program as a Distributional Clauses program:
\begin{adjustwidth}{1em}{0em}
	{
		\normalfont
		\begin{lstlisting}[breaklines,mathescape]
x ~ normal(20,3).

q(1)$\leftarrow$ $ {\simeq}(x)$<20.
q(2)$\leftarrow$  $ {\simeq}(y)$<20.

query(q(1)).
query(q(2)).
		\end{lstlisting}
	}
\end{adjustwidth}
The first query will again return $ p(\text{\lstinline|query(q(1))|}) =0.5$ (or an approximation thereof) but the second query will not throw an error but return a probability of zero. How does this work? The term \lstinline[mathescape]|$ {\simeq} (y)$| tries to unify with the value of the random variable \lstinline|y|. As there is no such random variable, this silently fails and the body of the clause whose head is \lstinline|q(2)| is simply not satisfied.

If we were to negate the comparison predicate in the body of the \lstinline|q(2)| clause, \ie write:
\begin{adjustwidth}{1em}{0em}
	{
		\normalfont
		\begin{lstlisting}[breaklines,mathescape]
q(2) $\leftarrow$  \+(${\simeq}(y)$<20).
		\end{lstlisting}
	}
\end{adjustwidth}
we would obtain for the second query: $ p(\text{\lstinline|query(q(1))|}) =1$. Negating the corresponding comparison predicate in \dcproblogsty would still result in an ill-typed term, \ie an error would be thrown.

The comparison we just performed does not only illustrate a semantic difference between \dcproblogsty and Distributional Clauses but also a conceptual one. In \dcproblogsty the values of of random variables never enter the realm of the logic program: values of random variables only live in the external arithmetic engine of \dcproblogsty. In contrast,  random variables in Distributional Clauses are part of the logic program: the values of random variables are stored in a logic data base and can be accessed through logical unification. Distributional Clauses currently does this via SLD resolution.

%
\section{Probabilistic Inference Tasks} \label{sec:inference-tasks}

In Section~\ref{sec:log_cons_bool_comp} we defined the probability distribution induced by a \dfplpsty program by extending the basic measure to logical consequences (expressed as logical rules). The joint distribution is then simply the joint distribution over all (ground) logical consequences. We obtain marginal probability distributions by marginalizing out specific logical consequences.
This means that marginal and joint probabilities of atoms in \dfplpsty programs are well-defined. 
Defining the semantics of probabilistic logic programs using an extension of Sato's distribution semantics gives us the semantics of probabilistic queries: the probability of an atom of interest is given by the probability induced by the joint probability of the program and marginalizing out all atoms one is not interested in.

The situation is more involved with regard to conditional probability queries. In contrast to unconditional queries, not all conditional queries are well-defined under the measure semantics (as well ass the distribution semantics). We will now give the formal definition of the PROB task, which lets us compute the (conditional) marginal probability of probabilistic events and which has so far not yet been defined in the PLP literature for hybrid domains under a declarative semantics (\eg \citep{azzolini2021semantics}).

After defining the task of computing conditional marginal probabilities, we will study how to compute these probabilities in the hybrid domain. Before defining the PROB task, we will first need to formally introduce the notion of a conditional probability with respect to a  \dcproblogsty program.


\begin{definition}[Conditional Probability]
\label{def:conditional_prob}
Let $\herbrandbase$ be 
% the Herbrand base, i.e,
the set of all ground atoms in a given \dcproblogsty program \dcpprogram.
Let $\evidenceset = \{\eta_1,\ldots,\eta_n\} \subset \herbrandbase$ be a set of observed atoms, and  $e=\langle e_1,\ldots, e_n\rangle$ a vector of corresponding  observed truth values, with $e_i\in \{\bot, \top \}$.
We refer to $(\eta_1=e_1) \wedge \ldots\wedge (\eta_n=e_n)$  as the evidence and write more compactly  $\evidenceset = e$.
Let  $\mu \in\herbrandbase$ be an atom of interest called the query.
If the probability of $\evidenceset = e$ is greater than zero, then the conditional probability of $\mu=\top$ given $\evidenceset=e$ is defined as:
\begin{align}
    P_\dcpprogram(\mu=\top \mid \evidenceset=e)= \frac{P_{\dcpprogram}(\mu=\top, \evidenceset=e)}{P_\dcpprogram(\evidenceset=e)} \label{eq:conditional_prob}
\end{align}
\end{definition}

\begin{definition}[PROB Task]
\label{def:prob_task}
Let $\herbrandbase$ be the set of all ground atoms of a given \dcproblogsty program \dcpprogram.
We are given the (potentially empty) evidence $\evidenceset=e$ (with $\evidenceset \subset \herbrandbase$)
and a set $\queryset \subset \herbrandbase$ of atoms of interest, called
query atoms.
The {\bf PROB task} consists of computing the conditional probability of the truth value of every atom in $\queryset$ given the evidence, \ie compute the conditional probability $P_{\dcpprogram} (\mu {=} \top \mid \evidenceset {=} e)$ for each $\mu \in \queryset$.
\end{definition}

\begin{example}[Valid Conditioning Set]
    Assume two random variables $\nu_1$ and $\nu_2$, where $\nu_1$ is distributed according to a normal distribution and $\nu_2$ is distributed according to a Poisson distribution. Furthermore, assume the following conditioning set $\evidenceset= \{\eta_1=\top, \eta_2=\top \}$, where $\eta_1 \leftrightarrow (\nu_1>0)$ and $\eta_2 \leftrightarrow (\nu_2=5)$. This is a valid conditioning set as none of the events has a zero probability of occurring, and we can safely perform the division in Equation~\ref{eq:conditional_prob}.
\end{example}

\subsection{Conditioning on  Zero-Probability Events}

A prominent class of conditional queries, which are not captured by Definition~\ref{def:conditional_prob}, are so-called  zero probability conditional queries. For such queries the probability of the observed event happening is actually zero but the event is still possible. Using Equation~\ref{eq:conditional_prob} does not work anymore as a division by zero would now occur.

\begin{example}[Zero-Probability Conditioning Set]
    Assume that we have a random variable $\nu$ distributed according to a normal distribution and that we have the conditioning set $\evidenceset= \{ \eta=\top \} $, with $\eta\leftrightarrow (\nu=20)$. In other words, we condition the query on the observation that the random variable $\nu$ takes the value $20$ -- for instance in a distance measuring experiment. This is problematic as the probability of any specific value for a random variable with uncountably many outcomes is in fact zero and applying Equation~\ref{eq:conditional_prob} leads to a division-by-zero. Consequently,  an ill-defined conditional probability arises.
\end{example}

In order to sidestep divisions by zero when conditioning on zero-probability (but possible) events, we modify Definition~\ref{def:conditional_prob}. Analogously to~\citet{nitti2016probabilistic}, we follow the approach taken in~\citep{kadane2011principles}. 
 
 
\begin{definition}[Conditional Probability with Zero-Probability Events]
\label{def:conditional_prob_zero_event}
Let $\nu$ be a continuous random variable in the \dcproblogsty program \dcpprogram with ground atoms \herbrandbase. Furthermore, let us assume that the evidence consists of $\evidenceset = \{ \eta_0 = \top  \}$ with $\eta_0\leftrightarrow (\nu=w)$ and $w\in \samplespace_\nu$.
The conditional probability of an atom of interest $\mu\in \herbrandbase$ is now defined as:
\begin{align}
    P_\dcpprogram (\mu= \top \mid \eta_0= \top)
    &= \lim_{\Delta w \rightarrow 0}
    \frac{P_\dcpprogram(\mu=\top, \nu \in [w-\nicefrac{\Delta w}{2}, w+\nicefrac{\Delta w}{2} ]) }{P_\dcpprogram(\nu \in [w-\nicefrac{\Delta w}{2}, w+\nicefrac{\Delta w}{2} ] )} \label{eq:condition_prob_zero_ev}
\end{align}
\end{definition}

To write this limit more compactly, we introduce an 
infinitesimally small constant $\delta w$ and two new comparison atoms  $\eta_1 \leftrightarrow ( w-\nicefrac{\delta w}{2} \leq  \nu)$ and $\eta_2 \leftrightarrow (\nu \leq w+\nicefrac{\delta w}{2} )$ that together encode the limit interval.
Using these, we rewrite Equation~\ref{eq:condition_prob_zero_ev} as
\begin{align}
    P_\dcpprogram (\mu = \top \mid \eta_0 = \top) =  \frac{P_\dcpprogram(\mu=\top,  \eta_1=\top, \eta_2=\top )}{P_\dcpprogram( \eta_1=\top, \eta_2=\top )}
\end{align}

Applying the definition recursively, allows us to have multiple zero probability conditioning events. More specifically, let us assume an additional continuous random variable $\nu'$ that takes the value $w'$ for which we define: 
$\eta^{\prime}_1 \leftrightarrow ( w'-\nicefrac{\delta w'}{2} \leq  \nu')$ and $\eta^{\prime}_2 \leftrightarrow (\nu' \leq w'+\nicefrac{\delta w'}{2} )$.
This then leads to the following conditional probability:
\begin{align}
    P_\dcpprogram(\mu=\top \mid \nu=w, \nu'=w') 
    &= \frac{P_\dcpprogram(\mu=\top, \eta_1=\top, \eta_2=\top \mid \nu'=w' )}{P_\dcpprogram(\eta_1=\top, \eta_2=\top \mid \nu'=w')} \nonumber \\
    &= \frac{ \frac{P_\dcpprogram(\mu=\top, \eta_1=\top, \eta_2=\top, \eta'_1=\top, \eta'_2=\top)}{ \cancel{P_\dcpprogram(\eta'_1=\top, \eta'_2=\top)}} }{\frac{ P_\dcpprogram(\eta_1=\top, \eta_2=\top, \eta'_1=\top, \eta'_2=\top)}{\cancel{ P_\dcpprogram(\eta'_1=\top, \eta'_2=\top) }}} \nonumber \\
    &= \frac{P_\dcpprogram(\mu=\top,\eta_1=\top, \eta_2=\top, \eta'_1=\top, \eta'_2=\top)}{P_\dcpprogram(\eta_1=\top, \eta_2=\top, \eta'_1=\top, \eta'_2=\top)} 
\end{align}
Here we first applied the definition of the conditional probability for the observation of the random variable $\nu$ and then for the observation of the random variable $\nu'$. Finally, we simplified the expression.

\begin{proposition}
\label{prop:existence_cond_prob_zero_evidence}
The conditional probability as defined in Definition~\ref{def:conditional_prob_zero_event} exists.
\end{proposition}
\begin{proof}
See ~\citep[Equation 6]{nitti2016probabilistic}.
\end{proof}

In order to express zero-probability events in \dcproblogsty we add a new built-in comparison predicate to the finite set of comparison predicates $\comparisonpredicates =\{ \mathprobloginline{<}, \mathprobloginline{>},\allowbreak \mathprobloginline{=<}, \mathprobloginline{>=}, \mathprobloginline{=:=},\allowbreak \mathprobloginline{=\=} \}$ (cf. Definition~\ref{def:reserved_vocabulary}).


\begin{definition}[Delta Interval Comparison]\label{def:delta_interval}
    For a random variable \probloginline{v} and a rational number \probloginline{w}, we define \probloginline{delta_interval(v,w)} (with $\text{\predicate{delta_interval}{2}} \in \comparisonpredicates$) as follows. If \probloginline{v} has a countable sample space, then  \probloginline{delta_interval(v,w)} is equivalent to \probloginline{v=:=w}. Otherwise, \probloginline{delta_interval(v,w)} is equivalent to the conjunction of
    the two comparison atoms \probloginline{w-@$\delta$@w=<v} and \probloginline{v=<w+@$\delta$@w},
    where \probloginline{@$\delta$@w} is an infinitesimally small number.
\end{definition}

The delta interval predicate lets us express conditional probabilities with zero probability conditioning events as defined in Definition~\ref{def:conditional_prob_zero_event}.

Zero probability conditioning events are often abbreviated as $P_\dcpprogram (\mu=\top\mid \nu=w)$.
This can be confusing as it does not convey the intent of conditioning on an infinitesimally small interval. To this end, we introduce the symbol `$\doteq$' (equal sign with a dot on top). We use this symbol to explicitly point out an infinitesimally small conditioning set. For instance, we abbreviate the limit
\begin{align*}
    \lim_{\Delta w \rightarrow 0}
    \frac{P_\dcpprogram(\mu=\top, \nu\in [w-\nicefrac{\Delta w}{2}, w+\nicefrac{\Delta w}{2} ]) }{P_\dcpprogram(\nu \in [w-\nicefrac{\Delta w}{2}, w+\nicefrac{\Delta w}{2} ] )}
\end{align*}
in Definition~\ref{def:conditional_prob_zero_event} as:
\begin{align}
    P_\dcpprogram(\mu=\top \mid \nu\doteq w )
\end{align}

More concretely, if we measure the height $h$ of a person to be $180cm$ we denote this by $h\doteq 180$. This means that we measured the height  of the person to be in an infinitesimally small interval around $180cm$. Note that the $\doteq$ sign has slightly different semantics for random variables with a countable support. For discrete random variables the $\doteq$ is equivalent to the $equal$ sign.



\begin{example} \label{ex:conditional_prob_zero}
    Assume that we have a random variable $\nu$ distributed according to a normal distribution and that we have the evidence set $\evidenceset= \{ \eta=\top \} $, with $\eta\leftrightarrow (\nu\doteq 20)$. This is a valid conditional probability defined through Definition~\ref{def:conditional_prob_zero_event}.
\end{example}


\begin{example}
    Assume that we have a random variable $\nu$ distributed according to a normal distribution and that we have the conditioning set $\evidenceset= \{ \eta=\top, \eta' = \top \} $, with $\eta_1\leftrightarrow (\nu\doteq 20)$ and $\eta'\leftrightarrow (\nu\doteq 30)$. This does not encode a conditional probability as the conditioning event is not a possible event: one and the same random variable cannot be observed to have two different outcomes.
\end{example}

The notation used to condition on zero probability events (even when using `$\doteq$') hides away the limiting process that is used to define the conditional probability. This can lead to situations where seemingly equivalent conditional probabilities have diametrically opposed meanings.

\begin{example} \label{ex:conditional_prob}
    Let us consider the conditioning set $\evidenceset = \{ \eta=\top, \eta'=\top \}$, with $\eta \leftrightarrow (\nu\leq 20)$ and $\eta' \leftrightarrow (20\leq \nu)$, which we use again to condition a continuous random variable $\nu$. In contrast to Example~\ref{ex:conditional_prob_zero}, where we directly observed $\nu\doteq 20$, here, Definition~\ref{def:conditional_prob} applies, which %leads to a division by zero 
    states that the conditional probability is undefined as $P(\nu\leq 20, 20\leq \nu )=0$.
\end{example}







\subsection{Discussion on the Well-Definedness of a Query}

The probability of an unconditional query to a valid \dcproblogsty program is always well-defined, as it is simply a marginal of the distribution represented by the program.  
This stands in stark contrast to conditional probabilities: an obvious issue are  divisions by zero occurring when the conditioning event does not belong to the set of possible outcomes of the conditioned random variable. Similarly to~\citet{wu2018discrete} we will assume for the remainder of the paper that conditioning events are always possible events, \ie events that have a non-zero probability but possibly an infinitesimally small probability  of occurring. This allows us to bypass potential issues caused by zero-divisions.\footnote{In general, deciding whether a conditioning event is possible or not is undecidable. This follows from the undecidability of general logic programs under the well-founded semantics~\citep{cherchago2007decidability}. A similar discussion is also presented in the thesis of Brian Milch~\citep[Proposition 4.8]{milch2006probabilistic} for the \blogsty language, which also discusses decidable language fragments~\citep[Section 4.5]{milch2006probabilistic}.}  

Even when discarding impossible conditioning events, conditioning a probabilistic event on a zero probability (but possible) event remains inherently ambiguous~\citep{jaynes2003probability} and might lead to the Borel-Kolmogorov paradox.
Problems arise when the limiting process used to define the conditional probability with zero probability events (cf. Definition~\ref{def:conditional_prob_zero_event}) does not produce a unique limit. 
For instance, a conditional probability  $P(\mu=\top \mid  2\nu \doteq \nu' )$, where $\nu$ and $\nu'$ are two random variables, depends on the parametrization used. We refer the reader to~\citep{shan2017exact} and \citep{jacobs2021paradoxes} for a more detailed discussion on ambiguities arising with zero probability conditioning events in the context of probabilistic programming.
We will sidestep such ambiguities completely by limiting observations of zero probability events to direct comparisons between random variables and numbers. This makes also sense from an epistemological perspective: we interpret a conditioning event as the outcome of an experiment, which produces a number, for instance the reading of a tape measure.






\subsection{Conditional Probabilities by Example}



\begin{example}\label{example:problog_machine}
The following ProbLog program models the conditions under which machines work. There are two machines (Line \ref{program:problog_machines_dec}), and three (binary) random terms, which we interpret as random variables as the bodies of the probabilistic facts are empty. The random variables are: the outside temperature (Line \ref{program:problog_machines_temp}) and  whether the cooling of each machine works (Lines \ref{program:problog_machines_cool1} and \ref{program:problog_machines_cool2}). Each machine works if its cooling works or if the temperature is low (Lines \ref{program:problog_machines_work_cool} and \ref{program:problog_machines_work_temp}).


	\begin{problog*}{linenos}
machine(1). machine(2). @\label{program:problog_machines_dec}@

0.8::temperature(low). @\label{program:problog_machines_temp}@
0.99::cooling(1). @\label{program:problog_machines_cool1}@
0.95::cooling(2). @\label{program:problog_machines_cool2}@

works(N):- machine(N), cooling(N). @\label{program:problog_machines_work_cool}@
works(N):- machine(N), temperature(low). @\label{program:problog_machines_work_temp}@
	\end{problog*}
We can query this program for the probability of \dcplpinline{works(1)} given that we have as evidence that \dcplpinline{works(2)} is true:
$$
P(\text{\dcplpinline{works(1)}}{=}1\mid\text{\dcplpinline{works(2)}}{=}1)\approx 0.998
$$
\end{example}


\begin{example}\label{example:dcproblog_machine}
In the previous example there are only Boolean random variables (encoded as probabilistic facts) and the \dcproblogsty program is equivalent to an identical \problogsty program. An advantage of \dcproblogsty is that we can now use an almost identical program to model the temperature as a continuous random variable.

	\begin{problog*}{linenos}
machine(1). machine(2).

temperature ~ normal(20,5). @\label{program:dcproblog_machines_temp}@
0.99::cooling(1).
0.95::cooling(2).

works(N):- machine(N), cooling(N).
works(N):- machine(N), temperature<25.0. @\label{program:dcproblog_machines_work_temp}@
	\end{problog*}
We can again ask for the probability of \probloginline{works(1)} given that we have as evidence that \probloginline{works(2)} is true but now the program also involves a continuous random variable:
$$
P(\text{\probloginline{works(1)}}{=}\top\mid\text{\probloginline{works(2)}}{=}\top)\approx 0.998
$$
\end{example}

In the two previous examples we were interested in a conditional probability where the conditioning event has a non-zero probability of occurring. However, \dcproblogsty programs can also encode conditional probabilities where the conditioning event has a zero probability of happening, while still being possible.
\begin{example}\label{example:dcproblog:observation}
	We model the size of a ball as a mixture of  different beta distributions, depending on whether the ball is made out of wood or metal (Line~\ref{program:dcproblog_machines_observation:ad}).
	We would now like to know the probability of the ball being made out of wood given that we have a measurement of the size of the ball.
	\begin{problog*}{linenos}
3/10::material(wood);7/10::material(metal).@\label{program:dcproblog_machines_observation:ad}@

size~beta(2,3):- material(metal)@\label{example:dcproblog:observation:beta23}@.
size~beta(4,2):- material(wood).
	\end{problog*}
Assume that we measure the size of the ball and we find that it is $0.4cm$, which means that we have a measurement (or observation) infinitesimally close to $0.4$. Using the `$\doteq$' notation, we write this conditional probability as:
	\begin{align}
		P\Bigl(\text{\probloginline{material(wood)}}{=}\top \mid \left(\text{\probloginline{size@$\doteq$@4/10}}\right) {=}\top \Bigr)
	\end{align}
\end{example}



The {\em Indian GPA problem} was initially proposed by Stuart Russell as an example problem to showcase the intricacies of mixed random variables.
Below we express the Indian GPA problem in \dcproblogsty.

\begin{example} \label{ex:indian_gpa}
The Indian GPA problem models US-American and Indian students and their GPAs. Both receive scores on the continuous domain, namely from 0 to 4 (American) and from 0 to 10 (Indian), cf. Line~\ref{ex:gpa:dens_am} and \ref{ex:gpa:dens_in}. With non-zero probabilities both student groups can also obtain marks at the extremes of the respective scales (Lines~\ref{ex:gpa:max_am}, \ref{ex:gpa:min_am}, \ref{ex:gpa:max_in}, \ref{ex:gpa:min_in}). 

\begin{problog*}{linenos}
1/4::american;3/4::indian.

19/20::isdensity(a).
99/100::isdensity(i).

17/20::perfect_gpa(a).
1/10::perfect_gpa(i).

gpa(a)~uniform(0,4):- isdensity(a). @\label{ex:gpa:dens_am}@
gpa(a)~delta(4.0):- not isdensity(a), perfect_gpa(a).  @\label{ex:gpa:max_am}@
gpa(a)~delta(0.0):- not isdensity(a), not perfect_gpa(a). @\label{ex:gpa:min_am}@
     
gpa(i)~uniform(0,10):- isdensity(i). @\label{ex:gpa:dens_in}@
gpa(i)~delta(10.0):- not isdensity(i), perfect_gpa(i).  @\label{ex:gpa:max_in}@
gpa(i)~delta(0.0):- not isdensity(i), not perfect_gpa(i).  @\label{ex:gpa:min_in}@
    
gpa(student)~delta(gpa(a)):- american.
gpa(student)~delta(gpa(i)):- indian.
\end{problog*}
Note that in order to write the probability distribution of \probloginline{gpa(a)} and \probloginline{gpa(i)} we used uniform and Dirac delta distributions. This allowed us to distribute the random variables \probloginline{gpa(a)} and \probloginline{gpa(i)} according to a discrete-continuous mixture distribution.
We then observe that a student has a GPA of $4$ and we would like to know the probability of this student being American or Indian. 
\begin{align}
P\Bigl(\text{\probloginline{american}}{=}\top \mid (\text{\probloginline{gpa(student)}}\doteq4)=\top \Bigr)&=1 
\nonumber \\
P \Bigl(\text{\probloginline{indian}}{=}\top \mid (\text{\probloginline{gpa(student)}}\doteq4)=\top \Bigr)&=0 \nonumber
\end{align}
\end{example}







%
\section{Inference via Computing Expectations of Labeled Logic Formulas} \label{sec:dc2smt}

In the previous sections we have delineated the semantics of \dcproblogsty programs and described the PROB task that defines conditional probability queries on \dcproblogsty programs. The obvious next step is to actually perform the inference. We will follow an approach often found in implementations of PLP languages in the discrete domain: reducing inference in probabilistic programs to 
performing inference on labeled Boolean formulas that encode relevant parts of the logic program.
Contrary to languages in the discrete domain that follow this approach~\citep{fierens2015inference,riguzzi2011pita}, we will face the additional complication of handling random variables with infinite sample spaces. We refer the reader to~\citep[Section 5]{riguzzi2018foundations} for a broader overview of this approach.

Specifically, we are going to define a reduction from \dcproblogsty inference to the task of computing the expected label of a propositional formula. The formula is a propositional encoding of the  relevant part of the logic program (relevant with respect to a query), where atoms become propositional variables, and the labels of the basic facts in the distribution database are derived from the probabilistic part of the program.
At a high level, we extend \problogsty's inference algorithm such that Boolean comparison atoms over (potentially correlated) random variables are correctly being kept track of. The major complication, with regard to \problogsty and other systems such as PITA~\citep{riguzzi2011pita}, is the presence of context-dependent random variables, which are denoted by the same ground random term. For instance, the random term \probloginline{size} in the program in Example~\ref{example:dcproblog:observation} denotes two different random variables but is being referred to by one and the same term in the program.

Inference algorithms for PLP languages often consider only a fragment of the language for which the semantics have been defined. A common restriction for inference algorithms is to only consider range-restricted programs\footnote{We call a \dcproblogsty program range-restricted if it holds that for every statement all logic variables occurring in the head also occur as positive literals in the body. This guarantees that all terms will become ground during backward chaining. Note that range-restrictedness implies that all facts (including probabilistic and distributional ones) are ground.}. 
% , the finite support assumption is usually presumed. That is, the queries and evidence depend only on a finite subset of Boolean comparison atoms (and thus a finite set of random variables),
Furthermore, we consider, without loss of generality only AD-free programs, cf.~Definition~\ref{def:ad_free_program}, as annotated disjunctions or probabilistic facts can be eliminated up front by means of {\em local} transformations that  solely affect the annotated disjunctions (or probabilistic facts).\footnote{For non-ground ADs, we adapt Definition~\ref{def:elim-ad} to include all logical variables as arguments of the new random variable. As this introduces non-ground distributional facts, which are not range-restricted, we also move the comparison atom to the end of the rule bodies of the AD encoding to ensure those local random variables are ground when reached in backward chaining.}

% \begin{example}[Eliminating Probabilistic Facts]
% \label{ex:eliminating_pf}
% Consider the non-ground \dcproblogsty program below, which is range-restricted but not AD-free.
% %containing ground probabilistic fact, ground distributional clauses, and a non-ground logic program. 
% 	\begin{problog*}{linenos}
% 0.2::hot.
% 0.99::cool(1).

% temp(1) ~ normal(27,5):- hot.
% temp(1) ~ normal(20,5):- not hot.

% works(N):- cool(N).
% works(N):- temp(N)<25.0.
% 	\end{problog*}
% We can eliminate the probabilistic facts independently form the rest of the program as this is a local transformation, which does not affect any part of the program but a probabilistic fact itself:
% 	\begin{problog*}{linenos}
% rv_hot ~ flip(0.2).
% hot:- rv_hot=:=1.
% rv_cool(1) ~ flip(0.99).
% cool(1):- rv_cool(1)=:=1.

% temp(1) ~ normal(27,5):- hot.
% temp(1) ~ normal(20,5):- not hot.

% works(N):- cool(N).
% works(N):- temp(N)<25.0.
% 	\end{problog*}
% \end{example}




The high level steps for converting a \dcproblogsty program to a labeled propositional formula closely follow the corresponding conversion for \problogsty programs provided by \citet[Section 5]{fierens2015inference}, \ie, given a \dcproblogsty program \dcpprogram,
		evidence $\evidenceset =e$ and a set of query atoms $\queryset$, the conversion algorithm performs the following steps:
\begin{enumerate}
    \item Determine the { relevant ground program} $\dcpprogram_g$ with respect to the atoms in  $\queryset\cup\evidenceset$ and obtain the corresponing \dfplpsty program. 
    \item Convert $\dcpprogram_g$  to an { equivalent propositional formula} $\phi_g$ and $\evidenceset =e$ to a propositional conjunction $\phi_e$.
%    \item {\bf Assert the evidence} $\evidenceset=e$.
    \item Define the { labeling function} for all atoms in $\phi_g$.
\end{enumerate}
Step 1 exploits the fact that ground clauses that have no influence on the truth values of  query or evidence atoms are irrelevant for inference and can thus be omitted from the ground program. Step 2 performs the conversion from logic program semantics to propositional logic, generating a formula that encodes \emph{all} models of the relevant ground program as well as a formula that serves to assert the evidence by conjoining both formulas. 
Step 3 completes the conversion by defining the labeling function. 
In the following, we discuss the three steps in more detail and prove correctness of our approach (cf. Theorem~\ref{thm:inference-by-expectation}).  

%Below we give the outline of the conversion algorithm to map a \dcproblogsty program to a weighted propositional logic formula, this is heavily inspired by~\citet[Section 5]{fierens2015inference},


%	\begin{algo}[\dcproblogsty to Weighted Propositional Formula] \label{algo:inference_dc_problog}
%		The algorithm takes as input a \dcproblogsty program \dcpprogram,
%		evidence $\evidenceset =e$ and a set of \emph{query atoms} $\queryset$.
%		The algorithm maps the queries to $\dcpprogram$ through the following steps to a labeled propositional  formula.
%		\begin{enumerate}
%			\item Ground out $\dcpprogram$ with respect to every query $Q\in \queryset$ and obtain the ground program $\dcpprogram_g$.
 %           Note that it is unnecessary to consider the full grounding of the program, as we only need the part that is relevant to the queries  (given the evidence). That is, the part that captures the probability $P(Q=1 \mid \evidenceset=e)$ for every $Q\in \queryset$. We refer to the resulting program $\dcpprogram_g$ as the {\bf relevant ground program} with respect to $\queryset$ and $\evidenceset=e$. 
		
  %          \item Convert the ground rules in $\dcpprogram_g$ to an {\bf equivalent propositional formula} $\phi_g$.

   %         \item {\bf Assert the evidence} by defining the formula $\phi$ as the conjunction of the formula $\phi_g$ for the rules and for the evidence $\phi_e$. The formula $\phi_e$ is obtained by setting the conjunction of the evidence atoms as the query of interests and perform steps 1. and 2., ie. ground the program \dcpprogram with respect to the conjunction of evidence atoms and obtain a propositional formula $\phi_e$ from that ground program.
            
    %        \item Define a {\bf labeling function} for all atoms in $\phi$.
	%	\end{enumerate}
	%\end{algo}

%We will now describe the four steps of Algorithm~\ref{algo:inference_dc_problog}  in more detail and prove its correctness by proving that each of the four steps does not alter the probability of the queried atoms. We will limit ourselves to a single query atom for the sake of exposition. Note, however, that the discussion below also holds for multiple queries by realizing that $\dcpprogram_g= \bigcup_{Q\in \queryset} \dcpprogram_g^Q$, where $\dcpprogram_g^Q$ is the ground program with respect to a single query atom $Q$.

\subsection{The Relevant Ground Program}

The first step in the conversion of a non-ground \dcproblogsty program to a labeled  Boolean formula consists of grounding the program with respect to a query set $\queryset$ and the evidence $\evidenceset=e$.
For each ground atom in $\queryset$ and $\evidenceset$ we construct its dependency set. That is, we collect the set of ground atoms and ground rules that occur in any of the proofs of an atom in $\queryset\cup \evidenceset$. The union of all dependency sets for all the ground atoms in $\queryset\cup\evidenceset$ is the dependency set of the \dcproblogsty with respect to the sets $\queryset$ and $\evidenceset$. This dependency set, consisting of ground rules and ground atoms, is called the relevant ground program (with respect to a set of queries and evidence).

\begin{example}
	\label{ex:grounded_nicely}
	Consider the non-ground (AD-free) \dcproblogsty program below.     
	\begin{problog*}{linenos}
rv_hot ~ flip(0.2).
hot:- rv_hot=:=1.
rv_cool(1) ~ flip(0.99).
cool(1):- rv_cool(1)=:=1.

temp(1) ~ normal(27,5):- hot.
temp(1) ~ normal(20,5):- not hot.

works(N):- cool(N).
works(N):- temp(N)<25.0.
	\end{problog*}
If we ground it with respect to the query \probloginline{works(1)} and subsequently apply the rewrite rules from Section~\ref{sec:eliminate_dc} we obtain:
	\begin{problog*}{linenos}
rv_hot ~ flip(0.2). 
hot:- rv_hot=:=1.
rv_cool(1) ~ flip(0.99).
cool(1):- rv_cool(1)=:=1.

temp(hot) ~ normal(27,5).
temp(not_hot) ~ normal(20,5).

works(1):- cool(1).
works(1):- hot, temp(hot)<25.0,
works(1):- not hot, temp(not_hot)<25.0.
	\end{problog*}
\end{example}

A possible way, as hinted at in Example~\ref{ex:grounded_nicely}, of obtaining a ground \dfplpsty program from a non-ground \dcproblogsty program is to first ground out all the logical variables. Subsequently, one can apply Definition~\ref{def:elim-ad} to eliminate annotated disjunctions and probabilistic facts, and Definition~\ref{def:adfree-to-core} in order to obtain a \dfplpsty program with no distributional clauses.
A possible drawback of such a two-step approach (grounding logical variables followed by obtaining a \dcproblogsty program) is that it might introduce spurious atoms to the relevant ground program. A more elegant but also more challenging approach is to interleave the grounding of logical variables and distributional clause elimination. We leave this for future research.


\begin{restatable}[Label Equivalence]{theorem}{theorgp}
	\label{theo:rgp}
	Let $\dcpprogram$ be a \dcproblogsty program and let $\dcpprogram_g$ be the relevant ground program for $\dcpprogram$ with respect to a query $\mu$ and the evidence $\evidenceset=e$ obtained by first grounding out logical variables and subsequently applying transformation rules from Section~\ref{sec:dcproblog}. The programs $\dcpprogram$ and $\dcpprogram_g$ specify the same probability:
	\begin{align}
		P_{\dcpprogram}(\mu=\top \mid \evidenceset=e)
		=
		P_{\dcpprogram_g}(\mu=\top \mid \evidenceset=e)
	\end{align}
    \end{restatable}

    \begin{proof}
        See Appendix~\ref{app:proof:rgp}.
        \end{proof}

\subsection{The Boolean Formula for the Relevant Ground Program}
\label{sec:relprog2boolfrom}

Converting a ground logic program, \ie a set of ground rules, into an equivalent Boolean formula is a purely logical problem and well-studied in the non-probabilistic logic programming literature. We refer the reader to \citet{janhunen2004representing} for an account of the transformation to Boolean formula in the non-probabilistic setting and to \citet{mantadelis2010dedicated} and \citet{fierens2015inference} in the probabilistic setting,
including correctness proofs. We will only illustrate the most basic case with an example here.


\begin{example}[Mapping \dcproblogsty to Boolean Formula] \label{example:dc2bool}
Consider the ground program in Example~\ref{ex:grounded_nicely}. 
To highlight the move from logic programming to propositional logic,	
	%We map  the logic program to a Boolean logic formula $\support_g$, where 
	we introduce for every atom \probloginline{a} in the program a %fresh and equivalent
	corresponding propositional variable  $\phi_\text{\probloginline{a}}$.
	As the program does not contain cycles, we can use Clark's completion for the transformation, \ie, a derived atom is true if and only if the disjunction of the bodies of its defining rules is true. The propositional formula $\phi_g$ corresponding to the program is then the conjunction of the following three formulas:
	\begin{align*}
	    \phi_\mathprobloginline{works(1)} &\leftrightarrow
		\Big( \phi_{\text{\probloginline{cool(1)}}} 
		\lor \phi_\text{\probloginline{hot}} \land  \phi_\text{\probloginline{temp(hot)<25.0}}  
		\lor\neg \phi_\text{\probloginline{hot}}  \land \phi_\text{\probloginline{temp(not_hot)<25.0}}   \Big) \\
		  \phi_{\text{\probloginline{cool(1)}}} &\leftrightarrow  \phi_{\text{\probloginline{rv_cool(1)=:=1}}}\\
		\phi_\text{\probloginline{hot}} &\leftrightarrow \phi_{\text{\probloginline{rv_hot=:=1}}}
	\end{align*}
% AK: 	I'm not quite sure what was going on in the rest of the complicated formula, so I couldn't fix it to fit the fix above, but it didn't seem relevant to the story, so I dropped it (and adapted subsequent examples)
% 	\begin{align}
% 		\support_q
% 		&\leftrightarrow \phi_{\text{\probloginline{cool(1)}}} \land \phi_{\text{\probloginline{rv_cool(1)=:=1}}}  \lor \nonumber \\
% 		&\phantom{{}\leftrightarrow{}}   \phi_\text{\probloginline{hot}} \land  \phi_\text{\probloginline{temp(hot)<25.0}} \land \phi_{\text{\probloginline{hot}=:=1}} \lor \nonumber \\
% 		&\phantom{{}\leftrightarrow{}} \phi_\text{\probloginline{machine(1)}} \land \phi_\text{\probloginline{temp(not_hot)<25.0}} \land  \neg \phi_\text{\probloginline{hot=:=1}} 
% % 		&\leftrightarrow (\text{\dcplpinline{cool(1)}}=1) \lor \nonumber \\
% % 		&\phantom{{}\leftrightarrow{}} (\text{\dcplpinline{temp(hot)}}<25.0) \land (\text{\dcplpinline{hot}}=1) \lor \nonumber \\
% % 		&\phantom{{}\leftrightarrow{}} (\text{\dcplpinline{temp(not_hot)}}<25.0) \land  (\text{\dcplpinline{hot}}=0)  
% 	\end{align}
%	\begin{align}
%		&\phantom{{}\leftrightarrow{}} \support_g  \nonumber \\
%		&\leftrightarrow
%		\Big( \phi_{\text{\probloginline{cool(1)}}} 
%		\lor \phi_\text{\probloginline{hot}} \land  \phi_\text{\probloginline{temp(hot)<25.0}}  
%		\lor\neg \phi_\text{\probloginline{hot}}  \land \phi_\text{\probloginline{temp(not_hot)<25.0}}   \Big) \land  \nonumber \\
%		&\phantom{{}\leftrightarrow{}} 
%		  \phi_{\text{\probloginline{cool(1)}}} \leftrightarrow  \phi_{\text{\probloginline{rv_cool(1)=:=1}}} \land \nonumber \\
%		&\phantom{{}\leftrightarrow{}}
%		  \phi_\text{\probloginline{hot}} \leftrightarrow \phi_{\text{\probloginline{rv_hot=:=1}}} \\
		%%%%%%%%%%%%%%%%%%%%%%%%%%%%%%%%%%%%
%    	&\leftrightarrow
%    	 \left( \phi_{\text{\probloginline{cool(1)}}}  \land \phi_{\text{\probloginline{rv_cool(1)=:=1}}} \land  \phi_\text{\probloginline{hot}} \land \phi_{\text{\probloginline{rv_hot=:=1}}} \right) \lor \nonumber \\
%		&\phantom{{}\leftrightarrow{}}
 %   	\left( \phi_{\text{\probloginline{cool(1)}}}  \land \phi_{\text{\probloginline{rv_cool(1)=:=1}}} \land  \neg \phi_\text{\probloginline{hot}} \land \neg \phi_{\text{\probloginline{rv_hot=:=1}}} \right) \lor \nonumber \\
  %      &\phantom{{}\leftrightarrow{}} 
%		\left( \phi_\text{\probloginline{hot}} \land  \phi_{\text{\probloginline{rv_hot=:=1}}} \land  \phi_\text{\probloginline{temp(hot)<25.0}} \land \phi_{\text{\probloginline{cool(1)}}}  \land \phi_{\text{\probloginline{rv_cool(1)=:=1}}}  \right) \lor \nonumber \\
 %       &\phantom{{}\leftrightarrow{}} 
%		\left( \phi_\text{\probloginline{hot}} \land  \phi_{\text{\probloginline{rv_hot=:=1}}} \land  \phi_\text{\probloginline{temp(hot)<25.0}} \land \neg \phi_{\text{\probloginline{cool(1)}}}  \land \neg \phi_{\text{\probloginline{rv_cool(1)=:=1}}}  \right) \lor \nonumber \\
 %       &\phantom{{}\leftrightarrow{}} 
%		\left( \neg \phi_\text{\probloginline{hot}} \land \neg \phi_{\text{\probloginline{rv_hot=:=1}}} \land  \phi_\text{\probloginline{temp(not_hot)<25.0}} \land \phi_{\text{\probloginline{cool(1)}}}  \land
%		\phi_{\text{\probloginline{rv_cool(1)=:=1}}}  \right) \lor \nonumber \\
 %       &\phantom{{}\leftrightarrow{}} 
%		\left( \neg \phi_\text{\probloginline{hot}} \land \neg \phi_{\text{\probloginline{rv_hot=:=1}}} \land  \phi_\text{\probloginline{temp(not_hot)<25.0}} \land \neg \phi_{\text{\probloginline{cool(1)}}}  \land
%		\neg \phi_{\text{\probloginline{rv_cool(1)=:=1}}}  \right) \label{eq:example_dc2bool_last}
    % 	
%	\end{align}

\end{example}

















Note that the formula obtained by converting the relevant ground program still admits \emph{any} model of that program, including ones that are inconsistent with the evidence. In order to use that formula to compute conditional probabilities, we still need to assert  the evidence into the formula by conjoining the corresponding propositional literals. 
The following theorem then directly applies to our case as well.
\begin{theorem}[Model Equivalence~\citep{fierens2015inference} (Theorem 2, part 1)]
\label{theo:model_equivalence}
Let $\dcpprogram_g$ be the relevant ground program for a \dcproblogsty program $\dcpprogram$ with respect to  query set \queryset\ and  evidence $\evidenceset=e$. Let $MOD_{\evidenceset=e}(\dcpprogram_g)$ be those models in $MOD(\dcpprogram_g)$ that are consistent with the evidence.
Let $\phi_g$ denote the  propositional formula derived from $\dcpprogram_g$, and set $\phi \leftrightarrow \phi_g \land \phi_e$, where  $\phi_e$ is the conjunction of literals that corresponds to the observed truth values of the atoms in $\evidenceset$. We then have {\bf model equivalence}, \ie, 
\begin{align}
    MOD_{\evidenceset=e}(\dcpprogram_g)
    =
    ENUM(\phi)
\end{align}
where $ENUM(\phi)$ denotes the set of models of $\phi$.
\end{theorem}


\subsection{Obtaining a Labeled Boolean Formula}

In contrast to a \problogsty program, a \dcproblogsty program does not explicitly provide independent probability labels for the basic facts in the distribution semantics, and we thus need to suitably adapt the last step of the conversion. 
We will first define the labeling function on propositional atoms and will then show that the probability of the label of a propositional formula is the same as the probability of the relevant ground program under the measure semantics from Section~\ref{sec:semantics}. We call this {\em label equivalence} and prove it in Theorem~\ref{theo:label_equivalence}. 


\begin{definition}[Label of Literal] \label{def:labeling_function}
The label $\alpha(\phi_{\rho})$ of a propositional atom $\phi_{\rho}$ (or its negation) is given by:
\begin{align}
    \alpha( \phi_{\rho })=
    \begin{cases} 
    \ive{c(vars(\rho)},  &  \text{if $\rho$ is a  comparison atom} \\
    1, & \text{otherwise}
% 
    \end{cases}
\end{align}
and for the negated atom:
\begin{align}
    \alpha(\neg \phi_{\rho})=
    \begin{cases} 
    \ive{\neg c(vars(\rho))},  &  \text{if $\rho$ is a  comparison atom} \\
    1, & \text{otherwise}
% 
    \end{cases}
\end{align}
We use Iverson brackets $\ive{\cdot}$~\citep{iverson1962programming} to denote an indicator function. Furthermore, $vars(\rho)$ denotes the random variables that are present in the arguments of the atom $\rho$ and $c(\cdot)$ encodes the constraint given by $\rho$.
\end{definition}

\begin{example}[Labeling function]
Continuing Example~\ref{example:dc2bool}, we obtain, inter alia, the following labels:
\begin{align*}
    & \alpha(\phi_\text{\probloginline{rv_hot=:=1}} ) = \ive{rv\_hot=1}  \\
    & \alpha(\neg \phi_\text{\probloginline{rv_hot=:=1}} ) = \ive{\neg(rv\_hot=1)} = \ive{rv\_hot=0} \\
    & \alpha(\phi_\text{\probloginline{hot}} ) = 1\\  
    & \alpha(\neg \phi_\text{\probloginline{hot}} ) = 1 
\end{align*}
\end{example}

\begin{definition}[Label of Boolean Formula]
\label{def:label_bool_formula}
Let $\phi$ be a Boolean formula and $\alpha(\cdot)$ the labeling function for the variables in $\phi$ as given by Definition~\ref{def:labeling_function}. We define the label of $\phi$ as
\begin{align*}
    \alpha(\phi) &= \sum_{\varphi\in ENUM(\phi)}\prod_{\ell\in\varphi}\alpha(\ell) 
\end{align*}
i.e. as the sum of the labels of all its models, which are in turn defined as the product of the labels of their literals.
\end{definition}

\begin{example}[Labeled Boolean Formula]
The label of the conjunction 
$$\neg \phi_\text{\probloginline{hot}} \land \neg \phi_{\text{\probloginline{rv_hot=:=1}}} \land  \phi_\text{\probloginline{temp(not_hot)<25.0}} \land \neg \phi_{\text{\probloginline{cool(1)}}}  \land
		\neg \phi_{\text{\probloginline{rv_cool(1)=:=1}}} \land \phi_{\text{\probloginline{works(1)}}}$$
which describes one model of the example formula, is computed as follows:
\begin{align*}
% 
    &\phantom{{}={}}
    \alpha( \neg \phi_\text{\probloginline{hot}}
    \land
    \neg \phi_{\text{\probloginline{rv_hot=:=1}}}
    \land
    \phi_\text{\probloginline{temp(not_hot)<25.0}}\\
    &\phantom{{}={}}\land
    \neg \phi_{\text{\probloginline{cool(1)}}}
    \land
	\neg \phi_{\text{\probloginline{rv_cool(1)=:=1}}}
	\land \phi_{\text{\probloginline{works(1)}}} )  \nonumber \\
	&=
    \alpha( \neg \phi_\text{\probloginline{hot}})
    \times \alpha( \neg \phi_{\text{\probloginline{rv_hot=:=1}}})
    \times  \alpha(\phi_\text{\probloginline{temp(not_hot)<25.0}}) \nonumber \\
    &\phantom{{}={}}
    \times \alpha(\neg \phi_{\text{\probloginline{cool(1)}}})
    \times \alpha(\neg \phi_{\text{\probloginline{rv_cool(1)=:=1}}} \times \alpha(\phi_{\text{\probloginline{works(1)}}}) )  \nonumber \\
    &= 1 \times \ive{rv\_hot=0} \times \ive{temp(not\_hot)<25} \times 1 \times \ive{rv\_cool(1)=0}\times 1\nonumber \\
    &= \ive{rv\_hot=0} \times \ive{temp(not\_hot)<25} \times \ive{rv\_cool(1)=0} 
\end{align*}

\end{example}

	


\begin{restatable}[Label Equivalence]{theorem}{Labelequivalence}
\label{theo:label_equivalence}

Let $\dcpprogram_g$ be the relevant ground program for a \dcproblogsty program $\dcpprogram$ with respect to a query $\mu$ and the evidence $\evidenceset=e$. Let $\phi_g$ denote the  propositional formula derived from $\dcpprogram_g$ and let $\alpha$ be the labeling function as defined in Definition~\ref{def:labeling_function}. We then have {\bf label equivalence}, \ie
\begin{align}
    \forall \varphi \in ENUM(\phi_g):  \E_{\randomvariableset \sim  \dcpprogram_g} [\alpha( \varphi )] = P_{\dcpprogram_g}(\varphi)
\end{align}
In other words, for all models $\varphi$ of $\phi_g$, the expected value ($\E_\cdot [\cdot]$) of the label of $\varphi$ is equal to the probability of $\varphi$ according to the probability measure of relevant ground program $\dcpprogram_g$.
\end{restatable}

\begin{proof}
See Appendix~\ref{app:proof:label_equivalence}.
\end{proof}

Theorem~\ref{theo:label_equivalence} states that we can reduce inference in hybrid probabilistic logic programs to computing the expected value of  labeled Boolean formulas, as summarized in the following theorem.
\begin{theorem}
\label{thm:inference-by-expectation}
Given a \dcproblogsty program \dcpprogram, a set \queryset\ of queries, and evidence $\evidenceset = e$, for every $\mu\in\queryset$, we obtain the conditional probability of $\mu = q$ ($q\in \{\bot,\top \}$) given $\evidenceset = e$ as
\begin{align*}
    P(\mu=q\mid\evidenceset = e) = \frac{\E_{vars(\phi) \sim  \dcpprogram_g} [\alpha( \phi \wedge \phi_q)] }{\E_{vars(\phi) \sim  \dcpprogram_g} [\alpha( \phi )] }
\end{align*}
where $\phi$ is the formula encoding the relevant ground program $\dcpprogram_g$ with  the evidence asserted  (cf.~Theorem~\ref{theo:model_equivalence}), and $\phi_q$ the propositional atom for $\mu$.
\end{theorem}
\begin{proof}
This directly follows from model and label equivalence together with the definition of conditional probabilities. 
\end{proof}

We have shown that the probability of a query to a \dcproblogsty program can be expressed as the expected label of a propositional logic formula.

































%
\section{Computing Expected Labels via Algebraic Model Counting}
\label{sec:alw}


In this section we will adapt the approach taken by~\citet{zuidbergdosmartires2019exact}, dubbed {\em Sampo} to compute the expected value of labeled propositional Boolean formulas.
The method approximates intractable integrals that appear when computing expected labels using Monte Carlo estimation. The main difference between Sampo and our approach, which we dub {\em infinitesimal algebraic likelihood weighting} (IALW) is that IALW can also handle infinitesimally small intervals, which arise when conditioning on zero probability events. 






\subsection{Monte Carlo Estimate of Conditional Query}

In Definition~\ref{def:conditional_prob} we defined the conditional probability as:
\begin{align}
    P_\dcpprogram(\mu=\top\mid\evidenceset=e)= \frac{P_{\dcpprogram}(\mu=\top, \evidenceset=e)}{P_\dcpprogram(\evidenceset=e)} 
\end{align}
and we also saw in Definition~\ref{def:conditional_prob_zero_event} that using infinitesimal intervals allows us to consider zero probability events, as well. Computing the probabilities in the numerator and denominator in the equation above is, in general, computationally hard. We resolve this using a Monte Carlo approximation. 

\begin{restatable}[Monte Carlo Approximation of a Conditional Query]{proposition}{mcapproxconditional}
\label{prop:mcapproxconditional}
Let the set 
\begin{align}
    \mathcal{S} = \left\{ \left(s_1^{(1)}, \dots, s_M^{(1)} \right), \dots , \left(s_1^{(\lvert \mathcal{S} \rvert)}, \dots, s_M^{(\lvert \mathcal{S} \rvert)} \right)  \right\} \label{eq:rejection_samples}
\end{align}
denote $\lvert \mathcal{S} \rvert$ i.i.d. samples for each random variable in $ \dcpprogram_g$.
A conditional probability query to a  \dcproblogsty program \dcpprogram can be approximated as: 
\begin{align}
P_\dcpprogram(\mu = q \mid \evidenceset = e) 
&\approx  \frac{ \sum_{i=1}^{\lvert \mathcal{S} \rvert}  \sum_{\varphi \in ENUM(\phi \land \phi_q) } \alpha^{(i)}(\varphi) } { \sum_{i=1}^{\lvert \mathcal{S} \rvert} \sum_{\varphi \in ENUM(\phi) } \alpha^{(i)}(\varphi) }, & \quad |\mathcal{S}|<\infty
\end{align}
The index $(i)$ on $\alpha^{(i)}(\varphi)$ indicates that the label of $\varphi$ is evaluated at the $i$-th ordered set of samples $ \left(s_1^{(i)}, \dots, s_M^{(i)} \right)$. 
\end{restatable}

\begin{proof}
    See Appendix~\ref{app:proof:mcapproxconditional}.
\end{proof}



In the limit $\lvert \mathcal{S} \rvert\rightarrow \infty$ this sampling approximation scheme is perfectly valid. However, in practice, with only limited resources available, such a rejection sampling strategy will perform poorly (in the best case) or even give completely erroneous results. After all, the probability of sampling a value from the prior distribution that falls exactly into an infinitesimally small interval given in the evidence tends to zero.
To make the computation of a conditional probability, using Monte Carlo estimates, feasible, we are going to introduce {\em infinitesimal algebraic likelihood weighting}. But first, we will need to introduce the concept of infinitesimal numbers.


\subsection{Infinitesimal Numbers}
Remember that infinitesimal intervals arise in zero probability conditioning events and describe an infinitesimally small interval around a specific observed value, \eg $\nu \in [w-\nicefrac{\Delta w}{2}, w+\nicefrac{\Delta w}{2} ]$ for a continuous random variable $\nu$ that was observed to take the value $w$ (cf. Definition~\ref{def:conditional_prob_zero_event}).
We will describe these infinitesimally small intervals using so-called {\em infinitesimal numbers}, which were first introduced by~\citet{nitti2016probabilistic} and further formalized in~\citet{wu2018discrete}, \citep{zuidberg2020atoms} and~\citep{jacobs2021paradoxes}. The latter work also coined the term {\em `infinitesimal number'} \fixed{and we refer the reader specifically to \citet[Section 5.2]{jacobs2021paradoxes} for an intuitive exposition of infinitesimal numbers.}

\begin{definition}[Infinitesimal Numbers]
    \label{def:inf_number}
An infinitesimal number is a pair \fixed{$(r, n) \in \mathbb{R} \times (\mathbb{N} \cup +\infty )$}, also written as $r\epsilon^n$, and which corresponds to a real number when $n=0$. We denote the set of all infinitesimal numbers by $\mathbb{I}$.
\end{definition}


\begin{definition}[Operations in $\mathbb{I}$]
\label{def:inf_number_ops}
Let $(r, n)$ and $(t,m)$ be two numbers in $\mathbb{I}$. We define the addition and multiplication as binary operators:
\begin{align}
	(r,n) \oplus
	(t,m)
	&\coloneqq
	\begin{cases}
		(r+t,n) &\quad  \text{if $n=m$} \\
		(r,n) &\quad  \text{if $n<m$} \\
		(t,m) &\quad  \text{if $n>m$}
	\end{cases} 
    \label{eq:infininumber_plus}
    \\
	(r,n) \otimes
	(t,m)
	&\coloneqq (r \times t , n+m)  &
    \label{eq:infininumber_times}
\end{align}
The operations $+$ and $\times$ on the right hand side denote the usual addition and multiplication operations for real and integer numbers.
\end{definition}

\begin{definition}[Neutral Elements]
\label{def:inf_number_neutral_elem}
The neutral elements of the addition and multiplications in $\mathbb{I}$ are, respectively, defined as:
\begin{align}
	e^\oplus  \coloneqq (0,+\infty)  &&
	e^\otimes  \coloneqq (1,0)
\end{align}
\end{definition}


Probabilistic inference and generalization thereof can often be cast as performing computations using commutative semirings~\citep{kimmig2017algebraic}. We will follow a similar strategy.

\begin{definition}\label{def:comm_semiring} 
	A {\bf  commutative semiring} is an algebraic structure $(\mathcal{A},\oplus,\otimes,\allowbreak e^{\oplus},e^\otimes)$ equipping a set of elements $\mathcal{A}$ with addition and multiplication such that
	\begin{enumerate}
		\item addition $\oplus$ and multiplication $\otimes$ are binary operations $\mathcal{A}\times \mathcal{A}\rightarrow \mathcal{A}$
		\item addition $\oplus$ and multiplication $\otimes$ are  associative and commutative binary operations over the set $\mathcal{A}$
		\item $\otimes$ distributes over $\oplus$
		\item  $e^\oplus \in \mathcal{A}$ is the neutral element of $\oplus$
		\item  $e^\otimes \in \mathcal{A}$ is the neutral element of $\otimes$
		\item $e^\oplus \in \mathcal{A}$ is an annihilator for $\otimes$
	\end{enumerate}
\end{definition}


\begin{lemma}
The structure $(\mathbb{I}, \oplus, \otimes, e^\oplus , e^\otimes )$ is a commutative semiring.
\end{lemma}
\begin{proof}
This follows trivially from the operations defined in Definition~\ref{def:inf_number_ops} and the neutral elements in Definition~\ref{def:inf_number_neutral_elem}.
\end{proof}
We will also need to perform subtractions and divisions in $\mathbb{I}$,

% \st{
% \fixed{
% \begin{definition}[Inverse Elements]
% Let $(r, n)$ be a number in $\mathbb{I}$. We define its inverse with respect to the addition $-(r,n)$, also called negation, as:
% \begin{align}
%     -(r,n) \coloneqq (-r, n)
% \end{align}
% Moreover, we define its inverse with respect to the multiplication $(r,n)^{-1}$, also called the reciprocal, as:
% \begin{align}
%     (r,n)^{-1}\coloneqq
%     \begin{cases}
%         (r^{-1}, -n) &\quad \text{if $r\neq 0$} \\
%         \text{undefined} &\quad \text{if $r=0$}
%     \end{cases}
% \end{align}
% \end{definition}
%     }
% }

\begin{definition}[Subtraction and Division in $\mathbb{I}$]
    \label{def:subdiv}
    Let $(r, n)$ and $(s,m)$ be two numbers in $\mathbb{I}$. We define the subtraction and division as:
    \begin{align}
            (r,n) \ominus (t,m) &\coloneqq  (r,n) \oplus (-t,m) \\
            (r,n) \oslash (t,m) &\coloneqq 
            \begin{cases}
                \text{undefined} &\quad \text{if $|n|=|m|=\infty$ and $sign(n) \neq sign(m)$} \\
                (\nicefrac{r}{t}, n-m)  &\quad \text{if $t\neq 0$} \\
                \text{undefined} &\quad \text{if $t=0$}
            \end{cases}
    \end{align}

\end{definition}

We would like to note that similar algebraic structures have been used for counting optimal variable assignments in graphical models~\citep{marinescu2019counting} and probabilistic inference in generating circuits~\citep{harviainen2023inference}.


\subsection{Infinitesimal Algebraic Likelihood Weighting}
The idea behind IALW is that we do not sample random variables that fall within an infinitesimal small interval, encoded as a delta interval (cf. Definition~\ref{def:delta_interval}), but that we force, without sampling, the random variable to lie inside the infinitesimal interval. 
To this end, assume again that we have $\lvert \mathcal{S} \rvert$ i.i.d. samples for each random variable. That means that we have again a set of ordered sets of samples:
\begin{align}
    \label{eq:ancestral_samples}
    \mathcal{S} = \left\{ \left(s_1^{(1)}, \dots, s_M^{(1)} \right), \dots , \left(s_1^{(\lvert \mathcal{S} \rvert)}, \dots, s_M^{(\lvert \mathcal{S} \rvert)} \right)  \right\}
\end{align}

This time the samples are drawn with the infinitesimal delta intervals taken into account. For example, assume we have a random variable $\nu_1$ distributed according to a normal distribution $\mathcal{N}(5,2)$ and we have an atom \probloginline{delta_interval(@$\nu_1$@,4)} in the propositional formula $\phi$. Each sampled value of $s_1^{(i)}$ will then equal $4$ ( $1\leq i\leq \lvert \mathcal{S} \rvert$). Furthermore, when sampling, we sample the parents of a random variable prior to sampling the random variable itself. For instance, take the random variable $\nu_2\sim \mathcal{N}(\nu_3=w,2)$, where $\nu_3$ is itself a random variable. We first sample $\nu_3$ and once we have a value for $\nu_3$ we plug that into the distribution for $\nu_2$, which we sample subsequently. In other words, we sample according to the ancestor relationship between the random variables.
We call the ordered set of samples $\varset{s}^{(i)} \in \mathcal{S}$ an {\em ancestral sample}.




\begin{definition}[IALW Label] \label{def:sample_labeling_function}
    \fixed{
Let $\delta_k$ denote the probability distribution of a random variable $\nu_k$.
Given an ancestral sample $\varset{s}^{(i)}= (s_1^{(i)}, \dots,  s_M^{(i)} ) $ for the random variables $\randomvariableset = (\nu_1,\dots, \nu_M)$, we denote by $\delta_k( \varset{s}^{(i)} )$ the evaluation of the density $\delta_k$ at $\varset{s}^{(i)}$, where $i$ specifies the $i$-th sample.
}
The IALW label of a positive literal $\ell$ is an infinitesimal number given by:
\begin{align}
    &\alpha_{IALW}^{(i)}( \ell) \nonumber \\
    &=\begin{cases}
    (\delta_k(\varset{s}^{(i)}) , 1),  &  \text{if $\ell$ is a \probloginline{delta_interval} whose first argument } \\
    & \text{is a continuous random variable} \\
    % (p_k(\varset{s}^{(i)}) , 0),  &  \text{if $\ell$ is a \probloginline{delta_interval} having as first argument the random} \\
                        % & \text{variable $\nu_k$, which is countable} \\
    ( \ive{ c_\ell(\varset{s}^{(i)}) }, 0), & \text{if $\ell$ is any comparison atom}
    \\
    (1, 0), & \text{otherwise} 
    \end{cases}
    \nonumber
\end{align}
The expression $\ive{ c_\ell(\varset{s}^{(i)}) }$ denotes the indicator function on the constraint that corresponds to the literal $\ell$ and which is evaluated using the samples $\varset{s}^{(i)}$. 

For the negated literals we have the following labeling function:
\begin{align}
    &\alpha_{IALW}^{(i)}( \neg \ell) \nonumber \\
    &=\begin{cases}
    (1 , 0),  &  \text{if $\ell$ is a \probloginline{delta_interval} whose first argument } \\
                        & \text{is a continuous random variable} \\
    % (1{-}w_k(\varset{s}^{(i)}) , 0),  &  \text{if $\ell$ is a \probloginline{delta_interval} having as first argument the random} \\
                        % & \text{variable $X_k$, which is countable} \\
    (1{-}\ive{ c_\ell(\varset{s}^{(i)}) }, 0), & \text{if $\ell$ is any other comparison atom}
    \\
    (1, 0), & \text{otherwise} 
    \end{cases}
    \nonumber
\end{align}
\end{definition}

Intuitively speaking and in the context of probabilistic inference,
the first part of an infinitesimal number accumulates (unnormalized) likelihood weights, while the second part counts the number of times we encounter a \probloginline{delta_interval} atom. This counting happens with $\oplus$ operation of the infinitesimal numbers (Equation~\ref{eq:infininumber_plus}). The $\oplus$ operation tells us that for two infinitesimal numbers $(r,n)$ and $(t,m)$ with $n<m$, the event corresponding to the first of the two infinitesimal numbers is infinitely more probable to happen and that we drop the likelihood weight of the second infinitesimal number (Equation~\ref{eq:infininumber_plus}). 
In other words, an event with fewer \probloginline{delta_interval}-atoms is infinitely more probable than an event with more such intervals.



\begin{example}[IALW Label of \probloginline{delta_interval} with Continuous Random Variable]
Let us consider a random variable \probloginline{x}, which is normally distributed: $p(\mathprobloginline{x}|\mu, \sigma)=\nicefrac{1}{(\sigma \sqrt{2 \pi})} \exp \left( - \nicefrac{(\mathprobloginline{x}-\mu)^2}{2 \sigma^2 } \right) $), 
where $\mu$ and $\sigma>0$ are real valued parameters that we can choose freely.
The atom \probloginline{delta_interval(x,3)} gets the label
$$
\left( \frac{1}{(\sigma \sqrt{2 \pi})} \exp \left( - \nicefrac{(\mathprobloginline{3}-\mu)^2}{2 \sigma^2 } \right)  , 1\right)
$$
The first element of the infinitesimal number is the probability distribution evaluated at the observation, in this case \probloginline{3}. As this is a zero probability event, the label also picks up a non-zero second element.

The label of $\neg \mathprobloginline{delta_interval(x,3)}$ is $(1,0)$. The intuition here being that the complement of an event with zero probability of happening will happen with probability $1$. As the complement event is not a zero probability event the second element of the label is $0$ instead of $1$. 
\end{example}

\begin{example}[IALW Label of \probloginline{delta_interval} with Discrete Random Variable]
    Let us consider a discrete random variable \probloginline{k}, which is Poisson distributed: 
    $$
    p(\mathprobloginline{k}|\lambda)=\nicefrac{\lambda^\mathprobloginline{k} e^{-\lambda}}{\mathprobloginline{k}!}
    $$
    where $\lambda>0$ is a real-valued parameter that we can freely choose.

    As a \probloginline{delta_interval} with a discrete random variable is equivalent to a \probloginline{=:=} comparison (\cf Definition~\ref{def:delta_interval}), we get for the label of the atom \probloginline{delta_interval(k,3)}:
    $(\ive{s_x^{(i)}=3}, 0)$, where $s_\mathprobloginline{k}^{(i)}$ is the $i$-th sample for \probloginline{k}.  
    % The atom \probloginline{delta_interval(x,3)} gets the label $( \nicefrac{\lambda^3 e^{-\lambda}}{3!}  , 0)$. The first element of the infinitesimal number is the probability of \probloginline{x} taking the the value 3. Because this is a non-zero probability event, the label does not pick up a non-zero second element. The label of $\neg \text{\probloginline{delta_interval(x,3)}}$ is now simply $(1-  \nicefrac{\lambda^3 e^{-\lambda}}{3!}  , 0)$.
\end{example}




% \begin{align}
%     P\Bigl(\text{\probloginline{american}}{=}\top \mid (\text{\probloginline{gpa(student)}}\doteq4)=\top \Bigr)&=1 
%     \nonumber \\
%     P \Bigl(\text{\probloginline{indian}}{=}\top \mid (\text{\probloginline{gpa(student)}}\doteq4)=\top \Bigr)&=0 \nonumber
%     \end{align}



\begin{definition}[Infinitesimal Algebraic Likelihood Weighting]
\label{def:alw}
Let $\mathcal{S}$ be a set of ancestral samples and let $ DI(\varphi)$ denote the subset of literals in $\varphi$ that are delta intervals. We then define IALW as expressing the expected value of the label of a propositional formula (given a set of ancestral samples) in terms of a fraction of two infinitesimal numbers:
\begin{align}
    \left( \E \left[ \sum_{\varphi\in ENUM(\phi)} \prod_{\ell \in \varphi}  \alpha \left(\ell \right) \bigg| \mathcal{S} \right] ,0 \right)
    \approx
    \frac
    {\displaystyle \bigoplus_{i=1}^{\lvert \mathcal{S} \rvert}  \bigoplus_{\varphi\in ENUM(\phi)} \bigotimes_{\ell \in \varphi}  \alpha_{IALW}^{(i)} \left(\ell \right)}
    {\displaystyle \bigoplus_{i=1}^{\lvert \mathcal{S} \rvert} \bigoplus_{\varphi\in ENUM(\phi)} \bigotimes_{\ell \in  DI(\varphi)}  \alpha_{IALW}^{(i)} \left(\ell \right)  } \label{eq:ALW}
\end{align}
The left hand side expresses the expected value as an infinitesimal number.
\end{definition}

% \begin{proposition}[Consistency of ALW]
\begin{restatable}[Consistency of IALW]{proposition}{alwconsistency}
\label{prop:alw_consistency}
Infinitesimal algebraic likelihood weighting is consistent, that is, the approximate equality in Equation~\ref{eq:ALW} is almost surely an equality for $\lvert \mathcal{S} \rvert \rightarrow \infty$.
% \begin{align}
%     \left( \E \left[ \sum_{\omega\in ENUM(\phi)} \prod_{\ell \in \omega}  \alpha \left(\ell \right) \right] ,0 \right)
%     =
%     \frac{\displaystyle \sum_{i=1}^N  \sum_{\omega\in ENUM(\phi)} \prod_{\ell \in \omega}  \alpha^{(i)} \left(\ell \right)}
%     {\displaystyle \sum_{i=1}^N  \sum_{\omega\in ENUM(\phi)} \prod_{\ell \in  DI(\omega)}  \alpha^{(i)} \left(\ell \right)  } 
% \end{align}
% holds almost surely for $N\rightarrow \infty$.
\end{restatable}
\begin{proof}
    See Appendix~\ref{app:proof:alw_consistency}.
\end{proof}



Likelihood weighting, the core idea behind IALW, is a well known technique for inference in Bayesian networks~\citep{fung1990weighing} and probabilistic programming~\citep{milch2005approximate,nitti2016probabilistic}, and falls within the broader class of self-normalized importance sampling~\citep{kahn1950random,kloek1978bayesian,casella1998post}.
Just like IALW, the inference approaches proposed by \citet{nitti2016probabilistic}, \citet{wu2018discrete}, and \citet{jacobs2021paradoxes}  generalize the idea of likelihood weighting to the setting with infinitesimally small intervals. What sets IALW apart from these methods is its semiring formulation. The semiring formulation will allow us to seamlessly combine IALW with knowledge compilation~\citep{darwiche2002knowledge}, a technique underlying state-of-the art probabilistic inference algorithms in the discrete setting. We examine this next.









Having proven the consistency of IALW, we can now express the probability of a conditional query to a \dcproblogsty program in terms of semiring operations for infinitesimal numbers $\mathbb{I}$.


\begin{restatable}{proposition}{alwapproximation}
\label{prop:alwapproximation}
% \begin{proposition}
A conditional probability query to a  \dcproblogsty program \dcpprogram can be approximated as: 
    \begin{align}
    P_\dcpprogram(\mu=q|\evidenceset=e) 
    \approx    
    \frac
    { \bigoplus_{i=1}^{\lvert \mathcal{S} \rvert}  \bigoplus_{\varphi\in ENUM(\phi \land \phi_{q})} \bigotimes_{\ell \in \varphi}  \alpha_{IALW}^{(i)} \left(\ell \right)}
    { \bigoplus_{i=1}^{\lvert \mathcal{S} \rvert}  \bigoplus_{\varphi\in ENUM(\phi)} \bigotimes_{\ell \in \varphi}  \alpha_{IALW}^{(i)} \left(\ell \right)}  
    \label{eq:prop:alwapproximation}
    \end{align}
    
% \end{proposition}
\end{restatable}

\begin{proof}
    See Appendix~\ref{app:proof:alwapproximation}.
\end{proof}



% In Example~\ref{ex:indian_gpa} we showed how to write down the Indian GPA problem in \dcproblogsty and also agave the probabilities for the conditional queries. After having introduced infinitesimal numbers we can now also explicitly perform the computation of the probabilities. 

% \begin{example}[Computing the Indian GPA Problem]
%     \label{example:compute_indian}
%     Let us first write the relevant ground program for the query
%     \begin{align*}       
%     P \Bigl(\mathprobloginline{american}{=}\top \mid (\text{\probloginline{gpa(student)}}\doteq4)=\top \Bigr)
%     \end{align*}
%     where we also transform distributional clauses into distributional facts and introduce an atom \probloginline{ev} for the evidence.
    

%     \begin{problog*}{linenos}
% rva ~ flip(1/4).
% american:- rva=:=1.
% rvisd ~ flip(19/20).
% isdensity(a):-  rvisd=:=1.
% rvp ~ flip(17/29).
% perfect_gpa(a):- rvp=:=1.

% gpa1~uniform(0,4).
% gpa2~delta(4.0).  
% gpa3~delta(0.0)

% ev:- american, delta_interval(gp1,4), isdensity(a).
% ev:- american, delta_interval(gp2,4), not isdensity(a),
%     perfect_gpa(a). 
% ev:- american, delta_interval(gp3,4), not isdensity(a),
%     not perfect_gpa(a).
% \end{problog*}
% The query for the transformed program is now:
% \begin{align*}
% P\Bigl(\text{\probloginline{american}}{=}\top \mid \mathprobloginline{ev} =\top\Bigr)
%     =
%     \frac{
%         P\Bigl(\mathprobloginline{american}{=}\top, \mathprobloginline{ev}=\top \Bigr)
%     }
%     {
%         P \Bigl(\mathprobloginline{ev}=\top \Bigr)
%     }
% \end{align*}




% \end{example}





\subsection{Infinitesimal Algebraic Likelihood Weighting via Knowledge Compilation}
\label{sec:ALWviaKC}


Inspecting Equation~\ref{eq:prop:alwapproximation} we see that we have to evaluate expressions of the following form in order to compute the probability of a conditional query to a \dcproblogsty program.
\begin{align}
    \bigoplus_{i=1}^{\lvert \mathcal{S} \rvert} \underbrace{\bigoplus_{\omega\in ENUM(\varphi)} \bigotimes_{\ell \in \varphi}  \alpha_{IALW}^{(i)} \left(\ell \right)}_{= \text{algebraic model count}} \label{eq:alw_show}
\end{align}
In other words, we need to compute $\lvert \mathcal{S} \rvert$ times a sum over products -- each time with a different ancestral sample. Such a sum over products is also called the algebraic model count of  a formula $\phi$~\citep{kimmig2017algebraic}. 
Subsequently, we then add up the $\lvert \mathcal{S} \rvert$ results from the different algebraic model counts giving us the final answer.

Unfortunately, computing the algebraic model count is in general a computationally hard problem~\citep{kimmig2017algebraic} -- \#P-hard to be precise~\citep{valiant1979complexity}.
A popular technique to mitigate this hardness is to use a technique called knowledge compilation~\citep{darwiche2002knowledge}, which splits up the computation into a hard step and a subsequent easy step. The idea is to take the propositional Boolean formula underlying 
an algebraic model counting problem (cf. $\varphi$ in Equation~\ref{eq:alw_show}) and compile it into a logically equivalent formula that allows for the tractable computation of algebraic model counts. The compilation constitutes the computationally hard part (\#P-hard). Afterwards, the algebraic model count is performed on the compiled structure, also called {\em algebraic circuit}~\citep{zuidbergdosmartires2019transforming}. Intuitively speaking, knowledge compilation takes the sum of products and maps it to recursively nested sums and products. Effectively, finding a dynamic programming scheme~\citep{bellman1957dynamic} to compute the initial sum of products.

Different circuit classes have been identified as valid knowledge compilation targets~\citep{darwiche2002knowledge} -- all satisfying different properties.
Computing the algebraic model count on an algebraic circuit belonging to a specific target class is only correct if the properties of the circuit class match the properties of the deployed semiring.
The following three lemmas will help us determining which class of circuits we need to knowledge-compile our propositional formula $\phi$ into.

\begin{lemma}
\label{lem:non_idem}
The operator $\oplus$ (c. Definition~\ref{def:inf_number_ops}) is not idempotent. That is, it does not hold for every $a \in \mathbb{I}$ that $a\oplus a =a$. 
\end{lemma}
\begin{lemma} The pair  $(\oplus, \alpha_{\ialw})$ is not neutral. That is, it does not hold that $\alpha_{\ialw}(\ell)\oplus \alpha_{\ialw}(\neg \ell) = e^{\otimes}$ for arbitrary $\ell$.
\end{lemma}
\begin{lemma} 
\label{lem:non_cons}
The pair  $(\otimes, \alpha_{\ialw})$ is not consistency-preserving. That is, it does not hold that $\alpha_{\ialw}(\ell)\otimes \alpha_{\ialw}(\neg \ell) = e^{\oplus}$ for arbitrary $\ell$.
\end{lemma}



From~\citep[Theorem 2 and Theorem 7]{kimmig2017algebraic} and the three lemmas above, we can conclude that we need to compile our propositional logic formulas into so-called smooth, deterministic and decomposable negation normal form (sd-DNNF) formulas~\citep{darwiche2001tractable}.\footnote{Note that we only require smoothness over derived atoms (otherwise case in Definition~\ref{def:sample_labeling_function}), as for the other cases the neutral sum property holds. Certain encodings of logic programs eliminate derived atoms. For such encodings the smoothness property can be dropped~\citep{vlasselaer2014compiling}. A more detailed discussion on the smoothness requirement of circuits in a PLP context can be found in \citep[Appendix C]{fierens2015inference}.}




% \begin{proposition}[ALW on d-DNNF]
\begin{restatable}[ALW on d-DNNF]{proposition}{alwonddnnf}
\label{prop:alwonddnnf}
    We are given the propositional formulas $\phi$ and $\phi_q$ and a set $\mathcal{S}$ of ancestral samples, we can use Algorithm~\ref{alg:prob_via_alw_kc} to compute the conditional probability $P_\dcpprogram(\mu=q|\evidenceset=e)$.
\end{restatable}

\begin{proof}
    See Appendix~\ref{app:proof:alwonddnnf}.
\end{proof}


Algorithm~\ref{alg:prob_via_alw_kc} takes as input a two propositional logic formulas $\phi$ and $\phi_q$, and a set of ancestral samples. It then knowledge-compiles the formulas $\phi \land \phi_q$ and $\phi$ into circuits $\Gamma_q$ and $\Gamma$. These circuits are then evaluated using Algorithm~\ref{alg:unormalize_alw}. The variables $ialw_q$ and $ialw$ hold infinitesimal numbers.
\fixed{
The ratio of these two numbers, which corresponds to the ratio in Equation~\ref{eq:prop:alwapproximation}, is an infinitesimal number having as second argument $0$ and as first argument the conditional probability.}
\begin{algorithm}[h]
    \SetKwFunction{ProbALW}{ProbALW}
    \SetKwFunction{KC}{KC}
    \SetKwFunction{IALW}{IALW}


    \SetKwProg{Fn}{function}{}{}
    \SetKwProg{ElseComment}{function}{}{}

    \caption{Conditional Probability via IALW and KC}
	\label{alg:prob_via_alw_kc}
\Fn{\ProbALW{$\phi$, $\phi_q$, $\mathcal{S}$ }}{
    $\Gamma_{q}$ $\leftarrow$ \KC{$\phi \land \phi_q$}\;
     \label{alg:prob_via_alw_kc:kc_qe} 
    $\Gamma$ $\leftarrow$ \KC{$\phi$}\;
    \label{alg:prob_via_alw_kc:kc_e}
    $ialw_{q}$ $\leftarrow$ \IALW{$\Gamma_{q}$,$\mathcal{S}$} \label{alg:prob_via_alw_kc:alw_qe}
    \tcp*[r]{cf. Algorithm~\ref{alg:unormalize_alw}} 
    $ialw$ $\leftarrow$ \IALW{$\Gamma$,$\mathcal{S}$}   \label{alg:prob_via_alw_kc:alw_e} 
    \tcp*[r]{cf. Algorithm~\ref{alg:unormalize_alw}}
    $(p,0)$ $\leftarrow$ $ialw_{q} \oslash ialw$\;
    \Return $p$ \label{alg:prob_via_alw_kc:conditional}
	}
\end{algorithm}


\begin{algorithm}[h]
    \SetKwFunction{IALW}{IALW}
    \SetKwFunction{Eval}{Eval}
    \SetKwProg{Fn}{function}{}{}
    \SetKwProg{ElseComment}{function}{}{}
    \caption{Computing the IALW}
	\label{alg:unormalize_alw}
\Fn{\IALW{$\Gamma$, $\mathcal{S}$}}{
    $ialw$ $\leftarrow$ (0,0) \;
    \For{$\varset{s}^{(i)} \in \mathcal{S}$  }{
        $ialw$ $\leftarrow$ $ialw$ $\oplus$ \Eval{$\Gamma$, $\varset{s}^{(i)}$}
        \tcp*[r]{cf. Algorithm~\ref{alg:eval}}  
    }
    \Return $ialw$
	}
\end{algorithm}

Algorithm~\ref{alg:unormalize_alw} computes the IALW given as input a circuit $\Gamma$ and a set of ancestral samples.
The loop evaluates the circuit (using Algorithm~\ref{alg:eval}) for each ancestral sample $\varset{s}^{(i)}$ and accumulates the result, which is then returned once the loop terminates. The accumulation inside the loop corresponds to the  $\bigoplus_{i=1}^{\lvert \mathcal{S} \rvert}$ summation in Equation~\ref{eq:alw_show}.
Algorithm~\ref{alg:eval} evaluates a circuit $\Gamma$ for a single ancestral sample $\varset{s}^{(i)}$ and is a variation of the circuit evaluation algorithm presented by~\citet{kimmig2017algebraic}.

\begin{algorithm}[h]
    \SetKwFunction{EvalFn}{Eval}
    \SetKwProg{Fn}{function}{}{}
    \SetKwProg{ElseComment}{function}{}{}

	\caption{Evaluating an sd-DNNF circuit $\Gamma$ for labeling function $\alpha^{(i)}$ (Definition~\ref{def:sample_labeling_function}) and semiring operations $\oplus$  and $\otimes$ (Definition~\ref{def:inf_number_ops})}
	\label{alg:eval}
\Fn{\EvalFn{$\Gamma$,$\varset{s}^{(i)}$}}{
		\If{  $\Gamma$ is a literal node $l$}{ \Return $\alpha^{(i)}(l)$}
		\ElseIf{$\Gamma$ is a disjunction $\bigvee_{j=1}^m \Gamma_j$}
		{\Return $\bigoplus_{j=1}^m$ \EvalFn{$\Gamma_j$, $\varset{s}^{(i)}$}}
		\Else( \tcp*[f]{$\Gamma$ is a conjunction $\bigwedge_{j=1}^m \Gamma_j$}){
		 \Return $\bigotimes_{i=j}^m$ \EvalFn{$\Gamma_j$, $\varset{s}^{(i)}$}}
	}
\end{algorithm}



\begin{example}[IALW on Algebraic Circuit]
\label{example:eval_observation}

Consider a version of the program in Example~\ref{example:dcproblog:observation} where the annotated disjunction has been eliminated and been replaced with a binary random variable $m$ and a \probloginline{flip} distribution.
	\begin{problog*}{linenos}
m~flip(0.3).

size~beta(2,3):- m=:=0.
size~beta(4,2):- m=:=1.
	\end{problog*}
We query the program for the conditional probability $P((\mathprobloginline{m=:=1}) =\top \mid \mathprobloginline{size}\doteq 4/10 )$.
Following the program transformations introduced in Section~\ref{sec:dc2smt} and then compiling the labeled propositional formula, we obtain a circuit representation of the queried program. Evaluating this circuit yields the probability of the query. To be precise, we actually obtain two circuits, one representing the probability of relevant program with the evidence enforced and with additionally having the value of the query atom set. 
In Figure~\ref{fig:circuit:ialw} we show the circuit where only the evidence is enforced.




	\begin{figure}[h]
	\resizebox{\linewidth}{!}{%

		\tikzstyle{distribution}=[rectangle, text centered, fill=white, draw, dashed,thick]
		\tikzstyle{leaf}=[rectangle, text centered, fill=gray!10, draw,thick]



		\tikzstyle{negate}=[
			rectangle split,
			rectangle split parts=3, 
			rectangle split horizontal,
			text centered,
			rectangle split part fill={gray!10,white,gray!10},
			draw,
			rectangle split draw splits=false,
			anchor=center,
			align=center,
			thick
		]
		\newcommand{\minus}{  ${\bm e^\otimes}$ \nodepart{second} ${{\bm \ominus}}$ \nodepart{third}  \phantom{${\bm e^\otimes}$}}

		\tikzstyle{sumproduct}=[
			rectangle split,
			rectangle split parts=3, 
			rectangle split horizontal,
			text centered,
			rectangle split part fill={gray!10,white,gray!10},
			draw,
			rectangle split draw splits=false,
			anchor=center,
			align=center,
			thick
		]
		\newcommand{\supr}[1]{  \phantom{${\bm e^\otimes}$} \nodepart{second} ${{\bm #1}}$ \nodepart{third} \phantom{${\bm e^\otimes}$}}
		
		\tikzstyle{circuitedge}=[ultra thick, thick,->]
		\tikzstyle{distributionedge}=[thick,->, dashed, in=-90]
		
		\tikzstyle{indexnode}=[draw,circle, inner sep=1pt]				
		
		\begin{tikzpicture}[remember picture]
			
			\node[sumproduct] (14) at (200.54bp,378.0bp) {\supr{\oplus}};
			\draw[ultra thick, thick,->] (14.90) to  ([shift={(0,1)}]14.90);
			
			\node[sumproduct] (9) [below left = of 14] {\supr{\otimes}};
			\node[sumproduct] (13) [below right = of 14]  {\supr{\otimes}};
			
			\node[negate] (m1) [below=of 9] {\minus};
			\node[sumproduct] (12)  [below  = of 13] {\supr{\oplus}};
			
			\node[sumproduct] (8)  [below=of 12]  {\supr{\otimes}};
			
			\node[indexnode, left=of 14, xshift=0.9cm, yshift=0.4cm] {\footnotesize {$6$}};			
			\node[indexnode, left=of 13, xshift=0.9cm, yshift=0.4cm] {\footnotesize {$5$}};			
			\node[indexnode, left=of 8, xshift=0.9cm, yshift=0.4cm] {\footnotesize {$2$}};			
			\node[indexnode, left=of 12, xshift=0.9cm, yshift=0.4cm] {\footnotesize {$4$}};			
			\node[indexnode, left=of 9, xshift=0.9cm, yshift=0.4cm] {\footnotesize {$3$}};			
			\node[indexnode, left=of m1, xshift=0.9cm, yshift=0.4cm] {\footnotesize {$1$}};	
			
			
			\node[leaf, below= of 8, xshift=-0.2cm] (4)   {$\subnode{var_m0}{$m$}=1$};
			\node[leaf, below= of 8, xshift=1.8cm] (2) {$\subnode{var_m1}{$m$}=0$};				
			\node[leaf] (size1obs) [left=of 4, xshift=0cm]  {$\subnode{var_s_11_un}{size_{1}} \doteq 0.4$};
			\node[leaf] (size0obs) [left=of 4, xshift=-4cm] {$\subnode{var_s_01_un}{size_{0}} \doteq 0.4$};				

			\node[distribution] (size0)  [below=of size0obs] {\probloginline{beta(2,3)}};
			\node[distribution] (size1)  [below=of size1obs] {\probloginline{beta(4,2)}};
			\node[distribution] (flip)  [below= of 4, xshift=1cm] {\probloginline{flip(0.3)}};				


			\draw[distributionedge,out=90]  (size1) to (var_s_11_un);
			\draw[distributionedge,out=90] (size0) to  (var_s_01_un);
			\draw[distributionedge,out=160] (flip) to  (var_m0);
			\draw[distributionedge] (flip) to  (var_m1);


			%https://tex.stackexchange.com/questions/447989/anchor-node-names-for-tikz-rectangle-split-horizontal			
			\draw[circuitedge] (9) to  (14.mid);
			\draw[circuitedge] (13) to  (14.three south |- 14.mid);
			\draw[circuitedge] (m1) to  (9.mid);
			\draw[circuitedge] (12) to  (13.three south |- 13.mid);
			\draw[circuitedge] (size0obs) to   (m1.three south |- m1.mid);	
			\draw[circuitedge] (2) to  (12.three south |- 12.mid);				
			\draw[circuitedge] (size1obs) to   (8.mid);
			\draw[circuitedge] (4) to  (8.three south |- 8.mid);	
			\draw[circuitedge] (8) to  (9.three south |- 9.mid);	
			\draw[circuitedge] (8) to   (12.mid);
			\draw[circuitedge] (size0obs) to  (13.mid);

		\end{tikzpicture}
	}
    \captionof{figure}{At the bottom of the circuit we see the distributions feeding in. The \probloginline{flip} distribution feeds into its two possible (non-zero probability) outcomes. The two \probloginline{beta} distributions feed into an observation statement each. We use the `$\doteq$' symbol to denote such an observation. Note how we identify each of the two random variables for the size by a unique identifier in their respective subscripts. The circled numbers next to the internal nodes, \ie the sum and product nodes, will allow us to reference the nodes later on and do not form a part of the algebraic circuit.}
    \label{fig:circuit:ialw}
\end{figure}



The probability of the query (given the evidence) can now be obtained by evaluating recursively the internal nodes in the algebraic circuit using Algorithm~\ref{alg:eval}.
We perform the evaluation  of the circuit in Figure~\ref{fig:circuit:ialw} for a single iteration of the loop in Algorithm~\ref{alg:unormalize_alw}, and we assume that we have sampled the value $m=0$ from the \probloginline{flip(0.3)} distribution.

\begin{minipage}{0.49\linewidth}
    \begin{align*}
        &\mathtt{Eval}(\footcircled{$1$})\\
        &=
        e^\otimes \ominus \alpha_{IALW}\big(size_0\doteq 0.4\big) \\
        &=
        (1,0) \ominus (1.728,1) \\
        &= (1,0)
        \\
        \hfill
        \\
        % 
        &\mathtt{Eval}(\footcircled{$2$})\\
        &=
        \alpha_{IALW}\big( size_1\doteq 0.4) \otimes \alpha_{IALW}\big(  m=1 \big) \\
        &=
        (0.768,1) \otimes (0,0) \\
        &=
        (0,1)\\
        \hfill
        \\
        % 
        &\mathtt{Eval}(\footcircled{$3$})\\
        &=
        \mathtt{Eval}(\footcircled{$1$}) \otimes \mathtt{Eval}(\footcircled{$2$}) \\
        &=
        (1,0) \otimes (0,1) \\
        &=
        (0,1)\\
    \end{align*}
\end{minipage}
\vline
\begin{minipage}{0.49\linewidth}
    \begin{align*}
        &\mathtt{Eval}(\footcircled{$4$})\\
        &=
        \mathtt{Eval}(\footcircled{$2$}) \oplus \alpha_{IALW}\big(  m=0 \big) \\
        &=
        (0,1) \oplus (1,0) \\
        &=
        (1,0)\\
        \hfill
        \\
        % 
        &\mathtt{Eval}(\footcircled{$5$})\\
        &=
        \alpha_{IALW}\big(  size_0\doteq 0.4 \big) \otimes \mathtt{Eval}(\footcircled{$2$})  \\
        &=
        (1.728,1) \otimes (1,0) \\
        &=
        (1.728,1)\\
        \hfill
        \\
        % 
        &\mathtt{Eval}(\footcircled{$6$})\\
        &=
        \mathtt{Eval}(\footcircled{$3$}) \oplus \mathtt{Eval}(\footcircled{$5$})  \\
        &=
        (0,1) \oplus (1.728,1) \\
        &=
        (1.728,1)\\
    \end{align*}
\end{minipage}


If we evalute the circuit for a sample $m=1$ we obtain in a similar fashion the result $\mathtt{Eval}(\footcircled{$6$})= (0.768,1)$. Moreover, if we evaluate the circuit multiple times we obtain (in the limit) 70\% of the time the outcome $(1.728,1)$ and 30\% of the time the value $(0.768,1)$. This yields an average of $(0.7\times 1.728, 1)\oplus (0.3\times 0.768, 1)= (1.440,1)$ and represents the unnormalized infinitesimal algebraic likelihood weight of the evidence.
The unnormalized infinitesimal algebraic likelihood weight of the query conjoined with the evidence is obtain again in a similar fashion but with the samples for $m=0$ being discarded. This then yields the result $(0.3\times 1.728, 1)$.
Dividing these two (unnormalized) infinitesimal algebraic likelihood weights by each other gives the probability of the query.
\begin{align*}
    &P((\mathprobloginline{m=:=1}) =\top \mid \mathprobloginline{size}\doteq 4/10 )\\
    &=(0.3\times 1.728, 1) \oslash \Big( (0.7\times 0.768,1 ) \oplus (0.3\times 1.728,1) \Big) \\
    &= (0.2304/1.440 , 1{-}1) \\
    &= (0.16,0)
\end{align*}
\end{example}



\subsection{Partial Symbolic Inference}

Evaluating circuits using binary random variables is quite wasteful: on average half of the samples are unused for one of the two possible outcomes ($0$ or $1$). We can remedy this by performing (exact) symbolic inference on binary random variables and replace the comparisons where they appear with their expectation. For instance, we replace \probloginline{m=:=1} by the infinitesimal number $(0.3,0)$ instead of sampling a value for \probloginline{m} and testing whether the sample satisfies the constraint. This technique is also used by other probabilistic programming languages such as \problogsty~\citep{fierens2015inference} and Dice~\citep{holtzen2020dice}. The main difference to \dcproblogsty is that those languages only support binary random variables (and by extension discrete random variables with finite support), while \dcproblogsty interleaves discrete and continuous random variables.

In a sense, the expectation gets pushed from the root of the algebraic circuit representing a probability to its leaves. This is, however, only possible if the circuit respects specific properties. Namely, the ones respected by \mbox{d-DNNF} formulas (cf. Section~\ref{sec:ALWviaKC}), which we use as our representation language for the probability.


\begin{definition}[Symbolic IALW Label of a Literal] \label{def:sample_probability_labeling_function}
Given an ancestral sample $\varset{s}^{(i)}= (s_1^{(i)}, \dots,  s_M^{(i)} ) $ for the random variables $\randomvariableset = (\nu_1,\dots, \nu_M)$.
The  Symbolic IALW (SIALW) label of a positive literal $\ell$ is an infinitesimal number given by:
\begin{align}
    \alpha_{SIALW}^{(i)}( \ell)
    =\begin{cases}
    (p_{\ell}, 0), & \text{if $\ell$ encodes a probabilistic fact} \\
    \alpha^{(i)}_{IALW}(\ell), & \text{otherwise} 
    \end{cases}
    \nonumber
\end{align}
For the negated literals we have the following labeling function:
\begin{align}
    \alpha_{SIALW}^{(i)}( \neg \ell)
    =\begin{cases}
    (1{-}p_{\ell}, 0), & \text{if $\ell$ encodes a probabilistic fact} \\
    \alpha^{(i)}_{IALW}(\neg \ell), & \text{otherwise}
    \end{cases}
    \nonumber
\end{align}
The number $p_\ell$ is the label of the probabilistic fact in a \dcproblogsty program.
\end{definition}

In the definition above we replace the label of a comparison that corresponds to a probabilistic fact with the probability of that fact being satisfied. This has already been shown to be beneficial when performing inference, both in terms of inference time and accuracy of Monte Carlo estimates~\citep{zuidbergdosmartires2019exact}. Following the work of~\citep{kolb2019exploit} one could also develop more sophisticated methods to detect which comparison in the leaves can be replaced with their expectation. We leave this for future work.








\begin{example}[Symbolic IALW on Algebraic Circuit]
\label{example:eval_observation_marginalized}

Symbolic inference for the random variable $m$ from the circuit in Example~\ref{example:eval_observation} results in annotating the leaf nodes for the different outcomes of the random variable $m$ with the probabilities of the respective outcomes. This can be seen in the red dashed box in the bottom right of Figure~\ref{fig:circuit:sialw}.

	\begin{figure}[h]
		\resizebox{\linewidth}{!}{%
			
			\tikzstyle{distribution}=[rectangle, text centered, fill=white, draw, dashed,thick]
			\tikzstyle{leaf}=[rectangle, text centered, fill=gray!10, draw,thick]
			
			
			
			\tikzstyle{negate}=[
			rectangle split,
			rectangle split parts=3, 
			rectangle split horizontal,
			text centered,
			rectangle split part fill={gray!10,white,gray!10},
			draw,
			rectangle split draw splits=false,
			anchor=center,
			align=center,
			thick
			]
			\newcommand{\minus}{  ${\bm e^\otimes}$ \nodepart{second} ${{\bm \ominus}}$ \nodepart{third}  \phantom{${\bm e^\otimes}$}}
			
			\tikzstyle{sumproduct}=[
			rectangle split,
			rectangle split parts=3, 
			rectangle split horizontal,
			text centered,
			rectangle split part fill={gray!10,white,gray!10},
			draw,
			rectangle split draw splits=false,
			anchor=center,
			align=center,
			thick
			]
			\newcommand{\supr}[1]{  \phantom{${\bm e^\otimes}$} \nodepart{second} ${{\bm #1}}$ \nodepart{third} \phantom{${\bm e^\otimes}$}}
			
			\tikzstyle{circuitedge}=[ultra thick, thick,->]
			\tikzstyle{distributionedge}=[thick,->, dashed, in=-90]
			
	    	\tikzstyle{indexnode}=[draw,circle, inner sep=1pt]				
			
			\begin{tikzpicture}[remember picture]
				
				\node[sumproduct] (14) at (200.54bp,378.0bp) {\supr{\oplus}};
				\draw[ultra thick, thick,->] (14.90) to  ([shift={(0,1)}]14.90);
				
				\node[sumproduct] (9) [below left = of 14] {\supr{\otimes}};
				\node[sumproduct] (13) [below right = of 14]  {\supr{\otimes}};
				
				\node[negate] (m1) [below=of 9] {\minus};
				\node[sumproduct] (12)  [below  = of 13] {\supr{\oplus}};
				
				\node[sumproduct] (8)  [below=of 12]  {\supr{\otimes}};
				
    			\node[indexnode, left=of 14, xshift=0.9cm, yshift=0.4cm] {\footnotesize {$6$}};			
    			\node[indexnode, left=of 13, xshift=0.9cm, yshift=0.4cm] {\footnotesize {$5$}};			
    			\node[indexnode, left=of 8, xshift=0.9cm, yshift=0.4cm] {\footnotesize {$2$}};			
    			\node[indexnode, left=of 12, xshift=0.9cm, yshift=0.4cm] {\footnotesize {$4$}};			
    			\node[indexnode, left=of 9, xshift=0.9cm, yshift=0.4cm] {\footnotesize {$3$}};			
    			\node[indexnode, left=of m1, xshift=0.9cm, yshift=0.4cm] {\footnotesize {$1$}};	
			
				
				\node[leaf, below= of 8, xshift=-0.2cm] (4)   {$3/10$};
				\node[leaf, below= of 8, xshift=1.8cm] (2) {$7/10$};
				\draw[red,ultra thick,dashed] ($(4.north west)+(-0.2,0.2)$)  rectangle ($(2.south east)+(0.2,-0.2)$);		
				
						
				\node[leaf] (size1obs) [left=of 4, draw,  xshift=0cm]  {$\subnode{var_s_11}{size_{1}} \doteq 0.4$};
				\node[leaf]  (size0obs) [left =of 4, xshift=-4cm] {$\subnode{var_s_01}{size_{0}} \doteq 0.4$};				
				
		    	\node[distribution] (size0)  [below=of size0obs] {\probloginline{beta(2,3)}};
		    	\node[distribution] (size1)  [below=of size1obs] {\probloginline{beta(4,2)}};

				
				
				\draw[distributionedge,out=90]  (size1) to (var_s_11);
				\draw[distributionedge,out=90] (size0) to  (var_s_01);
				
				
				%https://tex.stackexchange.com/questions/447989/anchor-node-names-for-tikz-rectangle-split-horizontal			
				\draw[circuitedge] (9) to  (14.mid);
				\draw[circuitedge] (13) to  (14.three south |- 14.mid);
				\draw[circuitedge] (m1) to  (9.mid);
				\draw[circuitedge] (12) to  (13.three south |- 13.mid);
				\draw[circuitedge] (size0obs) to   (m1.three south |- m1.mid);	
				\draw[circuitedge] (2) to  (12.three south |- 12.mid);				
				\draw[circuitedge] (size1obs) to   (8.mid);
				\draw[circuitedge] (4) to  (8.three south |- 8.mid);	
				\draw[circuitedge] (8) to  (9.three south |- 9.mid);	
				\draw[circuitedge] (8) to   (12.mid);
				\draw[circuitedge] (size0obs) to  (13.mid);
				
			\end{tikzpicture}
		}
        \captionof{figure}{Circuit representation of the SIALW algorithm for the probability $P(\mathprobloginline{size}\doteq 4/10 )$.}
        \label{fig:circuit:sialw}

	\end{figure}
	
Evaluating the marginalized circuit now returns immediately the unnormalized algebraic model count for the evidence without the need to draw samples and consequently without the need to sum over the samples. 

\begin{minipage}{0.49\linewidth}
    \begin{align*}
        &\mathtt{Eval}(\footcircled{$1$})\\
        &=
        e^\otimes \ominus \alpha_{SIALW}\big(size_0\doteq 0.4\big) \\
        &=
        (1,0) \ominus (1.728,1) \\
        &= (1,0)
        \\
        \hfill
        \\
        % 
        &\mathtt{Eval}(\footcircled{$2$})\\
        &=
        \alpha_{SIALW}\big( size_1\doteq 0.4) \otimes \alpha_{SIALW}\big(  m=1 \big) \\
        &=
        (0.768,1) \otimes (0.3,0) \\
        &=
        (0.2304,1)\\
        \hfill
        \\
        % 
        &\mathtt{Eval}(\footcircled{$3$})\\
        &=
        \mathtt{Eval}(\footcircled{$1$}) \otimes \mathtt{Eval}(\footcircled{$2$}) \\
        &=
        (1,0) \otimes (0.2304,1) \\
        &=
        (0.2304,1)\\
    \end{align*}
\end{minipage}
\vline
\begin{minipage}{0.49\linewidth}
    \begin{align*}
        &\mathtt{Eval}(\footcircled{$4$})\\
        &=
        \mathtt{Eval}(\footcircled{$2$}) \oplus \alpha_{SIALW}\big(  m=0 \big) \\
        &=
        (0.2304,1) \oplus (0.7,0) \\
        &=
        (0.7,0)\\
        \hfill
        \\
        % 
        &\mathtt{Eval}(\footcircled{$5$})\\
        &=
        \alpha_{SIALW}\big(  size_0\doteq 0.4 \big) \otimes \mathtt{Eval}(\footcircled{$2$})  \\
        &=
        (1.728,1) \otimes (0.7,0) \\
        &=
        (1.2096,1)\\
        \hfill
        \\
        % 
        &\mathtt{Eval}(\footcircled{$6$})\\
        &=
        \mathtt{Eval}(\footcircled{$3$}) \oplus \mathtt{Eval}(\footcircled{$5$})  \\
        &=
        (0.2304,1 \oplus (1.2096,1) \\
        &=
        (1.440,1)\\
    \end{align*}
\end{minipage}
	
	
	
	
	
\end{example}







\subsection{Experimental Evaluation}
\label{sec:experimental}

\new{

In order to demonstrate the benefits of adapting the technique of knowledge compilation to the discrete-continuous domain with zero-probability conditioning events, we model a machine that runs either in operating \probloginline{mode1} or \probloginline{mode2}; (with probability $0.2$ and $0.8$ respectively). Furthermore, the machine can be faulty with a small probability of $10^{-5}$. In this case we would like to switch the machine of and repair it.

If the machine is not faulty the temperature measurements we perform on the machine are distributed according to two Gaussian (Lines~\ref{line:ex:faulty:temp1} and \ref{line:ex:faulty:temp2}).
If the machine is faulty, however, we get a deterministic temperature reading of $2.0$ (Line \ref{line:ex:faulty:temp3}).

}

\begin{problog*}{linenos}
0.00001::faulty.
0.2::mode1;0.7::mode2.

temperature ~ normal(0.5,1.0) :- \+faulty, mode1. @\label{line:ex:faulty:temp1}@
temperature ~ normal(2.0,2.0) :- \+faulty, mode2. @\label{line:ex:faulty:temp2}@
temperature ~ delta(2.0) :- faulty. @\label{line:ex:faulty:temp3}@
\end{problog*}

\new{
We are now interested in computing $p( \mathprobloginline{faulty}{=}\top {\mid} \mathprobloginline{temperature}{\doteq} 2.0 )$. That is, what is the probability that the machine is faulty given that the temperature measurement is $2.0$.

In our experiment we compared the performance of SIALW (\cf Definition~\ref{def:sample_probability_labeling_function}) to the inference algorithm of \dcsty~\citep{nitti2016probabilistic}. The latter is equivalent to the algorithms presented by \Citet{wu2018discrete} and \Citet{jacobs2021paradoxes} as all three perform, in essence, likelihood weighting with infinitesimal numbers.   
Specifically, we study the sensitivity of the algorithms with regard to the probability of the machine being faulty.

In Figure~\ref{fig:experiment:faultVScorrectness} we see that using SIALW computes the correct probability regardless of the fault probability $\{10^{-5}, 10^{-4}, 10^{-3}, 10^{-2}, 10^{-1} \}$. We also see that the naivc likelihood weighting algorithm (without the exact symbolic inference of SIALW) needs a substantial amount of samples to infer the correct probability. Most notably, for a fault probability of $10^{-5}$ not even a sample size of $10^5$ is sufficient.

The large discrepancy between SIALW and the competing approach by \Citet{nitti2016probabilistic} is explained as follows: in order to correctly infer the queried probability one of the samples drawn from the Bernoulli distribution \probloginline{faulty ~ flip(0.00001)} needs to be true. This would then trigger the rule for \probloginline{temperature ~ delta(2.0) :- faulty}. As this is, however, extremely unlikely the crucial rule needed to perform correct likelihood weighting with infinitesimal numbers is never triggered and the returned probability is incorrect.
SIALW, in constrast, does not sample \probloginline{faulty ~ flip(0.00001)} but performs exact inference using knowledge compilation. As a result SIALW always computes the correct posterior probability.  





\begin{figure}
    \begin{center}
        \includegraphics[width=\linewidth]{experiment_dc/faultVScorrectness.pdf}
    \end{center}
    \caption{
        \new{
        We queried SIALW and (non-symbolci) likelihood weighting each $100$ times for the probability $p( \mathprobloginline{faulty}{=}\top {\mid} \mathprobloginline{temperature}{\doteq} 2.0 )$. On the $y$-axis we give the ratio $\nicefrac{\text{\#correct runs}}{\text{\#runs}}$.
        Die to the use of knowledge compilation, SIALW is insensitive to the probability of fault (on the x-axis).
        This is in contrast to the log-likelihood weighting (LLW) algorithm presented by~\Citet{nitti2016probabilistic}, which necessitates a considerable number of samples to compute the queried probability reliably. The different dotted lines indicate settings with varying sample sizes ($\{10^1, 10^2, 10^3, 10^4, 10^5 \}$).  
        }
    }
    \label{fig:experiment:faultVScorrectness}
\end{figure}
}

% \begin{figure}
%     \begin{center}
%         \includegraphics[width=\linewidth]{experiment_dc/samplesVScorrectness.pdf}
%     \end{center}
%     \caption{test}
%     \label{fig:experiment:samplesVScorrectness}
% \end{figure}


\section{\dcproblogsty and the Probabilistic Programming Landscape}\label{sec:related}

In recent years a plethora of different probabilistic programming languages have been developed. 
We discuss these by pointing out key features present in \dcproblogsty (listed below), which are missing in specific related works.
We organize these features along the three key contributions stated in Section \ref{sec:introduction}. Our first key contribution is the introduction of the measure semantics with the following features:


\begin{enumerate}[label=C1.\arabic*, leftmargin=2\parindent]
    \item \new{possibly infinite number (even uncountable) number of random variables} 
    \label{f:infinite_number_of rv}
    \item random variables with (possibly) infinite sample spaces
    \label{f:infinite_samplespaces}
    \item functional dependencies between random variables
    \label{f:rv_dependencies}
    \item uniform treatment of discrete and continuous random variables
    \label{f:dc_rv_uniform}
    \item negation
    \label{f:declarative_negation}
\end{enumerate}

\noindent Our second contribution is the  introduction of  the \dcproblogsty language, which  
\begin{enumerate}[label=C2.\arabic*, leftmargin=2\parindent]
    \item has purely discrete PLPs and their semantics as a special case,
    \label{f:discrete_special}
    \item supports a rich set of comparison predicates, and
    \label{f:comparison_predicates}
    \item is a Turing complete language (\dcplpsty)
    \label{f:turing_complete}
\end{enumerate}

\noindent Our last contributions concern inference, which includes 
\begin{enumerate}[label=C3.\arabic*, leftmargin=2\parindent]
    \item a formal definition of the hybrid probabilistic inference task,
    \label{f:definition_inference}
    \item an inference algorithm called IALW, 
    \label{f:inference_algorithm}
    \item that uses standard knowledge compilation in the hybrid domain.
    \label{f:inference_kc}
\end{enumerate}


% Our focus will lie with probabilistic logic programming languages, and we only touch lightly upon other paradigms of probabilistic programming
\subsection{\problogsty and \dcsty}

The \dcproblogsty language is a generalization of \problogsty, both in terms of syntax and semantics. A \dcproblogsty program that does not use distributional clauses (or distributional facts) is also a \problogsty program, and both define the same distribution over the logical vocabulary of the program. \dcproblogsty properly generalizes \problogsty to include random variables with infinite sample spaces (\ref{f:infinite_samplespaces}). 

On a syntactical level, \dcproblogsty is closely related to the \dcsty (DC) language, with which it shares the  \predicate{~}{2} predicate used in infix notation. 
In Appendix~\ref{sec:dcproblog-dc} we discuss in more detail the relationship between \dcproblogsty and the \dcsty language. Concretely, we point out that 
\dcproblogsty generalizes the original and negation-free version of DC~\citep{gutmann2011magic} (\ref{f:declarative_negation}).
However, \dcproblogsty differs in its declarative interpretation of negation from the procedural interpretation as introduced to DC by~\citet{nitti2016probabilistic}.
As a consequence, the semantics of DC and \problogsty differ in the absence of continuous random variables, while \dcproblogsty is a strict generalization of \problogsty (\ref{f:discrete_special}).



\subsection{Bayesian Logic Programs}
\label{sec:blp}

\new{
Bayesian logic programs (BLPs)~\citep{kersting2000bayesian} can be seen as a special case of \dcplpsty: while the semantics of \dcplpsty allow for a (possibly uncountable) infinite number of random variables~\ref{f:infinite_number_of rv}, BLPs are limited to finite distributional databases (expressed as Bayesian networks).

Moreover, using the construct of distributional clauses we introduce syntax to interleave logic programming statements and the declaration of the distributional database. This is not supported in the BLP language.
Lastly, IAWL equips \dcproblogsty with a sound inference algorithm for the discrete-continuous space. This is again in contrast to BLP's inference algorithm that only handles (discrete mixtures of) continuous random variables. 
}



\subsection{\extendedprismsty}

An early attempt of equipping a probabilistic logic programming language with continuous random variables can be found in~\citep{islam2012inference}, which was dubbed {\em \extendedprismsty}. 
Similar to \dcproblogsty, \extendedprismsty's semantics are based again on \citeauthor{sato1995statistical}'s distribution semantics.
However, \extendedprismsty assumes, just like \dcsty , pairwise mutually exclusive proofs (we refer again to Appendix~\ref{sec:dcproblog-dc} for details on this).
On the expressivity side, \extendedprismsty only supports linear equalities -- in contrast to \dcproblogsty, where also inequalities are included in the semantics of the language (\ref{f:comparison_predicates}). 
An advantage of restricting possible constraints to equalities is the possibility of performing exact symbolic inference. In this regard, \extendedprismsty, together with its symbolic inference algorithm, can be viewed as a logic programming language that has access to a computer algebra system. Swapping out the approximate Sampo-inspired inference algorithm in \dcproblogsty by an exact inference algorithm using symbolic expression manipulations would result in an inference approach closely related to that of \extendedprismsty. One possibility would be to use the Symbo algorithm presented in~\citep{zuidbergdosmartires2019exact}, which uses the PSI-language~\citep{gehr2016psi} as its (probabilistic)  computer algebra system.




\subsection{Probabilistic Constraint Logic Programming}

Impressive work on extending probabilistic logic programs with continuous random variables was presented by~\citet{michels2015new} with the introduction of Probabilistic Constraint Logic Programming (PCLP). The semantics of PCLP are again based on \citeauthor{sato1995statistical}'s distribution semantics and the authors also presented an approximate inference algorithm for hybrid probabilistic logic programs.
Interestingly, the algorithm presented in~\citep{michels2015new} to perform (conditional) probabilistic inference extends weighted model counting to continuous random variables using imprecise probabilities, and more specifically credal sets.

A shortcoming of PCLP's semantics is the lack of direct support for generative definitions of random variables, \ie, random variables can only be interpreted within constraints, but not within distributions of other random variables as is possible in \dcproblogsty (\ref{f:rv_dependencies}).
\citet{azzolini2021semantics} define a non-credal version of this semantics using a product measure over a space that explicitly separates discrete and continuous random variables, assuming that a measure over the latter is given as part of the input without further discussion of how this part of the measure is specified in a program. Furthermore, they do not define any inference tasks (\ref{f:definition_inference}), \eg computing conditional probabilities (\cf Section~\ref{sec:inference-tasks}), nor
do they provide an inference algorithm (\ref{f:inference_algorithm}).

A later proposal for the syntax of such programs~\citep{azzolini:iclp21} combines two classes of terms (logical and continuous ones) with typed predicates and functors, and defines mixture variables as well as arithmetic expressions over random variables through logical clauses. In other words, user-defined predicates define families of random variables through the use of typed arguments of the predicate identifying a specific random variable, arguments providing parameters for the distribution, and one argument representing the random variable itself.
In contrast, the syntax of \dcproblogsty clearly identifies all  random variables through explicit terms introduced through distributional facts or distributional clauses, explicitly exposes the probabilistic dependency structure by using random variable terms inside distribution terms, and avoids typing through argument positions.
Moreover, \dcproblogsty takes a uniform view on all random variables in terms of semantics, thereby avoiding treating discrete and continuous random variables separately (\ref{f:dc_rv_uniform}).

\subsection{\blogsty}


Notable in the domain of probabilistic logic programming is also the BLOG language~\citep{milch2005blog,wu2018discrete}. Contrary to the aforementioned probabilistic logic programming languages, BLOG's semantics are not specified using \citeauthor{sato1995statistical}'s distribution semantics but via so-called {\em measure-theoretic Bayesian networks} (MTBN), which were introduced in~\citep{wu2018discrete}. MTBNs can be regarded as the assembly language for BLOG: every BLOG program is translated or compiled to an MTBN.
With \dcproblogsty we follow a similar pattern: every \dcproblogsty program with syntactic sugar (\eg annotated disjunctions)  is transformed into \dfplpsty program. The semantics are defined on the bare-bones program. Note that the assembly language for \dcproblogsty (\dfplpsty) is Turing complete. This is not the case for MTBNs (\ref{f:turing_complete}).

% On the practical side, the default inference algorithm deployed by BLOG is similar in spirit to IALW inference  but does also provide alternatives, such as Metropolis-Hastings MCMC~\citep{milch2006general}. An important difference of \blogsty's inference algorithms with regard to \dcproblogsys is the absence of knowledge compilation. BLOG will, hence, struggle with problems that exhibit rich combinatorial structures, as shown by~\citet{zuidbergdosmartires2019exact}.


\subsection{Non-logical Probabilistic Programming}

As first pointed out by~\citet{russell2015unifying} and later on elaborated upon by~\citet{kimmig2017probabilistic}, probabilistic programs fall either into the {\em possible worlds semantics} category or the {\em probabilistic execution traces semantics} category. The former is usually found in logic based languages, while the latter is the prevailing view in imperative and functional probabilistic languages.

While, the probabilistic programming languages discussed so far follow the possible worlds paradigm,
many languages follow the execution traces paradigm, either as a probabilistic functional language~\citep{goodman2008church,wood2014approach} or as a imperative probabilistic language~\citep{gehr2016psi,salvatier2016probabilistic,carpenter2017stan,bingham2019pyro,ge2018turing}. Generally speaking, functional and imperative probabilistic programming languages target first and foremost continuous random variables, and discrete random variables are only added as an afterthought. A notable exception is the functional probabilistic programming language Dice~\citep{holtzen2020dice}, which targets discrete random variables exclusively.

Concerning inference in probabilistic programming, we can observe general trends in logical and non-logical probabilistic languages. While the latter are interested in adapting and speeding up approximate inference algorithms, such as Markov chain Monte Carlo sampling schemes or variational inference, the former type of languages are more invested in exploiting independences in the probabilistic programs, mainly by means of knowledge compilation. Clearly, these trends are not strict. For instance, \citet{obermeyer2019functional} proposed so-called {\em funsors} to express and exploit independences in \pyrosty~\citep{bingham2019pyro}, an imperative probabilistic programming language, and \citet{gehr2016psi} developed a computer algebra system to perform exact symbolic probabilistic inference.


\subsection{Representation of Probabilistic Programs at Inference Time}

Lastly, we would like to point out a key feature of the IALW inference algorithm that sets it apart from any other inference scheme for probabilistic programming in the hybrid domain. But first, let us briefly talk about computing probabilities in probabilistic programming. Roughly speaking, probabilities are computed summing and multiplying weights. These can for example be represented as floating point numbers or symbolic expressions. The collection of all operations that were performed to obtain the probability of a query to a program is called the computation graph. Now, the big difference between IALW and other inference algorithms lies in the structure of the computation graph. IALW represents the computation graph as a directed acyclic graph (DAG), while all other languages, except some purely discrete languages~\citep{fierens2015inference,holtzen2020dice}, use a tree representation. IALW is the first inference algorithm in the discrete-continuous domain that uses DAGs (\ref{f:inference_kc}). In cases where the computation graph can be represented as a DAG the size of the representation might be exponentially smaller compared to tree representations, which leads to faster inference times.

Note that \citet{gutmann2010extending} and more recently~\citet{saad2021sppl} presented implementations of hybrid languages where the inference algorithm leverages directed acyclic graphs, as well. However, the constraints that may be imposed on random variables are limited to univariate equalities and inequalities. In the weighted model integration literature it was shown that such probability computations can be mapped to probability computations of discrete random variables only~\citep{zeng2019efficient}.


\subsection{Probabilistic Neurosymbolic AI}


\new{
As noted by \Citet{desmet2023neural} a shortcoming of many neurosymbolic AI systems~\citep{garcez2019neural,marra2024statistical}, \ie systems that combine the function approximation power of neural networks with logic reasoning, is their restriction to only allowing discrete random variables.
Based on the semantics of \dcplpsty, \citet{desmet2023neural} extended distributional facts to so-called neural-distributional facts. 
That is, they allowed for neural networks to estimate the parameters of the distribution in the distributional fact.
Importantly, they showed that endowing a neurosymbolic system in the discrete-continuous domain with proper probabilistic semantics is advantageous when comparing to systems that exhibit, for instance, a fuzzy logic semantics~\citep{badreddine2022logic}.
}


\section{Conclusions}\label{sec:conclusions}


We introduced \dcproblogsty, a hybrid PLP language for the discrete-continuous domain and its accompanying measure semantics.
\dcproblogsty strictly extends the discrete \problogsty language \citep{de2007problog,fierens2015inference} and the negation-free \dcsty~\citep{gutmann2011magic} language.
In designing the language and its semantics we adapted ~\citet{poole2010probabilistic}'s design principle of percolating probabilistic logic programs into two separate layers: the random variables and the logic program.
% , and applied it to both the discrete and the continuous random variables. 
 Boolean comparison atoms then form the  link between the two layers.
It is this clear separation between the random variables and the logic program that has allowed us to use simpler language constructs and to write programs using a  more concise and intuitive syntax than alternative hybrid PLP approaches \citep{gutmann2010extending,nitti2016probabilistic,speichert2019learning,azzolini2021semantics}.

Separating random variables from the logic program also allowed us to develop the IALW algorithm to perform inference in the hybrid domain. 
IALW is the first algorithm based on knowledge compilation and algebraic model counting for hybrid probabilistic programming languages
and as such it generalizes the standard knowledge compilation based approach for PLP.
It is noteworthy that IALW correctly computes conditional probabilities in the discrete-continuous domain using the newly introduced infinitesimal numbers semiring.


Interesting future research directions include adapting ideas from functional probabilistic programming (the other declarative programming style besides logic programming) in the context of probabilistic logic programming. For instance, extending \dcproblogsty with a type system~\citep{Schrijvers2008TowardsTP} or investigating more recent advances, such as {\em quasi-Borel spaces}~\citep{heunen2017convenient} in the context of the measure semantics.



% We presented \dcproblogsty, a PLP language for the discrete-continuous domain. \dcproblogsty strictly extends the discrete \problogsty language \citep{de2007problog,fierens2015inference} and the negation-free \dcsty~\citep{gutmann2011magic} language.
% In designing the language we followed the approach of percolating probabilistic logic programs into two layers 1) random variables 2) logic program. We adapted this paradigm from~\citet{poole2010probabilistic} and applied it to the discrete-continuous setting. Boolean comparison atoms form the link between the two layers.

% When compared to other PLP languages following Sato's distribution semantics \citep{gutmann2010extending,nitti2016probabilistic,speichert2019learning,azzolini2021semantics}, the clear separation between the random variables and the logic program has allowed us to eliminate bloated language constructs and to write programs using a concise and intuitive syntax.

% With IALW, we have also presented the first algorithm based on knowledge compilation and algebraic model counting for hybrid probabilistic programming languages. It is noteworthy that IALW correctly computes conditional probabilities in the discrete-continuous domain using the newly introduced infinitesimal numbers semiring.

% Besides providing an efficient implementation of the \dcproblogsty language together with the IAWL inference algorithm, 
% interesting future directions of research might include incorporating ideas from functional probabilistic programming (the second declarative programming style besides logic programming). For instance, extending \dcproblogsty with a type system~\citep{Schrijvers2008TowardsTP} or investigating more recent advances, such as {\em quasi-Borel spaces}~\citep{heunen2017convenient} in the context of the distribution semantics.

\section*{Acknowledgement}

This research received funding from the Wallenberg AI, Autonomous Systems and Software Program (WASP) of the Knut and Alice Wallenberg Foundation, the Flemish Government (AI Research Program),
the KU Leuven Research Fund, the European Research Council (ERC) under the European Union’s Horizon 2020 research and innovation programme (grant agreement No [694980] SYNTH: Synthesising Inductive Data Models), and
the Research Foundation - Flanders.


\bibliography{references}

\newpage
\clearpage

\appendix

% \setcounter{section}{0}
% \renewcommand{\thesection}{\Roman{section}}
% \renewcommand{\theequation}{\Roman{section}.\arabic{equation}}
% \renewcommand{\thetheorem}{\Roman{section}.\arabic{theorem}}
% \renewcommand{\thedefinition}{\Roman{section}.\arabic{definition}}



% \input{files_appendix1/lp}

% \section{Weighted Model Counting and Algebraic Model Counting} \label{app:amc}



Formal logic can be used to describe problems that arise in artificial intelligence. Here we will focus on propositional logic formulas, which we define as follows
\begin{definition}\label{def:preplogic}
	Let $\bvars$ be a set of $M$ Boolean variables.
	We then define propositional logic formulas as Boolean combinations (by means of the standard Boolean operators $\{\neg, \land, \lor, \rightarrow, \leftrightarrow \}$) of {\bf Boolean variables} $b_i \in \bvars$.
\end{definition}

A fundamental question to ask is whether there exists an assignment to the Boolean variables present in the logic formula that satisfies the formula.
% \begin{definition}[Satisfying interpretation of propositional logic formula]
% \label{def:sat_interpretation}
% 	Let ${\bf j}$ and ${\bf k}$ be two disjoint sets of variables and $\phi({\bf j}, {\bf k})$ be a propositional formula over ${\bf j}$ and ${\bf k}$.
% 	The set of total interpretations (or total assignments) that satisfy $\phi$ is the set of assignments to the elements in ${\bf j}$ and ${\bf k}$ that satisfy $\exists {\bf j}, ${\bf k}$:\phi(${\bf j}$,${\bf k}$)$.
% 	We denote the set of total satisfying interpretations (or models) by $\mathcal{I}_{{\bf j},{\bf k}}(\phi)$.
% 	The set of partial interpretations is denoted by $\mathcal{I}_{{\bf j}}(\phi)$, which is the set of assignments to ${\bf j}$ that satisfy $\exists {\bf k}:\phi({\bf j},{\bf k})$.
% 	The set of total assignments to a partially interpreted formula is denoted by $\mathcal{I}_{{\bf j}} (\phi^{\bf k})$, which denotes the set of assignments to the elements in ${\bf j}$ that satisfy $\phi({\bf j},{\bf k}_\inter)$, with ${\bf k}_\inter \in \mathcal{I}_{{\bf k}}(\phi)$. 
% \end{definition}
Determining whether a propositional logic formula is satisfiable is a computational hard problem and falls in the NP-complete complexity class. Nevertheless, a plethora of practical solvers exists (\eg MiniSAT~\citep{sorensson2005minisat}, CryptoSAT~\citep{lafitte2018cryptosat}) that tackle the Boolean satisfiability problem and perform astoundingly well in practice by exploiting structure present in all problems but the most intricate ones. 

We can also ask an other questions: how many distinct assignments satisfy a formula? For propositional logic formulas this simply amounts to counting the number of satisfying assignment to the Boolean variables --  a \#P-complete problem~\citep{valiant1979complexity}. The number of distinct satisfying assignments is dubbed the \textit{model count} and the task of computing the model count is also referred to as $\#SAT$.

Weighted model counting further generalizes \#SAT tasks. Instead of simply counting the number of satisfying assignments, one performs a weighted sum over models.
\begin{definition}[Weighted model counting (WMC)] \label{def:wmc} Given a set $\bvars$ of $M$ Boolean variables, a weight function $w: \mathbb{B}^M \rightarrow \mathbb{R}_{\geq 0}$, and a propositional formula $\support$ over $\bvars$, the {\bf weighted model count} is
	\begin{align}\label{eqn:wmc}
	 \wmc(\support,\weight {\mid} \bvars) = \sum_{\omega \in \mathcal{M}(\phi)} \weight(\omega)
	\end{align}
	$\mathcal{M}(\phi)$ is the set of models of $\support$, \ie the set of interpretations of the Boolean $\bvars$ that satisfy $\support$.
\end{definition}
Traditionally, WMC is used when the weight function $\weight$ factorizes as a product of weights of literals:
\begin{equation}
 \wmc(\support, \weight |\bvars) {=} \sum_{\omega \in \mathcal{M}(\phi)} \prod_{\ell \in \omega} \weight(\ell)
\end{equation}
When performing probabilistic inference, we take
\begin{equation}
0 {\leq} w(b_i){\leq} 1 \quad\text{and}\quad w(b_i){+}w(\neg b_i)=1
\end{equation}
The resulting sum over products is then a computation in the probability semiring~\citep{kimmig2017algebraic}. Effectively, this computes the probability that a propositional logic formula is satisfied.

\citet{kimmig2017algebraic} further generalize probabilistic inference over propositional formulas, \ie over satisfying models, to arbitrary commutative semirings. The general setting of model counting over commutative semirings is called algebraic model counting.
\begin{definition}\label{def:comm_semiring} 
	A {\bf  commutative semiring} is an algebraic structure $(\mathcal{A},\oplus,\otimes,\allowbreak e^{\oplus},e^\otimes)$ equipping a set of elements $\mathcal{A}$ with addition and multiplication such that
	\begin{enumerate}
		\item addition $\oplus$ and multiplication $\otimes$ are binary operations $\mathcal{A}\times \mathcal{A}\rightarrow \mathcal{A}$
		\item addition $\oplus$ and multiplication $\otimes$ are  associative and commutative binary operations over the set $\mathcal{A}$
		\item $\otimes$ distributes over $\oplus$
		\item  $e^\oplus \in \mathcal{A}$ is the neutral element of $\oplus$
		\item  $e^\otimes \in \mathcal{A}$ is the neutral element of $\otimes$
		\item $e^\oplus \in \mathcal{A}$ is an annihilator for $\otimes$
	\end{enumerate}
\end{definition}







\begin{definition}\label{def:amc} (Algebraic model counting (AMC)) \citep{kimmig2017algebraic} Given:
	\begin{itemize}
		\item a propositional logic theory $\phi$ over a set of variables $\bvars$
		\item a commutative semiring $(\mathcal{A},\oplus,\otimes,e^{\oplus},e^\otimes)$
		\item a labeling function $\alpha:\mathcal{L}\rightarrow \mathcal{A}$, mapping literals $\mathcal{L}$ from the variables in $B$ to values from the semiring set $\mathcal{A}$
	\end{itemize} 
	The algebraic model count of a theory $\phi$ is then defined as:
	\begin{align}
	\textstyle \amc(\phi,\alpha | \bvars) = \bigoplus_{b \in \mathcal{I}_\bvars(\phi)} \bigotimes_{b_i \in b} \alpha (b_i)&  \nonumber
	\end{align}
\end{definition}
We use $\alpha$ instead of $w$ and the term label rather than weight to reflect that the elements of the semiring cannot always be interpreted as weights.

Algebraic model counting constitutes a general framework for many common inference tasks in artificial intelligence. Defining an appropriate semiring and labeling functions allows one, for instance, to perform sensitivity analysis, compute gradients or determine the provenance of queries in databases~\citep{kimmig2017algebraic}.



















% 



\section{Knowledge Compilation and Algebraic Model Counting} \label{app:compilation}

As mentioned in the previous section, model counting is a computationally hard problem, \#P-hard to be precise. Nevertheless practically useful model counters exist. State-of-the-art techniques for solving model counting problems, are based on exhaustive DPLL algorithms~\citep{birnbaum1999good}, which count the number of satisfying assignments to a formula. These solvers can be divided into two classes: the ones that build up a trace of the DPLL search, and the ones that do not. The latter return immediately the model count. The former builds up a diagrammatic representation of the propositional formula over which the model count can be obtained efficiently. By keeping a trace, such \#SAT solvers~\citep{huang2005dpll,oztok2018exhaustive} constitute, in fact, top-down {\em knowledge compilation} schemes.

Knowledge compilation~\citep{darwiche2002knowledge} has emerged as the go-to technique for dealing with the computational intractability of propositional reasoning (\#P-hard~\citep{valiant1979complexity}). The key idea is to split up inference on logical formulas into an \textit{off-line} and an \textit{on-line} step. In the off-line step, a propositional formula is compiled from its source representation into a target representation, in which repeated on-line poly-time inference is available.

\#SAT can also be performed by compiling formulas bottom-up~\citep{choi2013dynamic}. However, it has been shown ~\citep{huang2005dpll,oztok2018exhaustive} that top-down compilation, \ie knowledge compilation through exhaustive DPLL search, outperforms bottom-up compilers.

Similar to model counting, weighted model counting has as well been performed via knowledge compilation, for instance for probabilistic inference in Bayesian networks~\citep{chavira2008probabilistic} or probabilistic programming~\citep{fierens2015inference}.


A popular and well-known target language for the knowledge compilation of propositional formulas are {\em deterministic decomposable negation normal
form} (d-DNNF) formulas~\citep{darwiche2001tractable}
and variations thereof~\citep{bryant1986graph,darwiche2011sdd}.
Boolean formulas in {\em negation normal form}~\citep{darwiche1999compiling} are nested conjunctions and disjunctions that allow negations only to be applied directly to the atoms in the formula.
Determinism and decomposability are two additional properties that we need to impose in order to guarantee tractable probabilistic inference using algebraic model counting.

\begin{definition}[Determinism] An NNF formula is deterministic if and only if for every disjuntion the disjuncts are pairwise logical inconsistent.
\end{definition}

\begin{definition}[Decomposability] An NNF formula is decomposable if and only if for every conjunction the conjuncts do not share any variables.
\end{definition}

\begin{definition}[Smoothness] 
An NNF is smooth if and only if for each disjunction the disjuncts mention the same variables.
\end{definition}

\begin{algorithm}[t]
    \SetKwFunction{EvalFn}{Eval}
    \SetKwProg{Fn}{function}{}{}
    \SetKwProg{ElseComment}{function}{}{}

	\caption{Evaluating a d-DNNF circuit $\Gamma$ for a
		commutative semiring $\mathcal{S}$ and labeling function $\alpha$~\citep{kimmig2017algebraic}.}
	\label{alg:eval}
\Fn{\EvalFn{$\Gamma$,$\mathcal{S}$,$\alpha$}}{
		\If{  $\Gamma$ is a literal node $l$}{ \Return $\alpha(l)$}
		\ElseIf{$\Gamma$ is a disjunction $\bigvee_{i=1}^m \Gamma_i$}
		{\Return $\bigoplus_{i=1}^m$ \EvalFn{($\Gamma_i$,$\mathcal{S}$,$\alpha$}}
		\Else( \tcp*[f]{$\Gamma$ is a conjunction $\bigwedge_{i=1}^m \Gamma_i$}){
		 \Return $\bigotimes_{i=1}^m$ \EvalFn{($\Gamma_i$,$\mathcal{S}$,$\alpha$}}
	}
\end{algorithm}


% 		{$N$ is a conjunction $\bigwedge_{i=1}^mN_i$}

We will also need the notions of {\em sum-neutrality} and {\em consistency preserving}.

%In order to compute algebraic/weighted model counts on a d-DNNF, we additionally need the neutral-sum property.	
\begin{definition}[Neutral-sum property~\citep{kimmig2017algebraic}]\label{lem:nsp} A semiring addition and labeling function pair $(\oplus,\alpha)$ is neutral \emph{if and only if} $\forall b \in \bvars:  \alpha(b)\oplus \alpha(\neg b)=e^\otimes$.
\end{definition}

\begin{definition}[Consistency-preserving property~\citep{kimmig2017algebraic}]\label{lem:con_prev} A semiring multiplication and labeling function pair $(\otimes,\alpha)$ is consistency-preserving \emph{if and only if} $\forall b \in \bvars:  \alpha(b)\otimes \alpha(\neg b)=e^\oplus$.
\end{definition}

\begin{theorem}[AMC on d-DNNF~\citep{kimmig2017algebraic}]
	\label{theo:amc_ddnnf}
Evaluating an sd-DNNF representation of the propositional theory $\phi$, using Algorithm~\ref{alg:eval}
%, for a semiring and labeling function with neutral tuple $(\oplus, \alpha)$
is a correct computation (cf. \citep[Definition 10]{kimmig2017algebraic}) of the algebraic model count.
\end{theorem}

Under the premise that the $\otimes$ and $\oplus$ operations in Algorithm~\ref{alg:eval} can be perform in constant time and that intermediate results are cached, the computation of an algebraic model count on an sd-DNNF is performed in poly-time~\citep{darwiche2002knowledge}. 





\setcounter{section}{0}
\renewcommand{\thesection}{\Alph{section}}
\renewcommand{\theequation}{\Alph{section}.\arabic{equation}}
\renewcommand{\thetheorem}{\Alph{section}.\arabic{theorem}}
\renewcommand{\thedefinition}{\Alph{section}.\arabic{definition}}

\section{Logic Programming}
\label{app:lp_new}



We briefly summarize key concepts of the syntax and semantics of logic programming; for a full introduction, we refer to \citep{lloyd2012foundations}.

\subsection{Building Blocks}
The basic building blocks of logic programs are \emph{variables} (denoted by strings starting with upper case letters), \emph{constants}, \emph{functors} and
\emph{predicates} (all denoted by strings starting with lower case letters).  A \emph{term} is a variable, a constant, or a functor
$f$ of \emph{arity} $n$ followed by $n$ terms $t_i$, \ie,
$f(t_1,\dots,t_n)$. 
An \emph{atom} is a predicate $p$ of arity $n$ followed by $n$ terms $t_i$, \ie,
$p(t_1,\dots,t_n)$. A predicate $p$ of arity $n$ is also written as \mathpredicate{p}{n}. A \emph{literal} is an
atom or a negated atom $ not(p(t_1,\dots,t_n))$.

\subsection{Logic Programs}

A \emph{definite clause} is a universally quantified expression of the form $h
\lpif b_1, \dots, b_n$ where $h$ and the $b_i$ are atoms.
$h$ is called the \emph{head} of the clause, and $b_1, \dots, b_n$ its
\emph{body}. Informally, the meaning of such a clause is that if all
the $b_i$ are true, $h$ has to be true as well. 
 A \emph{normal clause}  is a universally quantified expression of the form $h
\lpif l_1, \dots, l_n$ where $h$ is an atom and the $l_i$ are
literals.
If $n=0$, a clause is called \emph{fact} and simply written
as $h$.
A \emph{definite clause program} or \emph{logic program} for
short is a finite set of definite clauses. A \emph{normal logic
  program} is a finite set of normal clauses. 

\subsection{Substitutions}

A \emph{substitution} $\theta$ is an expression of the form
$\{V_1/t_1,\dots,V_m/t_m\}$ where the $V_i$ are different variables and
the $t_i$ are terms. Applying a substitution $\theta$ to an expression
$e$ (term or clause) yields the \emph{instantiated} expression $e\theta$
where all variables $V_i$ in $e$ have been simultaneously replaced by
their corresponding terms $t_i$ in $\theta$. If an expression does not
contain variables it is \emph{ground}. Two expressions $e_1$ and $e_2$ can be \emph{unified} if and only if there are substitutions $\theta_1$ and $\theta_2$ such that $e_1\theta_1 = e_2\theta_2$.

\subsection{Herbrand Universe}

The \emph{Herbrand universe} of a logic program is the set of ground terms that can be constructed using the functors and constants occurring in the program.
% \footnote{If the program does not contain constants, one arbitrary constant is added.} 
The \emph{Herbrand base} of a logic program is the set of ground atoms that can be constructed from the predicates in the program and the terms in its Herbrand universe. 
A truth value assignment to all atoms in the 
Herbrand base is called \emph{Herbrand interpretation}, and is also represented as the set of atoms that are true according to the assignment. 
A Herbrand interpretation is a \emph{model} of a clause $h \lpif b_1,\ldots ,b_n\ldotp$ if for every substitution $\theta$ such that all $b_i\theta$ are in the interpretation, $h\theta$ is in the interpretation as well. It is a model of a logic program if it is a model of all clauses in the program. The model-theoretic semantics of a definite clause program is given by its smallest Herbrand model with respect to set inclusion, the so-called \emph{least Herbrand model} (which is unique). We say that a logic program $\dcpprogram$ \emph{entails} an atom $a$, denoted $\dcpprogram\models a$, if and only if $a$ is true in the least Herbrand model of $\dcpprogram$.   




% Normal logic programs use the notion of \emph{negation as failure}, that is, for a ground atom $a$, $not(a)$ is true exactly if $a$ cannot be proven in the program. They are not guaranteed to have a unique minimal Herbrand model. Various ways to define the canonical model of such programs have been studied; see, \eg, \citep[Chapter 3]{lloyd2012foundations} for an overview. We use the well-founded semantics here~\citep{van1991well}.

% The main inference task of a logic programming system is to determine
% whether a given atom, also called \emph{query} (or \emph{goal}), is true in the canonical
%  model of a  program. If the answer is yes (or no), we also say that the query \emph{succeeds} (or \emph{fails}). If such a query is not ground, inference asks for the existence of  \emph{answer substitutions}, that is,  substitutions that ground the query into atoms that are part of the canonical model. 




\section{Table of Notations}\label{app:table}
\begin{center}
    \rowcolors{2}{gray!15}{white}

\begin{tabular}{c|p{0.5\textwidth}|c}
symbol & meaning & for details, see \\ \hline
 
  $\distributionfunctors$  & set of distribution functors  & Definition~\ref{def:reserved_vocabulary}\\
  $\arithmeticfunctors$   &  set of arithmetic functors & Definition~\ref{def:reserved_vocabulary}\\
  $\comparisonpredicates$   &  set of comparison  predicates & Definition~\ref{def:reserved_vocabulary}\\
  \hline
  $\samplespace_\nu$ & sample space of random variable $\nu$ & Definition~\ref{def:samplespace}\\
  $\samplefunction(\cdot)$ & value assignment function & Definition~\ref{def:samplespace}\\
  $\distdb$ & distributional database & Definition~\ref{def:distDB}\\
  $\randomvariableset$ & set of random variables & Definition~\ref{def:distDB}\\
  $(\samplespace_\distdb, \sigmaalgebra_\distdb, \probabilitymeasure_\distdb)$ & probability space induced by $\distdb$ & Definition~\ref{def:well-distdb}\\
  % $\measurerandomvariableset$ & probability measure over $\randomvariableset$ defined by $\distdb$ & Proposition~\ref{prop:pv}\\
  \hline
  $\comparisonfacts$ & set of Boolean comparison atoms & Definition~\ref{def:comparison-atoms-set}\\
  $\samplespace_\comparisonfacts$ & sample space induced by $\comparisonfacts$ & Proposition~\ref{prop:omegaf}\\
  $\sigmaalgebra_\comparisonfacts$ & sigma algebra induced by $\comparisonfacts$ & Proposition~\ref{prop:pfsigma}\\
  $\probabilitymeasure_\comparisonfacts$ & probability measure induced by $\comparisonfacts$ & Proposition~\ref{prop:pf}\\
  \hline
  $\dfprogram=\distdb  \cup \logicprogram$ & \dfplpsty program & Definition~\ref{def:core-prog}\\
  $\comparisonfacts_{\samplefunction(\randomvariableset)}$ & consistent comparison database induced by $\samplefunction$ on the random variables in $\randomvariableset$ & Definition~\ref{def:consistent-fact-db}\\
  $\probabilitymeasure_\dfprogram$ & probability measure over Herbrand interpretations defined by $\dfprogram$ & Proposition~\ref{prop:pp}\\
  \hline
  $\program$  & \dcproblogsty program & Definition~\ref{def:fullprog}\\
  \adfreeprogram & AD-free \dcproblogsty program & Definition~\ref{def:ad_free_program}\\
  $\headsdc_\adfreeprogram$ & set of heads of distributional clauses in \adfreeprogram & Definition~\ref{def:ad_free_program}\\
  $\randomtermset_\adfreeprogram$ & random terms in $\headsdc _\adfreeprogram$ & Definition~\ref{def:ad_free_program}\\
  $\dclauses_{\adfreeprogram}$ & set of distributional clauses in $\adfreeprogram$ & Definition~\ref{def:dc-df-well-def}\\
  $\contextfunc(\cdot)$ &  contextualization function & Definition~\ref{def:context_function}\\
  $\dfadfreeprogram$ &  \dfplpsty program providing the semantics of $\adfreeprogram$ & Definition~\ref{def:adfree-to-core}\\
  \hline
  $MOD(\dcpprogram)$ & models of a program $\dcpprogram$ & Theorem~\ref{theo:model_equivalence} \\
  $ENUM(\phi)$ & models of a propositional formula $\phi$ & Theorem~\ref{theo:model_equivalence} \\
  $\alpha(\cdot)$ & labeling function of a propositional literal & Definition~\ref{def:labeling_function} \\
  $\ive{\cdot}$ & Iverson bracket denoting an indicator function & Definition~\ref{def:labeling_function} \\
  $\E[\cdot]$ & expected value & Theorem~\ref{theo:label_equivalence} \\
  \hline
  $\mathbb{I}$ & set of infinitesimal numbers & Equation~\ref{def:inf_number} \\
  $\mathcal{S}$ & set of ancestral samples & Equation~\ref{eq:ancestral_samples}
\end{tabular}
\end{center}


\section{Proofs of Propositions in Section~\ref{sec:semantics}}
\label{app:semantics_proofs}

% \subsection{Proof of Proposition~\ref{prop:pv}}
% \label{app:proof:pv}

% \proppv*

% \begin{proof}
%     The proof is analogous to that for the semantics of well-defined Bayesian Logic Programs (BLPs)~\cite[Theorem 4.9]{kersting2000bayesian}. They show that such a probability measure exists over a non-empty set of random variables if the ancestor structure of the random variables is acyclic and every random variable has a finite set of ancestors, which are exactly  conditions W2 and W1 in Definition~\ref{def:well-defd-facts}. The key idea is that under these conditions, 
%     for each finite subset of random variables closed under the ancestor relation, the joint distribution on that set has the form of a Bayesian network, and factorizes into the product of the individual variables' distributions. 
%     This family of distributions forms the basis of the unique measure over the potentially infinite set $\randomvariableset$. We refer to \cite[Theorem 4.9]{kersting2000bayesian} for  technical details.  
% \end{proof}

% Note that while BLPs also use LP syntax to define the random variables and structure of a Bayesian network, the way they use that syntax is fundamentally different from ours.



\subsection{Proof of Proposition~\ref{prop:omegaf}}
\label{app:proof:omegaf}

\new{

\propomegaf*


\begin{proof}
    Consider the set of comparison atoms $\comparisonfacts$=   $\{ \kappa_1,  \kappa_2, \dots\}$. 
    Each $\kappa_i$ depends on a finite subset $\randomvariableset_i$ of random variables, namely  those mentioned in $\kappa_i$.
    We write $\randomvariableset_{\leq n}=\bigcup_{1\leq j\leq n} \randomvariableset_j $ for the union of random variables that the first $n$ atoms in the enumeration depend on.
    We obtain the set of all random variables from the following limit:
    \begin{align}
        \mathcal{V}_\comparisonfacts = \lim_{n\rightarrow \infty} \mathcal{V}_{\leq n}.
    \end{align}
    We construct the sample space of $\randomvariableset_{\comparisonfacts}$  with a (countable) Cartesian product
    \begin{align}
        \samplespace_{\comparisonfacts}=\prod_{\nu \in \mathcal{V}_{\comparisonfacts}} \samplespace_\nu.
    \end{align}
\end{proof}
}




\subsection{Proof of Proposition~\ref{prop:pfsigma}}
\label{app:proof:pfsigma}



\new{



\proppfsigma*





\begin{proof}
    We construct the following cylinder set for each comparison atom in the set $\comparisonfacts=\{ \kappa_1,  \kappa_2, \dots \}$:
    \begin{align}
        K_j &= \{  \omega \in \samplespace_{\comparisonfacts} \mid \kappa_j(\omega)=\top \}.
        % \\
        % K_k^\bot &= \{  \omega \in \samplespace_{\comparisonfacts} \mid \mu_k(\omega)=\bot \}.
    \end{align}
    Here we use $\kappa_j(\omega)$ to explicitly denote the evaluation of the comparison atom at $\omega$. We denote the set of all such cylinder sets by $\mathcal{K}_{\comparisonfacts}= \lim_{n\rightarrow \infty} \bigcup_{j=1}^n K_j$.
    
    Finally, we form the sigma-algebra $\Sigma_{\comparisonfacts}$ as the sigma-algebra generated by the collection of cylinder sets $\mathcal{K}_{\comparisonfacts}$:
    \begin{align}
        \Sigma_{\comparisonfacts} = \sigma(\mathcal{K}_{\comparisonfacts}).
    \end{align}
    where $\sigma(\mathcal{K}_{\comparisonfacts})$ denotes the intersection of all sigma-algebras containing $\mathcal{K}_{\comparisonfacts}$.
    % Given that $\mathcal{U}_{\leq n}$ is closed under countable unions, intersection, and complements, $\Sigma_{\comparisonfacts}$ is also the smallest sigma-algebra.
    Given that $\mathcal{K}_{\comparisonfacts} \subseteq \sigmaalgebra_\distdb$ we also have that $\sigmaalgebra_\comparisonfacts \subseteq \sigmaalgebra_\distdb$.
    % This concludes the proof as we provide an explicit construction of $ \Sigma_{\comparisonfacts}$.
\end{proof}







    % If we now associate to each element in $\mathcal{K}_{\leq n}$ an element from the index set $I_{\leq n}= \{1, \dots, 2n  \}$, we can construct a larger collection of cylinder sets using the powerset of $I_{\leq n}$ (denoted by $\mathcal{I}_{\leq n}$):
    % \begin{align}
    %     \mathcal{U}_{\leq n }
    %     &=
    %     \bigcup_{J \in \mathcal{I}_{\leq n}  }
    %     \bigcap_{j \in J} 
    %     \bigg\{ \omega \in \samplespace_{\leq n} \bigg| \omega \in \mathcal{K}_{\leq n,j}  \bigg\}
    %     \\
    %     &=
    %     \bigcup_{J \in \mathcal{I}_{\leq n}  }
    %     \bigcap_{j \in J} 
    %     \bigg\{  \omega \in \mathcal{K}_{\leq n,j}  \bigg\},
    % \end{align}
    % where $\mathcal{K}_{\leq n,j}$ denotes an element from $\mathcal{K}_{\leq n}$.
    % Note that the intersection in the equation above denotes a set intersection and the union denotes a collection union. Moreover, as a finite intersection of cylinder sets is again a cylinder set we can rewrite the equation above as:



    % \begin{align}
    %     \mathcal{U}_{\leq n }
    %     =
    %     \bigcup_{J \in \mathcal{I}_{\leq n}  }
    %     \bigg\{ \omega \in  \bigwedge_{j \in J} \mathcal{K}_{\leq n, j}  \bigg\}, 
    % \end{align}
    % where $\bigwedge_{j \in J} \mathcal{K}_{\leq n, j}$ is a cylinder set for every $J \in \mathcal{I}_{\leq n}$.

    % It is now straightforward to show that the $\mathcal{U}_{\leq n }$ is closed under countable union, intersection and complements. This can easily be seen by realizing that our construction of $\mathcal{U}_{\leq n}$ makes sure that all the $2^n$ cylinder sets  $\{\omega \in \samplespace \mid \mu_1(\omega)=b_1, \dots , \mu_n(\omega)=b_n \}$ with $b_i\in \{\bot, \top\}$ are included in the collection, as well as arbitrary subsets thereof, \ie cylinder sets with arbitrarily ordered comparison atoms $\mu_i$ with a number of comparison atoms less or equal to $n$.




}


\subsection{Proof of Proposition~\ref{prop:pf}}
\label{app:proof:pf}


\new{

\proppf*

\begin{proof}
    To show existence of the  measure $\measurecomparisonfacts$, we need to show that 
    \begin{enumerate}
        \item non-negativity: $\probabilitymeasure_\comparisonfacts(A)\geq0 , \quad \forall A \in \sigmaalgebra_\comparisonfacts$
        \item normality: $\probabilitymeasure_\comparisonfacts(\samplespace_\comparisonfacts)=1$
        \item countably additivity: for any collection $\{A_i\}_{i=1}^\infty$ of disjoint sets in $\sigmaalgebra_\comparisonfacts$ we have
        \begin{align}
            \probabilitymeasure_\comparisonfacts
            \left(
                \bigcup_{i=1}^\infty A_i
            \right)
            =
            \sum_{i=1}^\infty \probabilitymeasure_\comparisonfacts(A_i)
        \end{align} 
    \end{enumerate}
    Using the fact that $\sigmaalgebra_\comparisonfacts\subseteq \sigmaalgebra_\distdb$ it is straightforward to show these three properties hold. Uniqueness of $\probabilitymeasure_\comparisonfacts$ is also inherited from the uniqueness of $\probabilitymeasure_\distdb$.
\end{proof}


}


% we fix an arbitrary enumeration  $\mu_1$, $\mu_2$, \dots, of the atoms in $\comparisonfacts$. 
% Each $\mu_i$ \emph{depending} on a subset $\randomvariableset_i\subseteq \randomvariableset$ of random variables, namely  those mentioned in $\mu_i$, as well as their ancestor sets. We write $\randomvariableset_{\leq n}=\bigcup_{1\leq j\leq n} \randomvariableset_j $ for the union of random variables that the first $n$ atoms in the enumeration depend on. By $\probabilitymeasure_{\randomvariableset_{\leq n}}$ we denote the measure restricted to this set.  

% By definition, all queries $\mu_i\in \comparisonfacts$ are  Lebesgue-measurable, and we thus get a family of distributions
% \begin{align}
% \measurecomparisonfacts^{(n)}(\mu_1=\boolval_1,\ldots,\mu_n=\boolval_n) = \int_{\samplespace(\randomvariableset_{\leq n})} \mathbf{1}_{[\mu_1=\boolval_1\wedge\ldots\wedge \mu_n=\boolval_n]}(\samplefunction(\randomvariableset_{\leq n})) \differential{\probabilitymeasure_{\randomvariableset_{\leq n}}}
% \end{align}
% where the $\boolval_i$ belong to the set  $\{ \bot, \top\}$, $\probabilitymeasure_{\randomvariableset_{\leq n}}$ factorizes over the random variables in $\randomvariableset_{\leq n}$, $\samplespace(\randomvariableset_{\leq n})$ denotes the space of possible assignments for variables in $\randomvariableset_{\leq n}$, and $\mathbf{1}_{[\varphi]}$ is the indicator function, \ie, equals $1$ if $\varphi$ is true and $0$ otherwise. The definition in terms of an indicator function and the measurability of the underlying Boolean queries ensures that this family of distributions is of the form required for the distribution semantics, \ie they are well-defined probability distributions satisfying the compatibility condition: $\measurecomparisonfacts^{(n)}$ can be obtained from $\measurecomparisonfacts^{(n+1)}$ by summing out $\mu_{n+1}$. There thus exists a completely additive probability measure $\measurecomparisonfacts$ over the space of truth value assignments to $\comparisonfacts$ such that for any $n$, we have
% \begin{align}
% \measurecomparisonfacts(\mu_1=\boolval_1,\ldots,\mu_n=\boolval_n) =\measurecomparisonfacts^{(n)}(\mu_1=\boolval_1,\ldots,\mu_n=\boolval_n)
% \end{align}





% \begin{proof}
%     We fix an arbitrary enumeration  $\langle \mu_1, \mu_2, \dots \rangle$, of the atoms in $\comparisonfacts$.
%     Each $\mu_i$ \emph{depending} on a subset $\randomvariableset_i\subseteq \randomvariableset$ of random variables, namely  those mentioned in $\mu_i$, as well as their ancestor sets. We write $\randomvariableset_{\leq n}=\bigcup_{1\leq j\leq n} \randomvariableset_j $ for the union of random variables that the first $n$ atoms in the enumeration depend on.
%     Furthermore, let $I_{\leq n}$ be the finite index set for the $n$ first atoms,  and $\mathcal{I}_{\leq n}$ its powerset.  


%     For an element $K\in \mathcal{I}_{\leq n}$ we define the following cylinder set:
%     \begin{align}
%         Cyl(K) = \{ \omega \in \samplespace_\randomvariableset \mid \forall k \in K: \mu_k(\omega)=\top \},  \quad K\in \mathcal{I}_{\leq n}  
%     \end{align}
%     where $\mu_k(\omega)$ evaluates the $k$-th comparison atom for the variable assignment $\omega$. We denote the collection of all such cylinder sets as follows:
%     \begin{align}
%         \mathcal{U}_{\leq n} = \bigcup_{K\in \mathcal{I}_{\leq n}} Cyl(K)
%     \end{align}
    

    
    

    

        
% \end{proof}






% \begin{proof}
%     We fix an arbitrary enumeration  $\langle \mu_1, \mu_2, \dots \rangle$, of the atoms in $\comparisonfacts$.
%     Each $\mu_i$ \emph{depending} on a subset $\randomvariableset_i\subseteq \randomvariableset$ of random variables, namely  those mentioned in $\mu_i$, as well as their ancestor sets. We write $\randomvariableset_{\leq n}=\bigcup_{1\leq j\leq n} \randomvariableset_j $ for the union of random variables that the first $n$ atoms in the enumeration depend on.
%     Furthermore, let $I_n$ be the finite index set for the $n$ first atoms,  and $\mathcal{I}_n$ its powerset.  

%     Let $I$ be an index set and $J\subseteq I$ be a finite subset. For each $j\in J$, let $A_j \subseteq \samplespace_j$. A cylinder set in $\samplespace$ is a set of the form:

%     For an element $i\in \mathcal{I}_n$ we define the following cylinder set:
%     \begin{align}
%         Cyl(i) = \{ \omega \in \samplespace_\randomvariableset \mid \in J: \mu_j(\omega)=\top \}    
%     \end{align}

    

%     we fix an arbitrary enumeration  $\langle \mu_1, \mu_2, \dots \rangle$, of the atoms in $\comparisonfacts$. 
%     Each $\mu_i$ \emph{depending} on a subset $\randomvariableset_i\subseteq \randomvariableset$ of random variables, namely  those mentioned in $\mu_i$, as well as their ancestor sets. We write $\randomvariableset_{\leq n}=\bigcup_{1\leq j\leq n} \randomvariableset_j $ for the union of random variables that the first $n$ atoms in the enumeration depend on.
    
%     We can now write the set of assignments to the random variables in $\randomvariableset_{\leq n}$ that respect the first $n$ comparison facts as:
%     \begin{align}
%         Cyl(\randomvariableset_{\leq n})
%         =
%         \{\omega \in \samplespace(\randomvariableset_{\leq n}) \mid \mu_1(\omega)=\top  \land \dots \land \mu_n(\omega) = \top  \}
%     \end{align}
    
    
%     We then denote by
%     \begin{align}
%         \boolval^{w_n} = \langle \boolval^{w_n}_1, \dots, \boolval^{w_n}_n \rangle \in \{\bot,\top\}^n, \quad 1\leq {w_n} \leq 2^n
%     \end{align}
%     the enumeration of Boolean values the $\mu_i$ take. We can now write the set of assignments to the random variables in $\randomvariableset_{\leq n}$ that respect the 

%     \begin{align}
%         Cyl(\randomvariableset_{\leq n}, \boolval^{w_n})
%         =
%         \{\omega \in \samplespace(\randomvariableset_{\leq n}) \mid \mu_1(\omega)=\boolval^{w_n}_1  \land \dots \land \mu_n(\omega) = \boolval^{w_n}_n  \}
%     \end{align}
    
%     The collection of all cylinder sets is:
%     \begin{align}
%         \mathcal{U}_\mathcal{V} =  \bigcup_{j=0}^n \bigcup_{w_j=1}^{2^j} Cyl(\randomvariableset_{\leq j}, \boolval^{w_j})
%     \end{align}
%     Here we also include the case $j=0$ in order to include the empty set. 
    
%     \begin{align}
%         \sigmaalgebracomparisonfact
%         =
%         \sigma (\mathcal{U}_\mathcal{V})
%     \end{align}
    
    

        
% \end{proof}







\subsection{Proof of Proposition~\ref{prop:pp}}
\label{app:proof:pp}

\proppp*


\new{
\begin{proof}
% To show this, we consider two cases. 
% If $\distdb$ is empty, \ie \dfprogram does not define any random variables,  the semantics of \dfprogram is the well-founded model of $\logicprogram$. Thus, normal logic programs (with total well-founded models) are a special case of \dfplpsty.

To show this, we follow Sato's construction to obtain the probability measure $\probabilitymeasure_\dfprogram$ over Herbrand interpretations from $\measurecomparisonfacts$. 
To this end we denote the set of atoms in the Herbrand base by $\mu_1, \mu_2, \dots$, which also includes those in $\comparisonfacts$.
As \dfprogram is valid, for every consistent comparison database $\comparisonfacts_{\samplefunction(\randomvariableset)}$ (\cf Definition~\ref{def:consistent-fact-db}), the logic program
$\comparisonfacts_{\samplefunction(\randomvariableset)}\cup \logicprogram$
has a total well-founded model $M_{\samplefunction(\randomvariableset)}$, and we can define   
\begin{align}
\probabilitymeasure_\dfprogram(\mu_1=\boolval_1,\mu_2=\boolval_2, \ldots)
:=
\measurecomparisonfacts
    \left(
    \left\{
    \samplefunction(\randomvariableset)
        ~|~
        M_{\samplefunction(\randomvariableset)}
    \right\}
    \right)
\end{align}
% It follows again that there is a completely additive probability measure $\probabilitymeasure_\dfprogram$ over Herbrand interpretations.
What remains, is to show that the set
$
    \left\{
    \samplefunction(\randomvariableset)
        ~|~
        M_{\samplefunction(\randomvariableset)}
    \right\}
$
is an element of the sigma-algebra $\sigmaalgebra_\comparisonfacts$. To this end, we rewrite the set as:

\begin{align}
    \Bigl\{
    \samplefunction(\randomvariableset)
        ~|~
        \mu_1(\samplefunction(\randomvariableset))=\boolval_1\land \mu_2(\samplefunction(\randomvariableset))=\boolval_2\land \ldots
    \Bigr\}
    \label{eq:proof:prop:semantics}
\end{align}
where we have that, 
\begin{align}
    M_{\samplefunction(\randomvariableset)} \models
    \mu_1(\samplefunction(\randomvariableset))=\boolval_1\land \mu_2(\samplefunction(\randomvariableset))=\boolval_2\land \ldots
\end{align}
We rewrite the set in Equation~\ref{eq:proof:prop:semantics} as


\begin{align}
    \bigcap_{j=1}^n
    \left\{
    \samplefunction(\randomvariableset)
        ~|~
        \mu_j(\samplefunction(\randomvariableset))={\boolval_j}       
    \right\}.
\end{align}
We now retain only those $ \mu_j$'s that depend on (a subset of) $\randomvariableset$, which we denote by $ \kappa_j$:
\begin{align}
    \bigcap_{j=1}
    \left\{
        \samplefunction(\randomvariableset)
            ~|~
            \kappa_j(\samplefunction(\randomvariableset))={\boolval_j}      
        \right\}.
\end{align}
The last line is an intersection of elements from $\comparisonfacts$ (or their complements) and thereby trivially part of $\sigmaalgebra_\comparisonfacts$, which concludes the proof.
\end{proof}

}


\section{Detailed Discussion on \dcproblogsty}
\label{sec:detaileddcproblog}




\subsection{Syntactic Sugar Semantics}
\label{sec:semantics_syntactic_sugar}

We now formalize the declarative semantics of \dcproblogsty, \ie \dfplpsty extended with probabilistic facts, annotated disjunctions and distributional clauses,
The idea is to define program transformations that eliminate these three modelling constructs from a \dcproblogsty program, resulting in a \dfplpsty program for which we have defined the semantics in Section~\ref{sec:semantics}.

Throughout this section, we will treat distributional facts as distributional clauses with empty bodies, and we will only consider ground programs for ease of notation. As usual, a non-ground program is shorthand for its Herbrand grounding.


\begin{definition}[Statement]
    A \emph{\dcproblogsty statement} is either a probabilistic fact, an annotated disjunction, a distributional clause, or a normal clause.
\end{definition}

\begin{definition}[\dcproblogsty program] \label{def:fullprog}
    A \dcproblogsty program $\program$ is a countable set of ground \dcproblogsty statements.
\end{definition}






\subsubsection{Eliminating Probabilistic Facts and Annotated Disjunctions}




\begin{example}\label{ex:running-sugar-full}
	We use the following \dcproblogsty program as running example.
	\begin{problog*}{linenos}
p ~ beta(1,1).
p::a.
b ~ normal(3,1) :- a.
b ~ normal(10,1) :- not a.
c ~ normal(b,5).
0.2::d; 0.5::e; 0.3::f :- not b<5, b < 10.
g :- a, not f, b+c<15.
	\end{problog*}
\end{example}



\begin{definition}[Eliminaton Rules for Probabilistic Facts and ADs]\label{def:elim-ad}
	Let $\program$ be  a \dcproblogsty program. We define the following elimination rules (ER) to eliminate probabilistic facts and annotated disjunctions.
	\begin{itemize}
		\item[ER1:] replace each probabilistic fact $p\prob \mu$ in $\program$ by
		      \begin{align*}
			       & \nu \sim flip(p).  \\
			       & \mu \lpif \nu=:=1.
		      \end{align*}
		      with a fresh random variable $\nu$ for each probabilstic fact.
		\item[ER2:] replace each AD $p_1\prob\mu_1;\dots;p_n \prob \mu_n \lpif \beta$ in $\program$ by
		      \begin{align*}
			       & \nu \sim finite([p_1 : 1, \dots, p_n : n]) \\
			       & \mu_1 \lpif \nu=:=1, \beta.                \\
			       & \dots                                      \\
			       & \mu_n \lpif \nu=:=n, \beta.
		      \end{align*}
		      with a fresh random variable $\nu$ for each AD.
	\end{itemize}
\end{definition}
Note that if the probability label(s) of a fact or AD include random terms, as in the case of \probloginline{p::a} in the Example~\ref{ex:running-sugar-full},  then these are  parents of the newly introduced random variable. However, the new random variable will not be a parent of other random variables, as they are only used locally within the new fragments. They thus introduce neither cycles nor infinite ancestor sets into the program.


\begin{definition}[AD-Free Program] \label{def:ad_free_program}
	An \emph{AD-free} \dcproblogsty program \adfreeprogram is a \dcproblogsty program that contains neither probabilistic facts nor annotated disjunctions. We denote by $\headsdc_\adfreeprogram$ the set of atoms $\tau \sim \delta$ that appear as head of a distributional clause in $\adfreeprogram$, and by $\randomtermset_\adfreeprogram$ the set of random terms in $\headsdc_\adfreeprogram$.
\end{definition}


\begin{example}\label{ex:running-sugar-adfree}
	Applying Definition~\ref{def:elim-ad} to Example~\ref{ex:running-sugar-full} results in
	\begin{problog*}{linenos}
p ~ beta(1,1).
x ~ flip(p).
a :- x =:= 1.
b ~ normal(3,1) :- a.
b ~ normal(10,1) :- not a.
c ~ normal(b,5).
y ~ finite([0.2:1,0.5:2,0.3:3]).
d :- y =:= 1, not b<5, b < 10.
e :- y =:= 2, not b<5, b < 10.
f :- y =:= 3, not b<5, b < 10.
g :- a, not f, b+c<15.
	\end{problog*}
	We have $\headsdc_\adfreeprogram=\{ \, \text{\probloginline{p~beta(1,1)}},\,\allowbreak \text{\probloginline{x~flip(p)}},\,\allowbreak  \text{\probloginline{b~normal(3,1)}},\,\allowbreak \text{\probloginline{b~normal(10,1)}},\allowbreak \text{\probloginline{c~normal(b,5)}},\,\allowbreak \text{\probloginline{y~finite[0.2:1,0.5:2.0.3:3])}} \, \}$. Furthermore, we also have $\randomtermset_\adfreeprogram = \{ \mathprobloginline{p}, \mathprobloginline{x}, \mathprobloginline{b}, \mathprobloginline{c}, \mathprobloginline{y} \}$.
\end{example}




\subsubsection{Eliminating Distributional Clauses}
\label{sec:eliminate_dc}

While eliminating probabilistic facts and annotated disjunctions is a rather straightforward local transformation, eliminating distributional clauses is more involved. The reason is that a distributional clause has a global effect in the program, as it defines a condition under which a random term has to be \emph{interpreted} as a specific random variable when mentioned in a distributional clause or comparison atom. Therefore, eliminating a distributional clause involves both introducing the relevant random variable explicitly to the program and pushing the condition from the body of the distributional clause to all the places in the logic program that interpret the original random term.

Before delving into the mapping from an AD-free \dcproblogsty to a \dfplpsty program, we introduce some relevant terminology.

\begin{definition}[Parents and Ancestors for Random Terms]
	\label{def:parentancestor2}
	Given an AD-free program \adfreeprogram with $\tau_p$ and $\tau_c$ in $\randomtermset_\adfreeprogram$. We call $\tau_p$   a \emph{parent} of  $\tau_c$ if and only if  $\tau_p$ appears in the distribution term $\delta_c$ associated with $\tau_c$  in $\headsdc_\adfreeprogram$ ($\tau_c\sim \delta_c \in \headsdc_\adfreeprogram$).
	We define \emph{ancestor} to be the transitive closure of \emph{parent}.
\end{definition}



\begin{figure}[h]
	\centering
	\tikzstyle{node}=[circle, text centered, draw,thick]
	\begin{tikzpicture}[remember picture]


		\node[node] (p) {\probloginline{p}};
		\node[node] (x) [below=of p] {\probloginline{x}};

		\node[node] (b) [right=of p] {\probloginline{b}};
		\node[node] (c) [below=of b] {\probloginline{c}};

		\node[node] (y) [right=of b] {\probloginline{y}};


		\draw[->,thick] (p) to  (x);
		\draw[->,thick] (b) to  (c);


	\end{tikzpicture}
	\caption{Directed acyclic graph representing the ancestor relationship between the random variables in Example~\ref{def:ad_free_program}.
		The random terms \probloginline{p}, \probloginline{b} and \probloginline{y} have the empty set as their ancestor set. The ancestor set of \probloginline{x} is $\{ \mathprobloginline{p} \}$ and \probloginline{c} is $\{ \mathprobloginline{b} \}$.}
\end{figure}






For random terms, we distinguish  \emph{interpreted occurrences} of the term that need to be resolved to the correct random variable from other occurrences where the  random term is treated as any other term in a logic program, \eg, as an argument of a logical atom.
\begin{definition}[Interpreted Occurrence]
	\label{def:interpreted_occ}
	An \emph{interpreted occurrence} of a random term $\tau$ in an AD-free program \adfreeprogram is one of the following:
	\begin{itemize}
		\item the use of $\tau$ as parameter of a distribution term in the head of a distributional clause in \adfreeprogram
		\item the use of $\tau$  in a comparison literal in the body of a (distributional or normal) clause in \adfreeprogram
	\end{itemize}
	We say that a clause \emph{interprets} $\tau$ if there is at least one interpreted occurrence of $\tau$ in the clause.
\end{definition}

\begin{definition}[Well-Defined AD-free Program]\label{def:dc-df-well-def}
	Given an AD-free program \adfreeprogram with $\dclauses_\adfreeprogram$ the set of distributional clauses in \adfreeprogram, we call $\dclauses_\adfreeprogram$ \emph{well-defined} if the following conditions hold:
	\begin{description}
		\item[DC1] For each random term $\tau \in \randomtermset_\adfreeprogram$,  the number of  distributional clauses $\tau \sim \delta \lpif \beta$ in $\adfreeprogram$ is finite, and these clauses all have mutually exclusive bodies. This means that only a single rule can be true at once.
		\item[DC2] All distribution terms in $\dclauses_\adfreeprogram$ are well-defined for all possible values of the random terms they interpret.
		\item[DC3] Each random term has a finite set of ancestors.
		\item[DC4] The ancestor relation is acyclic.
	\end{description}
\end{definition}



% Note that the arguments of a predicate \mathfunctor{\mu}{k} can contain elements of $\distdb$ and $\comparisonfacts$. In this case they will have a purely logical meaning. We discuss this in more detail in Definition~\ref{def:interpreted_occ}. 









We now discuss how to reduce a (valid) \dcproblogsty program to a \dfplpsty program. This happens in two steps. First, we eliminate distributional clauses and introduce appropriate distributional facts instead (see Definition~\ref{def:elim-dc}). Second, we {\em contextualize} interpreted occurrences of random terms in clause bodies (see Definition~\ref{def:core_encoding}).

The first step introduces a new built-in predicate \predicate{rv}{2} that associates random terms in a well-defined AD-free program with explicit random variables in the \dfplpsty program it is transformed into. This predicate is used in the bodies of clauses that interpret random terms (\cf Definition~\ref{def:interpreted_occ}) to appropriately contextualize those.

The idea behind the built-in \predicate{rv}{2} predicate is to restrict the applicability of a clause to contexts where all the random terms can be interpreted, \ie to contexts where the random terms are random variables. This implies that in contexts where such a random term cannot be interpreted, the \emph{entire} body evaluates to false.


\begin{definition}[Eliminating Distributional Clauses]\label{def:elim-dc}
	Let $\dclauses_{\adfreeprogram}$ be a well-defined set of distributional clauses.
	We denote by $\delta_{\rho_1, \dots, \rho_k}$ a distribution term that involves exactly $k$ different random terms $\rho_1,\ldots,\rho_k$.
	For each ground random term $\tau\in \randomtermset_{\adfreeprogram}$ we simultaneously define the following sets:
	\begin{itemize}
		\item the set of distributional facts for $\tau$
		      \begin{align*}
			      \distdb(\tau) =
			      \{
			      \tau^\beta_{\nu_1,\ldots,\nu_k}
			       & \sim \delta^\beta_{\nu_1,\ldots,\nu_k}
			      \\
			       & \mid
			      (\tau \sim \delta_{\rho_1,\ldots,\rho_k} \lpif \beta \in \dclauses_\adfreeprogram
			      ,
			      v_1\in \randomvariableset(\rho_1), \ldots, v_k\in \randomvariableset(\rho_k)
			      \}
		      \end{align*}
		\item    the set of (fresh) random variables for $\tau$
		      \begin{equation*}
			      \randomvariableset(\tau) = \{\nu \mid \nu\sim \delta \in \distdb(\tau)\}
		      \end{equation*}
		\item the set of context clauses for $\tau$
		      \begin{align*}
			       & \logicprogram^{c}(\tau)
			      =
			      \\
			       &
			      \quad \biggl\{\text{\probloginline{rv}}(\tau,\tau^\beta_{\nu_1,\ldots,\nu_k})
			      \lpif
			      \text{\probloginline{rv}}(\rho_1,\nu_1),\ldots,\text{\probloginline{rv}}(\rho_k,\nu_k), \beta
			      \\
			       &
			      \quad \bigm|
			      \tau \sim \delta_{\rho_1,\ldots,\rho_k} \lpif \beta \in \dclauses_{\adfreeprogram}, \nu_1\in \randomvariableset(\rho_1), \ldots, \nu_k\in \randomvariableset(\rho_k)
			      \biggr\}
		      \end{align*}
	\end{itemize}
\end{definition}

At first glance, Definition~\ref{def:elim-dc} seems to contain a mutual recursion involving $\distdb(\cdot)$ and $\randomvariableset(\cdot)$.
However, if we recall that for a well-defined set of distributional clauses $\dclauses_\adfreeprogram$ the ancestor relationship between random terms constitutes an acyclic directed graph, the apparent mutual recursion evaporates. We can now define the distributional facts encoding of the distributional clauses, which will give rise to a \dfplpsty program instead of \dcproblogsty program.

\begin{definition}[Distributional Facts Encoding]
	\label{def:core_encoding}
	Let $\adfreeprogram$ be an AD-free \dcproblogsty program and $\dclauses_{\adfreeprogram}$ its set of distributional clauses. We define the distributional facts encoding of $\dclauses_{\adfreeprogram}$  as
	$\dclauses_{\adfreeprogram}^{DF}\coloneqq \distdb\cup \logicprogram^{c}$, with
	\begin{align*}
		\distdb = \bigcup_{\tau\in \randomtermset_{\adfreeprogram}}\distdb(\tau)
		 &  &
		\logicprogram^{c} = \bigcup_{\tau\in \randomtermset_{\adfreeprogram}} \logicprogram^{c}(\tau)
	\end{align*}
	using $\distdb(\cdot)$ and $\logicprogram^{c}(\cdot)$ from Definition~\ref{def:elim-dc}.
\end{definition}



\begin{example}[Eliminating Distributional Clauses]
	\label{ex:running-sugar-core-encoding}
	We demonstrate the elimination of distributional clauses using the DCs in Example~\ref{ex:running-sugar-adfree}, \ie
	\begin{problog*}{linenos}
p ~ beta(1,1).
x ~ flip(p).
b ~ normal(3,1) :- a.
b ~ normal(10,1) :- not a.
c ~ normal(b,5).
y ~ finite([0.2:1,0.5:2,0.3:3]).
	\end{problog*}
	Here,  the distribution terms in Line 2 and Line 5 (\probloginline{flip(p)} and \probloginline{normal(b,5)}) contain one parent random term each (\probloginline{p} and \probloginline{b}, respectively), whereas all others have no parents. As \probloginline{b} is defined by two clauses, we get fresh random variables for each of them, which in turn  introduces different fresh random variables for the child \probloginline{c}.
	This gives us:
	\begin{problog*}{linenos}
v1 ~ beta(1,1).
rv(p,v1).
v2 ~ flip(v1).
rv(x,v2) :- rv(p,v1).
v3 ~ normal(3,1).
rv(b,v3) :- a.
v4 ~ normal(10,1).
rv(b,v4) :- not a.
v5 ~ normal(v3,5).
rv(c,v5) :- rv(b,v3).
v6 ~ normal(v4,5).
rv(c,v6) :- rv(b,v4).
v7 ~ finite([0.2:1,0.5:2,0.3:3]).
rv(y,v7).
	\end{problog*}
\end{example}







Eliminating distributional clauses (following Definition~\ref{def:elim-dc}) introduces the distributional facts and context rules necessary to encode the original distributional clauses. To complete the transformation to a \dfplpsty program, we further transform the logical rules. Prior to that, however, we need to define the {\em contextualization function}.

\begin{definition}[Contextualization Function] \label{def:context_function}
	Let $\beta$ be a conjunction of atoms and let its comparison literals interpret the random terms $\tau_1,\ldots,\tau_n$.
	Furthermore, let $\Lambda_i$ be a special logical variable associated to a random term $\tau_i \in \randomtermset_{\adfreeprogram}$ for each $\tau_i$.
	We define $\contextfunc(\beta)$ to be the  conjunction of literals obtained by replacing the interpreted occurrences of the $\tau_i$ in $\beta$ by their corresponding $\Lambda_i$
	and conjoining to this modified conjunction $\text{\probloginline{rv}}(\tau_i,\Lambda_i)$ for each $\tau_i$.
	We call $\contextfunc(\cdot)$ the contextualization function.
\end{definition}

\begin{definition}[Contextualized Rules]\label{def:adfree-to-core}
	Let \adfreeprogram be an AD-free program with logical rules $\logicprogram^\adfreeprogram$ and distributional clauses $\dclauses_\adfreeprogram$,
	and let $\dclauses_\adfreeprogram^{DF}=\distdb\cup \logicprogram^{c}$ be the distributional facts encoding of $\dclauses_\adfreeprogram$. We define the contextualization of the bodies of the rules $\logicprogram^\adfreeprogram \cup \logicprogram^{DF}$ as the sequential application of the following contextualization rules:
	\begin{enumerate}[label=\alph*.]
		\item[CR1:] apply the contextualization function $\contextfunc$ to all bodies in $\logicprogram^\adfreeprogram\cup \logicprogram^{c}$ and obtain:
		      \begin{align*}
			      \logicprogram^\Lambda = \{\eta \lpif \contextfunc(\beta) \mid \eta \lpif \beta \in \logicprogram^\adfreeprogram \cup \logicprogram^{c} \}
		      \end{align*}
		\item[CR2:] obtain the set of ground logical rules $\logicprogram$ by grounding each logical variable $\Lambda_i$ in $\logicprogram^\Lambda$  with random variables $\nu_i\in \randomvariableset(\tau_i)$ in all possible ways.
	\end{enumerate}
	We call $\logicprogram$ the contextualized logic program of $\adfreeprogram$, \fixed{Furthermore, we define $\dfadfreeprogram\coloneqq\distdb \cup \logicprogram $ to be the corresponding \dcplpsty program.}
\end{definition}

The contextualization function $\contextfunc(\cdot)$ creates non-ground comparison atoms, \eg $\Lambda_i>5$. Contrary to (ground) random terms, non-ground logical variables in such a comparison atom are not interpreted occurrences (\cf~Definition~\ref{def:interpreted_occ}) and the comparison itself only has a logical meaning. By grounding out the freshly introduced logical variables we obtain a purely logical program where the comparison atoms contain either arithmetic expressions or random variables (instead of random terms).


\begin{example}[Contextualizing Random Terms]
	\label{example:contextualization}
	Let us now study the effect of the second transformation step.
	Consider again the AD-free program in Example~\ref{ex:running-sugar-adfree} and the set of rules and distributional clauses obtained in Example~\ref{ex:running-sugar-core-encoding}.
	The contextualization rule CR1 (\cf Definition~\ref{def:adfree-to-core})  rewrites the logical rules in the AD-free input program to
	% \linelabel{line:comment_r2_t2a_rule}
	\begin{problog*}{linenos}
a :- rv(x,Lx), Lx =:= 1.
d :- rv(y,Ly), rv(b,Lb), Ly =:= 1, not Lb<5, Lb < 10.
e :- rv(y,Ly), rv(b,Lb), Ly =:= 2, not Lb<5, Lb < 10.
f :- rv(y,Ly), rv(b,Lb), Ly =:= 3, not Lb<5, Lb < 10.
g :- rv(b,Lb), rv(c,Lc), a, not f, Lb+Lc < 15.
	\end{problog*}
	These rules then get instantiated (rule CR2) to
	\begin{problog*}{linenos}
a :- rv(x, v2), v2 =:= 1.
d :- rv(y,v7), rv(b,v3), v7 =:= 1, not v3<5, v3 < 10.
e :- rv(y,v7), rv(b,v3), v7 =:= 2, not v3<5, v3 < 10.
f :- rv(y,v7), rv(b,v3), v7 =:= 3, not v3<5, v3 < 10.
d :- rv(y,v7), rv(b,v4), v7 =:= 1, not v4<5, v4 < 10.
e :- rv(y,v7), rv(b,v4), v7 =:= 2, not v4<5, v4 < 10.
f :- rv(y,v7), rv(b,v4), v7 =:= 3, not v4<5, v4 < 10.
g :- rv(b,v3), rv(c,v5), a, not f, v3+v5<15.
g :- rv(b,v3), rv(c,v6), a, not f, v3+v6<15.
g :- rv(b,v4), rv(c,v5), a, not f, v4+v5<15.
g :- rv(b,v4), rv(c,v6), a, not f, v4+v6<15.
	\end{problog*}
	Together with the distributional facts and rules obtained in Example~\ref{ex:running-sugar-core-encoding}, this last block of rules forms the \dcplpsty program that specifies the semantics of the AD-free \dcproblogsty program, and thus the semantics of the \dcproblogsty program in Example~\ref{ex:running-sugar-full}.
\end{example}




We note that the mapping from an AD-free program to a set of distributional facts and contextualized rules as defined here is purely syntactical, and written to avoid case distinctions.
Therefore, it usually produces overly verbose programs.
For instance, for random terms introduced by a distributional fact, the indirection via \probloginline{rv} is only needed if there is a parent term in the distribution that has context-specific interpretations.
The grounding step may introduce rule instances whose conjunction of \probloginline{rv}-atoms is inconsistent.
This is for example the case for the last three rules for \probloginline{g} in the Example~\ref{example:contextualization}, which we illustrate in the example below.


\begin{example}\label{ex:running-sugar-simplified}
	The following is a (manually) simplified version of the \dfplpsty program for the running example, where we propagated definitions of  \probloginline{rv}-atoms:
	\begin{problog*}{linenos}
v1 ~ beta(1,1).
v2 ~ flip(v1).
v3 ~ normal(3,1).
v4 ~ normal(10,1).
v5 ~ normal(v3,5).
v6 ~ normal(v4,5).
v7 ~ finite([0.2:1,0.5:2,0.3:3]).

a :- v2 =:= 1.
d :- a,     v7 =:= 1, not v3<5, v3 < 10.
e :- a,     v7 =:= 2, not v3<5, v3 < 10.
f :- a,     v7 =:= 3, not v3<5, v3 < 10.
d :- not a, v7 =:= 1, not v4<5, v4 < 10.
e :- not a, v7 =:= 2, not v4<5, v4 < 10.
f :- not a, v7 =:= 3, not v4<5, v4 < 10.
g :- a,     a,     a, not f, v3+v5<15.
g :- a,     not a, a, not f, v3+v6<15.  % inconsistent
g :- not a, a,     a, not f, v4+v5<15.  % inconsistent
g :- not a, not a, a, not f, v4+v6<15.  % inconsistent
	\end{problog*}
	In the bodies of the last three rules we have, inter alia, conjunctions of \probloginline{a} and \probloginline{not a}. This can never be satisfied and renders the bodies of these rules inconsistent.
\end{example}


\begin{definition}[Semantics of AD-free \dcproblogsty Programs]
	\label{def:semantics_adfree}
	The semantics of an AD-free \dcproblogsty program \adfreeprogram is the semantics of the \dfplpsty program $ \dfadfreeprogram = \distdb \cup \logicprogram$. We call $\adfreeprogram$ valid if and only if $\dfadfreeprogram$ is valid.
\end{definition}

\begin{definition}[Semantics of $\dcproblogsty$ Programs]
	The semantics of a $\dcproblogsty$ program $\program$ is the semantics of the AD-free \dcproblogsty program $\adfreeprogram$. We call $\program$ valid if and only if $\adfreeprogram$ is valid.
\end{definition}

Programs with distributional clauses can make programs with combinatorial structures more readable by grouping random variables with the same role  under the same random term. However, the programmer needs to be aware of the fact that distributional clauses have non-local effects on the program, as they affect the interpretation of their random terms also outside the distributional clause itself. This can be rather subtle, especially  if the bodies of the distributional clauses with the same random term are not exhaustive.
We discuss this issue in more detail in Appendix~\ref{sec:non-mixture-dc}.

\subsection{Syntactic Sugar: Validity}
\label{sec:dcvalidity}

As stated above, a DC-ProbLog program $\program$  is syntactic sugar for an AD-free program $\adfreeprogram$ (Definition~\ref{def:elim-ad}),
and is valid if $\dfadfreeprogram$ as specified in Definition~\ref{def:semantics_adfree} is a valid \dfplpsty program, \ie the distributional database is well-defined,
the comparison literals are measurable,
and each consistent fact database results in a two-valued  well-founded  model if added to the logic program (Definition~\ref{def:core-valid}).
For the distributional database to be well-defined (Definition~\ref{def:well-defd-facts}), it suffices to have $\dclauses_\adfreeprogram$ well-defined  (Definition~\ref{def:dc-df-well-def}), as can be verified by comparing the relevant definitions. Indeed, a well-defined $\dclauses_\adfreeprogram$ is a precondition for the transformation as stated in the definition.

The transformation changes neither distribution terms nor comparison literals, and thus maintains the measurability of the latter.
% \linelabel{line:comment_grammar_r2}
As far as the logic program structure is concerned, the transformation to a \dfplpsty adds rules for \probloginline{rv} based on the bodies of all distributional clauses, and uses positive \probloginline{rv} atoms in the bodies of all clauses  that interpret random terms to ensure that all interpretations of random variables are anchored in the appropriate parts of the distributional database. This level of indirection does not affect the logical reasoning for programs that only interpret random terms in appropriate contexts. It is the responsibility of the programmer to ensure that this is the case and indeed results in appropriately defined models.











\subsection{Syntactic Sugar: Additional Constructs}
\label{sec:Additionalsyntacticsugar}

\subsubsection{User-Defined Sample Spaces}
The semantics of \dcproblogsty as presented in the previous sections only alllows for random variables with numerical sample spaces, \eg normal distributions, or Poisson distributions. For categorical random variables, however, one might like to give a specific meaning to the elements in the sample space instead of a numerical value.
\begin{example}
	Consider the following program:
	\begin{problog*}{linenos}
color ~ uniform([r,g,b]).
red:- color=:=r.
	\end{problog*}
	Here we discribe a categorical random variable (uniformaly distributed) whose sample space is the set of expressions $\{\text{\probloginline{r}},\text{\probloginline{b}},\text{\probloginline{g}}\}$. By simply associating a natural number to each element of the sample space we can map the program back to a program whose semantics we already defined:
	\begin{problog*}{linenos}
color ~ uniform([1,2,3]).
r:- color=:=1,
red:- r.
	\end{problog*}
\end{example}

Swapping out the sample space of discrete random variables with natural numbers is always possible as the cardinality of such a sample space is either smaller (finite categorical) or equal (infinite) to the cardinality of the natural numbers.

\subsubsection{Multivariate Distributions}

Until now we have restricted the syntax and semantics of \dcproblogsty to univariate distributions, \eg the univariate normal distribution. At first this might seem to severely limit the expressivity of \dcproblogsty, as probabilistic modelling with multivariate random variables is a common task in modern statistics and probabilistic programming. However, this concern is voided by realizing that multivariate random variables can be decomposed into {\em combinations} of independent univariate random variables. We will illustrate this on the case of the bivariate normal distribution.

\begin{example}[Constructing the Bivariate Normal Distribution]
	Assume we would like to construct a random variable distributed according to a bivariate normal distribution:
	\begin{align*}
		\begin{pmatrix} \nu_1 \\ \nu_2 \end{pmatrix}
		\sim
		\mathcal{N}\left(
		\begin{pmatrix}
				\mu_1 \\  \mu_2
			\end{pmatrix}
		,
		\begin{pmatrix}
				\sigma_{11} & \sigma_{12} \\
				\sigma_{21} & \sigma_{22} \\
			\end{pmatrix}
		\right)
	\end{align*}
	The equation above can be rewritten as:
	\begin{align*}
		\begin{pmatrix} \nu_1 \\ \nu_2 \end{pmatrix}
		\sim
		\begin{pmatrix}
			\mu_1 \\  \mu_2
		\end{pmatrix}
		+
		\begin{pmatrix}
			\eta_{11} & \eta_{12} \\
			\eta_{21} & \eta_{22} \\
		\end{pmatrix}
		\begin{pmatrix}
			\mathcal{N}(0, \lambda_1) \\
			\mathcal{N}(0, \lambda_2)
		\end{pmatrix}
	\end{align*}
	where it holds that
	\begin{align*}
		\begin{pmatrix}
			\sigma_{11} & \sigma_{12} \\
			\sigma_{21} & \sigma_{22} \\
		\end{pmatrix}
		=
		\begin{pmatrix}
			\eta_{11} & \eta_{12} \\
			\eta_{21} & \eta_{22} \\
		\end{pmatrix}
		\begin{pmatrix}
			\lambda_1 & 0         \\
			0         & \lambda_2
		\end{pmatrix}
		\begin{pmatrix}
			\eta_{11} & \eta_{21} \\
			\eta_{12} & \eta_{22} \\
		\end{pmatrix}
	\end{align*}
	It can now be shown that the bivariate distributions can be expressed as:
	\begin{align*}
		\begin{pmatrix} \nu_1 \\ \nu_2 \end{pmatrix}
		\sim
		\begin{pmatrix}
			\mathcal{N}(\mu_{\nu_1}, \sigma_{\nu_1}) \\
			\mathcal{N}(\mu_{\nu_2}, \sigma_{\nu_2}) \\
		\end{pmatrix}
	\end{align*}
	where $\mu_{\nu_1}$, $\mu_{\nu_2}$, $\sigma_{\nu_1}$ and $\sigma_{\nu_2}$ can be expressed as:
	\begin{align*}
		\mu_{\nu_1} & = \mu_1 \qquad
		            & \sigma_{\nu_1} = \sqrt{ \eta_{11}\lambda_1^2 + \eta_{12}\lambda_2^2 } \\
		\mu_{\nu_2} & = \mu_2 \qquad
		            & \sigma_{\nu_2} = \sqrt{ \eta_{21}\lambda_1^2 + \eta_{22}\lambda_2^2 }
	\end{align*}
	We conclude from this that a bivariate normal distribution can be modeled using two univariate normal distributions that have a shared set of parameters and is thereby semantically defined in \dcproblogsty.

\end{example}

Expressing multivariate random variables in a user-friendly fashion in a probabilistic programming language is simply a matter of adding syntactic sugar for combinations of univariate random variables once the semantics are defined for the latter.

\begin{example}[Bivariate Normal Distribution]
	Possible syntactic sugar to declare a bivariate normal distribution in \dcproblogsty, where the mean of the distribution in the two dimensions is $0.5$ and $2$, and the covariance matrix is
	$
		\begin{bmatrix}
			2   & 0.5 \\
			0.5 & 1
		\end{bmatrix}
	$.
	\begin{problog*}{linenos}
(x1,x2) ~ normal2D([0.5,2], [[2, 0.5],[0.5,1]])
q:- x1<0.4, x2>1.9.
	\end{problog*}
\end{example}


On the inference side, the special syntax might then additionally be used to deploy dedicated inference algorithms. This is usually done in probabilistic programming languages that cater towards inference with multivariate (and often continuous) random variables~\citep{carpenter2017stan,bingham2019pyro}. Note that probability distributions are usually constructed by applying transformations to a set of independent uniform distribution. From this viewpoint the builtin-in \probloginline{normal/2}, denoting the univariate normal distribution, is syntactic sugar for such a transformation as well.

















\subsection{Beyond Mixtures}\label{sec:non-mixture-dc}
\label{sec:beyondmixtures}

By definition, we impose on distributional clauses mutual exclusivity of their bodies when they share a random term (\cf Definition~\ref{def:dc-df-well-def}). That is, if we have a set of distributional clauses: $\{ \tau \sim \delta_i\lpif \beta_1 \dots, \tau \sim \delta_n\lpif \beta_n  \}$ we impose that the conjunction of two distinct bodies $\beta_i$ and $\beta_j$ ($i \neq j$) is false.

A further condition that we might impose, which is, however, not necessary to define a valid distributional clause, is exhaustiveness. Let us consider again the set of distributional clauses  $\{ \tau \sim \delta_i\lpif \beta_1 \dots, \tau \sim \delta_n\lpif \beta_n  \}$. We call this set exhaustive if the disjunction of all the $\beta_i$'s is equivalent to true.

A set of exhaustive distributional clauses can be interpreted as a mixture models as they assign a unique distribution to the random term in any possible context. When, the bodies of such distributional clauses are not exhaustive, however, they may
interact with the logic program in rather subtle ways, especially if negation is involved. We demonstrate this in the examples below.

\begin{example}
	Consider the  following program fragments
	\begin{problog}
q :- not (x=:=1).
	\end{problog}
	and
	\begin{problog}
aux :- x=:=1.
q :- not aux.
	\end{problog}
	and now assume \probloginline{x} follows a mixture distribution, \eg,
	\begin{problog}
0.2::b.
x~flip(0.5) :- b.
x~flip(0.9) :- not b.
	\end{problog}
	With such a mixture model, as in the case of a distributional fact, "\probloginline{x} has an associated distribution" is always true, and both fragments agree on the truth value of \probloginline{q}.
\end{example}
In general, however, only the first of these two conditions is necessary, and it is thus possible to  associate a distribution with a random term in \emph{some} contexts only.

\begin{example}
	Consider again the two program fragments above with the following non-exhaustive definition of \probloginline{x}:
	\begin{problog}
0.2::b.
x~flip(0.5) :- b.
	\end{problog}
	With this definition, "\probloginline{x} has an associated distribution" is true if and only if \probloginline{b} is true, and the two fragments therefore no longer agree on the truth values of \probloginline{q}, as we more easily see after eliminating the distributional clause. We omit the transformation of the probabilistic fact for brevity. The fragment defining \probloginline{x} transforms to
	\begin{problog}
v1~flip(0.5).
rv(x,v1) :- b.
	\end{problog}
	The first program fragment maps to
	\begin{problog}
q :- rv(x,v1), not (v1=:=1).
	\end{problog}
	and the second one to
	\begin{problog}
aux :- rv(x,v1), v1=:=1.
q :- not aux.
	\end{problog}
	which clearly exposes the difference in how the negation is interpreted.
\end{example}
As this example illustrates, if random variables are defined through non-exhaustive sets of DCs, we can no longer refactor the logic program independently of the definition of the random variables in general, as it interacts with the context structure. The reason is that \dcproblogsty's declarative semantics builds upon the principle that the distributional database is declared \emph{independently} of the logic program, and can thus be combined modularly and declaratively with \emph{any} logic program over its comparison atoms. This is no longer the case with such arbitrary sets of DCs, which intertwine the definition of the two parts of a \dfplpsty program. We note that this differs from the procedural view on the existence of random variables taken in the \dcsty language~\citep{nitti2016probabilistic}, as we
discuss in more detail in Appendix~\ref{sec:dcproblog-dc}.





\section{Relation to the DC language}\label{sec:dcproblog-dc}

%%%%%%%%%%%%%%%%%%%%%%%%%%%%%%%%%%%%
Distributional clauses were first introduced in the language of the same name  by \cite{gutmann2011magic}, which at that point did not support negation. For negation-free programs, our interpretation of distributional clauses exactly corresponds to theirs, and \dcproblogsty thus generalizes both ProbLog (with negation) and the original (definite) distributional clause language.

% Indeed, our stochastic $W_P$-operator is a direct generalization of 
% Gutmann et al.'s stochastic $T_P$-operator, and also  incorporates the checks on when a comparison can use a random variable into the definition of the operator executing the program.
% Our declarative view, on the other hand, incorporates these into the transformed program itself, thus making the meaning of a program independent of its execution mechanism. 

In the following, we first discuss how the semantics of \dcproblogsty differs from \cite{nitti2016probabilistic}'s procedural view on negated comparison atoms, and then how \dcproblogsty's acyclicity conditions imposed on valid programs differ from those of \cite{gutmann2011magic}.


\subsection{Non-exhaustive sets of DCs}\label{sec:proc-nitti}
\cite{nitti2016probabilistic}  have extended the procedural view of the stochastic $T_P$ operator to locally stratified programs with negation under the perfect models semantics.\footnote{Local stratification is a necessary condition for perfect models semantics, and a sufficient one for well-founded semantics. On this class of programs, both semantics agree~\citep{van1991well}} In their view, a distributional clause \probloginline{x~d :- body} is informally interpreted  as "if \probloginline{body} is true, define a random variable \probloginline{x} with distribution \probloginline{d}". They then use the principle that "any comparison involving a non-defined variable will fail; therefore, its negation will succeed", \ie, they apply negation as failure to comparison atoms. 
In contrast, as already illustrated in Section~\ref{sec:non-mixture-dc}, we take a purely declarative view here, where all random variables are defined up front, independently of logical reasoning, and distributional clauses serve as syntactic sugar to compactly talk about a group of random variables. Then, truth values of comparison atoms are fully determined by their external interpretation, and do not involve reasoning about whether a random variable is defined or not. That is, we apply classical negation to comparison atoms, and restrict negation as failure to atoms defined by the logic program itself.

The following example adapted from \cite{nitti2016probabilistic} illustrates the difference.
\begin{example}\label{ex:dc-vs-dcproblog}
	Consider the following program about the color of certain objects, where the number of objects is given by the random variable \probloginline{n}:
	\begin{problog}
n ~ uniform([1,2,3]).
color(1) ~ uniform([red,green,blue]) :- 1=<n .
color(2) ~ uniform([red,green,blue]) :- 2=<n .
color(3) ~ uniform([red,green,blue]) :- 3=<n .
not_red :- not color(2)=:=red .
not_red_either :- color(2)=\=red.
	\end{problog}
	The \dcproblogsty semantics is given by the transformed program:
	\begin{problog}
v0 ~ uniform([1,2,3]).
v1 ~ uniform([red,green,blue]).
v2 ~ uniform([red,green,blue]).
v3 ~ uniform([red,green,blue]).

rv(n,v0).
rv(color(1),v1) :- rv(n,v0), 1=<v0.
rv(color(2),v2) :- rv(n,v0), 2=<v0.
rv(color(3),v3) :- rv(n,v0), 3=<v0.

not_red :- rv(color(2),v2), not v2=:=red.
not_red_either :- rv(color(2),v2), v2=\=red.
	\end{problog}
	If $\mathprobloginline{n}=1$ (i.e., $\mathprobloginline{v0}=1$), neither \probloginline{color(2)} nor \probloginline{color(3)} are associated with a distribution.  Thus, \probloginline{rv(color(2),v2)} fails, and both \probloginline{not_red} and \probloginline{not_red_either} therefore fail as well, independently of the values of the comparison literals.
	In contrast, under 
	the procedural semantics of \cite{nitti2016probabilistic}, \probloginline{color(2)=:=red} fails in this case, and \probloginline{not_red} thus succeeds. Similarly, \probloginline{color(2)=\=red} fails, and  \probloginline{not_red_either} thus fails.  
	Both views agree for $n>1$.
\end{example}

This example again illustrates that DC-ProbLog's semantics clearly follows the spirit of the distribution   semantics of defining a distribution over interpretations of basic facts (comparison atoms in this case) independently of the logic program rules. 
%In this view, all random variables "exist" and  "are defined" in all possible worlds, and  truth values of 
%comparison atoms thus exclusively depend on the external evaluation with respect to the current values of these random variables. This also implies that the negation of a comparison atom is equivalent to the comparison atom using the negated comparison operator, \eg, \probloginline{color(2)=:=red} and \probloginline{not color(2)=\=red} are equivalent, as are \probloginline{n>=2} and \probloginline{not n<2}.  \ak{we should probably stress these declarative aspects earlier already}
We note that the expressive power of logic programs allows the programmer to explicitly model the procedural view of "failure through undefined variable" in the program if desired, as illustrated in the following example. 

\begin{example}
	The following DC-ProbLog program is equivalent to the procedural interpretation of the program in  Example~\ref{ex:dc-vs-dcproblog}:
	\begin{problog*}{linenos}
n ~ uniform([1,2,3]).
color(1) ~ uniform([red,green,blue]) :- 1=<n .
color(2) ~ uniform([red,green,blue]) :- 2=<n .
color(3) ~ uniform([red,green,blue]) :- 3=<n .
not_red :- not color(2)=:=red.     
not_red :- not 2=<n.               
not_red_either :- 2=<n, color(2)=\=red.
	\end{problog*}
	We explicitly model that \probloginline{not_red} is true if either \probloginline{color(2)} can be interpreted and \probloginline{color(2)=:=red} is false (line 5, which is how \dcproblogsty interprets the first clause), or \probloginline{color(2)} cannot be interpreted (line 6, negating the body of the DC in line 3). Similarly, \probloginline{not_red_either} is true if and only if \probloginline{color(2)} can be resolved and \probloginline{color(2)=\=red} is true (line 7, repeating the body of the DC in line 3).
\end{example}


\subsection{Program validity}
To define valid programs, \citet{gutmann2011magic} impose acyclicity criteria based on the structure of the clauses in the program, whereas we use the ancestor relation between random variables in \dcproblogsty. This means that \dcproblogsty accepts certain cycles in the logic program structure that are rejected by \dcsty, as illustrated in the following example. 

\begin{example}
	We model a scenario where a property of a node in a network is either initiated locally with probability $0.1$, or propagated from a neighboring node that has the property with probability $0.3$. We consider a two node network with directed edges from each of the nodes to the other one, and directly ground the program for this situation.
	\begin{problog}
local(n1) ~ flip(0.1).
local(n2) ~ flip(0.1).
transmit(n1,n2) ~ flip(0.3) :- active(n1).
transmit(n2,n1) ~ flip(0.3) :- active(n2).
active(n1) :- local(n1)=:=1.
active(n2) :- local(n2)=:=1.
active(n1) :- transmit(n2,n1)=:=1.
active(n2) :- transmit(n1,n2)=:=1.
	\end{problog}
	This program is not distribution-stratified based on \citep{gutmann2011magic}, where in order to avoid cyclic probabilistic dependencies 1) DC heads have to be of strictly higher rank than any of their body atoms, 2) heads of regular clauses have to have at least the same rank as each body atom, and 3) atoms involving random terms have to have at least the same rank as the head of the DC introducing the random term. This is impossible with our program due to the cyclic dependency between \probloginline{active}-atoms and \probloginline{transmit}-random terms. The \dcproblogsty semantics, in contrast,  is clearly specified through the mapping:
	\begin{problog}
x1 ~ flip(0.1).
x2 ~ flip(0.1).
x3 ~ flip(0.3).
x4 ~ flip(0.3).
rv(local(n1),x1).
rv(local(n2),x2).
rv(transmit(n1,n2),x3) :- active(n1).
rv(transmit(n2,n1),x4) :- active(n2).
active(n1) :- rv(local(n1),x1), x1=:=1.
active(n2) :- rv(local(n2),x2), x2=:=1.
active(n1) :- rv(transmit(n2,n1),x4), x4=:=1.
active(n2) :- rv(transmit(n1,n2),x3), x3=:=1.
	\end{problog}
	We have four independent random variables, and a definite clause program whose meaning is well-defined despite of the cyclic dependencies between derived atoms. We can equivalently rewrite the logic program part to avoid deterministic auxiliaries:
	\begin{problog}
x1 ~ flip(0.1).
x2 ~ flip(0.1).
x3 ~ flip(0.3).
x4 ~ flip(0.3).
active(n1) :- x1=:=1.
active(n2) :- x2=:=1.
active(n1) :- active(n2), x4=:=1.
active(n2) :- active(n1), x3=:=1.
	\end{problog}
	Furthermore, \dcproblogsty agrees with the ProbLog formulation of the original program, \ie,
	\begin{problog}
0.1::local_cause(n1).
0.1::local_cause(n2).
0.3::transmit_cause(n1,n2) :- active(n1).
0.3::transmit_cause(n2,n1) :- active(n2).
active(n1) :- local_cause(n1).
active(n2) :- local_cause(n2).
active(n1) :- transmit_cause(n2,n1).
active(n2) :- transmit_cause(n1,n2).
	\end{problog}
	The AD-free program is
	\begin{problog}
v1 ~ flip(0.1).
local_cause(n1) :- v1=:=1.
v2 ~ flip(0.1).
local_cause(n2) :- v2=:=1.
v3 ~ flip(0.3).
transmit_cause(n1,n2) :- v3=:=1, active(n1).
v4 ~ finite(0.3).
transmit_cause(n2,n1) :- v4=:=1, active(n2).
active(n1) :- local_cause(n1).
active(n2) :- local_cause(n2).
active(n1) :- transmit_cause(n2,n1).
active(n2) :- transmit_cause(n1,n2).
	\end{problog}
	As this already is a \dfplpsty program, we can skip the further rewrites. While the definite clauses are factored differently compared to the earlier variants, their meaning is the same. 
\end{example}

% \section{Grounding via Resolution}
\label{app:resolution}

In Section~\ref{sec:ALWviaKC} we describe the conversion of a queried \dcproblogsty program to a labeled propositional formula through a sequence of deterministic steps. The first of which is obtaining the relevant ground program with respect to a set of query atoms and evidence atoms.

The goal of this first step of the conversion algorithm is to identify a ground core \dcproblogsty program $\dcpprogram_g$ that defines the same joint distribution over the query and evidence atoms as the original program, and contains only logical rules and distributional facts that are relevant for the query and evidence atoms. 

To achieve this goal, we define a procedure that constructs a ground core program for each potential logical derivation of an atom. We use this procedure to construct the full ground program by iterating over all such derivations for the query and evidence atoms. The procedure itself consists of two steps. First, it constructs a derivation using a set of dedicated resolution rules, where each application of a resolution rule produces a tag consisting of a set of logical rules and a set of rule manipulation instructions. If the first step succeeds, the sequence of tags is then translated to a ground program in the second step by applying the rule manipulation instructions of the tags.

The overall approach 
%as well as the two rules for  atoms defined by logical rules are 
is borrowed from the corresponding procedure introduced by \citet{fierens2015inference} for \problogsty. To generalize this approach to \dcproblogsty, we introduce rules to deal with comparison atoms, distributional clauses, and the context-dependent random variables they introduce, in addition to the original rules intriduced by~\citet{fierens2015inference}.\footnote{\citet{nitti2016probabilistic} also introduced modified resolution rules for hybrid probabilistic logic programs but for an inference algorithm that interleaves grounding and sampling, which does not require keeping track of rule groundings.}

%To obtain the relevant ground program for a \problogsty program, \citet{fierens2015inference} introduce a backward chaining procedure that applies SLD resolution~\citep{kowalski1974predicate,sterling1994art}, but ignores negation symbols, as the goal is to \emph{identify}   ground rules that are (potentially) relevant for proving query or evidence atoms rather than to actually  \emph{prove} those atoms. 

%We extend this procedure to \dcproblogsty, where we additionally need to 
%identify the \emph{relevant distributional facts}, \ie, the random variables that are interpreted in the relevant ground program and their distributions. As we have seen in Sections~\ref{sec:semantics} and~\ref{sec:dcproblog}, random variables can be interpreted both within comparison literals and within distributions of other random variables, and can be context-dependent, \ie, the same random variable term may be used to refer to several random variables introduced by a set of distributional clauses. We therefore  introduce additional  resolution rules to handle these cases.\footnote{\citet{nitti2016probabilistic} also introduced modified resolution rules for hybrid probabilistic logic programs but for an inference algorithm that interleaves grounding and sampling.}

%In order to obtain the relevant ground program we deploy, similarly to \problogsty, {\em selective linear definite clause} resolution~\citep{kowalski1974predicate,sterling1994art} with {\em negation as failure}~\citep{Clark1977NegationAF} (SLDNF). SLDNF resolution allows for computing the relevant ground program for a specific query to the logic program, while avoiding the grounding of unnecessary predicates~\citep{kersting2000bayesian,fierens2015inference}. In the case of range-restricted programs SLDNF resolution is guaranteed to terminate. We refer the reader to \citet{fierens2015inference} for a more detailed account of grounding procedure. However, in order to handle comparison predicates and distributional clauses, we will need to modify the SLDNF-resolution rules\footnote{\citet{nitti2016probabilistic} also introduced modified resolution rules for hybrid probabilistic logic programs but for an inference algorithm that interleaves grounding and sampling.}.

%Before we describe the the adapted SLDNF  resolution rules, which we dub {\em SLDNF-DC resolution}, we need specify two helper functions and a builtin predicate that are used during SLDNF-DC resolution:
In contrast to the declarative transformation used to define the semantics of full \dcproblogsty, cf.~Definition~\ref{def:adfree-to-core}, which introduces additional atoms and rules to describe the logical contexts in which specific instances of a random variable term are used, the grounding procedure directly incorporates the original context definitions into the bodies of ground rules, without introducing further atoms.  
Our resolution rules further rely on two global helper functions and an auxiliary predicate:
\begin{itemize}
	\item The injective helper function `$\mathrm{createRVname}$' maps a pair consisting of a ground random variable term and a logical context\footnote{We syntactically represent the context as a tuple of terms, which is interpreted as their logical conjunction (and thus independently of order) by the function.} to a unique constant. This constant serves as the name for the random variable term in the specific context, thus providing an external implementation of the \probloginline{refers2rv/2} predicate.
	\item The function `$\mathrm{symbols}$' maps a term~$t$  to the set of random variable terms in~$t$.  For instance, $\mathrm{symbols}(\text{\probloginline{a(N)+b(1)*2}})$ returns the set $\{ \text{\probloginline{a(N)}}, \text{\probloginline{b(1)}} \}$. This function is crucial to identify relevant random variable terms. 
	\item The auxiliary predicate \probloginline{rv/3}  takes as first argument a random variable term, as second argument a distribution term, and as third argument a context for the random variable term, which is given as a tuple of terms. \probloginline{rv/3} atoms are inserted into the current conjunctive goal when new random variables are identified, and trigger the creation of the corresponding random variable once its full ground context, which combines the body of the defining distributional clause with the contexts of the parent random variables, is determined.
\end{itemize}

We describe each resolution rule using three elements: the condition under which the rule applies (\cposlit to \crvone), the actual resolution step performed by the rule (\rposlit to \rrvonea and \rrvoneb), and the tag that labels this step (\tposlit to \trvone). We overload substitution notation to also allow for substitution of a random variable term by a newly introduced context dependent random variable name. The context-dependent rewrite operator $\twoheadleftarrow$ introduced in the last tag (\trvone) will be defined in Definition~\ref{def:context-substitution}.

\begin{definition}[Resolution Rules]
	\label{def:resolution_rules}
	
	%Let $T_{RV}$ be a hash table of $(key, value)$ pairs that keeps track of random variables (key) and their distributions (value), with insertion function $\mathrm{INSERT}(T_{RV},key,value)$. 
	Let $T_{RV}$ be a hash table that keeps track of random variables, with insertion function $\mathrm{INSERT}(T_{RV},var)$. 
	The six resolution rules for grounding \dcproblogsty programs with respect to a given goal are defined as follows. 
	
	\begin{enumerate}
		\item[\cposlit] If the goal is $(l_1, l_2, \dots, l_n)$ and  $\exists \theta$ such that  $h \lpif b_1, \dots, b_m \in \dcpprogram$, $\theta  = mgu(l_1, h)$ then:
		\begin{align*}
			&{\text{\bf \rposlit:}} \quad (l_1, l_2, \dots, l_n) \vdash (b_1, \dots, b_m, l_2, \dots, l_n) \theta
			\\
			&{\text{\bf \tposlit:}} \quad \tau = (\{ h \lpif b_1, \dots, b_m \}, \theta)
		\end{align*}
		
		\item[\bf \cneglit]  If the goal is $(not(l_1), l_2, \dots, l_n)$ and  $\exists \theta$ such that  $h \lpif b_1, \dots, b_m \in \dcpprogram$, $\theta  {=} mgu(l_1, h)$ then:
		\begin{align*}
			&{\text{\bf \rneglit:}} \quad (not(l_1), l_2, \dots, l_n) \vdash (b_1, \dots, b_m, l_2, \dots, l_n) \theta
			\\
			&{\text{\bf \tneglit:}} \quad \tau = (\{ h \lpif b_1, \dots, b_m \}, \theta)
		\end{align*}
		
		\item[\ccompone]  If the goal is $( l_{comp}, l_1, \dots, l_n )$ where  $l_{comp}$ is a comparison literal, and  there exists a substitution $\theta$ such that for some $s \in symbols(l_{comp})\setminus T_{RV}$ it holds that $h\sim d\lpif b_1, \dots, b_m\in \dcpprogram$, with $\theta = mgu(s,h)$, then:
		\begin{align*}
			&{\text{\bf \rcompone:}} \quad ( l_{comp}, l_1, \dots, l_n ) \vdash (b_1, \dots, b_m, rv(h, d, (b_1, \dots, b_m)),  l_{comp}, l_1,\dots ,l_n)\theta
			\\
			&{\text{\bf \tcompone:}} \quad \tau = (\emptyset, \theta)
		\end{align*}
		
		\item[\bf \ccomptwo]  If the goal is $( l_{comp}, l_1, \dots, l_n )$  where $l_{comp}$ is a comparison literal, and   $symbols(l_{comp})\subseteq T_{RV}$ then:
		\begin{align*}
			&{\text{\bf \rcomptwo:}} \quad  ( l_{comp}, l_1, \dots, l_n ) \vdash  (l_1, \dots, l_n )
			\\
			&{\text{\bf \tcomptwo:}} \quad \tau = (\emptyset, \emptyset)
		\end{align*}
		
		\item[\bf \crvtwo]  If the goal is $(rv(x, d, (b_1, \dots, b_{m})), l_1,\dots, l_n)$ and  there exists a substitution $\theta$ such that for some $s \in symbols(d)\setminus T_{RV}$ it holds that $h\sim d' \lpif b'_1, \dots, b'_{k}\in \dcpprogram$, $\theta = mgu(s,h)$, then:
		\begin{align*}
			{\text{\bf \rrvtwo:}} \quad 
			&(rv(x, d, (b_1, \dots, b_{m})), l_1, \dots, l_n) \\
			&~~~\vdash (b'_1, \dots, b'_{k}, rv(h, d', (b'_1, \dots, b'_{k})), rv(x, d, (b_1, \dots, b_{m})), l_1, \dots, l_n)\theta
			\\
			{\text{\bf \trvtwo:}} \quad &\tau =  (\emptyset, \theta)
		\end{align*}
		
		\item[\bf \crvone]  If the goal is $(rv(x, d, (b_1, \dots, b_{m})), l_1,\dots, l_n)$ and   $symbols(d)\subseteq T_{RV}$ then we further  distinguish two cases.
		$\text{\bf R6a}$ applies if $l_1 = rv(y, d_y, (b'_1, \dots, b'_k))$ and $x\in symbols(d_y)$. $\text{\bf R6b}$ applies in all other cases.  The tag is identical for both.
		\begin{align*}
			{\text{\bf \rrvonea:}} \quad
			&x_{new} = \mathrm{createRVname}(x,(b_1,\dots, b_m)) \\
			& \mathrm{INSERT}(T_{RV}, x_{new}) \\
			&\phi = \{ x/ x_{new} \} \\
			&(rv(x, d, (b_1, \dots, b_{m})), rv(y, d_y, (b'_1, \dots, b'_k)) , l_2, \dots, l_n) \\
			&~~~    \vdash (rv(y, d_y, (b_1, \dots, b_{m}, b'_1, \dots, b'_k) , l_2, \dots, l_n)\phi
			\\
			{\text{\bf \rrvoneb:}} \quad
			&x_{new} = \mathrm{createRVname}(x,(b_1,\dots, b_m)) \\
			& \mathrm{INSERT}(T_{RV}, x_{new}) \\
			&\phi = \{ x/ x_{new} \} \\
			&(rv(x, d, (b_1, \dots, b_{m})), l_1, \dots, l_n) \vdash (l_1, \dots, l_n)\phi
			\\
			{\text{\bf \trvone:}} \quad &\tau =  (\{x\sim d \}, \{x\twoheadleftarrow (x_{new},(b_1,\dots, b_m))\})
		\end{align*}
	\end{enumerate}
	% AK: for R6a, note that: R5 always inserts the parent rv/3-atom to the left of the child rv/3-atom, so if we need to propagate context a generation down, it is always to the goal that directly follows. However, not all pairs of neighboring rv/3-atoms are parent and child (we could also get an unrelated rv/3 from the last subgoal of the parent's context directly before the parent's rv/3), and we thus also need to check the child's distribution mentions the parent. 
\end{definition}
Intuitively, the first two rules (\rposlit and \rneglit) apply SLD resolution just as for \problogsty, and record both the rule and substitution used in the tag, while the remaining four rules are responsible for collecting information related to random variables and their distributions. These latter rules use the `$\mathrm{symbols}$' function to extract the interpreted random variable terms from the current subgoal, and compare that set against the hash table storing known random variables to distinguish between cases where all interpretations are already available and cases where the interpretation of some symbol still needs to be determined. Rules \rcompone and \rcomptwo are concerned with comparison literals, which can simply be dropped from the goal once all their random variables are resolved (\rcomptwo), and otherwise are re-considered after resolving the body of a relevant distributional clause and inserting the corresponding random variable via the auxiliary predicate (\rcompone). The last two rules ensure that the parents of a relevant random variable are included recursively using the same principle (\rrvtwo), and that all random variables are recorded in the hash table once their ancestors and context have been fully evaluated (\rrvonea and \rrvoneb). Rules \rcompone--\rrvtwo only record their substitutions in the tag, whereas \rrvonea and \rrvoneb record a new distributional fact as well as contextual rewrite information for its random variable term.  Additionally, \rrvonea propagates the parent random variable's context into the context of the child. 

%Given a query or literal to be proven, the resolution rules in Definition~\ref{def:resolution_rules} allow us to construct a resolution tree from a specific non-ground program with respect to the query. 
%The conversion algorithm uses these rules to construct a resolution tree for each query or evidence atom, using a shared hash map for the random variables.



\begin{example}[Derivations and tag sequences]
	\label{ex:resolution_tree}
	%Consider the AD-free program obtained in Example~\ref{ex:eliminating_pf}. 
	%We can construct an SLDNF-DC tree using the resolution rules from Definition~\ref{def:resolution_rules}. We depict the resolution tree in
	Figures~\ref{fig:sld_tree_left} and~\ref{fig:sld_tree_right} graphically depict the derivations obtained by applying the resolution rules from Definition~\ref{def:resolution_rules} to the AD-free program in Example~\ref{ex:eliminating_pf} as a resolution tree, where nodes represent goals and edges are labeled with the context number of the applied rule. We represent empty context conjunctions resulting from distributional facts using the term \probloginline{true} in \probloginline{rv/3}-atoms and labels. 
	
	The tag sequence for the leftmost derivation is\\
	$(\{$\probloginline{works(N):-cool(N)}$\}, \emptyset)$, \\
	$(\{$\probloginline{cool(1):-rv_cool(1)=:=1}$\}, \{N/1\})$,\\
	$(\emptyset, \emptyset)$,\\
	$(\{$\probloginline{rv_cool(1)~flip(0.99)}$\}, \{$\probloginline{rv_cool(1)}$\twoheadleftarrow ($\probloginline{rv_cool_1_1}, \probloginline{true}$)\})$,\\
	$(\emptyset, \emptyset)$.
	
	The tag sequence for the derivation in the middle is\\
	$(\{$\probloginline{works(N):-temp(N)<25.0}$\}, \emptyset)$, \\
	$(\emptyset, \{N/1\})$,\\
	$(\{$\probloginline{hot :- rv_hot =:=1}$\}, \emptyset)$, \\
	$(\emptyset, \emptyset)$,\\
	$(\{$\probloginline{rv_hot~flip(0.2)}$\}, \{$\probloginline{rv_hot}$\twoheadleftarrow ($\probloginline{rv_hot_1}, \probloginline{true}$)\})$,\\
	$(\emptyset, \emptyset)$,\\
	$(\{$\probloginline{temp(1)~normal(27,5)}$\}, \{$\probloginline{temp(1)}$\twoheadleftarrow ($\probloginline{temp_1_1}, \probloginline{hot}$)\})$,\\
	$(\emptyset, \emptyset)$.
	
	The tag sequence for the rightmost derivation is\\
	$(\{$\probloginline{works(N):-temp(N)<25.0}$\}, \emptyset)$, \\
	$(\emptyset, \{N/1\})$,\\
	$(\{$\probloginline{hot :- rv_hot =:=1}$\}, \emptyset)$, \\
	$(\emptyset, \emptyset)$,\\
	$(\{$\probloginline{rv_hot~flip(0.2)}$\}, \{$\probloginline{rv_hot}$\twoheadleftarrow ($\probloginline{rv_hot_1}, \probloginline{true}$)\})$,\\
	$(\emptyset, \emptyset)$,\\
	$(\{$\probloginline{temp(1)~normal(20,5)}$\}, \{$\probloginline{temp(1)}$\twoheadleftarrow ($\probloginline{temp_1_2}, \probloginline{not hot}$)\})$,\\
	$(\emptyset, \emptyset)$.
	
	
	
	\begin{figure}[h]
		\begin{center}
			\begin{tikzpicture}
				{\small
					\node {(\probloginline{works(N)})} [sibling distance = 0.5\linewidth]{
						child [] { node {(\probloginline{cool(N)}) }
							child [] { node {(\probloginline{rv_cool(1)=:=1})}
								child [] { node {(\probloginline{rv(rv_cool(1),flip(0.99),true)}, \probloginline{rv_cool(1)=:=1})}
									child [] { node  {( \probloginline{rv_cool_1_1=:=1})}
										child [] {node {()} edge from parent [] node [left] { \scriptsize \ccomptwo } }
										edge from parent [] node [left] { \scriptsize \crvone }
									}
									edge from parent [] node [left] { \scriptsize \ccompone }
								}
								edge from parent [] node [left] { \scriptsize \cposlit }
							}
							edge from parent [] node [left,above] { \scriptsize \cposlit }
						}
						child [] { node {(\probloginline{temp(N)<25.0})} [sibling distance = 0.4\linewidth]
							child [] { node {} edge from parent [dashed] edge from parent [] node [left, above] { \scriptsize \ccompone   } }
							child [align=center] { node {} edge from parent [dashed] edge from parent [] node [right, above] { \scriptsize \ccompone  } }
							edge from parent [] node [right, above] {  \scriptsize \cposlit  }
						}
					};
				}
			\end{tikzpicture}
		\end{center}
		\caption{Left subtree of the resolution tree for the program in Example~\ref{ex:eliminating_pf} with respect to \probloginline{query(works(N))}; dashed lines continue in Figure~\ref{fig:sld_tree_right}.}
		\label{fig:sld_tree_left}
	\end{figure}
	
	
	
	\begin{figure}
		\begin{center}
			\begin{tikzpicture}
				{\small
					
					\node {} [sibling distance = 0.55\linewidth]
					child [] {node {(\probloginline{temp(N)<25.0})} edge from parent [dashed,level distance = 2.5cm] 
						child [align=center] {node { \probloginline{(hot},\\\probloginline{rv(temp(1),normal(27,5),hot)},\\ \probloginline{temp(1)<25.0}) } edge from parent [solid] 
							child [align=center] { node {(\probloginline{rv_hot=:=1},\\ \probloginline{rv(temp(1),normal(27,5),hot)},\\ \probloginline{temp(1)<25.0})}
								child [align=center] { node {(\probloginline{rv(rv_hot, flip(0.2), true) },\\  \probloginline{rv_hot=:=1}, \\ \probloginline{rv(temp(1),normal(27,5),hot)}, \\ \probloginline{temp(1)<25.0}) }
									child [align=center] { node {( \probloginline{rv_hot_1=:=1}, \\ \probloginline{rv(temp(1),normal(27,5),hot)}, \\ \probloginline{temp(1)<25.0}) }
										child [align=center, ] { node {( \probloginline{rv(temp(1),normal(27,5),hot)}, \\\probloginline{temp(1)<25.0} ) }
											child [ level distance = 1.5cm] { node {(\probloginline{temp_1_1<25.0})}
												child { node {()}  edge from parent [] node [left,] {  \scriptsize \ccomptwo  } }
												edge from parent [] node [left,] {\scriptsize \crvone } 
											}
											edge from parent [] node [left,] { \scriptsize \ccomptwo }
										}
										edge from parent [] node [left,] { \scriptsize \crvone }
									}
									edge from parent [] node [left,] { \scriptsize \ccompone }
								}
								edge from parent [] node [left,] { \scriptsize \cposlit }
							}
							edge from parent [] node [left,above] { \scriptsize \ccompone }
						}
						child [align=center] {node { (\probloginline{not hot},\\\probloginline{rv(temp(1),normal(20,5),not hot)},\\ \probloginline{temp(1)<25.0} } edge from parent [solid] 
							child [align=center] { node {(\probloginline{rv_hot=:=1},\\ \probloginline{rv(temp(1),normal(20,5),not hot)},\\ \probloginline{temp(1)<25.0})}
								child [align=center] { node {(\probloginline{rv(rv_hot, flip(0.2), true) }, \\ \probloginline{rv_hot=:=1}, \\ \probloginline{rv(temp(1),normal(20,5),not hot)}, \\ \probloginline{temp(1)<25.0}) } 
									child [align=center] { node {( \probloginline{rv_hot_1=:=1}, \\ \probloginline{rv(temp(1),normal(20,5),not hot)},\\\probloginline{temp(1)<25.0}) }
										child [align=center] { node {( \probloginline{rv(temp(1),normal(20,5),not hot)}, \\\probloginline{temp(1)<25.0}) }
											child [ level distance = 1.5cm] { node {(\probloginline{temp_1_2<25.0})}
												child { node {()}  edge from parent [] node [right,] { \scriptsize \ccomptwo } }
												edge from parent [] node [right,]  { \scriptsize \crvone  }
											}
											edge from parent [] node [right]  { \scriptsize \ccomptwo  }
										}
										edge from parent [] node [right]  { \scriptsize \crvone  }
									}
									edge from parent [] node [right] { \scriptsize \ccompone  }
								}
								edge from parent [] node [right] { \scriptsize \cneglit }
							}
							edge from parent [] node [right,above] { \scriptsize \ccompone}
						}
						edge from parent [] node [right] {\scriptsize \cposlit }
					}
					;
				}
			\end{tikzpicture}
		\end{center}
		\caption{Right subtree of the resolution tree for the program in Example~\ref{ex:eliminating_pf} with respect to \probloginline{query(works(N))}; dashed line continued from Figure~\ref{fig:sld_tree_left}}\label{fig:sld_tree_right}
	\end{figure}
\end{example}


\newpage
\clearpage

%The resolution tree functions as an intermediate representation for obtaining the relevant ground program. Defining a tagging function for each edge in the resolution tree will allow us to specify an algorithms that maps a path in a tree (also called derivation) to its corresponding relevant ground program.

%\begin{definition}[Edge Tagging in  Resolution Tree]
%\label{def:sld_edge_tagging_rules}
%Let $\mathcal{T}$ be a resolution tree. We tag each edge in the tree with a tuple $\tau=(\tau_r,\tau_s)$, where $\tau_r$ represents a set of logic rules and $\tau_s$ a substitution. The tag of an edge of the tree $\mathcal{T}$ depends on the condition ({\bf C1-C6}) that held when the edge was created.
%We now define the six tagging rules for edges in a resolution tree in function of the conditions that apply.



%\begin{enumerate}
%    \item[\cposlit] If the goal is $(l_1, l_2, \dots, l_n)$ and if  $\exists \theta$ such that  $h \lpif b_1, \dots, b_n \in \dcpprogram$, $\theta  = mgu(l_1, h)$ then:
%   \begin{align*}
	%      {\text{\bf \tposlit:}} \quad \tau = (\{ h \lpif b_1, \dots, b_n \}, \theta)
	% \end{align*}

% \item[\bf \cneglit]  If th goal is $(not(l_1), l_2, \dots, l_n)$ and if $\exists \theta$ such that  $h \lpif b_1, \dots, b_n \in \dcpprogram$, $\theta  {=} mgu(l_1, h)$ then:
%\begin{align*}
%   {\text{\bf \tneglit:}} \quad \tau = (\{ h \lpif b_1, \dots, b_n \}, \theta)
%\end{align*}

%\item[\ccompone]  If the goal is $( l_{comp}, l_1, \dots, l_n )$ and if $l_{comp}$ is a comparison atom (or its negation) and if there exists a substitution $\theta$ such that for an arbitrary symbol $s \in symbols(l_{comp})\setminus keys(T_{RV})$ it holds that $h\sim d\lpif b_1, \dots, b_n\in \dcpprogram$, with $\theta = mgu(s,h)$, then:
%\begin{align*}
%   {\text{\bf \tcompone:}} \quad \tau = (\emptyset, \theta)\\
%\end{align*}

%\item[\bf \ccomptwo]  If the goal is $( l_{comp}, l_1, \dots, l_n )$  and if $l_{comp}$ is a comparison atom (or its negation) and if $symbols(l_{comp})\subseteq keys(T_{RV})$ then:
%\begin{align*}
%   {\text{\bf \tcomptwo:}} \quad \tau = (\emptyset, \emptyset)
%\end{align*}

%  \item[\bf \crvtwo]  If the goal is $(rv(x, d, (b_1, \dots, b_{n})), l_1,\dots, l_m)$ and if there exists a substitution $\theta$ such that for an arbitrary symbol $s \in symbols(d)\setminus keys(T_{RV})$ it holds that $h\sim d^h \lpif b^h_1, \dots, b^h_{n^h}\in \dcpprogram$, $\theta = mgu(s,h)$, then:
%  \begin{align*}
	%      {\text{\bf \trvtwo:}} \quad \tau =  (\emptyset, \theta)\\
	%  \end{align*}

%  \item[\bf \crvone]  If the goal is $(rv(x, d, (b_1, \dots, b_{n})), l_1,\dots, l_m)$ and if  $symbols(d)\subseteq keys(T_{RV})$ then:
%  \begin{align*}
	%          {\text{\bf \trvone:}} \quad \tau =  (\{x\sim d \}, \{ x/ x_{b_1,\dots,b_n} \})
	%   \end{align*}
%\end{enumerate}
%\end{definition}

%\begin{definition}[Derivation]
%Let $\dcpprogram$ be a \dcproblogsty program, and $Q$ a query atom. We call a path in the   resolution tree of $Q$ for \dcpprogram a derivation of $Q$. A derivation consists of a set of edges $\{e^0,\dots, e^N  \}$.
%Edges in a derivation are ordered, starting at the root.
%We call a derivation tagged if the rules in Definition~\ref{def:sld_edge_tagging_rules} have been applied.
%\end{definition} 

In order to define the ground program corresponding to a tag sequence, we first need to define the rewrite operation introduced by \trvone.
\begin{definition}[RV substitution]\label{def:context-substitution}
	Let $\theta=\{x\twoheadleftarrow (v, (b_1, \dots ,b_m))\}$ be a random variable substitution. Then,
	\begin{align*}
		(h \lpif l_1\dots, l_n)\theta &= (h\lpif b_1, \dots, b_m,l_1, \dots, l_n)\{x/v\}, &\text{~if~x~is~interpreted~in~some~}l_i\\
		s\theta &= s\{x/v\}, &\text{~otherwise}
	\end{align*}
	where we overload regular substitution notation for random variable terms, \ie, $c\{x/v\}$ is $c$ with every occurrence of $x$ replaced by $v$.
\end{definition}
That is, we add the  context to the bodies of all rules that interpret $x$, \ie, whose body contains a comparison literal involving $x$, and replace all occurrences of $x$ by the new variable $v$.
%replace the old random variable term $x$ by the context-dependent random variable $v$ in all statements,  and  
% we do not need to substitute parents in distribution terms, as (1) R5 ensures that R6 is applied to all parents before it is applied to the child, (2) the moment R6 is applied to a parent, there is a rv/3 atom for the child in the rest of the goal, and R6 substitutes the new parent name there, thus (3) by the time the T6 tag of the child is created, the distribution contains the parent's new name
Algorithm~\ref{alg:tags2gp} uses this rewrite operation, together with regular substitutions included in other types of tags, to extract a relevant ground program from a tag sequence. 


\begin{algorithm}[h!]
	\SetKwInput{KwOutput}{Output}              
	\DontPrintSemicolon
	\KwInput{Derivation tag sequence $S=((P_1,\theta_1),\ldots,(P_N,\theta_N))$}
	\KwOutput{Corresponding Relevant Ground Program}
	
	
	$\dcpprogram_S \leftarrow \emptyset$\;
	\For{$i \in \{0,\dots, N\}$  }{
		$\dcpprogram_S \leftarrow \dcpprogram_S \cup P_i$\;
		$\dcpprogram_S \leftarrow  \dcpprogram_S\theta_i$
	}
	
	\Return $\dcpprogram_S$
	\caption{Relevant Ground Program for a Derivation Tag Sequence}
	\label{alg:tags2gp}
\end{algorithm}
%Note that applying a tag's substitutions directly after adding its rules is crucial, as several tags may substitute the same random variable terms

%\begin{algorithm}[h!]
% \SetKwFor{For}{for (}{}
%
%\newlength\mylen
%\newcommand\myinput[1]{%
	%  \settowidth\mylen{\KwIn{}}%
	%  \setlength\hangindent{\mylen}%
	%  \hspace*{\mylen}#1\\}
%
%
%\SetKwInput{KwOutput}{output}              
%\DontPrintSemicolon
%\KwInput{Tagged derivation $\mathcal{R}$ (of depth $N$)}
%\myinput{Table $tags$ storing the tags of the edges in $\mathcal{R}$}
%\KwOutput{Relevant Ground Program corresponding to $\mathcal{R}$}
%
%
%$RelGroundProgram \leftarrow \emptyset$\;
%\For{$i \in \{0,\dots, N\}$  }{
	%    $\tau_r^i, \tau_s^i \leftarrow tags[i] $\;
	%    $RelGroundProgram \leftarrow RelGroundProgram \cup \tau_r^i$\;
	%    $RelGroundProgram \leftarrow  RelGroundProgram/\tau_s^i$
	%}
%
%\Return $RelGroundProg$
%\caption{Relevant Ground Program with Respect to a Derivation}
%\label{alg:tags2gp}
%\end{algorithm}

%\begin{definition}[Relevant Ground Program]
%Let \dcpprogram be an AD-free program, $Q$ a query atom, and  $\mathcal{T}$ the resolution tree obtained from applying the resolution rules in Definition~\ref{def:resolution_rules} to  \dcpprogram and $Q$, where all derivations have been tagged. We denote these derivations by  $\{\mathcal{R}_0, \dots,  \mathcal{R}_M \}$  and their ground programs as obtained from Algorithm~\ref{alg:tags2gp} by $\dcpprogram
%_g^{\mathcal{R}} =\{\dcpprogram_g^{\mathcal{R}_0}, \dots, \dcpprogram_g^{\mathcal{R}_M}  \}$. The relevant ground program of \dcpprogram with respect to the atom of interest $Q$, denoted by $\dcpprogram_g^Q$, is then given by: $ \dcpprogram_g^Q = \bigcup_{i\in \{0,\dots,M  \}} \dcpprogram_g^{\mathcal{R}_i}  $.
%\end{definition}

\begin{example}
	Algorithm~\ref{alg:tags2gp} constructs the following program for the rightmost tagging sequence in the example:
	\begin{problog*}{linenos}
works(1) :- not hot, temp_1_2<25.0.
hot :- true, rv_hot_1 =:= 1.
rv_hot_1~flip(0.2).
temp_1_2~normal(27,5).
	\end{problog*}
	Note that all random variable terms have been replaced by the new random variables, and appropriate contexts added to the rules interpreting them. 
\end{example}

\begin{definition}[Relevant ground program for an atom]
	Let \dcpprogram be an AD-free program, $A$ an atom and $\mathcal{S}$ be the set of all derivation tag sequences for $A$ in \dcpprogram. The \emph{relevant ground program with respect to $A$}, denoted by $\dcpprogram_g^A$, is given by $\dcpprogram_g^A = \bigcup_{S\in\mathcal{S}}\dcpprogram_S$.
\end{definition}

\begin{definition}[Relevant ground program]
	Let \dcpprogram be an AD-free program, $\queryset$ a set of queries of interest, and  $\evidenceset=e$ the evidence. The \emph{relevant ground program}  is $\dcpprogram_g = \bigcup_{A\in (\queryset\cup\evidenceset)}\dcpprogram_g^A$.
\end{definition}


\begin{example}[Relevant Ground Program] 
	%Applying the edge tagging and the definition of a relevant ground program  to the resolution tree in Example~\ref{ex:resolution_tree} results in the following ground program.
	The full ground program obtained from the derivations in Example~\ref{ex:resolution_tree} is
	\begin{problog*}{linenos}
rv_hot_1 ~ flip(0.2).
hot :- true, rv_hot_1=:=1.
rv_cool_1_1 ~ flip(0.99).
cool(1) :- true, rv_cool_1_1=:=1.

temp_1_1 ~ normal(27,5).
temp_1_2 ~ normal(20,5).

works(1):- cool(1).
works(1):- hot, temp_1_1<25.0.
works(1):- not hot, temp_1_2<25.0.
	\end{problog*}
	Note that lines 1 and 2 are generated by both the middle and the rightmost derivation, while all other statements stem from a single derivation in this example.	
	%Note how the `$\mathrm{createRVname}$' function created unique names for the random variables in the program, which are then also inserted into the arguments of the comparison terms. For instance \probloginline{temp_1_2} in the the comparison \probloginline{temp_1_2<25.0}. 
\end{example}


In order to implement an efficient resolving algorithm, one ought to avoid processing the same atom twice. This is usually done through tabling, which also avoids going into an infinite loop when cyclical rules are present in the (logic part of the) program. 
%Note also that if we had had multiple query atoms, instead of a single one, our approach would not change dramatically. The main difference would lie in having a multi-rooted resolution tree instead of a single root tree.
Note also that while the relevant ground program computed by this approach does not contain any rules that are disconnected from the reasoning process related to queries and evidence, it is not necessarily the smallest such program. It may for instance include rules whose bodies are logically inconsistent because of added conjunctions of contexts, or invalidated by the actual values of the evidence. 

\section{Proofs of Theorems and Propositions in Section~\ref{sec:dc2smt} and Section~\ref{sec:alw}}
\label{app:transformation_proofs}

\subsection{Proof of Theorem~\ref{theo:rgp}} \label{app:proof:rgp}


\theorgp*

\begin{proof}
    %The proof is analogous  to the one for Theorem~1 in~\citep[Appendix A]{fierens2015inference}, as it only concerns the logic part of the program.
    The semantics of $\dcpprogram$ is given by the ground program that is obtained by first grounding $\dcpprogram$ with respect to its Herbrand base and reducing it to a \dfplpsty program as specified in Section~\ref{sec:dcproblog}. The resulting program consists of distributional facts and ground normal clauses only, and includes clauses defining \probloginline{rv}-atoms as well as calls to those atoms in clause bodies. However, as the definitions of these atoms are acyclic, and each ground instance is defined by a single rule (with the body of the DC that introduced the new random variable), we can eliminate all references to such atoms by recursively applying the well-known \emph{unfolding} transformation, which replaces atoms in clause bodies by their definition. The result is an equivalent ground program using only predicates from \dcpprogram, but where rule bodies have been expanded with the contexts of the random variables they interpret. We know from Theorem~1 in \citep{fierens2015inference} that for given query and evidence, it is sufficient to use the part of this logic program that is encountered during backward chaining from those atoms. We note that in our case, this also includes the distributional facts providing the distributions for relevant random variables, \ie, random variables in relevant comparison atoms as well as their ancestors.
\end{proof}




\subsection{Proof of Theorem~\ref{theo:label_equivalence}}
\label{app:proof:label_equivalence}

\Labelequivalence*








\begin{proof} The probability of a model $\varphi$ of the relevant ground program $\dcpprogram_g$ is, according to the distribution semantics (cf. Appendix~\ref{app:proof:pf}), given by:
    \begin{align}
        P_{\dcpprogram_g}(\varphi) = \int \mathbf{1}_{[\mu_1=\boolval_1\wedge\ldots\wedge \mu_n=\boolval_n]}(\samplefunction(\randomvariableset)) \differential{\probabilitymeasure_{\randomvariableset}}
    \end{align}
    where the $\mu_i$ are the comparison atoms that appear (positively or negatively) in $\dcpprogram_g$ and the $b_i$ the truth values these atoms take in $\varphi$.
    We can manipulate the probability into:
    \begin{align}
        P_{\dcpprogram_g}
         & =
        \int \left( \prod_{i =1}^n \mathbf{1}_{[\mu_i=b_i]}(\samplefunction(\randomvariableset)) \right) \differential{\probabilitymeasure_{\randomvariableset}}                                                                                                         \\
         & =
        \int \left( \prod_{i: b_i=\bot} \mathbf{1}_{[\mu_i=b_i]}(\samplefunction(\randomvariableset)) \right)  \left( \prod_{i: b_i=\top} \mathbf{1}_{[\mu_i=b_i]}(\samplefunction(\randomvariableset)) \right)  \differential{\probabilitymeasure_{\randomvariableset}} \\
         & =
        \int \left( \prod_{i: b_i=\bot} \ive{\neg c_i(vars(\mu_i))} \right)  \left( \prod_{i: b_i=\top}  \ive{ c_i(vars(\mu_i))}  \right)  \differential{\probabilitymeasure_{\randomvariableset}}
        \label{eq:program_prob_label_equi}
    \end{align}
    Turning our attention now to the expected value of $\alpha(\varphi)$ we have:
    \begin{align}
        \E_{\randomvariableset \sim  \dcpprogram_g} [\alpha( \varphi )]
        =
        \int \alpha \left( \bigwedge_{\ell_i \in \varphi} \ell_i \right) \differential{P_\randomvariableset}
        =
        \int  \left( \prod_{\ell_i \in \varphi} \alpha \left(  \ell_i \right) \right)  \differential{P_\randomvariableset}
    \end{align}
    The literals $\ell_i \in \varphi$ fall into four groups: atoms whose predicate is a comparison and that are true in $\varphi$ (denoted by $CA^+(\varphi)$), non-comparison atoms that are true in $\varphi$ (denoted $NA^+(\varphi)$), and similarly the atoms that are false in $\varphi$ (denoted by $CA^-(\varphi)$ and $NA^-(\varphi)$). This yields:
    \begin{align}
         & \E_{\randomvariableset \sim  \dcpprogram_g} [\alpha( \varphi )] \\
         & =
        \int
        \left( \prod_{\ell_i \in CA^+(\varphi)} \alpha (  \ell_i ) \right)
        \left( \prod_{\ell_i \in CA^-(\varphi)} \alpha (  \neg \ell_i ) \right)
        \left( \prod_{\ell_i \in NA^+(\varphi)} \alpha (  \ell_i ) \right)
        \left( \prod_{\ell_i \in NA^-(\varphi)} \alpha (  \neg \ell_i ) \right)
        \nonumber
        \differential{P_\randomvariableset}
    \end{align}
    Plugging in the definition of the labeling function the last two products reduce to $1$ and we obtain for the remaining expression:
    \begin{align}
        \E_{\randomvariableset \sim  \dcpprogram_g} [\alpha( \varphi )]
        =
        \int
        \left( \prod_{i: \ell_i \in CA^+(\varphi)} \ive{  c_i(vars(\ell_i)) } \right)
        \left( \prod_{i: \ell_i \in CA^-(\varphi)} \ive{  \neg c_i(vars(\ell_i)) } \right)
        \differential{P_\randomvariableset}
        \label{eq:formula_prob_lable_equi}
    \end{align}
    Identifying now the set $\{ i: \ell_i \in CA^+(\varphi) \}$ with the set $\{i : \mu_i=\top   \} $ and the set  $\{ i: \ell_i \in CA^-(\varphi) \}$ with the set $\{i : \mu_i=\bot   \} $ proves the theorem, as this equates Equation~\ref{eq:program_prob_label_equi} and Equation~\ref{eq:formula_prob_lable_equi}
\end{proof}


\subsection{Proof of Proposition~\ref{prop:mcapproxconditional}}
\label{app:proof:mcapproxconditional}


\mcapproxconditional*

\begin{proof}
    First we write the conditional probability as a ratio of expected values invoking Theorem~\ref{thm:inference-by-expectation}, on which we then use Defintion~\ref{def:label_bool_formula}:
    \begin{align}
        P_\dcpprogram(\mu=q\mid\evidenceset = e)
         & = \frac{\E_{\randomvariableset \sim  \dcpprogram_g} [\alpha( \phi \wedge \phi_q)] }{\E_{\randomvariableset \sim  \dcpprogram_g} [\alpha( \phi )] }
        \\
         & =
        \frac{
            \E_{\randomvariableset \sim  \dcpprogram_g} \left[ \sum_{\varphi\in ENUM(\phi \land \phi_{q}) } \prod_{\ell \in \varphi} \alpha(\ell) \right]
        }
        {
            \E_{\randomvariableset \sim  \dcpprogram_g} \left[ \sum_{\varphi\in ENUM(\phi) } \prod_{\ell \in \varphi} \alpha(\ell)  \right]
        }
        \\
         & =
        \frac{
            \E_{\randomvariableset \sim  \dcpprogram_g} \left[ \sum_{\varphi\in ENUM(\phi \land \phi_{q}) } \alpha(\varphi) \right]
        }
        {
            \E_{\randomvariableset \sim  \dcpprogram_g} \left[ \sum_{\varphi\in ENUM(\phi) } \alpha(\varphi)  \right]
        }
    \end{align}



    We can now express the conditional probability in terms of the sampled values $\mathcal{S}$:
    \begin{align}
        P_\dcpprogram(\mu=q\mid\evidenceset=e)
           & =
        \frac{
        \lim_{ \lvert \mathcal{S} \rvert \rightarrow \infty} \nicefrac{1}{\lvert \mathcal{S} \rvert} \sum_{i=1}^{\lvert \mathcal{S} \rvert}  \sum_{\varphi\in ENUM(\phi \land \phi{q}) } \alpha^{(i)}(\varphi)
        }
        {
        \lim_{\lvert \mathcal{S} \rvert \rightarrow \infty} \nicefrac{1}{\lvert \mathcal{S} \rvert} \sum_{i=1}^{\lvert \mathcal{S} \rvert} \sum_{\varphi\in ENUM(\phi) } \alpha^{(i)}(\varphi)
        }                                                                   \\
           & \approx
        \frac{
        \sum_{i=1}^{\lvert \mathcal{S} \rvert}  \sum_{\varphi\in ENUM(\phi \land \phi_{q}) } \alpha^{(i)}(\varphi)
        }
        {
        \sum_{i=1}^{\lvert \mathcal{S} \rvert} \sum_{\varphi\in ENUM(\phi) } \alpha^{(i)}(\varphi)
        }, & \quad \lvert \mathcal{S} \rvert<\infty \label{eq:mc_prob_cond}
    \end{align}
\end{proof}


\subsection{Proof of Proposition~\ref{prop:alw_consistency}}
\label{app:proof:alw_consistency}


\alwconsistency*
\begin{proof}
    First we manipulate the expected value on the left hand side of Equation~\ref{eq:ALW}:
    \begin{align}
         & \E \left[ \sum_{\varphi \in ENUM(\phi)} \prod_{\ell \in \varphi}  \alpha \left(\ell \right) \bigg| \mathcal{S} \right]                                                           \\
         & =
        \lim_{ \lvert \mathcal{S} \rvert\rightarrow\infty} \sum_{i=1}^{ \lvert \mathcal{S} \rvert}  \sum_{\varphi \in ENUM(\phi)} \prod_{\ell \in \varphi}  \alpha^{(i)} \left(\ell \right) \\
         & =
        \lim_{ \lvert \mathcal{S} \rvert \rightarrow\infty} \sum_{i=1}^{ \lvert \mathcal{S} \rvert}  \sum_{\varphi \in ENUM(\phi)}
        \left(\prod_{\ell \in \varphi \setminus DI(\varphi)}  \alpha^{(i)} \left(\ell \right) \prod_{\ell \in DI(\varphi)}  \alpha^{(i)} \left(\ell \right)   \right)
    \end{align}
    As the samples are ancestral samples, they satisfy by construction the delta invervals appearing in the second product. This means that $\prod_{\ell \in DI(\varphi)}  \alpha^{(i)} \left(\ell \right) =1$ and that we can write the expected value in function of non delta interval atoms only:
    \begin{align}
        \E \left[ \sum_{\varphi\in ENUM(\phi)} \prod_{\ell \in \varphi}  \alpha \left(\ell \right) \bigg| \mathcal{S} \right]
        =
        \E \left[ \underbrace{\sum_{\varphi\in ENUM(\phi)} \prod_{\ell \in \varphi \setminus DI(\varphi)}  \alpha \left(\ell \right)}_{\coloneqq f(\phi)} \bigg| \mathcal{S} \right]  \label{eq:ALW_exp_simplified}
    \end{align}
    Let us now manipulate the expression in the numerator on the right hand side of Equation~\ref{eq:ALW}:
    \begin{align}
         & \bigoplus_{i=1}^{ \lvert \mathcal{S} \rvert}  \bigoplus_{\varphi \in ENUM(\phi)} \bigotimes_{\ell \in \varphi}  \alpha_{IALW}^{(i)} \left(\ell \right) \\
         & =
        \bigoplus_{i=1}^{ \lvert \mathcal{S} \rvert}  \bigoplus_{\varphi \in ENUM(\phi)}
        \underbrace{\left(  \bigotimes_{\ell \in \varphi \setminus DI(\varphi)}  \alpha_{IALW}^{(i)} \left(\ell \right) \right) }_{\left( r_\varphi^{(i)} , 0  \right)}  \otimes
        \underbrace{\left( \bigotimes_{\ell \in DI(\varphi)}  \alpha_{IALW}^{(i)} \left(\ell \right) \right) }_{\left( t_\varphi^{(i)} , m_\varphi^{(i)}  \right)} \label{eq:alw_proof_intermediate_1}
    \end{align}
    The expressions $\left( r_\varphi^{(i)} , 0  \right)$ and $\left( t_\varphi^{(i)} , m_\varphi^{(i)}  \right)$ denote infinitesimal numbers. Note how only the latter of the two picks up a non-zero second part.

    From the definition of the addition of two infinitesimal numbers we can see that only those infinitesimal numbers with the smallest integer in the second part {\em survive} the addition. This also means that in Equation~\ref{eq:alw_proof_intermediate_1} only those terms that have the smallest integer in their second part among all terms will contribute. We denote this smallest integer by:
    \begin{align}
        m^* =  \min_{ \substack{i\in \{ 1, \dots,  \lvert \mathcal{S} \rvert \} \\ \varphi\in ENUM(\phi) }} m^{(i)}_\varphi
    \end{align}
    We rewrite Equation~\ref{eq:alw_proof_intermediate_1} in function of $m^*$:
    \begin{align}
         & \bigoplus_{i=1}^{ \lvert \mathcal{S} \rvert}  \bigoplus_{\varphi \in ENUM(\phi)}
        \left(  \ive{m_{\varphi}^{(i)}{=} m^* } , 0  \right) \otimes
        \left( r_\varphi^{(i)} , 0  \right) \otimes
        \left( t_\varphi^{(i)} , m^*  \right)                                                                              \\
         & =
        \bigoplus_{i=1}^{ \lvert \mathcal{S} \rvert}  \bigoplus_{\varphi \in ENUM(\phi)}
        \left( \ive{m_{\varphi}^{(i)}{=} m^* } r_\varphi^{(i)}  t_\varphi^{(i)} , 0  \right) \otimes \left( 1, m^* \right) \\
         & =
        \left( \sum_{i=1}^{ \lvert \mathcal{S} \rvert}  \sum_{\varphi \in ENUM(\phi)}  \ive{m_{\varphi}^{(i)}{=} m^* } r_\varphi^{(i)}  t_\varphi^{(i)} , 0   \right)  \otimes \left( 1, m^* \right) \label{eq:ALW_num_simplified}
    \end{align}
    Similarly we get for the denominator in Equation~\ref{eq:ALW}:
    \begin{align}
         & \bigoplus_{i=1}^{ \lvert \mathcal{S} \rvert}  \bigoplus_{\varphi \in ENUM(\phi)} \bigotimes_{\ell \in  DI(\varphi)}  \alpha_{IALW}^{(i)} \left(\ell \right) \\
         & =
        \left( \sum_{i=1}^{ \lvert \mathcal{S} \rvert}  \sum_{\varphi\in ENUM(\phi)}  \ive{m_{\varphi}^{(i)}{=} m^* }  t_\varphi^{(i)} , 0   \right)  \otimes \left( 1, m^* \right)  \label{eq:ALW_den_simplified}
    \end{align}
    We can now plug Equation~\ref{eq:ALW_exp_simplified}, Equation~\ref{eq:ALW_num_simplified} and Equation~\ref{eq:ALW_den_simplified} back into Equation~\ref{eq:ALW} and obtain:

    \begin{align}
         & \phantom{{}\Leftrightarrow{}}  \left( \E \left[  f(\phi) | \mathcal{S} \right],  0 \right)
        =
        \left(
        \frac
        { \sum_{i=1}^{ \lvert \mathcal{S} \rvert} \sum_{\varphi \in ENUM(\phi)}  \ive{m_{\varphi}^{(i)}{=} m^* } r_\varphi^{(i)}  t_\varphi^{(i)} }
        { \sum_{i=1}^{ \lvert \mathcal{S} \rvert}  \sum_{\varphi\in ENUM(\phi)}  \ive{m_{\varphi}^{(i)}{=} m^* }  t_\varphi^{(i)}   }
        , 0
        \right) \quad ( \lvert \mathcal{S} \rvert\rightarrow \infty)                                  \\
         & \Leftrightarrow \E \left[  f(\phi) | \mathcal{S} \right]
        =
        \frac
        { \sum_{i=1}^{ \lvert \mathcal{S} \rvert}  \sum_{\varphi \in ENUM(\phi)}  \ive{m_{\varphi}^{(i)}{=} m^* } r_\varphi^{(i)}  t_\varphi^{(i)} }
        { \sum_{i=1}^{ \lvert \mathcal{S} \rvert}  \sum_{\varphi\in ENUM(\phi)}  \ive{m_{\varphi}^{(i)}{=} m^* }  t_\varphi^{(i)}   } \label{eq:llw}
    \end{align}
    We realize that $r_\varphi^{(i)}$ is actually $f(\phi)$ evaluated at the $i$-th sample at the instantiation $\varphi$ and evoke~\citep[Theorem 4.1]{wu2018discrete} to prove Equation~\ref{eq:llw}, which also finishes this proof.
\end{proof}

\subsection{Proof of Proposition~\ref{prop:alwapproximation}}
\label{app:proof:alwapproximation}


\alwapproximation*


\begin{proof}
    We start the proof by invoking Theorem~\ref{thm:inference-by-expectation}, which expresses the conditional probability as a ratio of expectations. In the numerator and the denominator we then write the label of the propositional logic formulas as the sum of the labels of the respective possible worlds.
    \begin{align}
        P_\dcpprogram(\mu=q|\evidenceset=e)  \label{eq:cond_amc_1}
         & =\frac{\E_{\randomvariableset \sim  \dcpprogram_g} [\alpha( \phi \wedge \phi_q)] }{\E_{\randomvariableset \sim  \dcpprogram_g} [\alpha( \phi )] } \\
         & =
        \frac
        {
            \E_{\randomvariableset \sim  \dcpprogram_g} \left[ \sum_{\varphi\in ENUM(\phi \land \phi_{q e}) } \alpha(\varphi) \right]
        }
        {
            \E_{\randomvariableset\sim  \dcpprogram_g} \left[ \sum_{\varphi\in ENUM(\phi) } \alpha(\varphi)  \right]
        }  \label{eq:cond_amc_2}
    \end{align}
    Next, we approximate the expectation using a set of ancestral samples $\mathcal{S}$, followed by pulling out the query from the summation index in the numerator:
    \begin{align}
         & \frac
        {
            \E_{\randomvariableset \sim  \dcpprogram_g} \left[ \sum_{\varphi\in ENUM(\phi \land \phi_{q}) } \alpha(\varphi) \right]
        }
        {
            \E_{\randomvariableset \sim  \dcpprogram_g} \left[ \sum_{\varphi\in ENUM(\phi) } \alpha(\varphi)  \right]
        }
        \\
         & \approx
        \frac
        {
            \E \left[ \sum_{\varphi\in ENUM(\phi \land \phi_{q})   } \alpha(\varphi) |  \mathcal{S}  \right]
        }
        {
            \E \left[ \sum_{\varphi\in ENUM(\phi) } \alpha(\varphi) |  \mathcal{S}   \right]
        }  \label{eq:cond_amc_approx}
        \\
         & \approx
        \frac{
            \E \left[ \sum_{\varphi\in ENUM(\phi)   } \ive{\varphi \models \phi_q} \alpha(\varphi) |  \mathcal{S}  \right]
        }
        {
            \E \left[ \sum_{\varphi\in ENUM(\phi) } \alpha(\varphi) |  \mathcal{S}   \right]
        }
        \label{eq:cond_amc_expanded}
    \end{align}
    We now rewrite the fraction of two real numbers in Equation~\ref{eq:cond_amc_expanded} as the fraction of two infinitesimal numbers and plug in the definition of the infinitesimal algebraic likelihood weight (cf. Definition~\ref{def:alw}):
    \begin{align}
         & \frac{
            \E \left[ \sum_{\varphi\in ENUM(\phi)   } \ive{\varphi \models \phi_q} \alpha(\varphi) |  \mathcal{S}  \right]
        }
        {
            \E \left[ \sum_{\varphi\in ENUM(\phi) } \alpha(\varphi) |  \mathcal{S}   \right]
        }
        \\
         & =
        \frac{
            \left(\E \left[ \sum_{\varphi\in ENUM(\phi)   } \ive{\varphi \models \phi_q} \alpha(\varphi) |  \mathcal{S}  \right], 0 \right)
        }
        {
            \left( \E \left[ \sum_{\varphi\in ENUM(\phi) } \alpha(\varphi) |  \mathcal{S}   \right],0 \right)
        }          \\
         & \approx
        \frac
        {
        \displaystyle \bigoplus_{i=1}^{\lvert \mathcal{S} \rvert}  \bigoplus_{\varphi\in ENUM(\phi)} \ive{\varphi \models \phi_q} \bigotimes_{\ell \in \varphi}   \alpha_{IALW}^{(i)} \left(\ell \right)}
        {
        \displaystyle  \cancel{ \bigoplus_{i=1}^{\lvert \mathcal{S} \rvert}  \bigoplus_{\varphi\in ENUM(\phi)} \bigotimes_{\ell \in  DI(\varphi)}  \alpha_{IALW}^{(i)} \left(\ell \right) }}
        \otimes
        \frac
        {
        \displaystyle \cancel{\bigoplus_{i=1}^{\lvert \mathcal{S} \rvert}  \bigoplus_{\varphi\in ENUM(\phi)} \bigotimes_{\ell \in  DI(\varphi)}  \alpha_{IALW}^{(i)} \left(\ell \right) } }
        {
        \displaystyle \bigoplus_{i=1}^{\lvert \mathcal{S} \rvert}  \bigoplus_{\varphi\in ENUM(\phi)} \bigotimes_{\ell \in \varphi}  \alpha_{IALW}^{(i)} \left(\ell \right)
        } \label{eq:cond_amc_canel}
    \end{align}

    In the last line the first factor corresponds to the numerator of the previous equation and the second factor corresponds to the reciprocal of the denominator.
    Note that the consistency of the infinitesimal algebraic likelihood weight of the numerator (first factor) is guaranteed by defining a new labeling function $\alpha^q(\varphi)\coloneqq \ive{\varphi \models \phi_q} \alpha(\varphi)$ and evoking Proposition~\ref{prop:alw_consistency} with $\alpha^q$.

    Finally, we push the expression $\ive{\varphi \models \phi_q}$ in the numerator back into the index of the summation ($\oplus$), which yields the following expression:
    \begin{align}
        \frac
        { \bigoplus_{i=1}^{\lvert \mathcal{S} \rvert}  \bigoplus_{\varphi\in ENUM(\phi \land \phi_{q})} \bigotimes_{\ell \in \varphi}  \alpha_{ALWI}^{(i)} \left(\ell \right)}
        { \bigoplus_{i=1}^{\lvert \mathcal{S} \rvert}  \bigoplus_{\varphi\in ENUM(\phi)} \bigotimes_{\ell \in \varphi}  \alpha_{IALW}^{(i)} \left(\ell \right)}
        \label{eq:cond_alw_last}
    \end{align}
    which proves the proposition. \end{proof}





\subsection{Proof of Proposition~\ref{prop:alwonddnnf}}
\label{app:proof:alwonddnnf}

\alwonddnnf*
\begin{proof}
    Algorithm~\ref{alg:prob_via_alw_kc} first compiles both propositional formulas into equivalent d-DNNF representations, cf. Lines~\ref{alg:prob_via_alw_kc:kc_qe} and~\ref{alg:prob_via_alw_kc:kc_e}.
    In Lines~\ref{alg:prob_via_alw_kc:alw_qe} and~\ref{alg:prob_via_alw_kc:alw_e} it then computes the (unnormalized) infinitesimal algebraic likelihood weight for both formulas by calling Algorithm~\ref{alg:unormalize_alw}. In other words, we compute the numerator and denominator in Equation~\ref{eq:cond_alw_last}. We observe that Algorithm~\ref{alg:unormalize_alw} evaluates a given d-DNNF formula for each conditioned topological sample using the \texttt{Eval} function, which evaluates a d-DDNF formula given a labeling function, cf. Algorithm~\ref{alg:eval}~\citep{kimmig2017algebraic}. The correctness of Algorithm~\ref{alg:prob_via_alw_kc} now hinges on the correctness of the \texttt{Eval} function, which was proven by~\citet{kimmig2017algebraic} for the evaluation of a d-DNNF formula using a semiring and labeling function pair that adheres to the properties described in Lemmas~\ref{lem:non_idem} to~\ref{lem:non_cons}. Effectively, Algorithm~\ref{alg:unormalize_alw} correctly computes the algebraic model count for each ancestral sample, adds up the results, and returns the unnormalized algebraic model count to Algorithm~\ref{alg:prob_via_alw_kc}. Line~\ref{alg:prob_via_alw_kc:conditional} finally return the ratio of the two unnormalized algebraic likelihood weights, which corresponds to the conditional probability $P_\dcpprogram(\mu=q|\evidenceset=e)$, as proven in Equations~\ref{eq:cond_amc_1} to~\ref{eq:cond_alw_last}.
\end{proof}

% \section{The Equivalence of Labeled Propositional Formulas and SMT Formulas}
\label{app:labeled_prop_to_SMT}








%\pedro{ to be adapted
%First, we rewrite the equation in terms of weighted model integration, which follows from our exposition in Section~\ref{sec:dc2smt}:
%\begin{align}
%    P_\dcpprogram(Q=1|\evidenceset=e) = \frac{\lwmi(\phi_{q\land e},w)}{\lwmi(\phi_{e},w)}
%\end{align}
%The formula $\phi_{e}$ denotes the SMT formula for the ground program with the evidence asserted and $\phi_{q\land e}$ its conjunction with the SMT variable for the query. The symbol $w$ denotes again the joint probability distributions for all the random variables present in $\phi_{q\land e}$ and $\phi_e$, which we refer to as $\{X_1, \dots,  X_V  \}$. This means that $\int w d(\lambda\times \xi \times \mu)=1$. We continue by expressing the conditional probability in terms of a ratio of expectations:
%\begin{align}
%    P_\dcpprogram(Q=1|\evidenceset=e)
%    &= \frac{\int_{\mathcal{M}(\phi_{q\land e})} w d(\lambda\times \xi \times \mu) }{\int_{\mathcal{M}(\phi_{e})} w d(\lambda\times \xi \times \mu)} \\
%    &= \frac{\int \ive{ \mathcal{M}(\phi_{q\land e})} w d(\lambda\times \xi \times \mu) }{\int \ive{\mathcal{M}(\phi_{e})} w d(\lambda\times \xi \times \mu)} \\
%    &= \frac{\E_{w} \left[  \ive{\mathcal{M}(\phi_{q\land e})} \right] }{\E_{w} \left[ \ive{\mathcal{M}(\phi_{e})}  \right]}\\
%    &= \frac{\E_{w} \left[ \sum_{\omega\in ENUM(\phi^\downarrow_{q\land e}) } \alpha(\omega) \right] }{\E_{w} \left[ \sum_{\omega\in ENUM(\phi^\downarrow_{e}) } \alpha(\omega)  \right]}
%\end{align}
%In the last line the $(\omega\in ENUM (\phi^\downarrow_{q\land e})$ denotes an abstracted version of the SMT formula $\phi_{q\land e}^\downarrow$ and similarly for $\phi_e^\downarrow$ (see Example~\ref{ex:abstraction} for an instance of an abstraction).
%Moreover, the last line expresses the initial conditional probability as the ratio of the expected values of $\phi_{q\land e}^\downarrow$ and $\phi_{e}^\downarrow$, respectively, with regards to the variables in the joint probability $w$.
%}




%\ak{this section needs to start with a sentence that provides context -- why are we discussing this? where are we headed?}

In Section~\ref{sec:dc2smt} where we describe the conversions performed by the inference algorithm on a queried \dcproblogsty program we used labeled propositional formulas as an intermediate representation. This, however, ignored the arithmetic meaning inherent to the comparison predicates. We will now refine this our understanding of labeled propositional logic formulas by making the link with weighted model integration~\citep{belle2015probabilistic}.

On the logic formula level we have been treating comparisons, \eg, \probloginline{(temp<25.0)}, as purely logical, with no inherent meaning. For instance, we never exploited facts like $(\text{\probloginline{temp<25}} \land \text{\probloginline{temp>30}}) \leftrightarrow \bot $. Only through the labeling function (cf. Definition~\ref{def:labeling_function}) have comparisons been assigned an arithmetic meaning. For instance, the labeling function {\em interprets} the predicate \probloginline{</2} as an arithmetic comparison and the two arguments (\probloginline{temp}, \probloginline{25.0}) as arithmetic expressions.\footnote{Note that logic programming languages, such as \problogsty, often include a non-logical external arithmetics engine. Through such engines logic programming languages are capable of interpreting arithmetic expressions and comparison predicates, \eg \probloginline{25>5+2} and assign a logical meaning to the comparison~\citep[Section 8]{sterling1994art}. In a sense, probabilistic logic programming languages extend logic programming with probabilistic arithmetics engines.}

Interpreting comparisons can also be performed on the formula level, without the use of labeled atoms. Such formulas are called satisfiability modulo theory (SMT) formulas and extend propositional logic with interpreted first-order predicates. 
% We give a formal definition of SMT formulas in Appendix~\ref{app:wmi} and refer the reader to~\citet{barrett2009handbook} for an in-depth treatment.
% Using the so-called refinement operation one can obtain SMT formulas from Boolean formulas~\citep{barrett2009handbook}.
We refer the reader to~\citet{barrett2009handbook} for a formal definition and in-depth treatment of SMT formulas.
Using the so-called refinement operation one can obtain SMT formulas from Boolean formulas.

\begin{example}[Refinement of a Boolean Formula]\label{example:refinment}
Consider the Boolean formula $\support_g $ from Example~\ref{example:dc2bool}, \ie,:
	\begin{align}
		%\support_g  \nonumber
		%&\leftrightarrow
		&\Big( \phi_{\text{\probloginline{works(1)}}} \leftrightarrow \phi_{\text{\probloginline{cool(1)}}} 
		\lor \phi_\text{\probloginline{hot}} \land  \phi_\text{\probloginline{temp(hot)<25.0}}  
		\lor\neg \phi_\text{\probloginline{hot}}  \land \phi_\text{\probloginline{temp(not_hot)<25.0}}   \Big)  \nonumber \\
	\land ~~	& \Big(
		  \phi_{\text{\probloginline{cool(1)}}} \leftrightarrow  \phi_{\text{\probloginline{rv_cool(1)=:=1}}} \Big) \nonumber \\
	\land ~~ 	& \Big(
		  \phi_\text{\probloginline{hot}} \leftrightarrow \phi_{\text{\probloginline{rv_hot}=:=1}} \Big)
	\end{align}
Refining the Boolean literals that encode a comparison in \dcproblogsty yields the following SMT formula $\support_g ^\uparrow$:
\begin{align}
	%	\support_g ^\uparrow \nonumber
	%	&\leftrightarrow
		&\Big( \phi_{\text{\probloginline{works(1)}}} \leftrightarrow \phi_{\text{\probloginline{cool(1)}}} 
		\lor \phi_\text{\probloginline{hot}} \land  (temp_{hot}{<}25.0)  
		\lor\neg \phi_\text{\probloginline{hot}}  \land  (temp_{not\_hot}{<}25.0)    \Big) \nonumber \\
	\land ~~ 		& \Big( 
		  \phi_{\text{\probloginline{cool(1)}}} \leftrightarrow (rv\_cool_1=1)\Big) \nonumber \\
		\land ~~ 	& \Big( 
		  \phi_\text{\probloginline{hot}} \leftrightarrow  (rv\_hot=1)\Big)
\end{align}
Here the $\uparrow$ in $\support_g ^\uparrow$ indicates that the formula is the refinement of $\support_g $. The formula $\support_g ^{\uparrow}$ is now an SMT formula over the propositional variables 
$\phi_{\text{\probloginline{works(1)}}}$, $\phi_{\text{\probloginline{cool(1)}}}$  and $\phi_\text{\probloginline{hot}}$, the binary SMT variables $rv\_cool_1$ and $rv\_hot$, and the real-valued SMT variables $temp_{hot}$ and $temp_{not\_hot}$.
\end{example}

\begin{example}[Abstraction of SMT Formula]\label{ex:abstraction}
The process of going from an SMT formula back to a propositional logic formula is called abstraction and we denote it by $\downarrow$. If we have, for instance, the SMT formula $\phi$ defined as 
\begin{align*}
    \phi_{\text{\probloginline{cool(1)}}} \lor \phi_\text{\probloginline{hot}} \land  (temp_{hot}{<}25.0)
\end{align*}
we obtain the corresponding propositional formula $\phi^\downarrow$ as:
\begin{align*}
		 \phi_{\text{\probloginline{cool(1)}}} 
		\lor \phi_\text{\probloginline{hot}} \land  \phi_\text{\probloginline{temp(hot)<25.0}}  
\end{align*}
\end{example}

An obvious query to pose is that of the satisfiability of an SMT formula, \ie to figure out whether there is a satisfying assignment to the variables in an SMT formula (\eg~\citep{demoura2009z3}). An equally important query, however, is the weighted counting problem on SMT formulas, which was dubbed {\em weighted model integration}~\citep{belle2015probabilistic}.
% We include a primer on weighted model integration in Appendix~\ref{app:wmi} and refer the reader to the survey paper of~\citet{morettin2021hybrid} for an overview of the field.
We refer the reader to the survey paper of~\citet{morettin2021hybrid} for an overview of the field.
% Recently, \citet{miosic2021measure} also gave a stringent measure theoretic formulation of weighted model integration. 




Under the assumption that the values of the variables in an SMT formula are distributed according to a given probability distribution, the weighted model integral of an SMT formula is identical to the expected value of the SMT formula with  respect to the probability distributions of its variables.
This observation gives us the following Lemma~\ref{lemma:expt_label2wmi}.

\begin{lemma}[The Expected Label as a WMI Problem] \label{lemma:expt_label2wmi}
Let $\phi$ denote the logic propositional formula derived from a relevant ground program $\dcpprogram_g$ with evidence asserted, and let $\alpha$ be the labeling function as defined in Definition~\ref{def:labeling_function}. Computing the expected value of the label of $\phi$ is a weighted model integration problem.
\end{lemma}
\begin{proof}[Proof.] We start by expressing the expected value of the label of a specific $\omega$ in terms of Iverson brackets (Equation~\ref{proof:label_wmieq:iverson}). We then rewrite the integral by putting the Iverson brackets in the boundary of the integral (Equation~\ref{proof:label_wmieq:boundary1}) and slightly rewrite the boundary (Equation~\ref{proof:label_wmieq:boundary2}).
\begin{align}
    &\E_{vars(\omega) \sim  \dcpprogram_g} [\alpha( \omega )] \nonumber \\
    &=
    \int
        \left( \prod_{i: a_i \in CA^+(\omega)} \ive{  c_i(vars(a_i)) } \right)
        \left( \prod_{i: a_i \in CA^-(\omega)} \ive{  \neg c_i(vars(a_i)) } \right) 
    dP_{vars(\omega)}(\omega) \label{proof:label_wmieq:iverson} \\
    &= \int_{\{ a_i^\uparrow: a_i \in CA^+(\omega) \} \cup \{a_i^\uparrow : a_j \in CA^-(\omega)\} }     dP_{vars(\omega)}(\omega)   \label{proof:label_wmieq:boundary1} \\
    &= \int_{ \{  a_i^\uparrow : a_i \in ( CA^+(\omega) \cup CA^-(\omega) )   \} } dP_{vars(\omega)}(\omega)  \label{proof:label_wmieq:boundary2}
\end{align}


To get the expected value of the label of $\phi$ we sum over all $\omega \in ENUM(\phi)$ (Equation~\ref{proof:label_wmieq:expected_label_formula}). We then rewrite the expected label in terms of the Lebesgue measure $\lambda$ and a counting measure $\xi$ for integer variables (Equation~\ref{proof:label_wmieq:product_measure}). We refer to~\citep[Section 4.2]{,miosic2021measure} for a stringent measure theoretic formulation of weighted model integration. Note how we now explicitly integrate over the function $w_\omega$ representing the probability distributions and which was hidden before in the measure $P_{vars(\omega)}$. Referring again to~\citep{miosic2021measure}, we equate Equation~\ref{proof:label_wmieq:product_measure} with the measure theoretic formulation of weighted model integration (Equation~\ref{proof:label_wmieq:wmi}).
\begin{align}
    &\sum_{\omega \in ENUM(\phi) } \int_{ \{  a_i^\uparrow : a_i \in ( CA^+(\omega) \cup CA^-(\omega) )   \} } dP_{vars(\omega)}(\omega) \label{proof:label_wmieq:expected_label_formula}   \\
    &= \sum_{\omega \in ENUM(\phi) } \int_{ \{  a_i^\uparrow : a_i \in ( CA^+(\omega) \cup CA^-(\omega) )   \} } \weight_{\omega} d( \lambda \times \xi) \label{proof:label_wmieq:product_measure}\\
    &=\int_{\mathcal{M}(\phi^\uparrow)} w d(\lambda\times \xi \times \mu)\\
    &= \lwmi(\phi^\uparrow,w) \label{proof:label_wmieq:wmi}
\end{align}
In the second but last line the integration goes over all models of the SMT formula $\phi^\uparrow$ using the product measure $\lambda\times \xi \times \mu$, where $\mu$ is a counting measure for the Boolean variables in $\phi^\uparrow$ (cf.~\citep[Definition 8]{,miosic2021measure}). This then corresponds to the definition of the measure theoretic formulation of WMI, denoted in Equation~\ref{proof:label_wmieq:wmi} by $\lwmi$ (the `L' stands for Lebesgue integration).
\end{proof}

Lemma~\ref{lemma:expt_label2wmi} has the delightful property of  allowing us to compute the expected value of the label of a Boolean formula by adopting inference algorithms that have been developed as approaches to weighted model integration~\citep{sanner2011symbolic,morettin2017efficient,kolb2018efficient,zuidbergdosmartires2019exact,kolb2019exploit,zeng2019efficient,derkinderen2020ordering,zeng2020probabilistic}.


% This means that we can reduce inference in \dcproblogsty to weighted model integration. 



% \section{Implementation} \label{sec:implementation}
\ak{needs an introduction (and I'd suggest reorganization into algorithms first, then system)}
We make two observations:
\begin{enumerate}
    \item Every implementation of \prologsty  (or any other logic programming language)  reserves a set of {\em  system predicates} executing code not written in pure \prologsty~\citep[Chapter 8]{sterling1994art}.
    System predicates complement pure \prologsty implementations and give users access to efficient algorithms that are cumbersome to implement in pure logic programming -- for instance algorithms for numeric computations.
    \item In Section~\ref{sec:semantics} we have seen that we can separate a probabilistic logic program into a set of independent random choices and a logic program. The independent choices and the logic program communicate to each other via comparison predicates. Comparison predicates are an instance of system predicates.
\end{enumerate}

These observations hint at the possibility of implementing \dcproblogsty as a deterministic logic programming language with access to an external engine with support for random variables.
On an implementation level, the key difference between a probabilistic logic programming language and a (deterministic) logic programming language is the availability of probability distributions in the external engine of the former.

Decoupling independent choices (random variables) and the deterministic system (logic program) does also mean that we should be able to take a probabilistic logic programming language that only supports finite discrete probability distributions and extend it with continuous probability distributions.
We will demonstrate how to achieve this by adding continuous random variables to the ProbLog2 system~\citep{dries2015problog2} (the most recent implementation of \problogsty). We dub our implementation \dcproblogsys.

% First, we will introduce a type system 
% (Section~\ref{sec:type_system}), followed by a description of how a type system aids the extension of the ProbLog2 system with continuous random variables (Section~\ref{sec:external_engine}).
% Followed by a description of the system architecture of \dcproblogsty in Section~\ref{sec:architecture}, which is based on the \problogsys architecture~\citep{dries2015problog2} and which we dub \dcproblogsys.








% To this end we will first introduce a type system (Section~\ref{sec:type_system}), followed by a description of how a type system aids the extension of the ProbLog2 system with continuous random variables (Sections~\ref{sec:multiple_dispatch} and~\ref{sec:external_engine}).
% Lastly, we will also discuss the implementation of ALW (cf. Section~\ref{sec:inference}) and how it uses the existing inference mechanisms already present in the ProbLog2 system.













\subsection{\dcproblogsty System Architecture}
\label{sec:architecture}

\dcproblogsys, is built on top of the existing \problogsys system. This means we inherit the general architecture of performing inference via a number of deterministic program transformations performed on the source program (written in \dcproblogsty). Figure~\ref{figure:dcproblog_sys} summarizes the program transformation steps in \dcproblogsys. These are grounding, cycle breaking, compiling, and evaluating. Conceptually the transformations are identical to the ones performed in the \problogsys implementation.
% We will briefly describe the four transformations and give a more detailed account of the compilation and evaluation steps in Section~\ref{sec:inference} as these are the steps where \dcproblogsys and \problogsys differ most.

\begin{figure}[h]
	\centering
	\includegraphics[width=\linewidth]{figures/dcproblog_sys.pdf}
	\caption[Overview of the program transformation steps in the \dcproblogsys system]{Overview of the program transformation steps in the \dcproblogsys system. (Figure inspired by~\protect\citep[Figure 2]{zuidbergdosmartires2019transforming}.}
	\label{figure:dcproblog_sys}
\end{figure}

%%% moved from elsewhere
\subsection{Conditional \dcproblogsty Programs}
\ak{this is not the right place for the sec; just moved it from the language section as it is about system}

This lets us, for instance, compute the probability of the query in Example~\ref{example:dcproblog_machine}, where we saw that we can express a conditional probability by using the reserved \probloginline{evidence/1} predicate. \dcproblogsty also has a binary predicate \probloginline{evidence/2}, where the second argument is either the \probloginline{true} symbol or the \probloginline{false} symbol. The former is equivalent to using the unary version, while the latter allows one to express negative evidence. These predicates are also present in \problogsty.



While the \probloginline{evidence} predicate allows us to express conditional probabilities where we condition on Boolean random variables, its semantics does not extend directly to conditioning on continuous random variables: a continuous random variable can neither be true nor false. In order to allow a user to condition on continuous random variables in \dcproblogsty, we introduce the \probloginline{observation/2} predicate.

\begin{example}\label{example:dcproblog:observation}
	We model the size of a ball as a mixture of  different beta distributions, depending on whether the ball is made out of wood or metal (Line~\ref{program:dcproblog_machines_observation:ad})\footnote{Annotated disjunctions are used to concisely write down mutually exclusive Boolean random variables. Internally they are compiled down to probabilistic facts and deterministic rules.}.
	We would now like to know the probability of the ball being made out of wood given that we have a measurement of the size of the ball. In order to condition on a continuous random variable we introduce the \probloginline{observation/2} predicate, which has an analogous functionality as the evidence predicates for Boolean random variables.
	\begin{problog*}{linenos}
3/10::material(wood);7/10::material(metal).@\label{program:dcproblog_machines_observation:ad}@

size~beta(2,3):-material(metal)@\label{example:dcproblog:observation:beta23}@.
size~beta(4,2):-material(wood).

observation(size,4/10).
query(material(wood)).
	\end{problog*}
	This \dcproblogsty program encodes the conditional probability:
	\begin{align}
		p(\text{\probloginline{material(wood)}}| \text{\probloginline{size=4/10}})
	\end{align}
\end{example}



%%% end moved from elsewhere

\ak{if we keep this overview, include pointers to the sections that discuss details}
\paragraph{\bf Grounding}

\dcproblogsys is an extension of the \problogsys system: the logic programming components in both systems are identical. This means that we can simply bootstrap the logic grounding engine of the \problogsys system for \dcproblogsys. Grounding is performed using {\em selective linear definite clause} (SLD) resolution~\citep{kowalski1974predicate,sterling1994art}. SLD resolution allows for computing the relevant ground program for a specific query to the logic program, which avoids grounding of unnecessary predicates~\citep{kersting2000bayesian,fierens2015inference}.\ak{what about negation? SLDNF or something else?}

\paragraph{\bf Cycle Breaking} Logic programs may contain cycles. In order to perform inference these need to be removed from the ground program. As these cycles only occur in the logic program part of a \dcproblogsty program we can again bootstrap the cycle breaking available in \problogsys, which is currently a variation of the algorithm proposed in~\citep{janhunen2004representing} \citet{mantadelis2010dedicated} have also proposed an alternative cycle breaking algorithm.

\paragraph{\bf Compiling} The next step that \problogsys performs to compute probabilities is the compilation of acyclic ground programs. Acyclic ground programs have a one-to-one mapping to propositional logic formulas, which are then compiled into decision diagrams in a process dubbed knowledge compilation~\citep{darwiche2002knowledge}. \dcproblogsys differs in that acyclic ground \dcplpsty programs are not mapped to propositional logic formulas but to so-called {\em satisfiability modulo theory} (SMT) formulas (\eg $x>3$)~\citep{barrett2009handbook}, which are then compiled to decision diagrams that support algebraic constraints~\citep{sanner2011symbolic,zuidbergdosmartires2019exact}.


\paragraph{\bf Evaluating} The probability of a query to a (hybrid) logic program is obtained by evaluating the compiled logic formula (decision diagram). While in \problogsys this is performed using the weighted model counting (WMC)~\citep{chavira2008probabilistic} framework, \dcproblogsys uses weighted model integration~\citep{belle2015probabilistic} -- an extension of WMC that allows also for algebraic constraints on random variables.

\bigbreak

We provide the necessary background on knowledge compilation, weighted model counting, weighted model integration, and SMT formulas in~\ref{app:amc}, \ref{app:compilation}, and
\ref{app:wmi}.


\subsection{Grounding}
As already mentioned, grounding in \dcproblogsys is performed by bootstrapping the existing grounding engine of \problogsys. The sole minor difference concerns the grounding of comparison predicates, \eg (\probloginline{1<2}), (\probloginline{x>20}).
During the grounding of a \dcproblogsty program, the types of the comparison's arguments are checked and the comparison is considered to be proven if the types of the arguments match the type signature of the comparison predicate.
The possible types of a comparison's predicate arguments are dictated by the external arithmetic engine. For instance, \prologsty only allows for (deterministic) numbers, in contrast to \dcproblogsty where the external arithmetic engine also allows for comparing random variables.\ak{what if typecheck fails?}

More concretely, if a comparison predicate of the form (\probloginline{x>20}) is encountered during the grounding, the engine checks whether there are distributional facts of the form \probloginline{x ~ Dist} present in the program. If such a clause is present, the type of \probloginline{x} is inferred from the distribution \probloginline{Dist}. The comparison predicate is proven if the external engines allows for comparing the inferred types, \eg can the external arithmetic engine compare continuous random variables to real valued numbers. We give a more detailed account of typing and the external arithmetic engine in \ref{app:types}.


\subsection*{Evaluation of Comparison Predicates}

A distinctive property of comparison predicates in \dcproblogsty is that they have a potential infinite and uncountable number of groundings, \ie an infinite number of possible assignment to the variables in the arguments that would satisfy the comparison\ak{careful with grounding (replacing logical variable by ground terms) vs evaluating (replacing random variable by its value)}.
This cannot to be carried out using a logic grounding engine.
Therefore, we have to defer the grounding of comparison predicates to a later stage of the inference mechanism and at the same time retain a symbolic representation of the set containing all possible groundings. Probabilistic inference then constitutes of {\em measuring} the size of these sets using integration.
We call the grounding of a system predicate also an evaluation, as the external engines performs an arithmetic evaluation in the background.

In the special case of a `deterministic' comparison, \ie a comparison not containing any random variables in its arguments, we can interleave the grounding with the evaluation of the predicate.
That is, we can replace the atom with its corresponding Boolean value. For the comparison (\probloginline{1<2}) we would get \probloginline{true}. Such a type dependent eager evaluation during grounding can be regarded as an instance of multiple dispatch: differently typed predicates induce different run time behavior~\citep{castagna1995calculus}.

Eager evaluation of arithmetic expressions makes most sense when this leads to simplifications of the ground program. More concretely, when a comparison predicate is not satisfied this results in a \probloginline{false} atom and proving a goal during grounding can be stopped pre-maturely.
Eager evaluation of the comparison predicates allows \dcproblogsty to retain the usual behavior of proving and grounding in \problogsty where a proof fails if a (deterministic) comparison is not satisfied.

Depending on the arithmetic engine deployed, eager evaluation is also possible for evaluations of non-deterministic arithmetic expressions. If the evaluation of arithmetic expressions is purely sampling-based, the random variables in an arithmetic expression are replaced by sampled values. For each sample the arithmetic expressions can then be eagerly evaluated. This is the approach taken by~\citet{nitti2016probabilistic}.\ak{we need to be more explicit about this entire "external engine" stuff; right now, remarks about alternative engines are mingled into the description of a pipeline that seems to be fixed to one such engine (sampling here, Prolog earlier)}

% For symbolic arithmetic engines it does also make sense to perform eager evaluations to some degree. Consider the comparison (\probloginline{3+5>2+1}). A purely symbolic external engine would evaluate two symbolic expression trees and wrap the comparison into a \probloginline{holds}\lstinline[columns=fixed]|/1| predicate. Eagerly evaluation the expression tree however lets us replace the \probloginline{holds(3+5>2+1)} with \probloginline{false}.


\begin{example}[\dcproblogsty grounding] \label{example:grounding}
Consider the non-ground \dcproblogsty program below.

	\begin{problog*}{linenos}
0.2::hot.
0.99::cooling(1).

temperature(1) ~ normal(27,5):- hot.
temperature(1) ~ normal(20,5):- \+hot.

works(N):- cooling(N).
works(N):- temperature(N)<25.0.

query(works(1)).
	\end{problog*}
The grounding mechanism in \dcproblogsys is the same one as in \problogsys and only works on pure logic programs. However, the logic program and the random variables are still interleaved. In a first step we remove syntactic sugar and rewrite the distributional clauses and probabilistic facts  as distributional facts, cf. Section~\ref{sec:semantics}.
	
	\begin{dcplp*}{linenos}
rv_1_hot ~ flip(0.2). @\label{line:dist_hot}@
map(rv_hot,rv_1_hot).
hot:- map(rv_hot,R), R=:=1.

rv_1_cooling(1) ~ flip(0.99). @\label{line:dist_cooling}@
map(rv_cooling(1),rv1_cooling(1)).
cooling(1):- map(rv_cooling(1),R), R=:=1.

rv_1_temperature(1) ~ normal(27,5). @\label{line:dist_temp_1}@
map(rv_temperature(1), rv_1_temperature(1)):- hot.

rv_2_temperature(1) ~ normal(20,5). @\label{line:dist_temp2}@
map(rv_temperature(1), rv_2_temperature(1)):- \+hot.

works(N):- cooling(N).
works(N):- map(rv_temperature(N),R) , R<25.0.

query(works(1)).
	\end{dcplp*}
After removing the syntactic sugar we have a \dcplpsty program constituted of distributional facts in Lines~\ref{line:dist_hot}, \ref{line:dist_cooling}, \ref{line:dist_temp_1} and \ref{line:dist_temp2}, and a (non-probabilistic) logic program in the remaining lines. The logic  part of the \dcplpsty program can now be grounded using the \problogsys grounding engine:
	
	\begin{dcplp*}{linenos}
rv_1_hot ~ flip(0.2). 
rv_1_cooling(1) ~ flip(0.99). 
rv_1_temperature(1) ~ normal(27,5).
rv_2_temperature(1) ~ normal(20,5).

map(rv_hot,rv_1_hot).
map(rv_cooling(1),rv_1_cooling(1)).
map(rv_temperature(1), rv_1_temperature(1)):- hot.
map(rv_temperature(1), rv_2_temperature(1)):- \+hot.

hot:- map(rv_hot,), rv_1_hot=:=1.
cooling(1):-
    map(rv_cooling(1),rv_1_cooling(1)), 
    rv_1_cooling(1)=:=1.
works(1):- cooling(1).
works(1):-
    map(rv_temperature(1),rv_1_temperature(1)) ,
    rv_1_temperature(1)<25.0.
works(1):-
    map(rv_temperature(1),rv_2_temperature(1)) 
    rv_2_temperature(1)<25.0.

query(works(1)).
	\end{dcplp*}
As there are no cycles present in the \dcplpsty program above, we do not need to perform cycle breaking on the logic program part.
\end{example}

Note that for, the sake of clarity, we performed the rewriting of a \dcproblogsty program to a \dcplpsty program in Example~\ref{example:grounding} as a separate and preliminary step. In the \dcproblogsys implementation the elimination of syntactic sugar is interleaved with the grounding and performed on the fly. This avoids rewriting parts of the program that are irrelevant to computing the probability of a query.\ak{why/when is grounding on the fly possible?}










%%%%%%%%%%%%%%%%%%%%

















% In probabilistic programming a recent trend has emerged of mixing eager evaluation (usually in combination with sampling based probabilistic inference) and lazy evaluation (also called delayed evaluation). Examples for this are the probabilistic programming languages Birch~\citep{murray2018automated} or the symbolic probabilistic programming language PSI~\citep{gehr2016psi}. Lazy evaluation of arithmetic expressions do also allow to simplify probabilistic programs such that inference becomes tractable after these simplifications. An obvious example for this is then rewriting a conjugate prior-likelihood pair into a single expression. Lazy evaluation combined with symbolic simplifications goes into the direction of compiling probabilistic programs, \eg~\citep{wu2016swift}.


% The key difference between the external arithmetic engine of \prologsty (or \problogsty) and the external arithmetic engine of \dcproblogsty is that the external engine of the latter is capable of performing arithmetic operations (perform evaluations) with random variables. Combining this innovation with multiple dispatch allows us to retain the same syntax (same functor) to perform arithmetic operations on \type{SymbolicRV} terms and/or \type{Number} terms.








% \subsubsection{Arithemtics of Random Variables}

% Evaluating arithmetic expressions that contain random variables should not to be confused with performing arithmetic operations on random variables. Writing \probloginline{x+y} in a comparison predicate, where \probloginline{x} and \probloginline{y} are random variables has the semantics of evaluating both random variables (for instance by drawing a sample) and adding up both evaluations.

% In order to perform arithmetic operations on random variables directly, we would need to introduce syntax and an arithmetic engine for random variables. This could, for example, be achieved in the following fashion:
% \begin{problog*}{linenos}
%  x ~ normal(20,5).
%  y ~ normal(25,6).
%  z ~ x+y.
% \end{problog*}
% In this case z would be a normal distribution with mean ($20{+}25$) and standard deviation ($5{+}6$). We leave arithmetic operations on random variables for future research.\pedro{maybe add citation of somebody who has done something like this already?}















%%%%%%%%%%%%%%%%%%%%







\subsection{Mapping Acyclic Ground \dcproblogsty to Weighted SMT Formulas}

In \problogsys, a query to a probabilistic program is mapped to a weighted propositional formula~\citep{fierens2015inference}. The probability of the query being satisfied is then the weight of the propositional formula. This weight is also called the {\em weighted model count}~\citep{darwiche2009modeling}.
Inference in \dcproblogsys follows a similar principle, with the difference that a hybrid probabilistic program is mapped to a weighted SMT formula, instead of a weighted propositional formula.
The weight of a weighted SMT formula is called the {\em weighted model integral}.
% Weighted model integration (WMI)~\citep{belle2015probabilistic} generalizes weighted model counting in that it also allows for continuous variables.

The probability of a query \probloginline{query(q)} in a \dcproblogsty program is obtained by computing the probability of an equivalent weighted SMT formula using weighted model integration:
\begin{align}
	p(\text{\probloginline{query(q)}})= p(\support_q) = \wmi(\support_q, w_q)
\end{align}
where $\support_q$ denotes the SMT formula and $\weight_q$ the associated weight.

% We first encode a query to a \dcproblogsty program as a weighed logic formula and present then algebraic likelihood weighting (ALW), a probabilistic inference algorithm that computes the weight of a logic formula in the discrete-continuous domain in the context of probabilistic logic programming.

% We just mapped the probability encoded by a \dcplpsty program to a weighted model integral. Inference in \dcplpsty/\dcproblogsty can hence be performed by any algorithm capable of computing the weighted model integral in question.


\begin{example}[Mapping \dcplpsty to WMI] Consider again the \dcplpsty program in Example~\ref{ex:dcproblog_program_without_dc_plpdc}. If we ground the purely logic component of the program we obtain the ground program .
\begin{dcplp*}{linenos}
hot ~ flip(0.2). 
cooling(1) ~ flip(0.99). 
temperature(hot) ~ normal(27,5).
temperature(not_hot) ~ normal(20,5).

machine(1).
works(1):- machine(1), cooling(1)=:=1.
works(1):-
	machine(1),
	temperature(hot)<25.0,
	hot=:=1. 
works(1):-
	machine(1),
	temperature(not_hot)<25.0,
	hot=:=0. 

query(works(1)).
	\end{dcplp*}
	We map the acyclic ground logic program and the Boolean queries to an SMT formula $\support_q$:
	\begin{align}
		\support_q
		&\leftrightarrow \text{\dcplpinline{machine(1)}} \land (\text{\dcplpinline{cooling(1)}}=1) \lor \nonumber \\
		&\phantom{{}\leftrightarrow{}} \text{\dcplpinline{machine(1)}} \land (\text{\dcplpinline{temperature(hot)}}<25.0) \land (\text{\dcplpinline{hot}}=1) \lor \nonumber \\
		&\phantom{{}\leftrightarrow{}} \text{\dcplpinline{machine(1)}} \land (\text{\dcplpinline{temperature(not_hot)}}<25.0) \land  (\text{\dcplpinline{hot}}=0) \nonumber \\
		&\leftrightarrow (\text{\dcplpinline{cooling(1)}}=1) \lor \nonumber \\
		&\phantom{{}\leftrightarrow{}} (\text{\dcplpinline{temperature(hot)}}<25.0) \land (\text{\dcplpinline{hot}}=1) \lor \nonumber \\
		&\phantom{{}\leftrightarrow{}} (\text{\dcplpinline{temperature(not_hot)}}<25.0) \land  (\text{\dcplpinline{hot}}=0)  
	\end{align}
	The weight function $w_q$ is simply obtained by multiplying together the functions corresponding to the random variables declared in the \dcplpsty program.
\end{example}


























\subsection{Correctness of the Transformations}

In Algorithm~\ref{algo:inference_dc_problog} we outline the steps that are performed to map a \dcplpsty program to a weighted SMT formula. They closely follows the steps presented in~\citep[Section 5]{fierens2015inference}. 
\ak{the query should be an inout of the alg as well (note that the Boolean queries in the program are the comparison atoms, not the inference query)}

\begin{mdframed}
	\begin{algo}[\dcplpsty to Weighted SMT] \label{algo:inference_dc_problog}
		The algorithm takes as input a \dcplpsty program \dcpprogram, consisting of a countable set $P$ of ground distributional facts, a countable set of measurable Boolean queries $Q$, and a purely logic program $L$. The algorithm then maps $\dcpprogram$ through the following steps to a weighted SMT formula $(\support_q, \weight_q)$.
		\begin{enumerate}
			\item Ground out the purely logic program $L$ of $\dcpprogram$ with respect to a query $q$ and obtain the ground program $L_g$. This will also ground the Boolean queries in $Q$, leading to $Q_g$.
    		\item Break cycles in the ground logic program $L_g$.
			\item Convert the ground logic program consisting of $L_g$ and the set of measurable queries $Q_g$ to an equivalent SMT formula $\support_q$.
			\item Define a weight function $\weight_q$, which corresponds to the countable set of random variables in $P$.
		\end{enumerate}
	\end{algo}
\end{mdframed}


\begin{theorem}
The probability of a query $q$ to a valid \dcplpsty program $\dcpprogram$ can be expressed as a weighted model integral of a weighted SMT formula $(\support_q, \weight_q)$.\ak{too vague for a thm? why not make it concrete?}
\end{theorem}

\proof{The proof consists of showing that we can formulate the relevant ground program $L_g$ of the program $\dcpprogram$ with respect to the query $q$ as an SMT formula, that we can map the  }



\todo{work this in after Section 2 and 3 are written}

Let us consider the definition of weighted model integration given in \ref{app:wmi} and in particular the formulation given in Equation~\ref{eqn:wmi_sum_ive_weight}.
\begin{align}
	\wmi(\support, \weight) = \sum_{\langle\wsupport_i,\wweight_i\rangle \in \wtuples_f} \sum_{\bvars} \int \ive{\wsupport_i(\xvars, \bvars)} \wweight_i(\xvars) d\xvars 
\end{align}


Comparing the definition of the probability of a \dcplpsty program to the definition weighted model integration, and in particular the form of WMI given in Equation~\ref{eqn:wmi_sum_ive_weight}:
\begin{align}
	\sum_{\langle\wsupport_i,\wweight_i\rangle \in \wtuples_f} \sum_{\bvars} \int \ive{\wsupport_i(\xvars, \bvars)} \wweight_i(\xvars) d\xvars 
\end{align}
We cannot but appreciate the striking similarity between both Equations. The only substantial difference is that Equation~\ref{eq:definition_probability_dcplp} uses Lebesgue integration to marginalize out Boolean random variables, whereas Equation~\ref{eqn:wmi_sum_ive_weight} uses a summation. Other than that it is straight forward to map one equation to the other:

\begin{itemize}
	\item We associate the sum over the models $l \in MOD(L)$ to the sum over different SMT(\lra) formulas $\wsupport_i$ in weighted model
	\item We associate a specific product of Iverson brackets of Boolean queries $\prod_{q \in Q(l)} \ive{q}$ to the Iverson bracket of a specific SMT(\lra) formula $\ive{\wsupport_i}$, which is a conjunction of atomic SMT(\lra) formulas.
	\item We associate the distributions in $ D_l$ with the weight function $\wweight_i$.
\end{itemize}
\todo{end}

































\





\end{document}