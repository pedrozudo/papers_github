\usetikzlibrary{chains}
%\usetikzlibrary{arrows}
\usepackage{pgfplots}

%%%%%%%%%%%%%%%%%%%%%%%%%%%%%%%%%%%%%%%%%%%%%%%%%%%%%%%%%%%%%%%%%

\pgfmathdeclarefunction{gauss}{2}{%
  \pgfmathparse{1/(#2*sqrt(2*pi))*exp(-((x-#1)^2)/(2*#2^2))}%
}

% int node style
\tikzstyle{int} = [circle,fill=white,draw=black,inner sep=0pt,line width=0.6pt,
minimum size=16pt]


% sum node style
\tikzstyle{sum} = [circle,fill=white,draw=black,inner sep=0pt,line width=0.6pt,
minimum size=16pt, path picture={\draw [shorten >=3pt,shorten <=3pt] (path picture bounding box.south) -- (path picture bounding box.north); \draw [shorten >=3pt,shorten <=3pt] (path picture bounding box.west) -- (path picture bounding box.east);}, node distance=1]

% product node style
\tikzstyle{prod} = [circle,fill=white,draw=black,inner sep=0pt,line width=0.6pt,
minimum size=16pt, path picture={\draw [shorten >=6pt,shorten <=6pt] (path picture bounding box.south east) -- (path picture bounding box.north west); \draw [shorten >=6pt,shorten <=6pt] (path picture bounding box.south west) -- (path picture bounding box.north east);}, node distance=1]

% \intnode [tikz-options] {int node name} 
\newcommand{\intnode}[2][]{
  \node[int, #1] (#2) {$\int$};
}

% \sumnode [tikz-options] {sum node name} 
\newcommand{\sumnode}[2][]{
  \node[sum, #1] (#2) {};
}

% \prodnode [tikz-options] {product node name} 
\newcommand{\prodnode}[2][]{
  \node[prod, #1] (#2) {$\times$};
}
\newcommand{\wprodnode}[3][]{
  \node[prod, #1] (#2) {$\color{#3}\times$};
}

% \maxnode [tikz-options] {max node name} 
\newcommand{\maxnode}[2][]{
  \node[circle,fill=white,draw=black,inner sep=0pt,line width=0.6pt,
    minimum size=16pt, node distance=1, #1] (#2) {\scriptsize $max$};
}

% \emptynode [tikz-options] {generic node name}
\newcommand{\emptynode}[2][]{
  \node[circle,fill=white,draw=black,inner sep=0pt,line width=0.6pt,
    minimum size=16pt, node distance=1, #1] (#2) {};
}

\newcommand{\varnode}[3][]{
  \node[circle,fill=white,draw=black,inner sep=0pt,line width=0.6pt, minimum size=16pt, font=\small, #1] (#2) {#3};
}

\newcommand{\varsqnode}[4][]{
  \node[rectangle, rounded corners=3pt,fill=white,draw=#4,inner sep=0pt,line width=0.6pt, minimum size=19pt, #1] (#2) {#3};
}

\newcommand{\evnode}[4][]{
  \node[circle,fill=#4,text=white,inner sep=0pt, minimum size=16pt, #1] (#2) {#3};
}

\newcommand{\evnodevar}[5][]{
  \node[circle,fill=#4,text=white,inner sep=0pt, minimum size=16pt,
    #1] (#2) {#3};
  \node[below=1pt of #2, xshift=1pt] (#2_vn) {#5};
}

\newcommand{\bernode}[3][]{
  \node[circle,fill=white,draw=black,inner sep=0pt,line width=0.6pt,
    minimum size=16pt, #1] (#2) {\tikz\draw circle (3pt);};
  \node[below=1pt of #2, xshift=1pt] (#2_vn) {#3};
}

\newcommand{\contnode}[3][]{
  \node[circle,fill=white,draw=black,inner sep=0pt,line width=0.6pt,
    minimum size=16pt, node distance=1, #1] (#2) {
    \begin{tikzpicture} %these distributions look really distorted, and are the ones I want to include, inside the minipage
      \begin{axis}[
          no markers, scale only axis, domain=-4:4,samples=50, smooth,
          axis lines=none,
          height=10pt]
        \addplot [mark=none] {gauss(0,1)};
      \end{axis}
    \end{tikzpicture}};
  \node[below=1pt of #2, xshift=1pt] (#2_vn) {#3};
}

\newcommand{\catnode}[3][]{
  \node[circle,fill=white,draw=black,inner sep=0pt,line width=0.6pt,
    minimum size=16pt, node distance=1, #1] (#2) {
    \begin{tikzpicture} %these distributions look really distorted, and are the ones I want to include, inside the minipage
      \begin{axis}[
          no markers, scale only axis, domain=-4:4, smooth,
          axis lines=none,
          height=10pt]
        \addplot [mark=none] coordinates { (0,0) (0,10)};
        \addplot [mark=none] coordinates { (-1,0) (-1,7)};
        \addplot [mark=none] coordinates { (-2,0) (-2,3)};
        \addplot [mark=none] coordinates { (1,0) (1,5)};
        \addplot [mark=none] coordinates { (2,0) (2,8)};
      \end{axis}
    \end{tikzpicture}};
  \node[below=1pt of #2] (#2_vn) {#3};
}


% \edge [options] {inputs} {outputs}
\newcommand{\edge}[3][]{ %
  % Connect all nodes #2 to all nodes #3.
  \foreach \x in {#2} { %
      \foreach \y in {#3} { %
          \path (\x) edge [-,line width=0.6pt,#1] (\y) ;%
        } ;
    } ;
}

\newcommand{\upedge}[3][]{
  \path (#2) edge [line width=1pt,->,>=stealth,#1] (#3);
}

\newcommand{\ledge}[4][]{
  \path (#2) edge [line width=1pt,->,>=stealth,#1,edge label=#4] (#3);
}

\newcommand{\weigedge}[4][]{
  \draw[line width=0.6pt,-,#1] (#2) edge node [midway,line width=0.6pt,#1] {#4} (#3) ;
}




%%% Local Variables:
%%% mode: latex
%%% TeX-master: t
%%% End:
